--- Mamãe ficou sozinha com cinco crianças, uma menor do que a outra ---
disse Maria ---, depois da morte do meu pai, nossa casa\footnote{No
  original, \emph{khata,} casebres de camponeses na Ucrânia e
  Bielorrússia.} desabou e os chefes do colcoz nos deram outra, perto da
\emph{tamba}... Mamãe ficou nessa casa, pois quase todas as crianças
incharam e caíram de cama. Não sobrou nada para trocar por comida,
tínhamos só a roupa do corpo e uns farrapos...

Maria calou-se. Petró Semiónovitch também estava calado. Todos estavam
calados. A garota mendiga contou o que todos sabiam e o que muitos já
tinham sofrido, mas, dito em voz alta e infantil, e ainda sob coação,
soou como uma prece, um lamento sobre suas dificuldades e desgraças.
Talvez por estarem muito tempo sem rezar muitos tenham derramado
lágrimas, mas Petró Semiónovitch ficou com o rosto pálido de tristeza e
fúria, e apenas sua cicatriz de espada matizou-se de sangue.

--- Eis o que os sanguessugas burgueses fizeram conosco --- disse ele
pesadamente, por entre os dentes ---, o cerco capitalista... Mas não faz
mal, nós aguentaremos... Não lhes daremos o prazer de se alegrarem com
nossa desgraça... Vamos botá-los na cova --- e levantando bruscamente a
cabeça ---, mas onde se meteu o sujeito que estava sentado perto da
janela, o que lhe deu o pão? Mostre já a sua esmola --- disse à Maria,
estendendo-lhe a enorme palma de sua mão, em que se espetavam os dedos
que pareciam barras de ferro, capazes de, num átimo, estrangular alguém
até a morte.

E nessa mão calosa pelo uso de utensílios de trabalho e de armas ela
depositou o pedaço dо pão impuro do exílio, marrom-escuro, preparado
segundo a receita do profeta Ezequiel.

--- Eu sabia --- disse Petró Semiónovitch ---, esse pão não é nosso,
veio de fora... Ah, não vigiamos o suficiente...

Com efeito, o lugar perto da janela estava vazio. Ninguém vira o
forasteiro ir embora.

--- É preciso correr ao soviete da aldeia --- gritou Petró Semiónovitch.
--- Stepan --- virou-se para o moreninho ---, vá depressa até o soviete,
ligue para Maksim Ivánovitch, o delegado do GPU... Quanto a nós, por ora
procuraremos aqui... Vamos, cinco sujeitos comigo... Você, você, você,
você... --- andando, ele enfiava o dedo no rosto dos frequentadores da
casa de chá e escolhia os mais apropriados para a busca e a perseguição.

Sempre andando, tirou da jaqueta um velho revólver Nagant, à vista de
todos, com a tinta descascada e muitas vezes consertado com suas mãos de
artesão autodidata. O grupo de perseguidores corria em fila espaçada,
sobre a lama e as poças d'água da rua da vila, diante das casas escuras
e em meio aos latidos dos cães.

Nesse meio-tempo, como se esperasse a alvorada, a chuva parou,
recarregando-se até o dia seguinte, quando os moradores famintos sairiam
de suas casas para cuidarem da vida, pública e pessoal. A lua apareceu,
uma lua ucraniana, que ali, na região de Khárkov, onde se sentia uma
grande mistura com a Rússia, poderia não ser tão lustrosa como a lua de
Poltava, mas se distinguia da lua de Riazán por ser menos soturna, mais
luminosa e por seus jogos de luz. Foi sob a luz dessa lua que saíram
correndo à \emph{tamba}, como ali chamavam a estrada para a cidade de
Dimítrov.

--- Deve ter ser metido no \emph{zakáz} --- disse o camponês desdentado,
também escolhido para participar da perseguição ---, e no \emph{zakáz}
não se pode pegá-lo de noite...

\emph{Zakáz},\footnote{\emph{Zakáz} significa ``encomenda'', mas também
  designa uma área florestal onde o desmatamento e a caça são proibidos.}
no jargão local, era a floresta que escurecia depois dos campos.

--- Okhrimienko, você está desorganizando o povo --- Petró Semiónovitch
dizia como em 1918, à moda revolucionária, batendo o maxilar ---, eu não
apenas arrancarei esse contrarrevolucionário do \emph{zakáz}, mas
arrancarei sua pele com minhas próprias unhas, caso ele esteja escondido
lá. Já persegui muita gente, de muita gente já limpei nossa terra
socialista...

Decerto Petró Semiónovitch, agora chefe da brigada, ao longo de sua vida
perseguira muitos homens até a morte. Os intelectuais seguidores de
Deníkin,\footnote{Seguidores do general Anton Deníkin
  (1872\emph{--}1947), líder do exército Branco, opondo-se aos
  bolcheviques.} que assassinaram cruelmente na prisão seu melhor e
único amigo, o operador de metralhadora e xará Petró Luchnó, e os
camponeses partidários de Petliura,\footnote{Membros do exército da
  República Popular da Ucrânia, sob a liderança de Simon Petliura
  (1879\emph{--}1929). Figura controversa, Petliura esteve ligado a
  \emph{pogroms} na Rússia tsarista. Foi morto em Paris por
  Shalom-Shmuel Schwarzbard (1886\emph{--}1938), anarquista e poeta
  judeu nascido na Bessarábia.} que lhe deixaram a marca de espada.
Petró Semiónovitch ainda lembrava o dia em que atacara na \emph{tamba,}
perto da vila Kom-Kuznetsóvskoie ou simplesmente Kuznetsovka, uma
carroça de \emph{petliurovistas}, carregada com os trastes roubados de
uns judeus da cidade de Dimítrov. Sem demora, ele os golpeou com o
sabre, apesar dos pedidos de clemência --- Petró Semiónovitch gostava de
dar golpes com o sabre, eventualmente atirava com o Nagant e quase nunca
com a carabina, pois o que mais apreciava era a luta corpo a corpo ---,
e assim os matou, então chegou a vez de lidar com os trastes dos judeus.
Soltou as plumas dos colchões, despedaçou os vestidos de veludo com
rendas, os lenços, os xales e também uns casaquinhos, e jogou no rio os
cálices e os castiçais de prata: não se interessava pela riqueza... Uma
vez um homem de seu destacamento tentou apropriar-se das tralhas de uns
judeus, e Petró na hora o colocou no paredão. O homem também chorou,
também implorou, mas não passava de uma pulga de cavalo. Por que um
sujeito desses deveria viver? Se é um ladrão e não sabe viver
honestamente, que roube uma peliça ou um cavalo. Mas que serventia teria
para um camponês o colchão de um judeu ou um vestido de veludo com
rendas? Além disso, deixam um cheiro desagradável em casa; nada melhor
que o cheiro de maçãs curtidas e de esterco de vaca, sem aquele fedor de
bombons adocicados... Eis que tipo de homem era Petró Semiónovitch, o
chefe da brigada. Guerreava duramente, mas tinha princípios: as mãos se
enchiam de sangue, mas estavam sempre limpas... No ano anterior, o
\emph{kulák} Mitka, o filho do moleiro, incendiou a estrebaria do
colcoz, e Petró Semiónovitch saiu em seu encalço, seguido por Maksim
Ivánovitch, o delegado do GPU. Petró alcançou Mitka no \emph{zakáz,}
pegando-o pela garganta e, quando Maksim Ivánovitch chegou correndo e
pronunciou seu costumeiro ``mãos ao alto'', não havia ninguém para
render... Redigiram uma ata, autenticaram-na no soviete da vila e a
enviaram a Dimítrov, enquanto o estrangulado foi entregue ao velho
moleiro para ser enterrado. Petró Semiónovitch perseguira e alcançara
muitos homens, mas nunca estivera no encalço do Anticristo, sob a lua de
Khárkov, mais opaca que a de Poltava, porém com mais jogos de luz que a
de Riazán.

Deve-se reconhecer que esses jogos de luz não existiam à toa, pois a
vila de Chagaro-Petróvskoie era bonita mesmo no outono... E o sítio
Lugovoi, onde vivia Maria, a garota mendiga, ficava bem perto. Sua casa
nova, dada pela direção do colcoz no lugar da que desabara, ficava
recuada, com um canteiro à frente, onde no verão colhiam morangos
silvestres e cogumelos. No canteiro só era possível entrar às
escondidas, correndo grande perigo, porque pertencia ao sanatório. O
sanatório situava-se sobre uma colina e sua mãe contava que ali vivia
uma velha que, depois da revolução, ficara muito zangada e tentava
acertar a bengala em qualquer camponês que passasse, e sua filha, uma
senhorinha bondosa e chorosa, continuamente a detinha. Um dia a filha se
descuidou e a velha correu para fora do portão e acertou a bengala em um
camponês que voltava da taberna, Volodka Sentchuk, o qual, embriagado,
deu-lhe um golpe em resposta que fez a alma da velha partir... Depois
disso, a filha dela foi embora e em sua casa organizaram o sanatório
para os trabalhadores de Dimítrov. Junto ao sanatório, ficava um grande
pomar, onde Maria se embrenhava se houvesse maçãs --- comia umas e
levava outras para casa. Bem ali se achava a igreja,\footnote{Durante a
  URSS, muitas igrejas foram desapropriadas, ganhando outros usos, como
  museus e moradias.} agora um depósito do colcoz, perto do clube do
colcoz, de onde se ouviam os ruídos do moinho d'água, e o rio, ao pé da
colina, conduzia-se para a outra vila, Kom-Kuznetsóvskoie. Do outro lado
da \emph{tamba} se avistava a floresta apelidada de \emph{zakáz} e,
depois dela, surgia a vila Popovka. Maria lembrava que, muito tempo
atrás, quando ela era menor que Vássia, Vássia estava no berço como
Jórik e Jórik não existia, sua mãe e seu pai, com trajes de festa e
alegres, a haviam levado a Popovka para visitar seus avós. Eles
caminharam pelo campo e depois pelo \emph{zakáz}. Chegaram a um grande
quintal e, de repente, um porquinho surgiu correndo de um barracão.
Maria se assustou e gritou, mas sua mãe levantou-a e a acalmou. Na casa
de sua avó havia um prato com ovinhos vermelhos, pois era Páscoa. Sua
avó disse:

--- Menininha, diga ``Jesus ressuscitou''\footnote{\emph{Khristós}
  \emph{voskries}, ``Jesus ressuscitou'', é a expressão usada na Rússia
  como felicitação pela Páscoa.} e eu lhe darei um ovinho.

Maria ficou assustada e não disse nada, mas sua avó mesmo assim lhe deu
o ovo. Fazia tempo que tudo isso acontecera. Maria nunca mais foi ver
sua avó, nem sabia se ela e seu avô tinham morrido ou ido embora. Desde
então, seu pai morreu, a fome chegou e, em meio à fome, seu irmão Vássia
crescia. No começo ele era alegre e carinhoso, e Maria passava o tempo
todo em sua companhia, porque Chura, Nikolai e sua mãe tinham outros
afazeres. Mas então a barriga de Vássia começou a inchar, as pernas
ficaram fininhas, e ele passava mais tempo sentado que de pé. Dava um ou
dois passos sobre a estufa\footnote{A estufa (\emph{piétka}) pode ser
  constituída de uma grande estrutura com fogão embutido. Em cima dela,
  costumavam dormir em dias frios.} e sentava. Tornou-se carrancudo e
bravo. Para beliscar Maria já não tinha forças, então a mordia, mas só
de vez em quando. Quando comia alguma coisa, ele ficava carinhoso de
novo. Maria não queria levá-lo para pedir esmola com ela, mas Chura
disse:

--- Leve, ele tem uma aparência doentia, vão dar mais esmola.

Maria não tentou discutir com Chura, pois por essa discussão poderia
apanhar, mas, quando chegaram à casa de chá, deixou Vássia na entrada,
sentou-o em um banquinho num canto, tirou o lenço que vestia e cobriu a
cabeça do irmão. Dessa vez, receberam coisas boas, apesar dos dois
sustos que levaram --- do forasteiro de aspecto citadino e do chefe da
brigada, que tomou o pão oferecido pelo primeiro. Ainda assim, ela
conseguiu as cascas de pão, as sementes de girassol, algumas balas de
goma e, o principal, o pedaço de toucinho. Maria saiu da casa de chá e
viu Vássia no mesmo lugar em que o deixara, como se estivesse dormindo,
só que ele não dormia, seus olhos estavam abertos e viam tudo.

--- Vamos, Vássia --- disse Maria ---, já está tarde, é noite.

--- Não quero --- disse Vássia ---, é longe para ir a pé, o melhor é
ficarmos aqui até amanhã, encoste-se em mim, Maria, e ficaremos
aquecidos.

--- Você é bobo --- disse Maria ---, vão expulsá-lo daqui. Assim que
chegarmos, vamos comer o que eu ganhei, e pode ser que mamãe ou Chura
deem mais alguma coisa.

--- O que você ganhou? --- perguntou. --- Dê o pão, senão não consigo
ir.

--- Vássia, eu ganhei algo melhor --- disse Maria, orgulhosa, e mostrou
o toucinho.

Vássia agarrou o toucinho e enfiou o pedaço inteiro na boca.

--- Por que você é assim, Vássia? --- perguntou Maria, mas depois pensou
e não lamentou. ``Que coma, de nós ele é o que mais sofre.''

Vássia terminou de comer, levantou-se e disse:

--- Vamos para casa.

Eles caminharam pela rua escura, depois pelo campo, atravessaram a
\emph{tamba} e passaram diante do \emph{zakáz}. Do \emph{zakáz} vinha o
ruído de galhos molhados, e pássaros noturnos os assustaram. Mas nem
Maria nem Vássia tinham medo da noite. Os lobos tinham sido extintos
havia tempo, e quem se atrairia por crianças miseráveis? Só se fosse por
diversão, mas, nesses tempos de fome, até malfeitores pararam de se
divertir, perderam o ideal de bandido e se tornaram práticos --- o
melhor seria atirar no comissário encarregado pelo confisco de provisões
ou assaltar um depósito de cereais. Apenas um intelectual
\emph{raznotchinets},\footnote{O \emph{raznotchinets} era um intelectual
  que não pertencia nem à nobreza nem ao povo. Desligado de suas raízes,
  o tipo se reflete em vários heróis de Fiódor Dostoiévski, como na
  personagem de Raskólnikov de \emph{Crime e Castigo} (1866).}
atormentado pelo desejo de compreender a ideia do sofrimento universal e
as razões de Deus para permiti-lo, um admirador do Messias de
Dostoiévski, seria capaz de esfaquear crianças miseráveis por motivos
doutrinários. Porém, devido à revolução, tipos assim morreram em grande
quantidade ou mudaram completamente. Além disso, em seus tempos áureos,
eles se encontravam em lugares em que a histeria religiosa pudesse
aflorar, onde havia mais ícones, e não na tediosa região de Khárkov.
Graças a essas circunstâncias, Maria e Vássia chegaram ao sítio em
segurança: ouviu-se o ruído da barragem do moinho e logo surgiu o muro
do sanatório. Eles bateram na porta, e Chura, abrindo-a, disse:

--- Chegaram... Mamãe já estava preocupada, mas eu disse que vocês
viriam.

A mãe abraçou e beijou Maria e Vássia e perguntou:

--- Crianças, vocês ganharam alguma coisa?

--- Ganhamos --- respondeu Maria.

--- Então, sentem ali no canto, jantem e vão dormir, porque preciso
conversar com Kólia e Chura.

--- Mamãe, ganhei um pedaço de toucinho --- disse Maria ---, mas Vássia
comeu tudo.

--- Não faz mal --- disse a mãe ---, Vássia está fraco, precisa comer.
Jantem, crianças; eu, Kólia e Chura estamos satisfeitos.

Maria e Vássia comeram toda a esmola dada pelos homens, lamentando o
fato de o chefe da brigada ter tomado o pedaço de pão que ganharam do
forasteiro, subiram na estufa e, encostados um no outro, adormeceram. E
a mãe continuou a conversar com seus filhos mais velhos, Chura e Kólia.

--- Não temos uma vaca nem roupa, nem mesmo pão --- dizia a mãe. ---
Pelo trabalho no colcoz neste verão só ganhei dez quilos de centeio, nem
batata temos o suficiente. Não nos restou nada e só vejo duas saídas: ou
morreremos de fome, ou viveremos, mas doentes... Não tenho como
alimentar vocês, meus filhos, por isso resolvi separá-los. Levarei os
pequenos embora, mas vocês, Kólia e Chura, irão trabalhar no campo do
colcoz e terão o que comer.

--- Está certo --- disse Chura ---, se carregarmos Maria, Vássia e Jórik
nas costas, não daremos conta. Quem sabe alguém fique com eles ou algum
orfanato os aceite, e eles consigam sobreviver.

--- Se tiverem que morrer --- disse a mãe ---, que pelo menos não seja
diante dos meus olhos. Não aguentaria ver meus filhos morrerem.

E decidiram: os pequenos seriam levados embora.

Ainda não havia clareado quando a mãe acordou Maria e Vássia, mas Jórik
já tinha sido tirado do berço e enrolado em um cobertor vermelho e
quente. Vássia, evidentemente, não queria se levantar.

--- Está frio lá fora --- dizia ---, o sol ainda não se levantou.

A mãe respondeu:

--- Crianças, vamos para a feira em Dimítrov, talvez eu troque ou compre
alguma coisa, e darei um presente a vocês. Quem sabe eu compre um
raminho com ameixas secas, nozes e balas. Lembram os raminhos que vocês
ganharam depois do enterro de seu pai?

Obediente, Maria não só se levantou como, para ajudar sua mãe, tentou
convencer Vássia a se levantar:

--- Vássia, você se lembra das ameixas secas? Só que temos que nos
apressar, porque a cidade é longe e, se chegarmos tarde, os outros
camponeses levarão tudo.

Saíram com o céu ainda cinza e vazio. Como de costume, passaram diante
do muro do sanatório, da igreja e do moinho; à medida que desciam a
colina até o campo, o céu se iluminava, e sobre o \emph{zakáz} se ergueu
o sol fresco da manhã.

Maria e Vássia andavam de mãos dadas, e a mãe levava nos braços o
pequeno Jórik, enrolado no cobertor vermelho, o que estava mais
aconchegado. Enquanto andavam pelo campo, Vássia várias vezes ameaçou
sentar-se para descansar, pois suas pernas eram muito finas e não
sustentavam o corpo, e a mãe e a irmã, para fazê-lo andar, ora o
ridicularizavam, ora o persuadiam, mas, assim que chegaram à
\emph{tamba}, até Vássia animou-se e começou a andar direito, sem
menear. Nesse ínterim, o sol se afastou do \emph{zakáz}, iluminou todo o
céu e ficou quente, uma revoada de pássaros migratórios sobrevoou, na
esperança de aproveitar as espigas espalhadas, e um inseto de asas
brilhantes saiu voando de debaixo de seus pés e sumiu numa valeta da
estrada. O outono visivelmente não tinha avançado, pois na mesma época
em anos anteriores ainda se podia banhar no rio, os veranistas da cidade
de Dimítrov ainda estavam em suas datchas e faziam conservas com as
frutinhas silvestres que sua mãe, Chura e outras mulheres lhes traziam.
Até Maria lembrava que ela e sua mãe iam colher frutinhas e as vendiam
aos veranistas, uma orquestra tocava no jardim do sanatório, e um
veranista de barbicha ria e dizia algo à sua mãe, que também ria,
afastando-o com um gesto, e uma vez ele pegou na sua mão, mas ela a
arrancou dele e foi para casa com a filha, sorrindo o caminho todo.
Naquela época, sua mãe tinha um rosto bonito e usava sobre os cabelos
pretos um lenço florido, que no inverno anterior haviam trocado por
painço.

O sol ia esquentando, o dia melhorando, o moinho de vento girando
preguiçosamente suas asas de madeira, e as carroças do colcoz com sacos
de cereais saíam da \emph{tamba} em direção ao moinho para saldar o
imposto em espécie cobrado pelo governo; tudo isso claramente perturbou
sua mãe, despertando algo prazeroso nela. Ela suspirou profundamente e
ficou pensativa, mas sem tristeza. Vássia, que fazia algum tempo que
andava com dificuldade, agora relinchava feito um potro num pasto
matutino e corria alegremente à valeta para apanhar um belo inseto e
esmagá-lo. Respirava com facilidade, seu cansaço havia desaparecido. Аs
primeiras casas que surgiram eram de pedra, diferentes das casas das
aldeias.

--- Chegamos, Vássia --- disse Maria, alegre ---, chegamos à feira a
tempo.

--- Não, crianças --- disse a mãe, como se despertasse de um sonho ---,
ainda não é a cidade de Dimítrov, é o povoado de Lípki. Fiquem de mãos
dadas, é tanta gente que vocês podem se perder.

O povoado de Lípki estava apinhado de pessoas e de carroças, e logo as
crianças ficaram com fome. Numa praça perto de uma grande casa de pedra,
via-se uma bandeira vermelha de pano pendurada, protegida do vento, e
sentia-se cheiro forte de painço cozido com banha. Vássia começou a
resmungar, pedindo painço e pão, e Maria disse:

--- Mamãe e Vássia, não fiquem tristes. Agora mesmo vou pedir esmola
naquela casa. Vão me dar alguma coisa.

Mas a mãe replicou:

--- Não temos tempo, crianças. Dimítrov está longe, vamos perder a
feira. É melhor sairmos do povoado, logo ali fica um poço com água tão
limpa que, ao beberem, perderão a fome.

Com efeito, assim que tomaram água, a fome diminuiu, e eles seguiram
adiante. De Lípki a Dimítrov, a \emph{tamba} ficou ainda mais larga e
povoada --- surgiam pessoas com carroças e a pé. De repente Maria
reconheceu num dos transeuntes o forasteiro que lhe oferecera pão na
casa de chá. Ele vestia um sobretudo surrado com mangas curtas e
estreitas que deixavam muito visíveis suas mãos ossudas, e uma espécie
de gorro de pele gasta. Os sapatos, que não tinham nada de especial,
sobressaíam apenas pela resistência e pela grossura das solas, algo raro
nessa época, como se fossem feitas especialmente para caminhadas longas
e frequentes. Seu sobretudo tinha uma gola de veludo, a mesma que os
janotas da aristocracia usavam no início do século e que mais tarde
muitos intelectuais passaram a usar, até os menos abastados. O
forasteiro estava vestido como um homem vivido, entretanto não passava
de um adolescente, quase um garoto. Mesmo que Petró Semiónovitch, o
chefe de brigada, corresse, mesmo com sua vasta experiência na
perseguição e no extermínio dos inimigos do estado socialista, ele não
poderia alcançar esse forasteiro. Para aumentar sua raiva e aflição, não
encontrara dele sequer rastros. O Senhor abandona muitos homens, por
seus pecados, à arbitrariedade do ímpio, abandona até o Redentor,
enviado para a Bênção, por pecados alheios, porém jamais abandonará à
arbitrariedade do ímpio a Áspide, o Anticristo, enviado para a Maldição.
O Anticristo é o juiz do ímpio е também о juiz de tudo o que existe. No
entanto, esse fardo era pesado demais para quem fora enviado pelos Céus,
mas transpunha caminhos terrenos. Não estava em seu poder salvar e
ajudar, mas julgar e executar. Andando pela estrada de Lípki para
Dimítrov, numa manhã ensolarada de outono, Dã da tribo de Dã, o
Anticristo, falava com o Senhor através do profeta Jeremias, cujo
espírito o concebeu, um pai espiritual. E disse o Senhor:

--- Antes que Eu te formasse no ventre materno, Eu te conheci; antes que
saísses do ventre, Eu te santifiquei.\footnote{Jeremias 1:5.}

--- Oh, Deus --- respondeu Dã ---, eu não sei falar, pois ainda sou
muito jovem.

Mas o Senhor disse:

--- Não digas: ``eu sou jovem'', pois irás a quem Eu te enviar e dirás
tudo o que Eu ordenar {[}...{]}\footnote{Jeremias 1:7.} Não tenhas medo,
pois Eu estou a teu lado para te livrar de tudo {[}...{]}\footnote{Isaías
  41:10.}

E ali, naquela estrada movimentada, apelidada de \emph{tamba}, Dã sentiu
Algo tocando em seus lábios e ouviu:

--- Ponho Minhas palavras em teus lábios {[}...{]}\footnote{Jeremias
  1:9.} Levanta a cabeça, olha para o povo que anda à tua volta com suas
preocupações... Eles renunciaram ao Senhor e disseram: ``Ele não existe,
nenhuma desgraça nos atingirá, não veremos a espada nem a
fome''.\footnote{Jeremias 5:12.}

E o Senhor se dirigiu a Dã por meio de outro profeta, Isaías, cujo
espírito concebeu o irmão de Dã, Jesus, da tribo de Judá, o Redentor:

--- Vê, eles irão reproduzir o feno, conceber a palha
{[}...{]}\footnote{Isaías 33:11.} Levanta teus olhos e olha em volta...
Uma mulher pode esquecer o bebê que alimentou ao seio? Pode não se
apiedar do filho que carregou no ventre? Mas ainda que ela dele se
esqueça, Eu não o esquecerei {[}...{]}\footnote{Isaías 49:15.}

Dã levantou a cabeça e viu à sua frente Maria lhe estendendo a mão como
na véspera na casa de chá; um pouco afastada, uma mulher com um bebê nos
braços, ainda jovem, mas castigada pela fome e pelo infortúnio; e também
um menininho, seu filho, em que se exprimiam pavor e esperança. Dã
voltou a tirar de sua bolsa de pastor o pão da fome e do exílio, feito
de uma mistura de trigo e cevada, favas e lentilhas, assado conforme a
receita do profeta Ezequiel, e deu um grande pedaço a Maria. Pela
primeira vez, algo tocou o coração de Dã, e ele se alegrou com seu ato
caridoso, mas o Senhor o preveniu:

--- Não te alegres com a tua bondade, Dã, pois tu não foste enviado para
isso. Esse povo destruiu o jugo de madeira do pescoço e em seu lugar
colocou um jugo de ferro.\footnote{Jeremias 28:13 (adaptado).} Há muita
coisa para ele aguentar antes que sua terra seja desposada.\footnote{Isaías
  62:4 (adaptado).}

O Senhor se calou, e Dã deu as costas a quem oferecera ajuda, afastou-se
a passos largos e logo desapareceu.

Maria, alegre, disse à mãe:

--- Que pedaço grande de pão, vamos dividir. Mamãe, divida em três
partes: uma para você, uma para Vássia e uma para mim, e podemos enrolar
um pouco de miolo no lenço para Jórik chupar.

Vássia estendeu a mão de imediato, para beliscar um pedaço além de sua
cota, antes da divisão, mas a mãe o deteve:

--- Jogue fora esse pão, Maria. Ele é impuro, um homem ruim o deu. Esse
pão não é russo.

--- Mas como jogar fora, mamãe? --- disse Maria. --- Estamos com fome e
não colocamos nada no estômago hoje além da água do poço... Deixe a
gente comer um pedacinho, apenas eu e Vássia.

--- Não, crianças --- disse a mãe ---, é melhor comerem taboa do que
esse pão. A taboa é uma planta comestível e cresce à vontade à beira do
riacho, depois do pântano. Assim que voltarmos da feira, vamos colher
algumas.

A mãe arrancou o pão legado pelo profeta Ezequiel das mãos de Maria e o
jogou bem longe, no campo enlameado do colcoz, amolecido pelas chuvas,
alvoroçando um bando de pássaros, que, no entanto, imediatamente começou
a bicá-lo.

Dã, a Áspide, o Anticristo, viu tudo isso, apesar de estar muito longe
dali, e disse através do fundador das profecias, do primeiro profeta do
Senhor, Amós, o pastor de Técua:\footnote{A cidade natal de Amós, Técua,
  pertencia à tribo de Judá e é identificada ao sul de Jerusalém.}

--- Por isso vos dei dentes esfomeados em todas as cidades, e carência
de pão em todos os lugarejos. {[}...{]} E impedi que chovesse por três
meses antes da colheita {[}...{]}\footnote{Amós 4:6, 7.}

Dã, o Anticristo, como todas as crianças judias, ficava facilmente
ofendido e rancoroso, e guardou rancor da mulher pecadora.

Passava muito do meio-dia quando a mãe e seus três filhos chegaram a
Dimítrov. Maria somente ouvira falar da cidade, mas nunca estivera lá,
Vássia tampouco, mas a mãe já a conhecia bem, por isso, sem perguntar o
caminho a ninguém, chegou aonde queria. Ela parou perto de uma casa
grande e bonita com uma escada de ferro coberta por uma trepadeira
selvagem. Ao redor, numa rua com calçamento de pedra, havia muitas casas
bonitas como essa, e os troncos das árvores haviam sido revestidos de
cal até a metade, como faziam nos casebres das aldeias. Carroças
passavam sem parar pela rua que levava até a feira, e o calçamento
estava abundantemente coberto pela palha que caía delas. A mãe juntou
uma braçada da palha caída na rua, forrou um banquinho perto da casa e
disse:

--- Crianças, sentem e esperem por mim aqui. Suas perninhas estão
doendo, vocês estão cansadas, e na feira está um tumulto, gente por todo
lado. Vou até lá, compro as ameixas secas e as balas, e volto para cá.

Vássia sentou-se no mesmo instante, sem que ninguém precisasse insistir,
e Maria sentou-se ao seu lado com Jórik nos braços. A mãe foi embora
rapidamente, sem sequer beijar as crianças, para não levantar suspeitas
de que estivesse se despedindo, de que iria abandoná-las. No início foi
agradável ficar ali, a palha estava macia e o sol os esquentava, e eles
também se alegravam ao imaginar que sua mãe traria ameixas secas da
feira. Eis que o vento começou a soprar, anunciando a noite, e as
carroças começaram a se arrastar na direção oposta da feira, quase todas
sem mercadoria, que havia sido vendida. Um cachorro magro correu até o
banquinho onde estavam sentados, assustando Vássia, mas não havia nem
sinal da mãe ou das ameixas que ela traria. Vássia ameaçou várias vezes
desatar no choro, mas Maria tentava acalmá-lo, dizendo que estavam em
uma época de fome e que achar boas ameixas secas era difícil e levava
tempo. No entanto, quando o pequeno Jórik começou a gritar em seus
braços, ela caiu em desespero. Jórik estava doente, cheio de erupções, e
faminto, precisava comer, mas Maria não tinha nada para dar, nem a ele
nem a Vássia. Ela mesma sentia as entranhas doendo de fome e também se
desfez em lágrimas, pois não podia substituir sua mãe, nem para Vássia
nem para Jórik. Assim eles estavam, ali sentados, chorando, e Jórik
começou a se debater, desenrolando o cobertor vermelho que o protegia.
Então uma porta se abriu, e um homem de óculos saiu e perguntou:

--- Vocês são de onde, crianças, e por que estão chorando?

--- Nós viemos do sítio Lugovoi --- disse Maria.

--- E onde estão seus pais? Seu pai e sua mãe? --- perguntou o homem de
óculos.

--- Nosso pai morreu no ano passado --- disse Maria ---, foi um ano de
fome. Mamãe ficou sozinha com cinco crianças. Depois da morte do meu
pai, nossa casa desabou e os chefes do colcoz nos deram outra, perto da
\emph{tamba}...

--- Claro, claro --- impaciente, disse o homem de óculos, interrompendo
a fala de Maria, que evidentemente não o interessava ---, e qual é o
sobrenome de vocês? Qual é o nome de sua mãe?

--- Nós não sabemos --- disse Maria ---, só sabemos que somos chamados
na aldeia de ``filhos da cidadã''.

Nesse momento, uma mulher muito bonita, vestindo uma camisa masculina e
uma gravata, apareceu na porta:

--- Pável, o que aconteceu?

--- Largaram umas crianças aqui... Ligarei agora mesmo para o orfanato.

--- Então convide-as para entrar --- disse a mulher. --- Já estão
espiando das janelas, vão pensar que as estamos maltratando... Entrem,
crianças --- acrescentou, abrindo a porta.

Com Jórik chorando em seus braços, Maria entrou no vestíbulo, onde havia
muitas roupas penduradas e cheiro de algo apetitoso. Era cheiro de
naftalina, mas, para Maria, qualquer cheiro, mesmo о de Jórik, lembrava
algo do \emph{kvás},\footnote{Bebida obtida da infusão de levedura e pão
  de centeio torrado.} lembrava o que ela comera e bebera na casa de sua
avó na vila de Popovka, na Páscoa. Do vestíbulo ascendia uma escada de
madeira com corrimão, pintada de verde e muito íngreme. Vássia deu dois
passos com suas perninhas finas e se agachou, pois a barriga inchada o
atrapalhava. Maria, porém, cochichou a ele:

--- Vamos para cima, Vássia, talvez nos deem alguma coisa para comer,
pão ou \emph{borsch}\footnote{Sopa de origem ucraniana feita com
  beterraba, repolho, batata e carne.} de ontem, que não farão falta a
niguém.

Ela ouvira certa vez de uma velha mendiga de sua aldeia,
Chagaro-Petróvskoie, que nas casas ricas da cidade davam \emph{borsch}
aos pobres, pois cozinhavam tamanha quantidade que as sobras eram
jogadas fora, e a velha não raro conseguia tomar um prato de sopa
excedente. Só que não lhes deram nem \emph{borsch} nem pão; enquanto
subiam, Vássia com as pernas finas e Maria com o pesado Jórik, a mulher,
na certa, teve tempo de tirar o \emph{borsch} da mesa, colocando livros
em seu lugar. O homem telefonou para alguém --- Maria já tinha visto um
telefone, havia um no soviete da aldeia. Num relance, como se morasse na
casa em frente, surgiu uma mulher zangada de cabelo curto, abriu o
cobertorzinho de Jórik de forma habitual e rude, examinou-o е perguntou
seu nome e sobrenome. Maria disse o nome e, em vez do sobrenome, começou
a contar a história da casa que desabou. Mas a mulher não se pôs a
ouvi-la, pegou Jórik e saiu.

--- Então, agora podem ir para casa --- disse o homem de óculos.

--- Não, não podemos ir para casa --- disse Maria ---, nós queremos ir
para a feira. Nossa mãe está lá. Como chegamos até a feira?

--- É muito simples --- disse o homem animadamente ---, mais simples
impossível. --- Sigam pela rua sempre à esquerda, atravessem a praça, e
lá acharão a feira.

Ele conduziu apressadamente Maria e Vássia pela escada de madeira e
fechou a porta atrás deles.

Maria e Vássia foram em direção à feira, achando-a rapidamente, e
procuraram sua mãe por longo tempo, mas ela não estava lá. Embora
tivesse anoitecido e as carroças se retirassem aos poucos, ainda havia
muitos sacos de painço e cebolas em tranças, e uma velhinha, parecida
com a mendiga da vila de Chagaro-Petróvskoie que contara à Maria sobre
as sobras de \emph{borsch} que recebia das casas ricas, vendia ameixas
secas ajeitadas em montinhos sobre um pano feito de saco. E ali, pela
primeira vez, passou pela cabeça de Vássia a ideia de roubar.

--- Vou pegar um monte de ameixas, com as duas mãos --- disse ele ---,
e, mesmo que minhas pernas sejam fracas, a vendedora é velha e não me
alcançará.

--- Deus me livre --- respondeu Maria. --- É um pecado grande. Não quero
ouvir nem mais um pio sobre isso. E também não conseguirá se safar. A
velha não alcançará você, mas fará um escândalo e outras pessoas o
pegarão. Você sabe como batem nos ladrões? Uma vez eu vi baterem num
cigano na nossa vila...

--- E por que mamãe não comprou as ameixas? --- disse Vássia. --- Para
não roubarmos?

--- Provavelmente porque na feira quiseram dar pouco pelo xale que ela
trouxe para vender --- disse Maria ---, e é um bonito xale de lã. O pai
deu o xale para mamãe no casamento. Ela deve ter ficado com pena de
vender por pouco e foi tentar algo nas casas ricas. Vamos dar uma volta
na cidade, Vássia, talvez achemos mamãe.

Dimítrov era uma cidade grande e bonita. Lá havia um bulevar cercado por
um gradil que, embora fosse de ferro, era baixo, e mesmo Vássia, com uma
ajudinha, conseguiria passar por cima dele. Inúmeras lâmpadas elétricas
iluminavam grandes vitrines com diversas mercadorias, roupas e sapatos,
mas não produtos alimentícios, pois era um ano de fome e os alimentos só
eram entregues aos moradores que tivessem cupons de racionamento. Os
transeuntes eram estranhos, desconhecidos, por isso, quando Maria
vislumbrou na multidão, perto do correio central, bem no centro da
cidade, o forasteiro que conhecia, cochichou a Vássia:

--- Veja, é o moço que nos deu pão duas vezes. Vamos, pode ser que nos
dê de novo. Nem mamãe nem o chefe da brigada estão aqui para tirá-lo de
nós, e poderemos matar a fome.

Na frente do correio central havia uma fonte de antes da revolução,
escurecida, com esculturas de crianças nuas montadas não em cavalos, mas
em sapos de cujos focinhos saíam jatos de água. Ao lado, via-se um ídolo
recém-esculpido em granito que fora fixado em um pedestal de pedra, de
modo que o pesado monolito ainda não tivera tempo de se amalgamar ao
solo que o sustentava, como acontece com os antigos ídolos das cidades
pagãs.

Durante sua breve permanência na terra, Dã da tribo de Dã, o Anticristo,
entendeu que estava entre pagãos cuja crença tinha sido adotada
recentemente ou atravessava um período de florescimento, pois havia
ídolos em demasia, fundidos em metal, talhados em madeira, esculpidos em
pedra, e também muitas figuras desenhadas ao redor. Os ídolos eram
variados, porém а imagem mais recorrente era de um homem bigodudo com as
maçãs do rosto tipicamente asiáticas,\footnote{Trata-se de Joseph
  Stálin.} lembrando os ídolos da Babilônia, cuja idolatria fora
censurada pelo profeta Jeremias... Dois grandes profetas, dois inimigos
dos ídolos, Isaías e Jeremias, preveniram o povo dos perigos da
veneração, mas o povo não os compreendeu.

--- Quem fabricou um deus, esculpiu um ídolo sem nenhum proveito? ---
exclamava Isaías com amargor. --- O ferreiro faz um machado e trabalha
sobre brasas; lhe dá forma com seu martelo e o manipula com seus braços
fortes, até ficar sem forças e faminto; ele não bebe água e se esgota.
Depois de escolher a madeira, o carpinteiro estende uma linha sobre ela
e, com uma ferramenta pontiaguda, delineia o contorno do rosto e o
aprimora com o buril, criando a imagem de um homem de belo semblante
para ser colocada em casa. Ele corta os cedros, arranca um pinheiro e um
carvalho, os quais escolhe entre as árvores na floresta, e planta um
freixo que a chuva fez crescer. Isso tudo serve de combustível ao homem,
que usa uma parte para se aquecer, para acender o fogo e assar seu pão.
Com a outra parte faz um deus e se curva, fabrica um ídolo e se ajoelha.
Uma parte da madeira ele queima no fogo, usa para preparar seu alimento,
faz um assado е come até se saciar, então se aquece e diz: ``Eu me
aqueci bem, senti o fogo''. E com аs sobras da madeira ele constrói um
deus, seu ídolo, e o adora, curva-se e faz uma prece, dizendo:
``Salva-me, pois tu és meu Deus''.\footnote{Isaías 44:10, 12--17.}

Não, o paganismo e a idolatria não são novidade nesta terra. Dã da tribo
de Dã, o Anticristo, vira num templo da cidade muitos velhos ajoelhados,
curvando-se à imagem, esculpida em madeira, de um monge eremita de
Alexandria pregado numa cruz que sacrificara sua carne devido à falta de
fé e era, por algum motivo, chamado pelo nome do Irmão de Dã, Jesus, da
tribo de Judá, robusto como seu ancestral, o jovem leão de Judá com
olhos ardentes como de seus irmãos Macabeus, Jesus, que pereceu pelas
mãos de idólatras de sua terra e de fora, assim como, sete séculos
antes, perecera seu profeta, Jeremias, que propusera quebrar a espinha
do ímpio por resignação. Nesse templo, em meio a estalos de incontáveis
velas e de cânticos solenes, olhando para esses velhos ombros curvados,
Dã da tribo de Dã pensava com amargor através do profeta Isaías:

--- Isso não atingirá seus corações, eles não têm conhecimento ou
inteligência para dizer: ``Eu queimei metade no fogo, sobre as brasas
fiz o pão, assei a carne e a comi; e com as sobras eu faria uma
indecência? Eu me curvaria a um pedaço de madeira?''.\footnote{Isaías
  44:19.}

Dã sabia que mesmo os primeiros cristãos, dos dois primeiros séculos do
cristianismo, apesar de terem muito de não divino, apesar de seu
paganismo, jamais cultuaram imagens e ídolos. No momento em que
começaram a cultuar a imagem do monge descarnado de Alexandria,
exatamente nesse instante, aconteceu a substituição, e o cristianismo
tornou-se inimigo de Cristo. Mas, se em épocas imemoriais substituíram o
Senhor, que tinha carne mas não forma, por elegantes ídolos gregos de
madeira, osso e mármore, agora passaram a substituir o Criador por
grosseiros ídolos babilônicos, feitos de materiais pesados --- metal ou
pedra. No entanto, esse processo é coeso e já dura mais de mil e
quinhentos anos, sendo essencialmente o mesmo. Só que a idolatria grega,
bela e refinada, conservada apenas em alguns lugares pelos mais velhos,
foi superada pela idolatria babilônica, com ídolos expostos em praças,
atraindo multidões de jovens que ensinavam até crianças a
reverenciá-los, e nessa noite muitas dessas crianças corriam em volta da
fonte e diante do ídolo bigodudo com as maçãs do rosto tipicamente
asiáticas, recentemente fixado em seu pedestal. Mas crianças são
crianças e, passado o primeiro susto do rosto de pedra severo e
endeusado, sentiram vontade de correr e de fazer travessuras. Nessas
brincadeiras infantis está o germe do que é próprio do Senhor, do que
Deus ensinou ao homem no sétimo dia da criação, porém a fome excessiva
destrói tudo o que é pueril --- uma criança faminta é como um velho
sábio; mas, se este existe unicamente porque pensa, os pensamentos de
uma criatura faminta se direcionam para um único ponto: como conseguir
pão? Movida por tais pensamentos, Maria aproximou-se de novo de Dã,
estendendo a mão para ganhar uma esmola, mas no ato sua mão foi agarrada
por um representante da lei, cujo posto de observação da ordem se
localizava ao lado do ídolo recém-instalado, onde qualquer mendicância,
jogos de azar e outras desordens eram proibidos.

--- Menina, você é de quem? --- perguntou o policial com firmeza, mas
sem raiva. --- Onde estão seu pai e sua mãe?

--- Nosso pai morreu no ano passado --- disse Maria ---, foi um ano de
fome. Mamãe ficou sozinha com cinco crianças, uma menor do que a outra.
Depois da morte do meu pai, nossa casa desabou, e os chefes do colcoz
nos deram outra, perto da \emph{tamba}. E mamãe ficou nessa casa, porque
quase todos estavam inchados e doentes.

--- Deixe a menina ir, camarada policial --- disse uma mulher, piedosa.

--- Eu não a estou segurando --- disse o policial ---, mas onde será que
ela mora? Onde você mora? Conhece o caminho de casa?

--- Conheço --- disse Maria, afoita ---, por Deus se conheço... O sítio
Lugovoi... É preciso ir sempre pela \emph{tamba}, sem nunca virar. A
gente passa pelo sanatório, pela igreja, depois pelo clube e pela
escola; há um riacho no pé da colina e também um moinho d'água. Ao lado
do moinho fica o canteiro onde no verão colhemos frutinhas silvestres e
cogumelos. E atrás do canteiro fica a nossa casa.

--- Então, vá para casa ---, disse o policial, que, mesmo sem crianças
miseráveis, tinha trabalho até o pescoço ---, vá logo e diga a sua mãe
que, se ela mandar você pedir esmola outra vez, eu prenderei tanto uma
como outra.

--- Muito bem --- apoiou um voluntário, no meio da multidão ---, em
lugar de trabalharem no colcoz, ficam mendigando e roubando, feito
ciganos.

--- Apenas não fale em nacionalidade, pois entre nós todas as nações são
iguais.

--- Desculpe pelo meu erro ---, disse apressadamente o voluntário,
embrenhando-se na multidão.

Impedida pela terceira vez de comer o pão legado pelo profeta Ezequiel,
mas contente em se ver livre, Maria pegou seu irmão faminto pelo braço e
foi embora, faminta também.

Vendo tudo isso, Dã da tribo de Dã, o Anticristo, passou a língua nos
lábios e sentiu um amargor. E ele disse através do profeta Jeremias:

--- Mais vale uma vasilha útil, da qual o dono da casa irá se servir,
que deuses falsos, ou mais vale uma porta em casa que proteja seus bens
que deuses falsos {[}...{]}\footnote{Baruc, Cartas de Jeremias 1:58.}

Isso foi dito pelo profeta que adorava o Senhor e, conforme о
entendimento atual, significa:

--- Se não houver forças para acreditar no Senhor, até o ateísmo é
melhor do que a idolatria. Mas um ateísmo saudável, maternal. O ateísmo
tolerado pelo Senhor só é acessível a trabalhadores honestos, de alma
endurecida, ou, ao contrário, a sábios passivos e contemplativos. Ou
seja, o verdadeiro ateísmo está ao alcance de poucos. Desde os tempos
imemoriais deste país e deste povo, os ateístas são tão poucos quanto os
que acreditam no Senhor. Mas muitos são os cantores de salmos apáticos e
os idólatras exaltados.

E Dã disse consigo:

--- Seus profetas profetizam mentiras e seus sacerdotes prevalecem por
meio deles, e o povo adora isso. O que farão depois de tudo, renegados?
Será que minha alma não se vingará de um povo como esse? O maravilhoso e
o terrível coexistem nesta terra...

Assim dizendo, Dã, o Anticristo, virou a esquina do correio central e se
afastou ao longo de uma travessa fracamente iluminada por escassos
lampiões.

Temendo perguntar o caminho a alguém e serem pegos de novo, Maria e
Vássia perambularam um bom tempo sob o entardecer da cidade, até
conseguirem encontrar a \emph{tamba}.

--- Bem, agora acharemos nossa casa ---, disse Maria, alegre. --- Iremos
sempre pela \emph{tamba,} sem desviar, até o \emph{zakáz}.

De novo essas pobres crianças andaram à noite sem nenhuma proteção, de
novo a ninguém essa vulnerabilidade incomodou, de novo a lua do céu de
Khárkov iluminou-lhes o caminho. Só que dessa vez o caminho era muito
longo e, ao chegarem à vila de Lípki, estavam esgotados. Como de
costume, Vássia começou a chorar e a lamuriar.

--- Maria, vamos passar a noite em algum saguão, na escada. Ou achamos
um banquinho numa viela sem vento. Encostaremos um no outro e dormiremos
até o sol nascer. E de manhã continuaremos.

--- Não, Vássia, Deus nos livre --- respondeu Maria ---, talvez mamãe já
tenha voltado para casa e, se não nos achar, ficará preocupada. Vamos,
falta pouco. O tempo que levamos para chegar até Lípki através do campo
do colcoz, onde mamãe jogou fora o pão que o forasteiro nos deu, é o
mesmo que levaremos andando pelo campo até o nosso riacho, e logo ali
estão o \emph{zakáz}, o moinho, a igreja e o sanatório. Assim que
passarmos pelo sanatório, veremos a nossa casa.

Maria convenceu o irmão e eles seguiram adiante, cansados, famintos e
indefesos. À noite tudo parecia diferente. O campo do colcoz parecia
mais exposto ao vento, as margens do rio não se distinguiam da água, o
\emph{zakáz} surgia como uma nuvem escura e densa, e eles mesmos estavam
tão pequenos e solitários que seriam um alvo tentador para uma criatura
mal-intencionada, que não veria nessa miséria um empecilho, pois sua
única recompensa é o sofrimento humano; de modo que, se não estivessem
em Khárkov, onde o ímpio vestia botas besuntadas de breu e não tinha um
semblante pálido, inspirado e imaginativo, é pouco provável que as
crianças chegassem até sua casa. Mas elas chegaram. Bateram na porta uma
vez, depois outra. Chura, sua irmã, abriu, fitou-os severamente e disse:

--- Onde vocês deixaram Jórik?

--- Uma tia veio e o levou --- respondeu Maria.

--- Mas vocês sabem --- perguntou o irmão, Kólia --- que mamãe se
recrutou e quer ir embora?

--- Ir embora para onde? --- perguntou Maria.

--- Isso não sabemos --- respondeu Chura ---, mas, já que vocês
voltaram, deitem-se no canto e durmam.

Maria e Vásia se deitaram perto da estufa gelada, no chão de terra,
encostaram um no outro, aquecendo-se como podiam, e dormiram, exaustos.
O sol ainda não havia se levantado quando sentiram alguém os sacudindo:
levantem! Maria levantou-se de um salto, pensando que fosse Chura com
alguma repreensão, pois tinha medo dela, mas não era Chura, mas sua mãe,
que parou diante deles vestindo um casaco acolchoado e segurando uma
trouxa.

--- Venham se despedir, crianças --- disse ela. --- Eu vou partir.

Ela beijou Maria e Vássia, ainda muito sonolento, beijou Chura e
Nikolai, e partiu. Maria não voltou a dormir, mas Vássia caiu no sono
novamente. Bastou, porém, o sol se levantar e ela o acordou.

--- Chega de dormir --- disse. --- Agora vamos cuidar de nosso sustento.

Ao saírem de casa, ainda fazia frio e os galos da vila de
Chagaro-Petróvskoie chamavam um ao outro. Maria e Vássia chegaram até a
\emph{tamba}, passaram diante do pântano e desceram a colina até a beira
do rio. A neblina ainda pairava na água agitada, e o local era úmido e
desagradável, mas lá havia planta comestível, a taboa.

--- Vá arrancando, Vássia --- disse Maria ---, pegue um punhado e
arranque assim --- continuou ela, mostrando como se fazia. --- Vamos
pegar bastante, o quanto conseguirmos levar, porque nem tudo da taboa é
comestível, uma parte é jogada fora.

Enquanto Maria e Vássia colhiam, a neblina dispersou, e ficou mais
quente. Voltaram com as folhas de taboa para casa e acomodaram-se num
canto ensolarado; Maria começou a limpá-las da película não comestível e
dos galhos secos, e o que podia ser comido ela separava para Vássia e
para si. Eles comeram até se fartarem, então começaram a refletir.

--- Escute, Vássia --- disse Maria ---, vamos para Dimítrov, até a
estação, pois já conhecemos o caminho.

--- Vamos --- respondeu Vássia.

--- Mas teremos que correr o caminho todo --- continuou Maria ---, pois
tenho medo de não encontrarmos mamãe... Concorda?

--- Sim! --- respondeu Vássia.

Saíram correndo, e correram o caminho todo, mas dessa vez a estrada lhes
pareceu mais curta, talvez por terem comido taboa à vontade e se
sentirem mais fortes. O sanatório, o moinho, a igreja e o \emph{zakáz}
ficaram para trás sem que notassem. Só pararam para tomar fôlego um
pouco antes de Lípki, no campo do colcoz, então continuaram a corrida.
Passaram pelo campo em que a mãe jogara fora o pão dado pelo forasteiro,
e Lípki ficou para trás... Eis que surgiu Dimítrov.

--- Tia --- disse Maria a uma moradora ---, qual é o caminho mais rápido
para a estação?

--- Ora essa --- disse a mulher, sorrindo ---, vai perder o trem?

--- Eu não sei o que é um trem --- respondeu Maria ---, mas nós
precisamos chegar até a estação depressa.

--- Se você não sabe o que é um trem, como sabe o que é uma estação?

--- Estação é onde as locomotivas apitam --- respondeu Maria.

--- Claro --- riu a mulher ---, o que é um trem você não sabe, mas o que
é uma locomotiva você sabe? --- e, continuando a rir, indicou a Maria e
Vássia o caminho para a estação.

Maria e Vássia atravessaram os trilhos e avistaram sua mãe sentada num
banquinho ao lado de sua trouxa. Eles correram até ela е ela levantou os
braços e começou a beijá-los entre lágrimas; foi com eles até a
lanchonete da estação e comprou-lhes pãezinhos. Maria e Vássia
terminaram de comer e a mãe disse:

--- Agora, crianças, corram para casa, antes que escureça.

Nesse momento, Maria e Vássia se desfizeram em lágrimas e pediram que
não fossem mandados embora com tanto desespero que chamaram a atenção
das pessoas em volta. Então a mãe disse:

--- Não chorem, crianças, fiquem sentadas perto de mim, eu não vou
mandá-las embora --- então ela se dirigiu a uma mulher, que também
vestia um casaco acolchoado, mas portava um bauzinho em vez de trouxa:
--- Sei que é proibido, mas eu não posso mandá-las para longe de mim.
Meu coração não suportaria.

--- Sim --- assentiu a mulher do bauzinho ---, uma mãe é sempre uma mãe
para seus filhos.

Maria e Vássia sentaram-se perto de sua mãe e se encostaram nela,
estavam felizes. No entanto Vássia olhava para os lados, curioso.

--- Oh, que montanhas grandes --- disse ele, apontando aos vagões.

--- Não são montanhas --- explicou a mãe ---, são vagões de areia.
Crianças, aqui não é como no sítio, aqui o perigo nos espreita em todo
lugar e pode nos esmagar. Nosso embarque será à noite, por isso, Maria,
fique de olho em Vássia. Você vai embarcar com ele, separada de mim, e
depois nos encontramos no vagão. Senão o recrutador irá notar e me
proibir de levar vocês.

Realmente, logo que escureceu, uma sensação de medo invadiu a estação.
Havia muitas pessoas, todas se empurravam e corriam, as locomotivas
apitavam, uma confusão generalizada, e ninguém se preocupava com
ninguém. E entrar no trem era ainda mais assustador. Quando ele
apareceu, todo de ferro, Vássia se apavorou, fincou os pés no chão,
tremendo, relutando em ir. Oh, como Maria sofreu para fazê-lo entrar no
vagão, mas ali dentro, apesar de apinhado, a mãe logo os localizou.
Colocou Vássia ao seu lado, no banco, e disse à Maria:

--- Esconda-se embaixo do banco.

Maria se enfiou embaixo do banco, mas ali era mais confortável, havia
menos gente, e sob o piso se ouvia algo batendo, como dois martelos numa
forja, mas não harmoniosamente, como ferro contra ferro, mas ferro
contra madeira. Bateu, bateu, então começou a apitar, depois a chiar,
até que Maria dormiu. Acordou quando sua mãe lhe passou uma chaleira de
lata.

--- Tome um pouco de água, filhinha.

Maria bebeu e voltou a adormecer. Estava dormindo quando, de repente,
sentiu, no sono, que algo ruim e terrível lhe acontecia. Ela acordou,
espiou ao redor e, nesse instante, os dedos de alguém agarraram seu
ombro, machucando-a, e a puxaram de debaixo do banco.

--- Mas resolveu escondê-la debaixo do banco! --- um vulto na escuridão
começou a gritar para a mãe, que se mantinha calada, com a cabeça baixa
e ar culpado. --- Eu avisei... Eu a proíbo de levar as crianças ---
disse ele ao sair.

--- Quem é esse? --- perguntou Maria.

--- É o recrutador --- respondeu a mãe ---, ele estava passando e viu
Vássia do meu lado. Oh, que desgraça, que desgraça --- ela ficou aflita,
mas não obrigou Maria a voltar para debaixo do banco, e seus filhos
dormiram o resto da noite sobre seus joelhos.

De manhã chegaram a Khárkov. Meu Deus, que riqueza! Maria e Vássia não
acreditariam que algo assim existisse se alguém lhes contasse. Dimítrov
era uma cidade grande e bonita, mas, diante de Khárkov, parecia uma
aldeia ou um sítio. Passaram com a mãe por algo que se assemelhava a uma
porta, mas se viram num lugar que não era nem uma casa, nem uma rua. Um
céu de vidro surgiu sobre eles, árvores estranhas cresciam em vasos de
madeira, entre as árvores ascendia uma escada de pedra branca e
reluzente, ao redor tudo brilhava, e Maria viu em um minuto mais gente
do que havia visto durante toda a sua vida. Maria e Vássia logo se
alegraram, querendo ver e tocar tudo. Maria pegou o irmão pela mão e
eles subiram correndo a escada branca e reluzente; em cima o piso era de
ladrilhos cor de framboesa e escorregava como gelo. Vássia, que no
inverno gostava de deslizar pelo morro, correu e caiu, mas, em vez de
chorar, começou a rir. Maria correu atrás dele e caiu, rindo também.
Assim ficaram eles, correndo e caindo, até que Maria inventou outra
brincadeira --- ela correria em volta dos vasos de árvores e ele deveria
alcançá-la. Mas é preciso notar que, por mais que Maria brincasse, de
tempos em tempos ia até o corrimão e olhava para baixo, certificando-se
que sua mãe continuava sentada no banco, ao lado de sua trouxa. Todas as
vezes em que ela foi até o corrimão, viu sua mãe sentada no mesmo lugar.
Só que, da última, sua mãe não estava lá. Maria e Vássia desceram
correndo, começaram a gritar e a chamar pela mãe, sem saber de onde
vinham forças para gritarem por tanto tempo, tão alto e sem intervalo,
pois, além do pãozinho da véspera em Dimítrov, não tinham comido nada.
Por mais que gritassem, não localizaram sua mãe em lugar algum. Os
gritos atraíram as pessoas em volta, que formaram uma roda ao redor de
Maria e Vássia e começaram a acalmá-los:

--- Agora mesmo vamos chamar um policial e ele achará a mãe de vocês.

Um policial se aproximou, pegou Maria e Vássia pelas mãos e disse com
ternura:

--- Vamos procurar a mamãe.

Maria gostou do policial de imediato, mas Vássia olhava para ele de
soslaio e tentava puxar sua mão, que, no entanto, o policial segurava
com força. Ele conduziu Maria e Vássia pelos trilhos até um vagão
isolado e desengatado. No vagão havia muitas crianças da idade de Maria
e de Vássia. Maria disse ao policial que os trouxe:

--- Tio, fique conosco até encontrarem nossa mãe, para sairmos juntos,
porque eles vão nos bater.

--- Não tenho tempo, menininha --- respondeu o polical, acariciando sua
cabeça ---, e vocês, peraltas --- dirigindo-se ao grupo de crianças ---,
prestem atenção, não mexam com eles. Eles ainda não estão acostumados
com essa vida. São da aldeia. Vocês são da aldeia, não é mesmo?

--- Do sítio --- disse Maria.

--- Caso aconteça algo, chamem a moça de plantão --- acrescentou o
policial ---, ela está atrás daquele tabique.

Bastou o policial sair e os meninos peraltas começaram a rir de Maria e
de Vássia e a imitar o homem:

--- Chamem, chamem... A moça de plantão, a moça de plantão... Ela está
atrás do tabique.

Quase todas as crianças estavam encardidas, cobertas de carvão e de
sujeira, e fazia tempo que haviam esquecido as carícias maternais ou
talvez jamais as tivessem conhecido, ao contrário de Maria e Vássia,
que, ainda nessa manhã, tinham sido abraçados por sua mãe, que os
apertara contra si. Maria disse a Vássia:

--- Sente-se perto de mim e não olhe para eles.

Mas um garoto da idade de Maria, vestido em uns farrapos ensebados, com
o pescoço imundo e as mãos sujas e arranhadas, mostrou um apito de barro
a Vássia, que se aproximou dele, esquecendo a irmã. Assim que ele se
avizinhou, o garoto deu-lhe um piparote na orelha, e a turma caiu na
risada.

--- Bem feito --- Maria disse a Vássia ---, agora sabe o que acontece
quando não obedece a sua irmã. Vou contar para a mamãe quando a
encontrarmos.

Depois disso, Vássia sentou-se perto de Maria e não se afastou mais. Em
seguida, um homem com uma pasta e uma mulher com uns papéis entraram no
vagão. O homem olhou em volta, franziu o rosto, na certa devido ao
cheiro ruim --- os meninos peraltas, sem cerimônia, soltavam sonoros
peidos seguidos de risos ---, e disse:

--- Parece que há mais gente aqui; onde vou metê-los... No orfanato não
há mais lugar... Que escândalo... Só se os mandar para o orfanato do
distrito.

Nesse momento, Maria, uma garota perspicaz, falou:

--- Tio, nós nos perdemos hoje da nossa mãe, precisamos achá-la.

--- Pois é --- disse o homem da pasta ---, Kalieria Vassílievna, destes
temos de sobra. Precisamos mandar todos para suas casas, para não
ocuparem os lugares dos órfãos.

A mulher disse a Maria e Vássia:

--- Vamos --- e os levou para trás do tabique.

Lá havia uma mesa e uma pequena estufa de ferro acesa. O homem colocou a
pasta sobre a mesa, tirou o sobretudo, tirou o chapéu, pendurou tudo no
canto e começou a interrogar Maria, enquanto a mulher anotava.

--- Qual é o seu sobrenome? --- perguntou o homem.

--- Não sei --- disse Maria.

--- E como o papai e a mamãe se chamam?

--- Também não sabemos, apenas papai e mamãe, só isso... Chamávamos
papai de pai, mas ele morreu no ano passado, porque foi um ano de fome.

--- Vocês têm irmãos e irmãs?

--- Temos.

--- Sabem os nomes?

--- Sabemos --- respondeu Maria. --- Meu irmão se chama Kólia, minha
irmã Chura, e eu tenho um irmãozinho, Jórik, mas ele não está mais em
casa.

--- Está bem --- disse o homem e trocou olhares com a mulher, que tomava
nota ---, e será que vocês sabem onde moram? A aldeia, o distrito ou a
região?

--- Não --- disse Maria ---, não sabemos nada disso, mas conhecemos a
nossa vila e o nosso sítio.

--- Qual é o nome da vila de vocês? --- perguntou o homem.

--- Vila de Chagaro-Petróvskoie, sítio Lugovoi --- respondeu Maria.

--- É pouco provável que fique longe daqui --- disse o homem ---, fora
da região de Khárkov.

--- Mas, Modest Fiéliksovitch --- disse a mulher ---, na região de
Khárkov há muitas vilas chamadas Petróvskoie... Eu mesma conheço três
vilas com esse nome.

--- Então --- disse o homem --- vamos arrumar um guia e ração seca para
que percorram as vilas e encontrem sua casa. Acho que o Departamento de
Educação Popular aprovará nossa iniciativa. Teremos somente os gastos
das passagens e da ração. E, para guia, escolheremos um entre os
ativistas locais do voluntariado social.

Maria, que prestava atenção em tudo, disse:

--- Rezarei eternamente pelos senhores se nos levarem para casa, se
pudermos rever nossos irmãos, Kólia e Chura, já que Jórik não está mais
em casa.

--- Agora, Kalieria Vassílievna --- disse o homem ---, mande-os para a
casa de desinfecção da estação.

De novo Maria se mostrou perspicaz e disse:

--- Querido senhor, dê um pedaço de pão para mim e Vássia, pelo amor de
Cristo, porque desde ontem à noite não comemos nada, nem taboa, uma
planta comestível que só dá na nossa vila.

O homem fixou os olhos em Maria, cujas súplicas podiam ser bastante
convincentes, como quando, na casa de chá, o \emph{tchekista}\footnote{Membro
  da Tcheká, primeira polícia secreta do governo bolchevique. Foi criada
  em 1917 e substituída pelo GPU em 1922.} de ferro, o chefe da brigada
Petró Semiónovtch, derramara lágrimas. E o homem da pasta, de repente,
também limpou os óculos com um lenço, dizendo:

--- Kalieria Vassílievna, ofereça uma xícara de água quente para cada um
e dê também isto --- e tirou da pasta um embrulho de papel engordurado,
estendendo-o à mulher.

--- Vou solicitar uma ração para eles --- disse Kaliéria Vassílievna.
--- Como o senhor ficará sem o desjejum, Modest Fiéliksovitch?

--- Não faz mal --- respondeu Modest Fiéliksovitch ---, dê às crianças.
Vejo que elas ainda não sabem roubar e, em geral, são totalmente
dependentes dos outros, como gatinhos. Ainda não são uns peraltas
endurecidos pela rua.

A mulher pegou uma chaleira de lata sobre a estufa, encheu canecas de
lata com água quente e abriu o papel engordurado. Oh, que dádiva Maria e
Vássia receberam! Era um pão branco fresco, cortado ao meio, e em cada
metade havia duas fatiazinhas de salame com gordura. Vássia engoliu sua
metade num minuto --- de sua felicidade restou apenas a lembrança --- e
começou a olhar com avidez para Maria, que comia seu pedaço devagar, com
sabedoria.

--- Vássia, tome água quente --- disse Maria, sem forças para retirar um
pouco de pão e de salame de sua parte e dar ao irmão. Justamente o que
ele queria!

Mais tarde, ela veria nisso um presságio e se repreenderia com
frequência. Assim, Maria não deu a Vássia nenhum pedacinho de sua
porção, comeu até as últimas migalhas, que recolheu dos joelhos.
Percebendo que não iria ganhar nada, Vássia começou a beber água quente.
Maria também bebeu a sua e ficou mole, com os olhos pesados, pois tinha
dormido por curtos intervalos, ora debaixo do banco, ora apoiada nos
joelhos da mãe. Mas a mulher não a deixou aproveitar seu momento,
sentadinha na cadeira, no calor.

--- Vamos para a casa de desinfecção --- disse ela ---, pois, além de
vocês, tenho outros afazeres.

Ela os conduziu de novo pelos trilhos. Maria ficou alegre por terem se
livrado dos garotos peraltas, que seriam capazes de bater neles e de
ensinar coisas ruins a Vássia.

Chegaram a um local abafado e molhado, e seus pés chapinhavam na água.

--- Tirem toda a roupa, será desinfetada --- disse a mulher.

Vássia despiu-se --- a barriguinha estava mais inchada e as pernas mais
finas, e sob sua pele cada osso era visível. O corpo de Maria, apesar de
extenuado, tinha uma forma regular, e já fazia tempo que ela sentia
vergonha de tirar a roupa na frente dos homens, até dе seu irmão Kólia.
Mas de Vássia ela não se envergonhava. A essa hora não havia ninguém na
casa de desinfecção, e as crianças, alegres, se lavaram com água quente,
que, depois do pão com salame, foi a segunda dádiva do dia, uma após a
outra... Maria achou no chão um pedaço de sabonete e ensaboou
abundantemente Vássia, que, deliciando-se, emitia rosnidos como um
cachorro agradecendo ao dono. Deram uma toalha com a textura de um
\emph{waffle}, uma para os dois. Logo que Maria começou a enxugar Vássia
no vestiário, sentiu que alguém os olhava. Virou-se e viu um rapaz
espiando da porta. Ela deu um grito e correu de volta ao banheiro, e o
rapaz caiu na risada.

--- O que há com você? --- disse ele. --- Eu sou o guia de vocês, fui
indicado para acompanhá-los e vocês devem me obedecer.

--- Feche a porta do banheiro --- disse Maria ---, deixe-me colocar
minha roupa, e depois vestirei Vássia.

--- Está bem, pode se vestir --- disse o guia e desapareceu dando uma
risadinha.

O guia parecia um Vássia mais crescido. Como Vássia, ele era magro,
tinha pequenos olhos cinzentos, um rosto comprido e um nariz levemente
arrebitado. Apesar da semelhança com o irmão, Maria antipatizou com ele
na hora, ao contrário de Vássia, que se sentiu atraído pelo rapaz. De
maneira que Maria, pela primeira vez, experimentou uma sensação estranha
e vaga, mas era de descontentamento com o irmão e de inveja do guia,
como se este possuísse algo para Vássia que ela, sua própria irmã, não
possuísse. No entanto, ela não poderia mostrar abertamente sua antipatia
pelo guia, que se chamava Gricha, pois ele estava com o cesto de
provisões --- pão e toucinho. Na verdade, o guia, Gricha, não lhe deu
toucinho nenhuma vez, apenas pão.

Assim começaram a percorrer as vilas chamadas Petróvskoie da região de
Khárkov. Chegaram a uma grande vila com muitas casas de pedra e uma
igreja branca na praça.

--- Aqui está --- disse Gricha --- sua vila Petróvskoie.

Vássia, para agradar ao guia, disse:

--- É a nossa, é a nossa!

Mas Maria olhou em volta e disse:

--- Não, não é a nossa... A igreja da nossa vila ficava sobre uma
colina, ao lado do sanatório, e embaixo da colina corria um riacho.

--- Está bem --- disse Gricha ---, se não é a de vocês, não é.

Voltaram a subir no trem e partiram, depois desembarcaram e
conduziram-se pela estrada numa carroça. Na carroça, Gricha cochichava o
tempo todo com Vássia, e Maria olhava para eles com desaprovação, mas em
silêncio, pois o cesto de provisões estava com Gricha. Maria reparou que
Gricha cortava pão e toucinho para si e para Vássia, para si um pouco
mais e para Vássia um pouco menos, enquanto a ela dava somente pão, e um
pedaço pequeno. ``Que seja,'' pensava Maria, ``pelo menos Vássia comerá
toucinho, já que eu não ganharei de qualquer jeito.'' Assim, apesar de
amargurada por si mesma, alegrava-se por Vássia.

Finalmente chegaram à outra vila. Havia uma igreja na colina e um riacho
embaixo.

--- É a vila Petróvskoie de vocês? --- perguntou Gricha.

--- É a nossa! --- respondeu o tolo Vássia, para lhe agradar.

--- Não, não é a nossa --- disse Maria ---, ela pode ter uma igreja
sobre a colina e um riacho, mas onde está o sanatório? E não dá para ver
o \emph{zakáz} que leva para a vila Popovka, onde vovó e vovô tinham uma
casa.

Seguiram viagem, primeiro de carroça, depois de trem, depois de novo de
carroça.

--- É a vila de vocês? --- perguntou Gricha.

--- É a nossa --- disse Vássia.

--- Se é a nossa --- Maria não se conteve ---, onde está o sítio
Lugovoi? Então, Vássia, encontre a nossa casa, onde moram Chura e
Kólia... Será que você não lembra que a nossa casa era mais afastada,
que tinha um canteiro na frente onde no verão colhíamos morangos
silvestres e cogumelos?

--- Está bem --- disse Gricha, sorrindo ---, só não briguem. Seguiremos
em frente.

Chegaram a um apeadeiro.

--- Hoje não passa mais trem --- disse Gricha ---, de modo que
dormiremos aqui. E também não podemos procurar a vila Petróvskoie de
noite. Nem de dia vocês conseguem reconhecê-la.

Mas Maria replicou:

--- Eu reconheceria nossa vila até de noite se a visse. Um moinho fica
na colina, ao pé da colina corre o riacho que leva até a outra vila,
Kom-Kuznetsóvskoie; a \emph{tamba} vai até a cidade de Dimítrov, e, no
meio do caminho, fica a vila de Lípki.

--- Então amanhã, com essas indicações, você irá encontrá-la --- disse
Gricha, sorrindo como de costume ---, mas está na hora de jantar --- e
cortou para si uma grande fatia de pão e uma de toucinho, para Vássia
fatia menores e para Maria apenas uma fatiazinha de pão.

Vássia mordia o pão e lambia o toucinho, mordia o pão e lambia o
toucinho, sempre cochichando com Gricha. Por fim, Gricha disse:

--- Por que devemos passar a noite neste apeadeiro? Aqui venta muito e
não vamos conseguir dormir --- os trens fazem barulho e as locomotivas
apitam. Eu conheço os arredores... Vamos, restou um celeiro por perto de
um proprietário de terra, cheio de palha. Expulsamos os ratos a grito e
dormimos.

Maria se opôs, não por preferir ficar no apeadeiro, mas simplesmente
porque sentia vontade de retrucar a tudo o que Gricha dizia. Vássia,
porém, o apoiou.

--- Aqui vou passar frio --- disse ---, não vou conseguir dormir. Quero
ir para o celeiro...

Que fazer se Vássia também queria ir? Eles saíram do apeadeiro, onde ao
menos havia um lampião aceso, e rumaram para a escuridão, pois nessa
noite nem a lua esquálida de Khárkov nem as estrelas eram visíveis no
céu. Embora o céu estivesse escuro, não chovia, tudo estava silencioso,
não se ouviam sequer latidos de cães, e também não havia vento, parecia
até esquentar. Maria quis pegar o irmão pela mão, mas ele a puxou e se
aproximou do guia, e ela caminhou sozinha, um pouco para trás. Não havia
estrada, e sob os pés sentiam apenas ressaltos e buracos e tinham a
impressão de andarem por um campo aberto, sem nenhuma habitação ao
redor. Finalmente algo surgiu.

--- Aqui está o celeiro --- disse Gricha. --- A porta está trancada, é
preciso afastar uma tábua, aqui temos uma solta.

Enfiaram-se pelo buraco e realmente encontraram palha.

--- Oh, como é macio aqui --- disse Vássia ---, e quente.

--- Pois é, Maria --- disse Gricha ---, e você não queria vir.

--- Vamos, Vássia --- disse Maria ---, deite-se ao meu lado, encoste-se
em mim, ficará mais quente; mesmo com a palha, de madrugada irá esfriar.

--- Não --- disse Vássia ---, eu vou me deitar com Gricha.

Vássia já não o chamava de ``tio Gricha'' ou de ``guia'', mas
simplesmente de Gricha, como se fosse seu irmão Kólia.

--- Deite-se onde quiser --- respondeu Maria, irritada ---, você é
ruim...

--- Você é que é ruim --- respondeu Vássia.

Maria ficou perplexa.

--- Vássia --- disse ---, meu irmãozinho, quem ensinou isso a você? Se
mamãe ouvisse você falar desse jeito ou se Chura ou Kólia vissem quem
você se tornou, pensariam que lhe ensinei coisas ruins, pois sou eu que
cuido de você o tempo todo. Vássia, você ainda é uma criança e deve
obedecer a mim como obedece à mamãe, uma vez que estamos longe dela...

--- Você não é minha mãe --- disse Vássia. --- À mamãe eu obedeceria,
mas a você não.

Então Gricha se intrometeu da escuridão.

--- Já chega --- disse. --- Vássia, realmente, não fale grosserias à sua
irmã.

Ao ouvir isso, Vássia parou com as grosseiras. Mas o motivo que o deteve
não tranquilizou Maria, ao contrário, deixou-a ainda mais triste. ``Se
Vássia se tornar um homem ruim,'' pensava ela, ``mamãe, Chura e Kólia
não irão me perdoar.''

Assim, entre tristes pensamentos, ela adormeceu sozinha, sem o irmão,
que começou a roncar do outro lado do celeiro. De repente, mergulhada no
sono, ela sentiu alguém ao seu lado.

--- Vássia --- disse Maria, alegre e sonolenta ---, deite-se perto de
mim.

E realmente alguém se deitou ali e, apertando-se nela, arrastou a mão
entre seus joelhos, pois Maria dormia de lado, com os joelhos unidos.
Ela rapidamente percebeu que não era Vássia, empurrou a mão estranha e
deu um pulo.

--- O que você quer?

--- Quieta --- disse Gricha. --- Vai acordar Vássia.

--- O que você quer? --- repetiu Maria, mais baixo.

--- Eu trouxe toucinho --- disse Gricha. --- Você ainda não comeu
toucinho, somente pão. Então eu trouxe a sua porção inteira.

Maria pegou o toucinho e, ao tateá-lo, sentiu que era realmente um
pedaço grande, deu uma mordida, saboreou --- um toucinho muito bom,
suculento, macio ---, deu outra mordida, e a tristeza que sentira ao
adormecer aos poucos desapareceu. ``E com Vássia,'' pensava, ``tudo vai
se arranjar, ele fez isso por bobeira.''

--- Está bom o toucinho? --- perguntou Gricha, rindo.

--- Está sim --- respondeu Maria.

--- Pois é --- disse Gricha ---, mas você está sempre contra mim. Se
você me amar, não precisará de mãe nenhuma.

--- Como não? --- indagou Maria. --- Ela é sangue do meu sangue...

--- Isso mesmo --- respondeu Gricha. --- Sua mãe com certeza abandonou
você e seu irmão de propósito... Para se livrar de vocês... Não é de uma
mãe que você precisa, mas de um rapaz, porque agora você está na idade
certa para o verdadeiro prazer; depois que ficar adulta, tiver seios
grandes e engravidar, o prazer não será o mesmo.

Mal Gricha terminou de falar, Maria compreendeu o que ele queria, embora
ninguém a tivesse ensinado e tudo lhe acontecesse pela primeira vez.

--- Saia daqui, sem-vergonha ---, eu logo entendi quem é você, assim que
o flagrei me espiando nua no banho.

--- Se entendeu, melhor ainda --- disse Gricha.

De repente ele agarrou Maria pelas axilas, como se quisesse sentá-la em
algum canto, e com seus joelhos férreos de homem desuniu os joelhos
infantis dela, e ela ficou completamente sob seu poder no celeiro
escuro, trancado por um cadeado do lado de fora, longe de tudo, no meio
de um campo vazio que se ligava à linha férrea de um ermo apeadeiro. E
nessa noite nem a lua esquálida de Khárkov iluminava o céu.

Só havia uma alma viva por perto --- seu irmão Vássia ---, mas ele
dormia entre roncos. E, mesmo se estivesse acordado, o que poderia
fazer? Era apenas uma criança... Não havia a quem pedir socorro e, se
ela o fizesse, somente assustaria Vássia, por isso Gricha não tapou a
boca de Maria, como não se tapa a boca de um animal sendo abatido: quem
ouviria seus berros? Maria tentou se defender calada, mas, a cada
tentativa, Gricha torcia seu braço e ela sentia muita dor; quando ela
parava de resistir, ele a soltava. Assim Gricha conseguiu de Maria o que
queria, gemendo como um doente de tifo, mas Vássia continuava a dormir
--- não acordou quando Maria gritou da dor incomum e desconhecida que
Gricha lhe causava para obter prazer, não acordou mesmo quando Gricha
gemeu com muita força, como se, ao dilacerar o corpo de Maria, ele
dilacerasse o próprio corpo. Ela só entendeu o que acontecera quando
tudo havia terminado. Ouviam-se apenas a forte respiração de Maria e de
Gricha e o ronco de Vássia. Maria alegrou-se por seu irmão não ter
ouvido nada e não ter se assustado. A respiração de Gricha ficou
tranquila, e ele disse a Maria, que continuava ofegante:

--- Não se preocupe... Com sua vida, cedo ou tarde, você seria
violentada por um velho qualquer... Melhor comigo... Pegue isso --- e
lhe deu um pedaço de pão.

Ela pegou o pão e aquietou-se, e Gricha se afastou, indo para o outro
lado do celeiro, e logo começou a roncar como Vássia.

Não se poderia dizer que Maria adormeceu, mas antes que entrou em um
estado de torpor, pois nem por um momento deixou de distinguir as vigas
salientes do celeiro e de sentir a palha sob seu corpo. Seu ventre e
baixo-ventre doíam, como se, em vez da taboa, tivesse comido uma erva
venenosa, como acontecera a uma vizinha do sítio, que, intoxicada,
morrera no mesmo dia em que o pai de Maria. Mas a dor diminuía aos
poucos e, quando as vigas ficaram claramente visíveis no celeiro
iluminado, a dor era insignificante, algo como uma alusão ao que
acontecera de noite. Maria acordou, sentou-se e viu que no celeiro
estavam só ela e o irmão, que o guia, Gricha, havia sumido. Isso a
alegrou, mas logo a entristeceu, pois ele havia levado a cesta de
provisões... No entanto ela voltou a se alegrar quando tateou no bolso
um pedaço de toucinho e um pedaço de pão --- embora, à luz do dia, não
fossem tão grandes como pareciam no escuro, teriam do que viver por
algum tempo.

--- Vássia, levante-se --- disse Maria ---, o guia que deveria nos levar
para casa fugiu, e agora temos que ir sozinhos. E ele levou toda a
provisão... Pois é, irmão, agora se convença, veja quem você tomou por
um homem bom, desobedecendo a sua irmã, a única pessoa próxima de você,
já que mamãe não está conosco e Chura e Kólia também estão longe.

Vássia ficou em silêncio, sentindo-se evidentemente culpado.

Saíram pelo buraco e olharam em volta. O campo estendia-se para os dois
lados: para onde ir? Caminharam a esmo, mas chegaram precisamente à
estrada de ferro e ao apeadeiro, onde Gricha não poderia ter feito com
Maria o que fizera no celeiro isolado, pois aqui o vigia de plantão
poderia ver e sempre havia gente sonolenta zanzando pela plataforma.
Nada teria acontecido se não fosse por Vássia, mas Maria não quis
acusá-lo, não lhe contou nada sobre o que ocorrera e apenas lhe disse:

--- Eu não conheço o caminho para casa, para a vila de
Chagaro-Petróvskoie, mas sei que devemos ir até uma estação maior, onde,
em todo caso, é mais fácil pedir comida... Quando o trem chegar, suba
logo depois de mim.

--- Vou subir --- disse Vássia.

Assim que o guia desapareceu, Vássia voltou a obedecer a Maria, sem mais
ter medo dos trens, como acontecera em Dimítrov.

No trem, Maria e Vássia comeram o pão e o toucinho que ela ganhara de
Gricha como uma compensação pelo que ele lhe fizera no celeiro. Mas não
tudo, uma parte Maria escondeu de Vássia, para comerem mais tarde, ou o
irmão comeria tudo de uma vez. Maria e Vássia chegaram a uma grande
estação e saíram com uma multidão de passageiros, pois o trem não ia
adiante. Os irmãos olharam em volta e gritaram de alegria.

--- Mas é a cidade de Dimítrov... Daqui a \emph{tamba} leva direto para
o nosso sítio.

No entanto, um velho explicou:

--- Crianças, não é Dimítrov, mas Izium... Vocês já comeram uvas secas
doces? Então, em sua honra chamaram a cidade de Izium\footnote{\emph{Izium,}
  em russo, significa ``uva-passa''.} --- e sorriu.

Mesmo desapontada por não estar em Dimítrov, Maria ponderava sobre o
velho: ``Os velhos raramente sorriem, mas, se este está sorrindo,
significa que é bondoso, e um velho bondoso nos daria alguma coisa,
porque não restou quase nada de pão e toucinho''.

--- Vovô --- disse ---, nós não comemos nem uva doce, nem uva seca,
porque nos perdemos da nossa mãe... Dê, pelo amor de Cristo, o que
puder...

--- Nós conhecemos vocês --- disse o velho, que imediatamente se zangou
---, ficam perambulando pelos trens à procura de uma mala dando sopa...
Pois eu vou...

Maria pegou Vássia pela mão e eles fugiram do velho maldoso, correndo
pela plataforma em direção ao prédio da estação.

A estação de Izium não era como à de Khárkov, não tinha um telhado de
vidro nem uma escada branca e reluzente, mas também era bonita, grande e
aquecida, com muitos bancos e até uma árvore estranha dentro de um vaso
como em Khárkov, mas, de fato, apenas uma...

--- Não faz mal, Vássia --- disse Maria ---, aqui viveremos muito bem
por enquanto. Eu sei mendigar, minha voz é chorosa, e, se um der esmola,
outro dará também. E veja o mundaréu em volta. Vou pedir comida e na
certa darão alguma coisa. Se alguém tentar nos bater, aqui mesmo de
noite é cheio de gente e iluminado. Só que, Vássia, Deus o livre de
roubar... Você viu como o velho ficou bravo? Ele não ficou bravo com a
gente, mas com os ladrões. Não ofenda o povo, Vássia, e ele о defenderá
em qualquer momento, mas o ofenda e ele o abandonará à própria sorte...
Por acaso estávamos bem naquele celeiro escuro, em um campo escuro e
vazio, com um sujeito ruim por perto, de quem você, Vássia, resolveu
gostar por bobeira?

Assim Maria falava seu sermão a Vássia, que a escutava por depender do
que ela receberia de esmola. E, realmente, o que ela conseguiu na
estação de Izium não era de desprezar.

--- Senhor --- dizia ela ---, Jesus Cristo... Filho de Deus...

Por tais súplicas ela recebia esmola de velhos e jovens, de homens e
mulheres. Até alguns membros do partido se sentiam incapazes de recusar
às súplicas de uma criança, mesmo que ela usasse termos religiosos
obsoletos do antigo regime. Um passageiro, sem dúvida um membro do
partido, porque vestia um sobretudo de couro e tinha uma cicatriz em
forma de espada como a do chefe da brigada Petró Semiónovitch, deu à
Maria um pacote com cinco pãezinhos recheados de ervilhas. Acontecia
também de lhe darem arenque e embutidos, sem falar de pão; ali, na
estação de Izium, Maria e Vássia, pela primeira vez, comeram pão, se não
até se fartarem, ao menos a ponto de não passarem fome. À noite, dormiam
em bancos num canto aquecido. Eles estavam muito contentes com a vida.

Mas toda sorte casual, não preparada pelo destino, é passageira,
efêmera. Um dia, Maria voltava após ter juntado algumas esmolas quando
notou, perto de Vássia, uma mulher zangada que lembrava a que tinha ido
atrás de Jórik em Dimítrov.

--- Aqui está ela, minha irmã --- disse Vássia, apontando para Maria.

--- Muito bem --- disse a mulher ---, e onde está a mãe de vocês?

--- Nós nos perdemos de mamãe --- respondeu Maria.

--- Então vamos.

A mulher conduziu Maria e Vássia da estação quente a uma praça cheia de
vento, onde havia outras crianças, que, por sorte, não eram peraltas
como as do vagão de Khárkov. Maria já tinha aprendido a distingui-las.
Enfileiraram as crianças aos pares e as levaram dali. Maria,
evidentemente, andava com Vássia e segurava-o pela mão. Se fosse antes,
quando ela ainda morava no sítio, estaria olhando para todos os lados,
para as casas e para as pessoas. Mas ela não prestava atenção em Izium,
e preferia saber para onde estavam sendo levados e o que lhes dariam
para comer. Conduziram-nos a um pátio de cavalos, onde havia algumas
estrebarias e, em uma área de terra batida, postes com correntes e
estacas para prender os cavalos e muito estrume. А mulher de cabelo
curto apresentou-se como instrutora, mas não disse seu nome: era
simplesmente ``a instrutora''. Ela abriu o portão de uma das estrebarias
--- o chão estava coberto de palha podre e havia poucos cavalos, que
ficavam na extremidade oposta, deixando o espaço vazio.

--- Ajeitem-se --- ordenou a instrutora --- e esperem até eu voltar e
levá-los para almoçar. Mas não saiam sozinhos, porque os cavalos podem
ferir até a morte.

Ela falou e partiu. Maria e Vássia sentaram-se longe das outras
crianças, atrás de um monte de palha, e começaram a comer o que Maria
tinha recebido na estação. De repente, ela notou um garoto se aproximar,
um pouco mais novo que ela, porém mais velho que Vássia.

--- Meu nome é Vánia...\footnote{Apelido de Ivan.} --- disse ele.

--- E daí? --- disse Maria.

--- Daí --- disse --- passem algo para comer.

--- Vá embora --- disse Maria ---, nem para nós é suficiente... Quando
servirem o almoço, você comerá...

Ele se afastou, sem dizer nada.

O almoço coletivo demorou a acontecer. Só algumas horas depois a
instrutora apareceu, agrupou-os de dois em dois e levou-os para um
refeitório que ficava do lado do pátio. Provavelmente, quando passava
fome no sítio, Maria teria comido esse almoço com prazer, mas, depois de
experimentar os arenques, os embutidos e os pãezinhos de ervilhas que
lhe deram na estação de Izium, comia-o com esforço e por necessidade.
Ela percebeu que Vássia também se esforçava para comer. ``Pois é,''
pensava Maria, ``será difícil vivermos sem pedir esmola. É preciso
ensinar Vássia a mendigar, porque ele está ficando preguiçoso e, de uma
hora para outra, pode dar de roubar.''

E assim tudo se deu. Uma vez por dia, sempre na mesma hora, depois do
meio-dia, a instrutora chegava e os levava para o refeitório, onde
invariavelmente serviam uma sopa aguada --- farinha em água quente ---,
mingau de trigo sem manteiga e um pedaço de pão. O restante do tempo,
todos saíam atrás de sustento --- havia quem esmolasse, havia quem
roubasse. Maria não deixava Vássia se afastar, mesmo percebendo que ele
não gostava de mendigar. Como não gostava de fazê-lo, raramente recebia
alguma coisa, pois qualquer ofício exige dedicação e habilidade. Se ele
não queria esmolar, que ao menos ficasse perto dela ou a esperasse numa
esquina ou sentado num banco. Para que Vássia a escutasse e para lhe
despertar algum interesse, Maria, ao receber uma boa esmola, a entregava
a ele. Ela costumava mendigar em bares e perto das casas ricas, mas
raramente ia à estação e nunca à feira, por causa de Vássia. Maria sabia
que na feira havia muitos ladrões, os quais poderiam ser uma influência
ruim para ele. Dessa forma eles passavam os dias e depois iam dormir na
estrebaria.

No pátio dos cavalos ficava um vigia noturno, um velho apelidado Moskal.
Bondoso e terno, ele gostava de crianças e elas gostavam dele. Reunia-as
em volta de si na estrebaria e contava-lhes histórias até que
adormecessem. Algumas crianças dormiam logo, outras o ouviam até tarde.
Maria e Vássia, por exemplo, o ouviam até de madrugada. O velho tinha
várias histórias. Sobre o tsarévitche Ivan, sobre a órfã Márfucha, sobre
Iliá Múromets, o destruidor dos infiéis.\footnote{Personagem do folclore
  eslavo, Iliá Múromets é herói de muitas \emph{bylinas} (narrativas
  épicas medievais eslavas).} Mas a história mais interessante era sobre
um filho de Deus, Jesus Cristo. O velho apoiava o rosto enrugado de
barba branca na palma da mão, com ar pensativo e triste, e começava:

--- Em um reino longínquo de um país longínquo havia um grande pecado. O
Senhor decidiu salvar o povo do pecado e enviou para a terra seu filho
querido, Jesus Cristo. Assim que Jesus apareceu entre os homens, ficaram
todos contentes. Ele deu pão a todos, que comeram até se fartarem, e os
borrifou com água do rio Jordão, dizendo: ``Agora vocês serão um povo
batizado, ortodoxo, mas para os judeus-\emph{jides},\footnote{\emph{Jid},
  termo pejorativo para designar judeus.} que não querem trabalhar e só
se ocupam do comércio de seus templos sagrados, não haverá reino de
Deus''. E os judeus-\emph{jides} planejaram destruir a criança querida
de Deus, seu filho Jesus Cristo. O judeu mais importante entre eles era
Judas-Anticristo --- nesse instante, o velho levantou o dedo, como se
ameaçasse alguém no escuro, pôs-se a escutar os cavalos roncando e
mexendo os pés do outro lado da estrebaria e prosseguiu: ---
Judas-Anticristo reuniu o \emph{kahal}\footnote{\emph{Kahal,} do
  hebraico\emph{,} era a forma da organização da antiga sociedade de
  Israel e hoje designa uma congregação de pessoas.} mundial de judeus,
um bando de bandidos, e disse: ``Enquanto Jesus Cristo estiver vivo, não
venceremos o povo ortodoxo, não conseguiremos fazer homens e mulheres
ortodoxos trabalharem para nós nem pegaremos o sangue das crianças
ortodoxas para cozinhar nossa \emph{matsá}''. Assim são chamadas suas
panquecas impuras. Um dia, Jesus Cristo caminhava por um jardim,
enquanto Judas e outros judeus o espreitavam atrás de moitas. Eles
pegaram Jesus Cristo, arrastaram-no até uma montanha e pregaram suas
mãos e os seus pés em uma cruz, pensando que ele morreria. Mas ele não
morreu, subiu ao céu por força divina e de lá voltou para o povo
ortodoxo, dizendo: ``Aqui está ele, aqui estou eu. Não acreditem nos
\emph{jides}, que dizem que eu morri, e vinguem-se deles por meus
infortúnios divinos...''.

Apesar de ser um conto muito interessante, era muito comprido, de
maneira que, no fim, a maioria das crianças já tinha dormido. No
entanto, Maria e Vássia não dormiam, nem Vánia, o menino que no primeiro
dia lhes pedira comida, ouvindo tudo até terminar. O velhinho sempre
contava o fim da história de um modo diferente. Ora ao chamado de Jesus
Cristo vinham Iliá Múromets e Aliocha Popóvitch, ora Stepán Rázin e
Emelian Pugatchióv, ora Ermák Timoféievitch, о conquistador da
Sibéria...\footnote{Aliocha Popóvitch é um personagem das lendas
  populares eslavas. Stepán Rázin (1630--1671), cossaco de Don, liderou
  as insurreições dos anos 1670 e 1671. Emelian Pugatchióv (c.
  1742--1775) comandou um levante cossaco no reinado de Catarina II,
  acontecimento descrito na novela \emph{A filha do capitão} (1836), de
  Aleksándr Púchkin. Ermák Timoféievitch (1531--1585) foi o líder
  cossaco que conquistou a Sibéria.} Toda noite era assim. Os cavalos
roncavam, uma lua aparecia na janelinha sob o telhado da estrebaria...
Finalmente, uma noite, Vássia não aguentou, baixou a cabeça sobre o
peito e dormiu emitindo sons pelo nariz.

--- Vássia dormiu --- disse então Maria e cuidadosamente levou o irmão
para um canto que ela já havia coberto com palha, deitou-o e acomodou-se
ao lado.

Maria gostava dessas histórias noturnas, mas depois lamentou ter
permitido a Vássia ouvi-las, pois foi assim que ele se aproximou do
garoto que lhes pedira comida.

Uma vez, quando Maria estava saindo para pedir esmola na cidade, Vássia
lhe disse:

--- Eu não irei com você, irei com Vánia.

--- Vássia, meu irmãozinho --- disse Maria ---, por acaso eu o ofendi?
Dou o melhor do que recebo para você... Mas Vánia vai lhe ensinar a
roubar, eu sei que ele fica zanzando pela feira.

--- E daí se ele vai à feira? --- retrucou Vássia. --- Na feira dão mais
esmolas e melhores.

--- Eu sei muito bem como dão esmolas na feira, lá o povo é avarento ---
um quer comprar mais barato, outro vender mais caro... Não há lugar
melhor para mendigar do que num bar ou numa casa rica. Na estação também
dão boas esmolas, mas lá o povo é desconfiado, tem medo de ladrões. Se
despertar simpatia, ganha algo, senão pode até apanhar. Vamos comigo,
irmãozinho, você comerá bem.

Vássia não escutou Maria e saiu com Vánia. À noite, ao voltar, ele
disse:

--- Maria, dê um pedaço de pão, eu não ganhei nada.

Maria respondeu com uma repreensão:

--- Não tem nada que correr à feira, mas pedir esmola, trabalhar... ---
mesmo assim, ela deu-lhe pão.

No dia seguinte, ele não dirigiu palavra a Maria, nem apareceu no
almoço. Voltou com Vánia bem tarde e ambos estavam contentes е chupando
balas. Maria logo compreendeu tudo, mas não interrogou Vássia, apenas
levou Vánia para o lado e perguntou:

--- Vocês estão roubando na feira?

--- Sim --- respondeu Vánia.

--- Vánia --- disse então Maria ---, você é responsável por si mesmo,
mas eu sou responsável por Vássia perante a nossa mãe, de quem nos
perdemos no caminho... E perante nossos irmãos, Chura e Kólia... Não
leve Vássia para o roubo.

--- Mas nós não roubamos, pedimos --- respondeu Vánia, sorrindo de modo
insolente ---, eu enganei você.

--- Você está mentindo pra cachorro --- disse Maria, brava, e
afastou-se, pensando: ``Agora a única esperança é que nos transfiram
para orfanatos diferentes e separem Vánia e Vássia''.

Fazia tempo que corriam rumores sobre a transferência e, numa manhã, a
instrutora reuniu as crianças e disse:

--- Crianças, um carro chegará hoje e todos irão partir, mas eu não sei
para onde. O carro não levará todos de uma vez, mas em grupos, por isso
os irmãos devem permanecer juntos, para caírem no mesmo grupo.

Assim que ela o disse, Maria correu para avisar Vássia, mas não havia
nem sinal dele. O carro chegou --- um caminhão. Levou um grupo, e Maria
esperou. O carro voltou, reuniu um segundo grupo, e Maria começou e se
preocupar --- Vássia não aparecia. O que fazer? Se fosse procurá-lo na
feira, poderiam se desencontrar. Еle voltaria ao pátio, seria colocado
no carro e levado sem ela. Maria se afligia e amaldiçoava Vánia por
instigar seu irmão a roubar, principalmente num dia como esse. E se
amaldiçoava por tê-lo deixado ouvir as histórias noturnas do velho
vigia, ocasião em que ele e Vánia fizeram amizade. O carro retornou pela
terceira vez, reunindo mais um grupo, e restaram poucas crianças, que
seriam levadas todas juntas. Maria não aguentou: correu à feira,
procurou-o, chamou por ele, mas não o encontrou em lugar algum. Ela
correu também pela cidade, perto dos bares onde costumava pedir esmola
com Vássia --- ele poderia, quem sabe, ter tomado juízo, largado o roubo
e começado a esmolar ---, então correu à estação. Suada e cansada, Maria
voltou depressa ao pátio dos cavalos. Vássia não estava lá, mas o
caminhão já tinha chegado e as últimas crianças estavam sendo acomodadas
nele. Maria pediu para ser deixada ali, para que pudesse encontrar seu
irmão, mas a instrutora disse:

--- Nós sabemos que seu irmão está roubando, e você quer ficar para
roubar com ele? Nós o encontraremos e o levaremos para onde você
estiver...

Maria começou a chorar, tentou explicar que era responsável por Vássia
perante sua mãe, mas a instrutora e um homem grisalho a pegaram pelas
axilas, como Gricha lhe fizera no celeiro, colocaram-na no carro e
ordenaram às outras crianças que a segurassem. No entanto, se na noite
do celeiro ela fora obrigada a se resignar, pois Gricha torcia seu
braço, aqui ela lutou por seu irmão até o fim, tentando se soltar, mesmo
sentindo a dor que aquelas mãos estranhas lhe causavam, e gritava como
havia gritado na estação de Khárkov, quando ela e Vásia se perderam de
sua mãe. Finalmente Maria conseguiu se soltar e pular do carro, mas a
instrutora e o homem grisalho a alcançaram, pegaram-na pelas axilas e a
colocaram de volta. O caminhão partiu sob o choro e as maldições de
Maria, que só se acalmou um pouco quando saíram de Izium, atravessando a
ponte e o campo. Apenas bem longe de Izium, Maria se cansou,
resignou-se, e aí pararam de segurá-la. De novo, como acontecera depois
do que Gricha lhe fizera, ela entrou em um estado que não era de
dormência, mas de torpor. Parecia ver tudo, mas não compreendia nada.
Lembrava somente que, em alguma vila, de todo o grupo de crianças haviam
sobrado duas --- ela e uma menina mais velha. Levaram a menina para
algum lugar e disseram à Maria:

--- Espere aqui.

No entanto, agora ninguém a vigiava e, logo que se deparou sozinha, ela
fugiu.

Maria saiu correndo da vila e chegou a uma estrada e, assim que se viu
no campo pela primeira vez totalmente sozinha --- embora fosse raro
alguém da família estar ao seu lado além de Vássia, este estivera sempre
por perto ---, sentiu que algo havia mudado e, ао olhar em volta,
percebeu que nevava... ``Ah, meu Deus,'' pensou, ``como nesse frio e
faminta acharei Izium, onde Vássia ficou?'' Ela se apertou na velha
blusa que vestia, enfiou o rosto na gola, para aquecer o peito com a
própria respiração, e pôs-se a andar.

Enquanto caminhava, o campo embranquecia --- a neve caía e caía, e,
quanto mais caía, mais sua fome acentuava. A terra sob seus pés estava
branca e limpa e o céu, mesmo um pouco escurecido, também estava branco,
nevado, e somente Maria se deslocava como uma mancha preta miserável em
meio a essa brancura. Se pudesse entender a si mesma, sentiria nesse
instante como sua vida era desnecessária para o mundo e até como
comprometia a beleza dele. Porém, para sua felicidade, Maria não podia
se ver com a primeira neve caída ao fundo, nem compreender a si mesma
com um olhar distanciado, como as pessoas dadas a filosofar. Se soubesse
filosofar, ficaria horrorizada ao perceber que não era necessária a
ninguém, nem mesmo a Vássia, e que sua existência só gerara prazer a um
homem ruim como Gricha, que a violentara em um celeiro. Esses
pensamentos humanos desesperadores, que não saíram de um tratado,
possuíam aquele raro ateísmo fecundo que mais agrada ao Criador do que o
canto frio dos salmos ou a idolatria difundida. No entanto, a alma e a
razão estavam separadas de Maria por um espaço infinito, mas o coração
silencioso, privado do dom divino da palavra, estava com ela, e ela
chorou, sem palavras ou discernimento, apenas emitindo sons sem sentido.

Esse choro não era igual ao de sempre, como quando fora separada de
Vássia, nem o choro insano que esbraveja e amaldiçoa e que em nada
resulta. Era o pranto divino, vindo do coração, com o qual às vezes o
Senhor recompensa os insensatos ao substituí-lo por grandes verdades
acessíveis apenas aos profetas. E Maria, a miserável criatura que fora
renegada pela mãe e pelos irmãos mais velhos, que perdera o irmão mais
novo e cuja ausência neste mundo privaria de prazer apenas aquele que
violara seu corpo, elevou-se através desse pranto, sob esse céu branco e
essa terra branca, obteve através desse choro insensato mas sincero, о
consolo do Senhor, pronunciadо através do profeta Isaías:

--- Como uma mãe que consola Eu vos consolarei {[}...{]} Vós о vereis, e
vosso coração se alegrará e vossos ossos florescerão como ervas
frescas{[}...{]}\footnote{Isaías 66:13, 14.}

Ela leu o sermão do Senhor sem ajuda de palavras, compreendeu-o sem o
uso da razão. Confortada por esse precioso dom que recebera --- o pranto
divino ---, tranquilizada e com o coração aliviado, Maria percorreu o
campo nevado e se aproximou dos edifícios de uma estação, cobertоs de
neve. Não era a estação de Izium, mas de Andréievka.

``Não faz mal,'' pensou Maria, com o coração em paz, ``aqui sempre
poderei me alimentar de esmolas e talvez invente algo mais... Será que
devo mesmo ir até Izium? Pode ser que Vássia não esteja mais lá, pode
ser que ele já não estivesse lá de manhã, enquanto eu corria feito uma
louca pela feira e pela cidade. Talvez ele tenha ido embora com seu
amigo Vánia, para roubar em outro lugar. Mesmo que eu sinta a
consciência pesada por não ter cuidado dele, talvez minha mãe, quando
ela for encontrada, e meus irmãos, Kólia e Chura, entendam que eu não
consegui defender nem a mim mesma e sejam condescendentes com minha
falta.''

Pensando assim, Maria tranquilizou-se totalmente e decidiu mendigar aos
passageiros da estação de Andréievka, pois estava faminta. Da mesma
forma que não mendigava em todas as casas, não mendigava em todos os
trens. Quando via chegar um trem abarrotado de passageiros vestidos com
farrapos como ela, com suas coisas embaladas em trouxas e cestos, ela
não se mexia, preferia ficar em um banco no calor. Mas se via um trem
rico, com poucas pessoas e com malas, ia atrás de esmola.

Eis que chegou um trem rico como esse, e Maria foi esmolar. Notou um
moço saindo de um vagão com uma mala reluzente, acompanhado por uma
jovem sem bagagem. Maria queria aproximar-se, mas de repente se
intimidou. Nunca tinha visto pessoas tão bonitas, e pareciam ter cheiro
de mel. Sem saber por quê, começou a andar atrás deles. Seguindo-os,
ouviu o moço dizendo:

--- Eu não vou pegar o trem para Khárkov, vou a Lgov por Kursk.

Nesse momento, Maria levou as mãos à cabeça. ``Meu Deus... Nossa irmã
mais velha, Ksiénia, trabalhava numa casa de repouso em Lgov.'' Antes
era como se Maria soubesse e não soubesse disso. Porém, ao falar de
Lgov, o jovem subitamente reavivou a memória dela.

Enquanto isso, a moça se afastou, deixando-o sozinho. Maria começou a
chorar. Certamente não chorou como tinha chorado no campo nevado, por si
mesma, agora era um choro intencional, para atrair a atenção. O jovem
olhou para ela e perguntou:

--- Menina, por que está chorando?

--- Eu me perdi de minha mãe --- disse Maria ---, e minha irmã mais
velha, Ksiénia, trabalha numa casa de repouso em Lgov, mas eu não tenho
dinheiro para ir lá...

--- E você está com fome? --- perguntou ele.

--- Sim, estou com fome --- respondeu Maria.

--- Então vamos até a lanchonete, comprarei alguma coisa para você comer
--- disse o jovem.

A pequena lanchonete da estação de Andréievka não era como а de Izium,
mas o jovem disse alguma coisa ao garçom, que logo trouxe frango assado
e uma deliciosa garrafa de água açucarada. Enquanto comia, Maria
encarava o moço e, de tão atraída pela beleza dele, nem mesmo sentiu o
gosto do frango.

É preciso notar que, depois do que Gricha lhe fizera no celeiro, algo
mudara dentro dela. Maria vivia normalmente --- comia, bebia, dormia,
sem nenhuma alteração ---, mas de repente sentia que havia algo
diferente e isso a agradava. Era tão agradável que por vezes sonhava
voltar ao celeiro escuro e isolado, no meio de um campo vazio, mas não
com Gricha, e sim com outro homem, que ainda desconhecia... Ao ver o
jovem, ela soube com quem gostaria de ir ao celeiro e, mesmo se fosse
dolorido, não tentaria se proteger nem gritar. Surgiu-lhe a ideia de, em
vez de ir a Lgov atrás de Ksiénia, ligar-se a esse jovem. Mas ela não
sabia como lhe dizer isso. Nesse ínterim, ele falou:

--- Menina, coma depressa, tenho pouco tempo. Agora você irá comigo.

Maria se alegrou, roeu os ossinhos, tomou toda a garrafa de água
açucarada e só então voltou a si, sentindo-se envergonhada:

--- Perdão --- disse ---, eu comi tudo e para o senhor não sobrou nada.

O jovem riu com seus dentes brancos, alinhados e brilhantes.

--- Tudo bem --- disse ---, eu aguento.

Maria foi atrás do jovem com tanta alegria que, pela primeira vez em
muitos dias, sentiu vontade de cantar. É preciso notar que Maria cantava
frequentemente com sua mãe e Chura --- gostavam de músicas como ``Que
bela noite de luar''\footnote{No original, em ucraniano: \emph{Nitch
  iaka missiátchna} (1870). Canção popular com música de Mikola Líssenko
  (1842-1912) e letra de Mikhail Starítski (1840-1904).} ou ``Ponham chá
na caneca; adeus, eu vou partir''. A canção sobre o chá, pelo visto, não
era apropriada para momentos felizes, mas de outras canções Maria se
lembrava com prazer. Assim ela seguia o jovem e se perdia em lembranças
agradáveis. Aproximaram-se do vagão, e, logo que a jovem viu seu
acompanhante pela janela, saiu para a plataforma, abraçou-o e chorou
como se não o visse havia muito tempo. E o jovem lhe disse:

--- Vália,\footnote{Apelido de Valentina.} acompanhe esta menina até
Khárkov; de lá ela pedirá que a levem para Kursk e então para Lgov, onde
tem uma irmã.

A jovem imediatamente tirou as mãos dos ombros dele, enxugou as lágrimas
de seu rosto com um lencinho rendado e disse:

--- Mas é você que está indo para Kursk e Lgov.

--- Eu não viajarei tão cedo --- respondeu o moço ---, e essa menina
precisa ir logo... Meu lugar ficará livre... Pegue o dinheiro --- e o
esticou a ela.

--- Eu não preciso de dinheiro --- respondeu a jovem. --- Ela que viaje
no seu lugar.

Maria entrou em um vagão de beleza indescritível, todo forrado de seda,
com um espelho e bancos estofados. Sentou-se perto da janela com uma
cortina cor creme e ficou admirando o jovem. A moça sentou-se no banco
oposto e fingia não notá-lo, mas Maria percebeu que vez ou outra ela o
espiava. ``Ah,'' pensou Maria com raiva, ``pelo menos não sou só eu que
ficarei longe dele, você também ficará... Ele não será meu, mas também
não será seu.''

Nesse momento, o trem colocou-se em marcha e, de tão macio e silencioso,
Maria teve a impressão de que alguém a carregava nos braços.

--- Como você se chama? --- perguntou a jovem.

--- Maria.

--- E quantos anos você tem?

--- Eu não sei.

--- Você é da aldeia?

--- Sim --- respondeu Maria ---, da vila de Chagaro-Petróvskoie, do
sítio Lugovoi.

--- Você, com certeza, não tem nem quatorze anos --- disse a jovem ---,
deve ter uns doze... Que idade feliz, sem homens nem sofrimentos.

E não falou mais com Maria, ficou sentada no canto, calada, às vezes
encostava o lencinho rendado nos olhos com os dedos finos como agulhas e
as unhas pintadas de vermelho. Só quando chegaram a Khárkov, a jovem
trocou algumas palavras com Maria:

--- Pegue esse dinheiro --- disse. --- Compre uma passagem para Kursk e
lá compre uma para Lgov.

--- Que Deus a proteja --- agradeceu Maria, como sua mãe lhe ensinara
---, mas dê também um pedacinho de pão, pelo amor de Cristo... O caminho
é longo e sabe-se lá com quem vou deparar.

--- Aqui há mais do que o suficiente para as passagens --- respondeu a
jovem ---, você pode comprar pão e salame... Eu não tenho pão, e também
estou com fome...

Maria agradeceu mais uma vez e foi embora, e nunca mais viu a jovem. Foi
até o prédio da estação, que, mesmo bonito como antes, agora não lhe
parecia tão grande. Reconheceu o banco em que sua mãe se sentara ao lado
de sua trouxa e a escada branca e reluzente pela qual correra com seu
irmão. Viu também as estranhas árvores plantadas em vasos... Sentiu um
nó na garganta e se pôs a chorar amargamente, mas não como no campo
nevado, a caminho da estação de Andréievka, de modo que, após o choro,
continuou a sentir tristeza e um peso no peito. Ela nunca havia ganhado
dinheiro em notas, apenas moedas de cobre, e sabia onde poderia comprar
pão e salame, mas não passagens. Mas o jovem que ela escolheu entre
tantas pessoas para perguntar lhe mostrou onde as passagens eram
vendidas e ela comprou um cartão verde e duro.

Esse jovem não era tão bonito como o da estação de Andréievka, no
entanto sua aparência a agradava e ela sentiu que, se ficasse com ele no
celeiro escuro, talvez não gritasse também...

O salame que comprou era duro e escuro e, depois do frango assado da
lanchonete de Andréievka, não satisfez Maria, que já tinha se habituado
a esmolas ricas --- não é que o número de ricos tivesse aumentado, mas
ela aprendera a mendigar nos lugares certos e às pessoas certas.

--- Que foi, não gostou do salame? --- perguntou um homem vestindo um
capote militar e com bandas de pano enroladas nas pernas, de rosto
vermelho, como se congelasse de frio. --- Um dia eu cavalguei num
salame... Como diz a canção: ``A cavalaria de Budiónni virou
salame''\footnote{Sátira da canção ``Carta a Vorochílov'' (1938).
  Kliment Vorochílov (1881-1969) e Semion Bedióni (1883-1973), militares
  soviéticos, participaram da Guerra Civil (1918-1921).} --- e riu. ---
Se não gostou do salame, passe para cá...

Maria arrancou-lhe um pedaço, e pela primeira vez na vida, em vez de
receber algo, ela oferecia, e entendeu como isso era agradável e que
prazer as pessoas propiciavam a si mesmas ao darem esmolas. Os
miseráveis não deveriam agradecer a quem oferece uma caridade, é este
que deveria agradecer aos miseráveis, que lhe proporcionam tanto prazer
por existirem.

Embora o homem estivesse sujo, sua água-de-colônia era cheirosa, como a
do jovem de Andréievka. Com as mãos trêmulas, ele pegou o salame de
carne de cavalo e começou o roê-lo. Maria gostou de dar-lhe o embutido,
mas, quando ele se pôs a comê-lo, ficou desapontada e se afastou,
pensando com tristeza: ``Vássia nem um salame desses tem para comer.
Será que com o roubo ele ganhará alguma coisa além de tabefes, já que
não lhe ensinei a esmolar?''. No entanto, a amargura da separação havia
diminuído. Se Maria estudasse filosofia, entenderia que sua amargura
agora era otimista, pois qualquer otimismo, mesmo o universal, existe em
função de interesses próprios. ``Não faz mal,'' pensou ela, ``logo
acharei Ksiénia, e ela encontrará Vássia mais rápido do que eu, porque
faz tempo que ela mora na cidade, que deixou de ser uma mulher da
aldeia.'' Maria tinha pão, salame, mesmo que fosse de cavalo, e uma
passagem para Kursk. Viajou a noite toda em seu próprio assento, sentada
como uma dama, empurando com os cotovelos aqueles que avançavam para o
lugar dela.

Em Kursk também havia muita gente e árvores dentro de vasos, mas Maria
já estava acostumada e não mais se interessava por coisas supérfluas,
pensando apenas em como chegar a Lgov e conseguir comida, porque o pão e
o embutido de Khárkov tinham acabado. No entanto, mimada pelas esmolas
fáceis de Izium, pelo encontro afortunado com o jovem na estação de
Andréievka e pela viagem no rico vagão, Maria tornou-se preguiçosa e
começou a esmolar como Vássia, sem vontade. Em Kursk ninguém lhe
ofereceu esmola, e uma mulher, da qual Maria se aproximou mendigando em
nome de Cristo, deu-lhe subitamente uma bofetada. Maria fugiu e se
escondeu atrás de uns caixotes no fim da plataforma, mas não chorou. Só
pensava em como chegar até a casa de Ksiénia em Lgov, pois já não tinha
dinheiro para a passagem: gastara no salame à toa e poderia ter comprado
menos pão ou mesmo nada, já que era possível mendigá-lo. Quanto à mulher
que lhe batera, Maria se tranquilizou: ``Não faz mal, ela me confundiu
com uma ladra...''. Mas logo desanimou: ``Isso provavelmente acontece
com Vássia todo dia. É preciso chegar rápido até a casa de Ksiénia, para
que ela ache Vássia''.

De repente Maria viu dois meninos sujos, de sua idade, e uma menina.

--- Foi você quem roubou a mala daquela dona? --- perguntou o menino
mais alto.

--- Não, não fui eu --- respondeu Maria.

--- Então por que está sentada aqui? --- perguntou a menina.

--- Onde mais eu sentaria --- respondeu Maria --- se preciso chegar até
Lgov e não tenho dinheiro para a passagem?

Os dois meninos e a menina deram risada e disseram:

--- Pode viajar conosco para Lgov... O trem está para sair --- e
apontaram para os vagões de areia.

``Na certa, são uns peraltas,'' pensava Maria, ``mas preciso viajar...
Se me importunarem, começo a gritar.''

Subiram no vagão e partiram.

--- Vamos --- disse o garoto mais alto ---, encoste-se em nós, senão
você vai congelar.

Maria se sentou separada, mas o vento do vagão descoberto atravessava os
ossos e ela se enfiou entre eles. Logo que se acomodou, o garoto mais
alto começou a tocá-la, enquanto o mais baixo já cuidava da outra
garota, enfiando-lhe a mão por baixo da saia. ``Ele até pode me tocar,
mas debaixo da saia não deixo,'' pensou Maria e apertou os joelhos. Ela
percebeu que ele não tinha força de homem, como Gricha, e não
conseguiria desunir seus joelhos. O menino também o compreendeu e disse:

--- Vamos namorar. Para que você precisa de uma irmã em Lgov? Meu pai
mora em Khárkov e mesmo assim eu fugi dele. Viajaremos de trem, teremos
uma boa vida.

Maria certamente entendeu a insinuação, mas se fez de boba.

--- Não --- disse ---, eu preciso achar minha irmã em Lgоv para que ela
me ajude a encontrar meu irmão Vássia.

Enquanto conversavam, chegaram a Lgоv.

--- Peço desculpas --- disse Maria ---, obrigada pela companhia --- e
saltou do vagão-plataforma.

--- Sua miserável --- disse o menino e ameaçou correr atrás dela.

Мas Maria o advertiu:

--- Eu vou fazer um escândalo --- e ele desistiu.

Quando Maria estava em Izium, o que a chuva molhava o sol secava, mas
ali, em Lgov, ela entendeu que não poderia dormir na rua, que estava
coberta de neve, nem mesmo na estação, que era fria e menor do que a de
Andréievka. ``Caso eu não encontre Ksiénia, será o meu fim... Quem me
deixaria entrar em casa para me aquecer? Ou eu terei que ir sozinha para
um orfanato, e é disso que eu tenho mais medo,'' pensava Maria.

Ela perguntou a um transeunte sobre a casa de repouso.

--- De qual casa de repouso quer saber, menina? --- disse ele.

--- Como qual?... É onde a minha irmã Ksiénia trabalha.

--- E em qual casa ela trabalha? Temos a casa de repouso ``A Ladeira'' e
a do Décimo Congresso do Partido.

--- Eu não sou daqui --- disse Maria ---, não sei.

--- Então vá para a ``Ladeira'' e lá você verá --- e ele indicou o
caminho.

Maria partiu em meio a montes de neve, pois as ruas de Lgov eram
estreitas e as casas baixinhas, e à noite, pelo visto, caíra uma
nevasca. Maria caminhava e tremia de frio, e estava tão frio que nem se
podia parar e olhar em volta para ponderar sobre onde seria mais
proveitoso mendigar, já que os últimos vestígios das esmolas ricas
tinham evaporado e os últimos frutos das vitórias passadas tinham sido
gastos. Maria tornou-se de novo uma criatura esfomeada, como era no
sítio... Agora até se alegraria com a taboa, isso se fosse época de
taboa... Apesar de tudo, Maria não perdia as esperanças de estar perto
de sua irmã rica... Maria imaginava Ksiénia rica: ``Já que ela ignora
sua pobre família da aldeia e não dá nenhuma notícia, deve ser rica,''
pensava ela.

Maria chegou até os confins da cidade, onde o rio estava congelado e só
se distinguia dos campos brancos ao redor por uma margem íngreme. Ela
viu uma cerca como a que havia no sanatório perto de sua casa... Quis
atravessar o portão, mas um velho a impediu:

--- Ei, você, vá embora daqui.

--- Vovô --- disse Maria ---, não estou pedindo esmola... Eu tenho uma
irmã que trabalha aqui, vim de longe para vê-la.

--- Que irmã?

--- Ksiénia.

--- Qual é o sobrenome?

--- Eu não sei o sobrenome.

--- Então vá embora.

--- Vovô --- disse Maria ---, eu vim de longe, do sítio Lugovoi... Nosso
pai morreu no ano passado, porque foi um ano de fome. Mamãe ficou
sozinha com cinco crianças. A nossa casa desabou, e os chefes do colcoz
nos deram outra, perto da \emph{tamba}; mamãe ficou nessa casa, porque
todos nós inchamos, e não sobrou nada para trocar por comida, tínhamos
só a roupa do corpo e uns farrapos.

--- Está bem --- disse o velho ---, vá para a recepção e pergunte pela
sua irmã --- e ele deixou Maria passar.

Ela entrou e viu ao redor: uma casa bonita, antiga, branca, em volta da
qual havia um jardim coberto de neve onde velhinhos e velhinhas
passeavam. Maria teve receio de perguntar a eles. Pensava: ``Aquele
velho acreditou em mim, mas estes podem não acreditar e me expulsar. E
onde vou me meter?''. E saiu andando a esmo ou, mais precisamente,
farejando cheiro de sopa e de cebola frita. Ao se aproximar da entrada,
ela avistou uma mulher gorda vindo ao seu encontro com um balde de
lavagem na mão, do qual saía vapor de tão quente. Maria sempre confiou
mais nas pessoas gordas do que nas magras, pois os gordos têm sempre
algo de sobra, enquanto os magros raramente têm o que dividir, de modo
que ele mesmo é que recebe.

--- Tia --- disse Maria ---, onde está minha irmã Ksiénia?

--- Korobko? --- perguntou a mulher.

--- Sim --- respondeu alegremente Maria, pensando: ``Se é Ksiénia, é a
minha irmã, mesmo que seja Korobko''.

--- Não trabalha mais aqui --- respondeu a mulher gorda ---, se demitiu
ainda no Primeiro de Maio e foi embora da cidade.

Então Maria começou a chorar, e chorava de forma tão amarga, soluçava
tanto, que a mulher gorda logo se pôs a chorar também, sem largar o
balde. Depois disse:

--- Não chore, menina. Como eu era amiga de Ksiénia, sei o endereço
dela... Ela partiu para Vorónej, casou-se com um hóspede.

--- Mas como irei até Vorónej? --- indagou Maria e continuou a chorar.

--- Vamos --- respondeu a mulher gorda ---, vou lhe dar um pouco de sopa
e depois veremos.

Ela levou Maria, congelada e trêmula, a um aposento onde lavavam pratos,
sentou-a em um banquinho e lhe serviu sopa quente numa tigela de ferro.
Por muito tempo Maria se recordou daquelas mãos rechonchudas, enrugadas
devido à água quente, com dedos curtos, estendendo-lhe um prato de sopa,
acomodando-a num banquinho, em um canto quente, pois para Maria nessas
mãos havia algo de divino... Ela não se lembrou disso para sempre, pois
para sempre só é necessário se lembrar Dele, e não de Suas
manifestações, mas por muito tempo se recordou... Existe um tipo de
bondade que provém dos homens e não é consagrada pelo Altíssimo. Maria
comera o frango assado na estação de Andréievka sem qualquer sentimento
e recebera o dinheiro da bela mulher do trem como se recebesse uma
esmola qualquer --- uma casca de pão ou cinco copeques... Mas ela tomou
o prato de sopa velha, num aposento onde lavavam pratos, de modo solene
--- o caráter solene que envolvera seu pranto, no campo nevado, a
caminho de Andréievka, também envolvia a gratidão que sentia pela sopa
requentada que lhe foi servida em Lgov. Aqui não se tratava de bondade
humana, mas de bondade divina...

Pela segunda vez, Maria leu o sermão do Senhor sem ajuda de palavras e
absorveu, sem o uso da razão, o que é revelado permanentemente aos
profetas através de sua justiça e razão. Ela escutou, sem palavras, e
compreendeu, sem razão, o que foi dito através do profeta Isaías:

``Pobre, foi castigada por tempestades, desconsolada. Eu revestirei tuas
pedras com rubis e erguerei tua fundação com safiras. Tuas janelas farei
de rubis, tuas portas de pérolas e todas as tuas muralhas de pedras
preciosas''.\footnote{Isaías 54:11, 12.}

A mulher gorda, chamada Sofia, uma lavadora de pratos analfabeta, de uma
bondade que não era humana, mas divina, muitas vezes tinha escutado o
Senhor sem ajuda de palavras e O compreendido sem razão. Agora ela
também O compreendeu sem razão, que era confusa, mas com o coração
silencioso, através do profeta Isaías:

``Reparte teu pão com quem tem fome e leva para casa os pobres errantes;
veste o nu quando o encontrares e não te escondas de quem possui teu
próprio sangue...''.\footnote{Isaías 58:7.}

Sofia pegou seu casaco acolchoado que estava pendurado num canto e o deu
a Maria, dizendo:

--- Vista o casaco, senão irá congelar.

E acrescentou:

--- Assim que acabar meu plantão, vamos até a estação e eu pedirei ao
condutor que leve você até Vorónej, porque eu não tenho dinheiro para a
passagem.

O plantão de Sofia terminou perto do meio-dia e, nesse intervalo, ela
ofereceu comida mais duas vezes à Maria, painço cozido com cebola frita
e macarrão e, para a viagem, deu-lhe um pedaço de pão e um pedaço de
arenque.

Assim que Maria e Sofia chegaram à estação, esta falou à menina que se
mostrasse triste e até começasse a chorar. Porém, por mais que Maria se
esforçasse, dessa vez as lágrimas não vieram.

--- Está bem --- disse Sofia ---, talvez a gente consiga convencer o
condutor mesmo sem choro.

Ela escolheu um condutor a dedo, e não o calmo, que afavelmente recusava
um lugar a qualquer pessoa por o vagão estar cheio, mas aquele que
xingava e empurrava todo mundo. Sofia aproximou-se dele e, sem
preâmbulo, começou a falar dos sofrimentos de Maria.

--- O que você quer? --- o condutor a interrompeu, bravo. --- Para que
está me contando essas histórias? Eu mesmo posso lhe contar algumas.

--- Leve esta menina até Vorónej, para ela encontrar sua irmã --- disse
Sofia.

--- E você é o que dela?

--- Ninguém --- disse ela ---, mas agora nós lhe serviremos de família.

O condutor ficou em silêncio, mas Sofia se postou a seu lado, sem
arredar dali, e ordenou a Maria que ficasse com ela. Quando terminou o
embarque, o condutor disse:

--- Vou permitir, mas que ela se ajeite em qualquer lugar.

Sofia abraçou e beijou Maria, fez o sinal de cruz e disse palavras
vazias, acessíveis, pronunciadas por muitos, pois não eram palavras
divinas, mas humanas:

--- Que o Nosso Senhor, Jesus Cristo, a guarde.

E ela repetiu as mesmas palavras ao condutor, que respondeu:

--- Deixe disso, tia, faz tempo sue seu Cristo foi revogado por decreto,
o melhor seria rezar para que a menina não cruze com um inspetor no
caminho...

O condutor do vagão nº 7 tinha razão. A força do homem comum não está na
palavra divina, mas na ação divina --- somente a força dos profetas se
acha na palavra de Deus.

Assim, graças à ação divina de Sofia, a lavadora de pratos, e do
condutor do vagão nº 7, Maria foi levada a Vorónej. O trem chegou no
meio da madrugada e, no início, Maria pensou em esperar na estação até
amanhecer, mas depois mudou de ideia: ``Afinal, ela é minha irmã''. Ela
perguntou a um policial de plantão onde ficava a rua, que, como se
revelou, era bem próxima. ``Eu vou,'' decidiu.

As ruas de Vorónej eram mais largas que as de Lgov, e, à primeira vista,
a nova cidade lhe agradou. ``É um pouco parecida com Izium,'' pensava
Maria, ``em Kursk não davam boas esmolas, mas em Izium davam. Mesmo se
eu não achar minha irmã, passarei o inverno aqui.'' Assim pensando,
Maria andou conforme o endereço que tinha, pois ela sabia ler e escrever
--- aprendera antes do ano da fome ---, entrou num quintal, que estava
silencioso e escuro, pois era noite e todos dormiam. Aproximou-se da
porta, sempre conforme o endereço, e começou a bater. Bateu, bateu, mas
ninguém abriu. ``Será que ela viajou?'' pensou Maria com tristeza,
``talvez ela não esteja ouvindo, quem sabe se eu bater na janela do
outro lado...''

De repente uma janelinha se abriu sozinha, e de lá saltou um homem
vestindo umas calças brancas enfiadas em botas de feltro. Assim que ele
passou a toda por ela, como se fugisse de cachorros, Maria entendeu que
não eram calças, mas ceroulas. Ficou perplexa e, nesse momento, ouviu a
porta se abrir. Foi correndo até lá e deparou com sua mãe, só que muito
mais jovem e bonita, em algo lembrando a moça da estação de Andréievka.
Com os lábios pintados e o rosto muito pálido, ela segurava uma vela
numa mão e a gola de um penhoar azul com fios dourados na outra.

--- Quem está aí? --- perguntou.

Assim que ela começou a falar, Maria compreendeu que não era sua mãe,
mas Ksiénia, sua rica e linda irmã.

--- Ksiénia --- disse Maria ---, sou eu, sua irmã Maria.

Então Ksiénia deu um grito, deixou cair a vela, abraçou Maria e,
desfazendo-se em lágrimas, levou-a para dentro. Tudo o que Maria havia
suposto se confirmou. A casa era rica e num dos quartos havia um
guarda-roupa e um divã --- tudo isso Maria conhecera nas casas que lhe
ofereciam boas esmolas. No outro quarto havia uma cama larga e
desarrumada e dois travesseiros enormes.

--- Eu pensei em quem estaria batendo à noite --- disse Ksiénia e
chorou. --- Como você me achou, irmãzinha?

Maria começou a contar sobre a morte do pai, a casa que desabou e
Vássia. Ksiénia perguntou:

--- Quem é Vássia?

--- É o nosso irmãozinho --- respondeu Maria.

--- Mas nós temos um irmão mais novo? --- admirou-se Ksiénia. --- Eu
conheço Kólia, Chura e você, mas, quando eu fui embora, você era tão
pequena, ainda engatinhava.

--- Nós temos um irmão ainda mais novo, o Jórik --- disse Maria ---, só
que ele não está mais em casa --- e contou também sobre Jórik.

Ksiénia desatava no choro a cada história que a irmã contava. Maria
contou tudo, exceto o que Gricha lhe fizera naquela noite no celeiro, e
também não mencionou o homem vestido em ceroulas que ela tinha acabado
de ver saltar da janela.

--- Está bem --- disse Ksiénia ---, está bem, irmãzinha. Amanhã de manhã
chega o meu marido, Aleksei Aleksándrovitch; ele é técnico ferroviário,
um homem bom, generoso. Vamos convencê-lo а deixar você passar o inverno
conosco, e depois veremos...

Realmente, pela manhã chegou Aleksei Aleksándrovitch, o técnico
ferroviário. Maria viu um homem bem agasalhado, com uma peliça curta,
calças acolchoadas e botas de feltro, e, ao tirar seu chapéu de pele com
abas para orelhas, descobriu-se uma cabeça calva. Ksiénia começou a
abraçá-lo e a beijá-lo tanto que Aleksei Aleksándrovitch disse:

--- Deixe-me antes me lavar, gatinha, estou cheirando a graxa.

--- Essa é minha irmã Maria --- disse Ksiénia ---, ela veio passar um
tempo aqui.

--- Que fique --- respondeu Aleksei Aleksándrovitch ---, a casa é ampla,
lugar não faltará.

Maria passou a morar lá. Levantava-se de madrugada, ainda no escuro, e
precisava acender uma vela na cozinha quente em que dormia para começar
a limpeza. Aleksei Aleksándrovitch comprava caixas de velas baratas e
com seu uso economizava eletricidade. Primeiro, Maria recolhia sua roupa
de cama do chão --- uns xales e uns paletós velhos --- para lavar o piso
da cozinha, depois engraxava os sapatos; assim que clareava, ela ia para
dentro, sentava-se à mesa com Aleksei Aleksándrovitch e Ksiénia para
tomar chá adocicado e comer pão com banha de porco ou geleia de ameixa,
e voltava para a faxina, mas agora nos quartos... Mal se dava conta, já
era hora do almoço, quando Aleksei Aleksándrovitch voltava para casa. O
almoço era sempre farto e delicioso. Ksiénia cozinhava bem, pois tinha
sido cozinheira na casa de repouso. Fazia \emph{borsch}, como o que a
velha mendiga do sítio sonhava receber numa casa rica, macarrão com
molho de carne, almôndegas com painço cozido, e panquecas. Maria comia e
pensava: ``Ah, Vássia deveria estar aqui... Chura e Kólia também...''.

Depois do almoço, Maria lavava a louça por longo tempo, sob o olhar
atento de Ksiénia. Inicialmente fervia um tambor com água para tirar a
gordura dos pratos e talheres, enxaguando com água fria peça por peça.
Assim Maria passou o inverno, satisfeita. Em seu tempo livre, ia à feira
com Ksiénia ou passeava pela cidade. ``Vorónej é um bom lugar, não é
como Kursk,'' pensava, ``agora, além de estar perto da minha irmã, é
possível viver de esmola, sem emagrecer.'' Assim Maria passava o inverno
quando, certo dia, Ksiénia lhe disse:

--- Engraxe as botas de serviço de Aleksei Aleksándrovitch, porque ele
fará uma viagem de trabalho.

Maria começou a engraxar as botas, que eram pesadas, de couro grosso,
com forro de flanela e pequenas ferraduras pregadas na sola. Como Maria
sofreu para engraxá-las, quantos panos velhos gastou, quanta graxa, para
que as botas adquirissem brilho e o couro amaciasse! Aleksei
Aleksándrovitch calçou as botas, bateu com os pés no chão e disse:

--- Agora eu não vou molhar os pés. Às vezes, que diabo, os canos
explodem e, com as botas de feltro, os pés ficam molhados.

Aleksei Aleksándrovitch partiu e Ksiénia disse:

--- Maria, hoje você não vai mais varrer o chão, pois é de mau agouro
quando alguém viaja. Se você quiser, pode passear um pouco e depois se
deitar.

Maria passeou por Vorónej até começar a anoitecer, depois voltou e viu
Ksiénia sentada diante de um espelho, e seu rosto era tão belo que não
devia nada ao da jovem da estação de Andréievka. ``Se nossa mãe pudesse
ver Ksiénia agora, ficaria feliz'' pensava Maria.

--- Maria --- disse Ksiénia, alegre, cantarolando ---, coma as
almôndegas com pão e pode ir se deitar. Hoje você não vai limpar nada.

Maria jantou fartamente na cozinha, acomodou-se sobre os xales macios e
adormeceu depressa. No meio da noite foi despertada por sussurros e
risadinhas. ``Aleksei Aleksándrovitch deve ter voltado,'' pensou.

De repente, a conversa silenciou e Maria ouviu gemidos de Ksiénia.
``Está doente,'' pensou Maria, ``Ksiénia adoeceu.'' Ela se levantou e
foi até a porta, que estava trancada, e não conseguiu sair da cozinha.
Ficou junto à porta ouvindo os gemidos de Ksiénia. Mas sua irmã gemia
tão melodiosamente que era como se, movida por uma dor aguda, cantasse
uma canção alegre. E de súbito Maria se lembrou de como Gricha gemera no
celeirо escuro ao violentá-la. ``Será que sentirei isso algum dia?''
pensava Maria. ``Comida é possível mendigar e, para alguém habilodoso,
não é difícil conseguir um abrigo para passar a noite, mas experimente
pedir por esse prazer...'' De repente Maria sentiu um calafrio, como se
estivesse no meio de um campo gelado, mesmo estando em casa, na cozinha
quente. Desejou voltar ao celeiro escuro, sobre a palha, se não com o
jovem bonito da estação de Andréievka, na pior das hipóteses com Gricha.
``Será que aprenderei a gemer tão agradavelmente?'' pensava Maria,
febril.

No entanto, os gemidos sonoros de Ksénia repentinamente cessaram e logo
soou um barulho indescritível, como se alguém quisesse tirar o
guarda-roupa de casa e tivesse entalado na porta. Maria ouvia gritos de
diferentes vozes, dentre as quais a de Ksiénia. Se Maria entendesse algo
de música, perceberia que as vozes soavam no mesmo tom, sem articular um
discurso compreensível. De repente, a porta da cozinha escancarou e dela
irrompeu alguém familiar, o homem que Maria, na noite de sua chegada a
Vorónej, tinha visto saltar da janela de Ksiénia, fato que havia
escondido da irmã. E de novo ele vestia ceroulas brancas enfiadas em
botas de feltro. Irrompeu e correu para a janela, seguido por Aleksei
Aleksándrovitch, bem agasalhado, de calças acolchoadas, metidas nas
botas engraxadas por Maria. Logo após, Ksiénia entrou correndo,
totalmente nua. Embora Maria estivesse apavorada, ficou tão admirada com
a visão de Ksiénia nua que não podia desviar os olhos esbugalhados da
irmã e involuntariamente se comparava a ela. Os seios de Ksiénia eram
leitosos, encorpados, empinados, e na ponta de cada um se via um mamilo
vermelho e comprido, igual aos dedinhos de bebê de Jórik. Já Maria, em
vez de seios, tinha dois montículos que precisavam ser apalpados para
serem descobertos, e cada mamilo parecia uma espinha. O corpo de Ksiénia
também era leitoso, sem ossos aparentes, e a barriga e as pernas se
fundiam completamente; como ela tinha perdido todo o pudor em função da
briga feroz de dois homens cobertos, não foi sua vergonha que se
desnudou, mas sua beleza e, de vez em quando, os dois lhe lançavam
olhares, já não brigando com tanta veemência. Ksiénia apareceu nua, mas
era como se estivesse vestida; se Maria aparecesse nua, permaneceria nua
e seria motivo de zombarias. As pernas de Maria eram ossudas e a barriga
saliente, e onde Ksiénia exibia beleza Maria exibia uma cicatriz. Antes
Maria não pensava nessas coisas, mas, depois que Gricha a violentara,
ela começou a pensar e agora, olhando para Ksiénia, entendia que, se ela
um dia permitisse que alguém se aproximasse, seria no escuro. Já Ksiénia
poderia fazê-lo também na claridade...

Assim, através de Gricha e dos episódios que se desenrolaram depois,
Maria conheceu o terceiro pesado flagelo enviado aos homens pelo Senhor,
do qual falava o profeta Ezequiel. O terceiro flagelo do Senhor, um
animal chamado luxúria, é peculiar, pois os profetas não temem a espada,
a fome ou a doença, mas temem o animal. O rei Salomão, que era justo,
fora castigado com о terceirо flagelo. E Dã, a Áspide, o Anticristo,
sabia que, ao transpor caminhos terrenos, não precisava temer o primeiro
flagelo, a espada, pois era imortal; nem o segundo, a fome, pois sua
bolsa de pastor estava repleta do pão impuro do exílio; tampouco o
quarto, a doença, pois ele era suscetível somente aos castigos divinos;
mas do terceiro flagelo, o adultério, ele deveria se precaver.

Moisés, que havia conduzido o povo para longe da opressão egípcia, dizia
que o terceiro flagelo ainda viria, enquanto Jeremias, que muitos
séculos depois conduziu o povo para longe da opressão babilônica, dizia
que o terceiro flagelo já tinha vindo. Se para Moisés o destino do rei
Salomão, o justo, ainda era desconhecido, para Neemias era conhecido e
funcionava como uma parábola.

--- Não pecou por isso Salomão, o rei de Israel? --- dizia Neemias. ---
Poucos povos tiveram um rei como Salomão; ele era amado por seu Deus, e
Deus o tornou rei de todos os israelitas; no entanto, até ele foi levado
a pecar por mulheres estrangeiras...\footnote{Neemias 13:26.}

Mas qual é o mistério do terceiro flagelo do Senhor? Por que estão
fadados a ele não somente os pecadores, mas também os justos? Porque a
espada, a fome e a doença apenas dilaceram, enquanto o animal feroz, ao
dilacerar, frutifica. Porque no terceiro flagelo não há somente o joio,
mas também o trigo. Porque nem a razão nem a justiça nos salvarão dele.
Mesmo os ascetas, que violam o próprio corpo, não se salvarão: eles
alcançarão somente a deformidade, como mostra o exemplo dos monges de
Alexandria, aos quais os cristãos da Idade Média haviam substituído a
imagem de Jesus, da tribo de Judá.

Em sua indignação com a mulher pela maçã dо Éden, o Senhor transferiu o
terceiro flagelo para as mãos poderosas do ímpio, por isso só é possível
resistir a ele pela não violência, como ensinara o profeta Jeremias e
como, sete séculos depois, reforçou Jesus, da tribo de Judá. No entanto,
a salvação só é possível segundo a ressalva do profeta Jeremias:
entregue tudo ao ímpio, mas conserva, como um butim, a sua própria alma.
O amor foi inventado justamente para que se pegue o butim, a alma, do
ímpio adúltero. Para que, ao separar o joio do trigo e ao pagar o
tributo ao animal-luxúria, se conserve a frutificação. Mas, em virtude
de tal amor, é necessário que se cumpra a maldição de Deus e se suplante
a razão seduzida pela serpente. Mesmo se a razão for grande, o justo
ficará enfraquecido, como ficara o rei Salomão, em quem a mulher superou
Deus. No entanto, uma razão pequena é um estorvo ainda maior para o
trabalho espiritual, pois, à medida que se diminui a razão, diminui-se a
necessidade de dominá-la e se cresce a atração pelo ócio. A mais elevada
manifestação do ócio espiritual é o animal-luxúria. Assim era também na
Antiguidade.

--- Levanta teus olhos para as alturas --- disse com tristeza Jeremias,
o profeta sofredor --- e vê: onde tu não te profanaste? Tu sentavas para
eles na estrada e maculavas a terra com tua perversão e malícia. Por
isso foram retidas as águas e não vieram as chuvas tardias, mas tu
exibias a face de prostituta e não te envergonhavas.\footnote{Jeremias
  3:2, 3.}

--- A cada começo de estrada, tu construías para ti um lugar alto ---
disse também o profeta do exílio Ezequiel ---, para desonrar tua própria
beleza e abrir as pernas a qualquer um que passava, assim multiplicaste
a tua libertinagem...\footnote{Ezequiel 16:25.}

Assim também pensava o técnico ferroviário Aleksei Aleksándrovitch, mas
à maneira de Vorónej. Ele brandiu a mão e esbofeteou os dentes do homem
de ceroulas.

--- Por que está me batendo? --- perguntou ele, limpando o sangue.

--- Por sua baixeza --- explicou Aleksei Aleksándrovitch.

--- Então bata nela --- disse o homem de ceroulas ---, eu somente
satisfiz o seu pedido.

Nesse momento, Aleksei Aleksándrovitch chutou-lhe com a bota que Maria
engraxara, a das ferraduras, pesada como uma pedra do calçamento. Após o
golpe, o homem das ceroulas enfiadas em botas de feltro passou a andar
como numa parada, com passadas solenes e de trás para a frente, golpeou
a janela, quebrou o vidro e atirou-se pelo vão, lançando as botas para
cima, de maneira que, num instante, ele desapareceu da cozinha, onde
restaram Aleksei Aleksándrovitch, bem agasalhado e de botas, e Ksiénia,
nua e descalça, já que Maria, esquecida por todos num canto, parecia não
existir. Marido e mulher ficaram praticamente face a face. Sem nem tirar
o chapéu de pele, ele fixou os olhos injetados em Ksiénia por um ou dois
minutos. Depois estendeu os braços para segurá-la e castigá-la. Ksiénia
não o impediu, mas apenas, com um movimento das coxas largas, evitou que
ele a pegasse pela garganta e, em vez disso, Aleksei Aleksándrovitch,
visivelmente em choque, começou, com uma mão, a estrangular o seio
encorpado e leitoso de Ksiénia, motivo pelo qual seu mamilo, comprido
como um dedinho de bebê, tensionou; com a outra mão, Aleksei
Aleksándrovitch agarrou Ksénia por sua beleza exuberante, deu um grito,
ergueu-a como uma pesada caixa de ferramentas da ferrovia, pressionando
a palma contra o ventre redondo da esposa, e levou-a da cozinha. Com seu
braço rechonchudo que tinha uma covinha no cotovelo, Ksiénia fechou a
porta atrás de si.

Por algum tempo se podia ouvir um ruído atrás da porta, o choro fraco de
Ksiénia, mas logo tudo silenciou. De repente, Ksiénia voltou a gemer
melodiosamente. Tendo perdido o amante, ela seduziu o marido... E de
novo Maria sentiu um calafrio, mas muito mais forte que o primeiro, um
calafrio não só por tudo o que havia acontecido como também pelo vento
que vinha da janela quebrada.

Maria não dormiu a noite inteira --- tentando se aquecer, andou de um
lado para outro enrolada nos xales velhos sobre os quais dormia. De
manhã, bem tarde, Ksiénia finalmente entrou na cozinha, de rosto
amassado, sonolento e feio, justo ela que, antes disso, tinha o rosto
sempre bonito.

--- Escute --- disse Ksiénia, com o rosto feio e enrugado ---, eu e
Aleksei Aleksándrovitch decidimos mandá-la para casa, para a aldeia.
Vamos colocá-la no trem, daremos dinheiro e provisões... Você concorda?

--- Concordo --- respondeu Maria.

Ao dizer isso, sentiu de fato vontade de ver o sítio, com o canteiro em
frente da casa, onde se podia colher morangos silvestres e cogumelos.
Adiante ficava a igreja, ao lado o clube e, no pé da colina, o riacho
com o moinho d'água. O riacho corria para a outra vila,
Kom-Kuznetsóvskoie, a \emph{tamba} ia até a cidade de Dimítrov e, depois
da \emph{tamba}, via-se o \emph{zakáz}.

--- Eu já queria voltar para casa --- disse Maria ---, só que eu e
Vássia não conseguimos achar a nossa vila de jeito nenhum. Passamos por
muitas vilas, mas não encontramos a nossa. Tínhamos até um guia
particular --- Maria, porém, não estendeu a conversa sobre o guia,
Gricha.

--- Como não? --- disse Ksénia. --- Por acaso não sabe que a nossa vila
pertence ao distrito de Dimítrov?

--- Eu sabia que se pode chegar até Dimítrov pela \emph{tamba}, mas não
que Dimítrov é um distrito --- respondeu Maria. --- Nossos pais nunca
nos explicaram, tinham outras coisas para fazer.

Por Ksénia, ela soube que sua mãe também se chamava Maria e seu pai
Nikolai, como seu irmão Kólia. \footnote{Kólia é apelido de Nikolai.}

--- Mas eu tenho medo --- disse Maria --- de que Kólia e Chura me culpem
por eu ter abandonado Vássia numa terra estranha e por não ter cuidado
bem dele.

--- Você não tem culpa --- respondeu Ksiénia ---, não foi por sua culpa
que nossos destinos foram separados.

Dizendo isso, deu uma espiada na porta da cozinha, para se certificar
que estava fechada, e sussurrou:

--- Pegue isto e, para que ninguém descubra, esconda bem e não gaste,
porque é dinheiro. Para você, em separado, darei dinheiro e provisões,
conforme foi combinado com Aleksei Aleksándrovitch, mas este dinheiro é
meu, de minha parte, para mamãe. Caso ela não esteja em casa, entregue
para Kólia e Chura --- disse estendendo um pacote à Maria. --- Esconda
na sua calcinha, mas, na hora de fazer as necessidades, não o perca.

Assim fez Maria, e Ksiénia acompanhou-a diretamente da cozinha até o
trem, de modo que Maria não mais entrou nos quartos e não se despediu de
Aleksei Aleksándrovitch. Já de Ksiénia se despediu afetuosamente.
Chorando, Ksiénia abraçou-a e beijou-a, acenando-lhe até ficar fora de
vista. E com Ksiénia desapareceu a bela cidade de Vorónej.

Maria viajava em assento próprio e com provisão própria; além disso,
levava entre as pernas, apertado por um elástico, o pacote com o
dinheiro que Ksiénia mandara à mãe. Durante a viagem, não falou com
ninguém, para proteger seu dinheiro, e também não dividiu o pão e o
embutido. Se alguém lhe pedisse, até poderia dar, mas oferecer de
vontade própria, isso ela não faria. ``É melhor levar os sobras para
Kólia e Chura,'' pensava, ``passam fome no sítio.'' Mas ninguém lhe
pediu pão, nem suspeitou do dinheiro.

Maria chegou a Dimítrov e, ao se ver entre lugares conhecidos, derramou
lágrimas de alegria. ``Todos os lugares são diferentes,'' pensava,
``Kursk é uma cidade ruim, Lgоv um pouco melhor, Izium e Vorónej são
cidades muito boas, mas não existe nada como a casa da gente, em parte
alguma.'' Maria começou a andar por Dimítrov e reconheceu a casa diante
da qual sua mãe, também chamada Maria, acomodara-os sobre um punhado de
palha e os deixara para ir à feira atrás de ameixas secas, e, nesse
meio- tempo, uma mulher estranha levara embora o pequeno Jórik. Maria
saiu da cidade e caminhou pela \emph{tamba}, reconhecendo o lugar onde o
forasteiro lhe oferecera pão, mas sua mãe, assustada, tirou-o dela e o
jogou no campo. À medida que Maria avançava, reconhecia lugares
queridos. Lá estava o \emph{zakáz}, todo branco, brilhando ao sol, com
galhos curvados sob a neve. Lá estavam o riacho e as rodas do moinho
presas no gelo. Logo surgiu a igreja sobre a colina. Maria não notou
nada fora do comum, como quando procurava sua vila com Vássia e Gricha,
o guia que a violentara no celeiro. Aquela vez, ela só vira o que era
estranho, desconhecido; agora, tudo lhe era familiar. Ali estava a vila
de Chagaro-Petróvskoie, invernal, nevada, bela, com fumaça saindo das
casas. Na rua uma multidão de pessoas andava segurando bandeiras. Então
Maria ouviu uma conversa entre um homem que vagamente reconhecia e um
camponês local que conhecia de vista.

--- Que festa é esta? --- perguntou o homem.

--- Camarada, não é uma festa, é o enterro de um membro do partido ---
respondeu o camponês.

--- Quem estão enterrando?

--- Petró Semiónovitch, o chefe da brigada. Ele foi assassinado ---
respondeu o camponês.

--- Mas quem o matou?

--- Parece que o moleiro, não pode ser outro --- disse o camponês ---, o
moleiro e seu caçula, Lióchka, por causa do seu filho mais velho, Mitka,
que Petró Semiónovitch estrangulou... Mas o moleiro fez de um jeito que
Petró Semiónovitch não parecia um cadáver, mas uma peça de carne bovina
ou suína largada em cima do balcão do açougue de Dimítrov, nos tempos de
fartura. Agora o moleiro e seu filho Lióchka serão condenados e
fuzilados, levaram o dois para Khárkov.

Maria ouviu a conversa, mas não reconheceu o homem que indagava.
Lembrava-se, porém, de Petró Semiónovitch. Apesar de lamentar sua morte,
ela não chorou. Reparou que todos em volta lamentavam a morte de Petró
Semiónovitch e também não choravam, levando-o em silêncio dentro de um
caixão fechado. Maria passou pelo cortejo, e logo surgiram o sítio
Lugovoi, a cerca do sanatório, o canteiro coberto de neve e, atrás dele,
sua querida casa. Ah, como seu coração disparou, como ela gostaria que
sua mãe, Maria, lhe abrisse a porta e a abraçasse entre lágrimas, como
Ksiénia fizera ao se despedir, e que Vássia estivesse ao seu lado, que,
sem mais roubar, teria voltado para casa antes dela...

Mas foi Chura, sua irmã, quem abriu.

--- De onde você veio? --- perguntou.

--- Eu estive em Vorónej --- respondeu Maria ---, na casa de Ksiénia.

--- E onde está Vássia?

--- Vássia --- disse Maria --- se perdeu em Izium.

E aconteceu o que Maria mais temia: Chura começou a repreendê-la.

--- Como foi abandonar Vássia num lugar estranho? --- disse ela.

Maria ficou em silêncio, não sabia o que responder. A voz de Kólia soou
de dentro de casa:

--- Quem está aí?

--- É Maria, que acabou de voltar --- disse Chura ---, mas Vássia se
perdeu no caminho.

E Kólia também a repreendeu:

--- Por que você não cuidou de Vássia? O que vamos responder para mamãe,
que mandou uma carta perguntando de você e de Vássia?

Assim que Maria ouviu falar da carta de sua mãe, esqueceu-se de qualquer
mágoa.

--- Onde está mamãe? --- perguntou.

--- Ela está em Kertch --- respondeu Kólia. --- Mas por que você não
ficou com Ksiénia? Por acaso ela vive na pobreza, não poderia
alimentá-la?

--- Não --- respondeu Maria ---, Ksiénia não vive na pobreza, até mandou
dinheiro para vocês; eu voltei porque estava com saudade de casa ---
depois de dizer isso, tirou o pacote da calcinha e o entregou a Chura.

Chura pegou o pacote e começou a contar o dinheiro com Kólia. Depois ela
disse:

--- Ksiénia sempre soube se arrumar na vida, com abundância e riqueza,
mas aqui estamos perdidos... Ela fugiu em 1923, aos catorze anos, com um
fotógrafo que passou por aqui, e não voltou mais... Ela ainda está com
esse fotógrafo?

--- Não --- respondeu Maria ---, ela mora com Aleksei Aleksándrovitch,
um técnico ferroviário, mas, antes disso, trabalhava numa casa de
repouso em Lgov... Mamãe tinha me contado.

--- Mamãe sempre a mimou mais do que os outros filhos --- disse Chura
--- e o pai, quando estava vivo, amava-a muito, eu me lembro... Bem, já
que você voltou, sente-se, eu vou lhe servir um prato de \emph{borsch}.

E ela deu a Maria um prato de \emph{borsch} frio, de gosto entre o
amargo e o salgado, pensando que a irmã ficaria agradecida, já que,
antes de sua partida, não comiam um \emph{borsch} como esse --- desde
que Chura e Kólia passaram a trabalhar no campo do colcoz, começou a
haver alguma comida na mesa, embora, com certeza, não sobrasse nada.

Disso Maria pôde se convencer depressa, pois voltou a passar fome, só
que ali não havia a quem pedir esmola, não era Vorónej ou Izium; ali, em
sua terra natal, a esmola que davam era ainda pior que a de Kursk. Mas
ali as paragens eram bonitas no verão, até no inverno eram bonitas, só
era ruim no outono e na primavera, quando chovia, mas, no inverno e no
verão se vivia muito bem...

Maria atravessou a \emph{tamba} coberta de neve, marcada por trilhas de
carros e de carroças e, por um atalho, dirigiu-se ao \emph{zakáz}. As
botas de feltro e o xale dados por Ksiénia a aqueciam, assim como o
casaco acolchoado de Sofia, a lavadora de pratos de Lgov; ela caminhava
agradavelmente e respirava com leveza. Um pássaro levantou voo, deixando
cair neve dos galhos de um abeto, e ela se sentia bem. Apenas tinha fome
e se agustiava com a solidão. Antes nem sua mãe nem seu pai, quando ele
estava vivo, se ocupavam dela, tampouco Chura ou Kólia, mas ela estava
sempre ao lado de Vássia. Pode-se dizer que Maria fez as vezes de mãe
para Vássia, e ele, quando pequeno, trouxera muitas alegrias à irmã. E
Maria começou a se acusar em pensamento de ter se descuidado de Vássia.
Talvez devesse ter ido atrás dele em Izium, em vez de ter seguido o
jovem em Andréievka...

Angustiada, Maria saiu do \emph{zakáz} e viu brilhando sobre o campo
branco um sol cor de framboesa. Ela se ajoelhou, sem que ninguém lhe
houvesse ensinado, virou o rosto em direção ao sol, estendeu as mãos,
como fazia ao esmolar, e disse:

--- Senhor! Jesus Cristo! Filho de Deus!

Maria se lembrou da história que o velhinho bondoso, o vigia noturno da
estrebaria de Izium, contara sobre os judeus-\emph{jides} que tinham
matado o Filho de Deus, e chorou copiosamente por não saber quem poderia
salvar Vássia, pois, embora o Filho de Deus continuasse vivo, estava no
céu, enquanto Vássia estava na terra, em Izium...

Nesse meio-tempo, um rapaz que passava pelo campo nevado perguntou a
Maria o mesmo que o jovem da estação de Andréievka havia perguntado:

--- Menina, por que estás chorando?

Maria respondeu:

--- Estou chorando porque os judeus-\emph{jides} mataram o Filho de Deus
e ele agora está no céu, mas meu irmão Vássia está na terra, em Izium, e
não há ninguém para ajudá-lo.

E Dã da tribo de Dã, a Áspide, o Anticristo, disse as palavras do Senhor
que haviam sido pronunciadas pelo profeta Isaías, palavras celestias em
que se encontra o sentido de tudo e que Dã havia reservado para o
momento final, mas, ao passar por ali, compreendera que havia chegado a
hora de usá-las, para depois somente repeti-las inúmeras vezes.

--- Eu me revelei aos que não perguntavam por mim --- através de Isaías,
Dã pronunciava as palavras do Senhor ---, e fui encontrado pelos que não
estavam à minha procura...\footnote{Isaías 65:1.} Já os que me procuram
--- acrescentou Dã, após uma pausa --- não me encontrarão... Eu me
revelei àqueles que eu mesmo escolhi, e não aos que me escolheram... Já
os que me escolheram devem lembrar as palavras do meu Irmão, Jesus, da
tribo de Judá, sobre seus filhos malvados e os cães bondosos dos
outros... Os cães não tomarão a parte que cabe às crianças, mesmo que
sejam crianças malvadas... Somente através da fé eles poderão receber
seu quinhão... As crianças, ao contrário, mesmo sem rezar, receberão sua
parte... Assim diz o Cristo... Pois ou tu tens uma matilha forte, que
toma, ou tu tens Deus, que oferece...

Dizendo isso, ele atravessou o campo em direção ao \emph{zakáz} e
desapareceu. Só quando ele ficou fora de vista, Maria deu-se conta de
que era o forasteiro que havia lhe oferecido pão duas vezes, e lamentou
não ter lhe pedido novamente, pois agora o chefe da brigada, Petró
Semiónovitch, estava morto e sua mãe vivia em Kertch, e não havia
ninguém para lhe tirar o pão --- ela poderia se saciar. Pois não sabia
se Chura lhe daria algo, mesmo um prato de \emph{borsch} amargo e gelado
ou uma tigela de mingau ralo. As lembranças de uma vida farta em terras
estranhas só aumentavam a fome em terra natal. O mendigo que vagueia
pelo mundo, longe de sua morada, não se surpreende com nenhum alimento.
Prova de tudo, venha do pobre ou do rico. ``Pena que eu não lhe pedi
pão,'' Maria voltou a refletir, ``eu não entendi o que ele disse; esse
homem, pelo visto, vem de longe, mas o pão eu bem que comeria.''

Escurecia rapidamente. Na região de Khárkov, as noites de inverno são de
um frio gelado e, quando as estrelas aparecem e a lua começa a brilhar,
esfria ainda mais.

Maria voltou para casa depressa e lá encontrou Chura, que disse:

--- Vá se deitar, Maria, porque amanhã você se levantará cedo... Kólia e
eu resolvemos mandá-la para a casa de mamãe em Kertch. Você concorda?

--- Concordo --- respondeu Maria, pensando consigo: ``Eu estou com
saudade de mamãe e quem sabe lá eu não passe fome''.

De manhã, no escuro, enquanto ainda brilhava uma lua gelada, Maria se
aprontou, beijou Kólia e Chura e partiu. Parecia estar triste, mas não
muito... Ela havia procurado sua terra natal e a encontrado, portanto,
abandonava-а sem angústia, partindo a um lugar desconhecido, à cidade de
Kertch, para encontrar sua mãe. Seria bom ir à casa dela, ela sentiria
pena da filha e a alimentaria com o que tivesse. Adeus, \emph{zakáz};
adeus, igreja da colina; adeus, moinho d'água... A vila de
Chagaro-Petróvskoie desaparecia, assim como desapareceram Izium, onde
ela vivera tão bem, Kursk, onde passara tão mal, e Vorónej, onde vivera
ainda melhor, perto de Ksiénia. Maria seguia pela \emph{tamba,} na
contramão, rumo à estação de Dimítrov. Kólia e Chura lhe deram dinheiro
para a passagem, mas não pão; só que não adiantaria mendigar na
\emph{tamba}, ela teria que passar fome até chegar a Dimítrov. Os
mendigos possuem uma lei: quando a fome apertar, se abasteça de
paciência. Com efeito, foi em Dimítrov que ela conseguiu pão seco, perto
de uma casa rica. Ela o comeu na estação e comprou uma passagem até
Khárkov, porque não vendiam para Kertch. O trem havia se transformado
para Maria em algo habitual, e Khárkov também não a surpreendeu como da
primeira vez. Lá ganhou mais um pouco de pão de uns passageiros ricos,
comprou uma passagem e partiu para Kertch, que ela acreditava ser
parecida com Izium ou Vorónej, pois sua mãe, Maria, não escolheria morar
em uma cidade ruim como Kursk.

A viagem para Kertch durou um dia e uma noite. De manhã ela acordou e
olhou pela janela: a neve tinha desaparecido, o sol brilhava e no
horizonte se estendia um campo azul infinito.

--- É o mar --- explicaram ---, suas águas vão até a Turquia, um estado
estrangeiro.

E Maria olhou ao redor: a terra se encostava no céu!

--- É o monte Mitrídates --- explicaram...

Kertch em nada se parecia com Izium, onde a esmola era boa, com Kursk,
onde a esmola era ruim, ou com Vorónej, onde vivera tão bem com
Ksénia... ``Como será este lugar?'' pensava Maria, descendo do trem para
a terra quente. Ela caminhava e se surpreendia a todo instante, quase
como da primeira vez em que estivera em Khárkov, quando correra com
Vássia por entre as árvores estranhas plantadas em vasos. As ruas de
Kertch não eram como em outros lugares, mas pedregosas e íngremes, e o
mar, que, visto do vagão, de longe, parecia limpo e largo como um campo,
era ruidoso, enfumaçado e até pequeno, apenas um pouco maior que um rio,
porque se avistava a outra margem com inúmeras casinhas que, em vez de
se fincarem no chão, se empilhavam, umas sobre as outras. ``Como é
possível? Que esquisito!'' pensou Maria e perguntou ao redor.

--- Não é o mar --- disseram ---, é uma baía e um porto. O mar fica mais
adiante, virando a esquina.

Maria caminhou pela rua de pedra e realmente viu o mar sem fim...
``Mesmo diferente,'' pensou Maria, ``parece uma boa cidade, mamãe fez
bem em se recrutar aqui.''

No entanto, a cidade de Kertch só deixou de lhe causar uma impressão
insólita quando ela chegou à periferia, onde, conforme o endereço que
tinha, sua mãe morava. Longe da periferia, as casas eram alegres, de
pedra branca, mas, onde sua mãe vivia, eram severas, de tijolo vermelho,
cobertas de fuligem, como as casas perto da estrada de ferro de Vorónej.
Ela entrou num prédio e perguntou onde morava Maria Korobko. Explicaram.
Não por sua mãe ser conhecida por todos, mas porque, por acaso, ali se
achava uma mulher que a conhecia. Maria aproximou-se de uma porta, bateu
e ouviu a voz de sua mãe. Ao ouvir essa voz, os pés e as mãos de Maria
começaram a tremer, lágrimas jorravam de seus olhos, e ela irrompeu
gritando: ``Mãezinha!''. Sua mãe estava sentada na cama remendando uma
camisa masculina militar. Ao ver a filha, ela empalideceu e disse a
outras três mulheres, que também estavam sentadas em suas camas,
ocupadas com seus afazeres:

--- É minha filha Maria...

A mãe começou a chorar com mais força que Maria, e elas choravam tanto
que as três mulheres, que também se desfaziam em lágrimas, mal
conseguiram acalmá-las. Quando, enfim, sossegaram, a mãe disse:

--- Você vai morar comigo... Ali na panela sobrou mingau de ontem,
coma...

As mulheres se chamavam Olga, Klávdia e Matvéievna. E cada uma ofereceu
algo à Maria... Uma deu pão, outra uma bala e Matvéievna duas
maçãzinhas.

--- Pegue --- disse ela ---, estamos na Crimeia, aqui as frutas são
nossa fonte de alimentação.

Depois as três mulheres saíram.

--- Vamos, mulheres --- disse Matvéievna ---, vamos passear um pouco...
Deixem mãe e filha conversarem.

Maria começou a contar sobre sua vida à mãe, contou tudo, exceto sobre a
violência que Gricha lhe fizera no celeiro e também sobre o marido de
Ksiénia, Aleksei Aleksándrovitch, que flagara a esposa no meio da noite
com um homem de ceroulas. E sua mãe a repreendeu:

--- Como você pôde abandonar Vássia numa terra estranha?

--- Eu sei que sou culpada --- respondeu Maria com tristeza. --- Não o
vigiei bem.

--- Pelo menos Ksiénia, minha filha mais velha, conseguiu o que queria
--- disse a mãe ---, e eu me alegro com isso... Quando eu tive Ksiénia,
ainda havia pão e toucinho à vontade; no almoço, comíamos uma libra
inteira de toucinho... Naquele tempo, eu era jovem, bem alimentada, e
seu pai, Kólia, também era jovem e bonito. Chura e Kólia nasceram quando
nossa vida ainda era farta, mas foram criados durante a fome... Mas
você, Vássia e, especialmente, Jórik vieram da fome...

Então se ouviram batidas na porta e um homem de meia-idade entrou:

--- Remendou minha camisa militar? --- perguntou ele.

--- Remendei --- respondeu a mãe ---, mas veja, Saviéli, que alegria
inesperada eu tive... Minha filha Maria chegou...

--- Que bom --- disse Saviéli ---, as crianças pequenas devem ficar
perto da mãe...

Maria passou a morar com a mãe, habituou-se à nova vida e à cidade de
Kertch. E era uma cidade tão bonita que, depois dela, seria difícil
morar em qualquer outra. Ela não lamentaria se tivesse que viver ali até
morrer. As pessoas em volta tinham pena de Maria. Saviéli e Matvéievna
eram bons e Klávdia gentil, apenas Olga era um pouco desagradável. Sua
mãe também não gostava de Olga. Sem querer Maria a ouviu dizer a
Matvéievna:

--- Olga está descontente por minha filha ter vindo morar comigo. Ela
diz que tem três filhos na aldeia e também deveria trazê-los para cá...
E que aqui ficou apertado...

--- Não faz mal --- respondeu Matvéievna ---, não é Olga quem manda
aqui, mas a sociedade... Deixe Maria ficar por enquanto... Mas no outono
que vem será preciso colocá-la na escola, senão seria como se ela
vivesse no antigo regime... Será que foi para isso que a revolução
bolchevique e o finado Lênin lutaram?

--- Mas, quando ela morava comigo, estudou três anos --- respondeu
timidamente a mãe.

--- É pouco --- disse Matvéievna. --- Eu e você, por exemplo, somos
ignorantes... E o que conseguimos? Cavar terra... Nossos filhos deveriam
ser médicos e engenheiros...

--- Uma das minhas filhas, Ksiénia, vive ricamente em Vorónej ---
gabou-se a mãe. --- Ela é uma beldade, como eu fui nos anos de
juventude... Seu marido é um técnico ferroviário. Maria foi visitá-la,
comia até se fartar, e ganhou dela um lenço, botas de feltro e um casaco
acolchoado.

--- Não foi Ksiénia quem deu o casaco acolchoado, foi a tia Sofia, de
Lgov --- disse Maria.

--- Não interrompa quando os mais velhos estão falando --- disse a mãe,
brava. --- Eu a mimei muito, Matvéievna... Ela foi viajar com Vássia,
meu filho mais novo, descuidou-se dele e ele se perdeu.

--- Vássia não é o mais novo, é Jórik --- disse Maria ---, que uma tia
levou embora, em Dimítrov, quando você nos deixou para vender seu xale e
comprar ameixas secas na feira.

--- Está querendo voltar amanhã mesmo para nossa vila? --- perguntou a
mãe, brava. --- Viajou bastante, eu vejo, e se encheu de malícia.

Matvéievna apoiou a mãe:

--- Não irrite sua mãe, ela trabalha por você.

Então Maria se desculpou por sua grosseria, e elas a perdoaram.

Essa conversa aconteceu por volta do terceiro mês de convívio de Maria
com sua mãe, e foi a primeira vez que ela ficou zangada com a filha. Já
a havia repreendido por causa de Vássia, mas era a primeira vez que se
zangava de fato. Na manhã seguinte ao ocorrido, Maria dirigiu-se à
cidade de Enikalé, que era contígua a Kertch, como Dimítrov à vila de
Chagaro-Petróvskoie. É preciso notar que, àquela altura, Maria já
frequentava Enikalé em busca de esmolas, pois, embora morasse com sua
mãe, continuava com fome. Em Kertch Maria tinha medo de que Matvéievna
pudesse vê-la; já em Enikalé ninguém a conhecia. Ela escolhia as casas
mais ricas para pedir comida, onde moravam judeus, gregos ou tártaros, e
eles lhe davam. Costumava caminhar pela areia da praia, respirava o
vento marítimo e alegrava-se, revigorando-se, pois mesmo quem sofre de
tifo ganha um aspecto saudável perto do mar. Nessa época, Maria aprendeu
a nadar no mar e o fazia tão bem quanto no rio. Ela já tinha seu próprio
caminho de Kertch a Enikalé. A areia longe do mar era macia e quente;
perto do mar, dura e fresca; a água era limpa, deixando visíveis todas
as pedrinhas ao fundo; ao longe, se elevavam duas rochas e, mais ao
longe, se avistava o monte Mitrídates. Nesse dia, Maria resolveu
banhar-se, pois era primavera e, nessa época do ano, o sol ali era tão
quente como no verão de Khárkov. Maria olhou em volta e não viu ninguém;
tirou o vestido --- não usava calcinha devido ao calor --- e correu para
a água, e de repente notou que seus seios, embora não fossem iguais aos
de Ksiénia, já não eram dois montículos e o mamilo estava saliente, não
parecia mais uma espinha. As pernas se fundiam à barriga de forma tão
bela que ela sentiu vontade de acariciá-las, sem mais ter vergonha de se
olhar à luz do dia... Maria, porém, não se deu conta de que um grego a
espivava, antigo proprietário de uma cafeteria de Enikalé, agora
funcionário de uma associação de alimentação popular. Após o desjejum, o
grego gostava de passear ao longo da praia, munido de um binóculo
marítimo, na esperança de flagrar uma mulher nua. Ele viu Maria e a
desejou. Quando Maria terminou de se banhar e se vestiu, fresca, limpa e
cheirando a mar, o grego se aproximou:

--- Menina, aonde você vai?

--- Eu vou esmolar em Enikalé --- disse Maria.

--- Você não tem vergonha? --- disse o grego. --- Uma menina tão
bonita... Ah, isso não é bom... Venha comigo, eu vou lhe dar carne
assada, você quer?

Maria olhou para o homem, que era bonito e rico e não era russo, e
sentiu vontade de comer carne assada com ele. Foi até a casa dele, na
cidade de Enikalé. Lá havia tapetes por toda parte e um cheiro doce e
agradável, que não era russo. Uma velha trouxe um prato quente de carne
assada, salpicada com pó vermelho. Maria deu uma mordida e sentiu sua
garganta queimar, o que fez o grego desatar numa risada:

--- É pimenta grega... Fogo em pó...

Maria se empanturrou de carne e bebeu tanto vinho doce que o grego
mandou a velha tirar a garrafa, que se tornou inútil. A menina se deitou
no tapete macio, e o grego deitou-se ao lado dela. E Maria conseguiu do
grego o que queria, o que Ksiénia tomava do amante e do marido e o que
Gricha, o guia, tomara dela no celeiro escuro. Ouvindo sua própria voz,
melodiosa, emitindo gemidos de alegria, Maria agarrou-se ao grego, um
homem bem nutrido, belo, não russo, usufruindo de sua força para obter
prazer, o dia todo, a noite toda, a madrugada toda.

--- Como foi fraco o teu coração --- disse o Senhor através do profeta
Ezequiel ---, tu fizeste essas coisas como uma meretriz
indômita.\footnote{Ezequiel 16:30.}

Desde a infância, Maria vivenviou o segundo flagelo do Senhor, a fome;
mas a fome prazerosamente saciada tem poder de embriagar, de excitar o
corpo, e o segundo flagelo foi superado pelo terceiro: o animal feroz, a
luxúria, o adultério.

Maria não se afastou do grego até a manhã seguinte, e teria ficado ali
por mais tempo se ele não tivesse dito:

--- O normal é que o homem viole a mulher, e não o contrário... Você é
uma menina tola, comeu muito da minha carne e tenta violar a mim, um
homem grego...

O grego expulsou Maria, sem sequer lhe oferecer comida na despedida.
Triste e faminta, Maria regressou a Kertch, pois a energia obtida com a
carne assada havia sido gasta no grego. Maria chegou receosa à caserna
dos trabalhadores --- o alojamento de tijolos vermelhos onde sua mãe
morava --- e pensava numa forma ardilosa de esconder de sua mãe o que
havia acontecido, assim como escondera a violência que Gricha lhe fizera
no celeiro e o sujeito de ceroulas que o marido de Ksiénia havia
flagrado. No entanto, esconder isso era mais fácil por ter acontecido
quando Maria estava sozinha, em terra estranha, mas agora ela morava com
sua mãe. Com tais pensamentos, Maria se aproximou da caserna, subiu a
escada de ferro e no corredor encontrou Matvéievna, que disse, em
pranto:

--- Onde você se meteu? Nós a estávamos procurando. Sua mãe foi
atropelada por um trem, e agora você é órfã.

No início, ela não pôde atinar com o que Matvéievna dizia. Quando caiu
em si, sentou-se no chão, no corredor perto da porta. Enquanto isso, sua
mãe jazia em um caixão de pinho, colocado sobre a mesa de refeições do
alojamento, entre as quatro camas. Muitos vieram se despedir,
principalmente mulheres, mas também alguns homens, amigos de Saviéli,
que construíra pessoalmente o caixão.

--- Por que você está sentada aqui? --- Olga perguntou, brava, assoando
o nariz com um lencinho e enxugando os olhos. --- Por que não vai se
despedir de sua mãe?

Maria, porém, continuava sentada no corredor, perto da porta, sem ter
nada a dizer. Apenas entreabriu a porta, uma pequena fresta, pela qual
viu o topo da cabeça imóvel de sua mãe, envolvido nо lenço branco de
Matvéievna. Olhou por um minuto ou dois e fechou a porta. Passou um bom
tempo, talvez uma hora, até que ela alargasse um pouco mais a fresta e
visse a testa branca dе sua mãe sob o lenço de Matvéievna. Maria voltou
a fechar a porta e ficou longo tempo sentada, sem nada dizer, então
tornou a abri-la, um pouco mais, e viu uma vela acesa colocada nas mãos
de sua mãe, cruzadas sobre o peito. Encostou a porta novamente e
continuou sentada no corredor, mesmo com as súplicas de Matvéievna e
Saviéli para que ela entrasse e se despedisse. Maria abriu a porta ainda
três ou quatro vezes, cada vez a abria mais, até encarar sua mãe de
frente, deitada no caixão com o lenço branco de Matvéievna, a vela nas
mãos, e o vestido preto de lã que usava nas festas de família, no sítio
Lugovoi... E Maria lembrou que, ao atravessarem o \emph{zakáz} a caminho
da aldeia Popovka, indo à casa de seus avós, na Páscoa, quando seu pai
ainda vivia, Vássia estava em casa e era pequeno como Jórik e Jórik não
havia nascido, era aquele vestido preto de lã que sua mãe usava... Assim
que viu sua mãe por inteiro, Maria aceitou o fato, escancarou a porta e
entrou no quarto para se despedir. Os pés descalços da mãe estavam
brancos, assim como o rosto e as mãos. Apareceram muitas crianças do
alojamento com seus pais e até de outros prédios, e a todas elas
Matvéievna deu maçãs, bolachas doces e avelãs da Crimeia.

Assim Maria ficou sem mãe, e ninguém sabia o que fazer com ela. Embora
todos que a rodeassem fossem gentis, para Maria não passavam de
estranhos, assim como Maria em relação a eles.

--- É preciso mandá-la para a casa de seus irmãos --- disse Saviéli. ---
Você quer ficar com seus irmãos? --- perguntou a Maria.

--- Não --- ela respondeu. --- Chura e Kólia, que moram no sítio, estão
passando fome; em relação a Ksiénia, que mora em Vorónej, seu marido, o
técnico ferroviário Aleksei Aleksándrovitch, não gosta de mim.

--- Então pode ir para um orfanato --- disse Matvéievna ---, aqui em
Kertch temos um bom orfanato.

Maria começou a chorar.

--- O orfanato é a coisa que mais temo na vida --- disse ela.

--- E o que você quer? --- disse Matvéievna. --- Nessa idade, não se
pode ficar sozinha, porque você pode seguir um caminho ruim e acabar no
roubo ou na prostituição, ou até nas duas coisas.

Maria respondeu:

--- Nunca em minha vida roubei nada, somente pedi. Eu não consegui
livrar Vássia do roubo e sei que sou culpada. Mas prostituição eu não
sei o que é.

Saviéli riu e disse:

--- É quando uma mulher devassa faz por dinheiro o que uma mulher
legítima faz de graça.

--- Ah, sem-vergonha --- disse Matvéievna. --- Dizer isso na presença de
uma menina...

No entanto, Maria compreendeu do que se tratava, pois agora era
entendida nesse assunto, e pensou: ``Então o que Ksiénia faz com Aleksei
Aleksándrovitch é uma coisa, e o que eu fiz com o grego é outra...
Aquilo é permitido, mas isto é comparável ao roubo e deve ser bem
escondido''.

Saiu do quarto apavorada com a ideia de suspeitarem do grego de Enikalé
e angustiada ante a necessidade de evitar o orfanato de Kertch. De toda
maneira, havia decidido ficar em Kertch, porque era uma cidade boa,
quente e próxima do mar, que, até então, ela nunca tinha visto. Até sair
de Dimítrov pela primeira vez, com sua mãe e Vássia, ela nem sabia o que
era uma locomotiva. Agora Maria sabia o que era navio a vapor, chalana e
muitas outras coisas, pois passou a pedir esmola no porto. Algumas vezes
ela fazia com os marinheiros o que deveria ser bem escondido e era
comparável ao roubo, mas depois uma mulher bateu nela com mais força que
a mulher de Kursk, e Maria parou de frequentar o porto. Além disso, os
marinheiros faziam tudo muito depressa, sobre bancos duros ou no chão, e
Maria não conseguia usufruir de sua força viril para obter prazer, como
se dera com o grego. E não lhe pagavam com carne assada, mas com pão ou
peixe seco, o que ela poderia conseguir esmolando, sem fazer o que era
comparável ao roubo. Após ter apanhado no porto, ela até perdeu a
vontade de fazer tais coisas, mas não o desejo de, pelo menos mais uma
vez, gemer melodiosamente de prazer, como gemia sua irmã Ksiénia com o
marido e o amante e como ela mesma gemera com o grego, que, sem que ela
soubesse o porquê, na manhã seguinte ficara zangado e descontente com
ela.

Maria não voltou ao alojamento depois da morte de sua mãe, por medo de
que Matvéievna a forçasse a ir para um orfanato. Dormia onde podia, já
que a primavera em Kertch era quente e, quando chovia, sempre era
possível se proteger embaixo de um alpendre.

Em uma noite mais quente, ela resolveu pernoitar na praia, sob um
alpendre, porque, com o céu estrelado, às vezes caía uma chuva rápida,
que sussurava no teto por um ou dois minutos, silenciava e então tornava
a sussurrar por cinco ou dez minutos. A lua sobre o mar em nada lembrava
a lua sorturna e toldada de Khárkhov, que, quando brilhava, era como se
fosse acometida por uma febre de tifo, o que só agradava na ausência de
coisa melhor, e cujos jogos de luz só eram comparáveis aos da lua de
Kursk, esta totalmente esquálida e severa. Em seu esplendor, a lua sobre
o mar não era inferior à de Poltava e ainda a superava muitas vezes em
dimensão. Assim como a lua de Khárkov e a de Kursk, que se firmavam
sobre o campo ou sobre o \emph{zakáz}, a lua de Poltava se fixava
solidamente no céu, mas a lua sobre o mar parecia sempre estar a ponto
de cair. Como se a qualquer instante pudéssemos ouvir o estrondo de sua
queda no mar. Ela não caía, mas a expectativa de uma queda imimente
atormentava o coração.

Nessa noite, o coração de Maria estava especialmente atormentado, o que
poderia ser explicado pelo fato de quase não ter recebido esmola nesse
dia e por estar com fome, mas também pelo ruído diferente que a chuva
fazia, como se estivesse conversando com o alpendre, se calasse para
refletir e então voltasse a falar. O céu estava coberto por grandes
estrelas do sul e a lua se mostrava tão instável e tão grande que
parecia se rescostar no mar --- se alguém fechasse os olhos, poderia
ouvir o estrondo da queda e, ao abri-los, a lua não estaria mais no céu.
Nesse estado de espírito, Maria não conseguia dormir. De repente, ela
ouviu passos de alguém à beira do mar que estalavam com pedrinhas
molhadas. Ela olhou e percebeu que era um homem. ``Irei até ele e
pedirei pão,'' pensou, ``se não der, talvez eu me deite com ele sob o
alpendre e, em troca, ele me dará pão ou peixe seco.'' Maria se
aproximou do homem e reconheceu o forasteiro de Khárkov que ali, em
Kertch, onde ela, depois da morte de sua mãe, vivia em completa solidão,
não parecia um estranho. E disse Maria, estendendo as mãos para ganhar
uma esmola:

--- Senhor! Jesus Cristo! Filho de Deus!

E Dã da tribo de Dã, a Áspide, o Anticristo, respondeu:

--- Não é a mim que chamas, mas a meu Irmão, da tribo de Judá. Eu sou
Dã, da tribo de Dã, o Anticristo, o Filho de Deus enviado para a
Maldição, pronunciada pela primeira vez na montanha de Ebal. Ainda não
chegou a hora da Bênção, pronunciada pela primeira vez na montanha de
Garizim, por isso meu Irmão Jesus, da tribo de Judá, não responderá...

Maria não entendeu o que foi dito, porque não possuía razão, e não havia
nada nas palavras de Dã, o Anticristo, que pudesse ser compreendido sem
o uso da razão. E ela começou a chorar. Então Dã, a Áspide, o
Anticristo, perguntou:

--- Por que estás chorando?

--- Meu pai morreu há muito tempo --- disse Maria ---, e minha mãe
acabou de morrer. Meus irmãos me rejeitaram e eu perdi meu irmão mais
novo, Vássia, em Izium, e agora não há ninguém para cuidar de mim e eu
não tenho de quem cuidar... Estou sozinha...

Dã respondeu a ela:

--- Tem piedade de tua mãe, chora por ela, mas esse choro não trará
alívio. Ela não morreu por ação dos homens, pois o Senhor também julga o
pobre sem indulgência... Até monstros dão o seio a suas crias para
amamentá-las, mas a filha do povo tornou-se cruel como avestruzes no
deserto. A língua do bebê gruda na garganta de tanta sede, e as crianças
pedem pão, mas ninguém o dá.\footnote{Lamentações de Jeremias 4:3, 4.}

Assim disse Dã, a Áspide, o Anticristo, através do profeta Jeremias,
então tirou de sua bolsa de pastor o pão impuro do exílio, legado pelo
profeta Ezequiel, e o estendeu à Maria. Dessa vez, ninguém lhe tirou o
pão, sobre o qual o Senhor disse:

``Come-o como se fosse pães de cevada e asse-o à vista deles com
excremento humano...''\footnote{Ezequiel 4:12}

Então o Senhor disse:

--- Assim os filhos de Israel comerão o pão impuro, entre os povos pelos
quais Eu os espalharei.\footnote{Ezequiel 4:13}

O profeta Ezequiel, no entanto, convenceu o Senhor a assar o pão do
exílio em estrume de vaca, e não em excremento humano.

E Maria, a menina mendiga, através do pedaço de pão impuro do exílio,
uniu-se ao desígnio de Deus, e todos que a conheceram e a depravaram,
mesmo por ordem de Deus, tornaram-se detestáveis ao Senhor, e todos que
a ajudaram, mesmo por vontade própria, e não de Deus, tornaram-se
agradáveis a Ele. Através do pão impuro do exílio, Maria uniu-se a um
povo estranho, como Tamar\footnote{Personagem do Velho Testamento, Tamar
  casou-se com dois filhos de Judá (filho de Jacó): Her e Onã, ambos
  mortos por castigo de Deus. Depois, Judá a encontrou e, sem
  reconhecê-la e confundindo-a com uma prostituta, engravidou-a.}se
unira a Judá e Rute, uma moabita, se unira a Boaz, que era de Belém, uma
cidade judia. Maria não escolheu, mas foi escolhida. E Dã, a Áspide, o
Anticristo, uniu-se à Maria através do terceiro flagelo do Senhor, o
único dos quatro flagelos a que ele estava vulnerável em seu caminho
terreno.

Eles se deitaram sob o alpendre, o mar rumorejava na escuridão, a chuva
sussurrava por alguns minutos e silenciava, e Maria respondia a esses
sons vindos de várias direções com gemidos alegres e melodiosos. De
repente, ela ouviu um estrondo na água, como se um peso insustentável
tivesse caído no mar. Maria olhou por baixo do ombro ossudo de Dã, o
Anticristo, e percebeu que a lua tinha sumido. De repente tudo
silenciou, o mar e a chuva se calaram, como se os dois refletissem;
Maria enrodilhou-se como uma rosca, como dormem os desabrigados sob o
friozinho da manhã, e adormeceu, acalentando a semente ainda fresca do
sexto filho de Jacó, estranha ao seu ventre eslavo. Аfastando-se da
menina adormecida, Dã, o Anticristo, saiu andando ao longo da praia.

Dã se encontrava perto de sua terra natal, pressentia isso, e seu
coração batia como o coração do filho pródigo diante da porta paterna.
Caminhava sobre Panticapeu,\footnote{Antiga colônia grega (séc. VI a.
  C.), a cidade de Penticapeu, na Crimeia, foi fundada junto ao monte
  Mitrídates.} terra helênica, onde, antes do nascimento de seu Irmão,
Jesus, os helênicos haviam construído seus povoados ao pé do monte
Mitrídates. E por haver algo de helênico ele já se sentia em terra
própria, pois, embora fossem hostis ao povo de Dã, os helênicos não eram
estranhos a ele; da mesma forma, dois povos podem ser estranhos um ao
outro, mas não hostis ou podem ser tanto estranhos quanto hostis...
Assim como não existem dois homens iguais, cada um é bom para si mesmo,
não existem dois povos iguais, e os povos, como os homens, dependem de
seu destino. Аlguns povos convivem pacificamente, enquanto outros não
têm uma relação agradável, apesar de estarem ligados entre si pelo
destino, assim como se dá entre os homens...

Ainda não tinha amanhecido, mas o trabalho da alvorada estava no auge de
suas forças quando Dã, da tribo de Dã, a Áspide, o Anticristo, parou
para descansar perto de Enikalé. No mesmo lugar em que Maria se banhara
e pela primeira vez admirara seu corpo repleto de força femínea. O lugar
realmente era magnífico, o mar translúcido da manhã fazia reluzirem como
joias as pedras que surgiam da água nos bancos de areia, mas a força
inerte das rochas, vinda da água, lembrava a quem o doce rumor das ondas
na calma matinal deslumbrava que na beleza do mar, como em toda beleza
desmedida, predomina a crueldade, e a crueldade só pode ser admirada por
uma alma abatida. A beleza do mar não é humana, assim como a beleza do
cosmo. Para o homem, a grandeza espiritual não se acha no mar, nas
rochas ou no mundo absoluto, mas no campo, na relva, no riacho, no céu
terreno... A bíblia surgiu perto do mar, mas quase todas as suas ações
acontecem longe dele, entre vales e rios, entre pastagens, em cidades do
interior, e não da costa. E será que os feudos das principais tribos dos
filhos de Jacó, que desencadearam as grandes paixões bíblicas, estavam
distantes do mar por acaso?... O Senhor apareceu a Abraão numa colina e
a Moisés numa sarça,\footnote{Do Velho Testamento, arbusto que ardia sem
  se consumir através do qual Deus surgiu a Moisés, que foi encarregado
  de libertar os hebreus da escravidão egípcia.} Moisés conversou com o
Senhor no monte Sinai, no deserto, e um Anjo apresentou-se a Jacó numa
sarça... O homem mora ao lado do mar, vive graças a ele, admira-o, mas o
mar só pode lhe ensinar o que é ser forte, porém cruel; belo, porém mau;
grandioso, porém sem coração... Não foi por acaso que, entre todas as
tribos de Israel, somente a de Dã, incumbida de conceber o Anticristo,
possuía seu feudo perto do mar. De Hetalon а Emat e até
Haser-Enã,\footnote{Ezequiel 47: 15, 16, 17: ``Eis os limites da terra:
  do lado do norte, desde o Grande Mar: o caminho de Hetalon até a
  entrada de Emat, Sedada, Berota, Sabarim, que fica entre os limites de
  Damasco e os de Emat, Haser-Ticon, junto à fronteira de Aurã. Os
  limites irão desde o mar até Haser-Enã, tendo ao norte o território de
  Damasco e o território de Emat.'' (\emph{Bíblia de Jerusalém}, ed.
  Paulus, 2016, p. 1549)} do leste ao oceano estendiam-se as terras de
Dã, criado para amaldiçoar os feitos dos homens. Depois de cinco feudos,
achava-se o de Judá, de onde surgiu Cristo, enviado para abençoar.
Jesus, da tribo de Judá, foi levado ao mar pelo infortúnio; veio para
operar milagres, para andar sobre as águas como se andasse sobre a
terra, mas sua alma continuava no deserto, no rio Jordão, na Cidade
Santa...

Seu irmão, Dã, da tribo de Dã, o Anticristo, cujo feudo ficava perto do
mar, não se acalmou na costa. No mar só é possível se alegrar desprovido
de razão: a razão absorve o tormento das águas.

Dã, o Anticristo, olhou para o monte Mitrídates: atrás das ruínas da
Fortaleza Genovesa medieval, o sol se levantava numa faixa de luz
estreita e afiada, cortando, como uma espada, nuvens sombrias,
encharcadas de sangue, de modo que, se fossem espremidas, despejariam no
mar uma chuva rubra, até se esgotarem e se transformarem em nuvenzinhas
tão leves que mesmo uma brisa seria capaz de levá-las... Ele viu os
últimos pingos de sangue caírem da espada ao mar e as faixas sangrentas
da alvorada surgirem sobre as ondas. E Dã, a Áspide, o Anticristo, disse
consigo, através do profeta Jeremias:

--- Minhas entranhas, minhas entranhas! Sinto umа tristeza no fundo do
coração, meu coração se agita dentro de mim, não posso calar-me, pois tu
ouves, minha alma, o som da trombeta, o clamor da batalha...\footnote{Jeremias
  4:19.}

Então ele pronunciou de novo as palavras do profeta Jeremias:

--- Eles negaram a existência do Senhor e disseram: ``Ele não existe;
nenhuma desgraça nos atingirá, não veremos a espada nem a
fome''.\footnote{Jeremias 5:12.}

E Dã, o Anticristo, uma criança judia que amadureceu através de caminhos
terrenos e se transformou num jovem, disse:

--- Faz anos que o Senhor os castiga com o segundo flagelo --- a fome
---; que se sentem sempre indefesos diante do terceiro flagelo --- o
animal-adultério ---; e que se torturam com o quarto --- a espada ---,
que agrega todos os outros... Como disse o Senhor: ``Toda a terra será
assolada, mas a destruição não será completa''.\footnote{Jeremias 4:27.}

Dizendo isso, Dã, o Anticristo, seguiu adiante, para continuar
executando a Maldição prescrita pelo Senhor. O caminho que lhe fora
indicado o conduzirá à cidade de Rjév, um local totalmente diverso, para
visitar outros destinos humanos. Embora ele só devesse aparecer em Rjév
após seis anos, no ano 1940 depois do nascimento de seu Irmão Jesus, da
tribo de Judá, Dã sumiu rapidamente daquelas paragens e, por mais que
Maria o procurasse, não pôde encontrá-lo.

Condenada por prostituição e vadiagem, Maria teve o filho de Dã, o
Anticristo, no hospital da prisão. Conforme a medicina, o bebê não
sobreviveria, pois a mãe era uma menina e estava exaurida; no entanto,
ele sobreviveu e Maria o chamou de Vássia, como seu irmão perdido. O
bebê tinha olhos profundamente pretos e seu narizinho não eslavo quase
tocava no lábio quando ele sorria à mãe. Quando Maria dava o peito à sua
boca ávida, entregando-lhe toda a seiva de seu corpo, obtida com
dificuldade da pobre sopa prisional, ela gemia de forma tão alegre e
melodiosa que o médico da prisão dizia:

--- Isso é algo doentio... Será que ela não esconde uma sífilis?

--- Ela deve ter se deitado com um \emph{jid,} ou com algum georgiano ou
armênio --- dizia a auxiliar de enfermagem da prisão, que tinha aversão
à Maria por ela ter tido um filho judeu...

Tiraram Vássia de Maria, seu filhinho de olhos pretos, e o entregaram a
um orfanato. Depois disso, Maria não quis mais viver; morreu aos quinze
anos no hospital da prisão, em 23 de fevereiro de 1936, e foi enterrada
sem caixão. No mesmo dia, seu nome foi retirado da lista de racionamento
da prisão e seu processo arquivado.
