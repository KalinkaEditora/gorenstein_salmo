\begin{quote}
\emph{Dedicado à minha mãe.}

\emph{Não seguirás a multidão para fazeres o mal; não deporás, em um
processo, em nome da multidão, torcendo o direito. Não favorecerás o
pobre em seu processo.}

Segundo Livro de Moisés. Êxodo\footnote{Êxodo 23: 2,3.}

\emph{Seguir os pensamentos de um grande homem é a mais fascinante das
ciências.}

Púchkin\footnote{\emph{O negro de Pedro, o Grande}, tradução de Boris
  Schnaiderman. (Editora 34, 1999)}

\emph{Já ouvi falar também, e muito, de como você se pinta. Deus te deu
uma cara e você faz outra. E você ondula, você meneia, você cicia, põe
apelidos nas criaturas de Deus, e procura fazer passar por inocência a
sua volúpia.}

Shakespeare. Hamlet\footnote{\emph{Hamlet}, tradução de Millôr
  Fernandes. (L\&PM, 1997)}.
\end{quote}

I

``Que desgraça! Um alvoroço de numerosos povos! Fazem tanto barulho como
o mar. O clamor dos povos! Seu clamor é como o clamor de águas
violentas!''.\footnote{Isaías 17:12.} Assim falou Isaías, filho de Amós
e o profeta que, oito séculos antes da estrela de Belém, previu o
Nascimento da Criança, do Filho, ao seu povo amado, embora desobediente
e teimoso. Seu povo era exaurido por brados e tropéis vindos de todos os
lados. Assim falou o profeta, cujo ouvido sensível distinguiu o tropel
mais perigoso, que vinha do Norte.

Sim, tumulto e inquietação se espalharam pela terra. Mas, quanto mais
nos elevamos ao céu, menor o tumulto, quanto mais nos aproximamos do
Senhor, menor a piedade Dele pelos homens. Eis por que o Senhor, por
piedade dos homens, mandou seus mensageiros à terra. O Senhor não os
escolheu sozinho, mas enviou os eleitos e designados pelos profetas.
Esse direito foi concedido ao homem somente no princípio da existência,
na criação do mundo. ``O Senhor Deus moldou da terra todos os animais
dos campos e todos os pássaros do céu e os levou ao homem para ver como
ele os chamaria, e, como o homem designasse cada alma vivente, esse
seria o nome dela.''\footnote{Gênesis 2:19.} Dessa maneira, o Senhor
introduziu no homem a força do Criador e o iniciou nos mistérios da
arte. No sétimo dia da criação, houve o Nascimento da arte, no sétimo
dia foi dado ao homem esse dom divino, até hoje reservado para seus
eleitos. Entre os eleitos, Ele separou os profetas adivinhos, menores e
maiores, e, entre estes,\footnote{Segundo classificação de Agostinho, os
  profetas maiores, os que escreveram mais, são Isaías, Jeremias
  Ezequiel e Daniel; os menores, Oseias, Joel, Amós, Abdias, Jonas,
  Miqueias, Naum, Habacuc, Sofonias, Ageu, Zacarias e Malaquias (os doze
  últimos livros proféticos do Velho Testamento).} elegeu somente três
--- Moisés, o criador da Lei de Deus; Isaías, que previu a vinda de
Cristo, da tribo de Judá, o Messias; e Jeremias, que previu a vinda de
Anticristo, da tribo de Dã, o Antimessias.

Em seu leito de morte, Jacó, o fundador de Israel, anunciou o futuro a
seus doze filhos, para que eles saciassem a curiosidade por seu próprio
destino e empregassem suas forças apenas na realização do Testamento. Ao
quarto filho, Judá, ele disse:

--- Tu, Judá, serás louvado por teus irmãos! Tua mão pousará sobre as
costas de teus inimigos. Os filhos de teu pai se curvarão a ti. Um jovem
leão é meu filho Judá, que se ergue de sua presa. Ele se inclinou e se
deitou como um leão e uma leoa: quem o despertará? Um cetro não se
afastará de Judá nem um legislador de seus pés até que venha Siló e os
povos Lhe obedeçam...\footnote{Gênesis 49:8--10.}

Ao seu sexto filho, Dã, Jacó disse:

--- Dã julgará seu povo como uma das tribos de Israel. Dã será uma
serpente na estrada. Uma áspide no caminho que fere a pata do cavalo, de
maneira que o cavaleiro cairá para trás...\footnote{Gênesis 49:16, 17.}

Do leão no auge de sua força e de seu caminho surgiu Cristo, o Messias.
Da Áspide, a serpente, a arma de morte dos antigos carrascos e suicidas,
surgiu Anticristo, o Antimessias. E no grande dia da Bênção e da
Maldição, quando Moisés, da tribo de Levi, ensinou ao povo amar a Deus e
temer a maledicência, eles estavam separados. A tribo de Judá no monte
da Bênção, Garizim; a tribo de Dã no monte da Maldição, Ebal.\footnote{O
  monte Garizim (atual Cisjordânia) é separado do monte Ebal (hoje
  chamado \emph{Jebel et-Tor)} por um vale; ficam frente a frente.
  Moisés ordenou que os judeus, após atravessarem o rio Jordão, fossem
  aos montes Ebal e Garizim: ``Após ter atravessado o Jordão erigireis
  estas pedras, conforme hoje vos ordeno, sobre o monte Ebal, e as
  caiarás {[}...{]}'' (Deuteronômio 27: 4). (\emph{Bíblia de Jerusalém,}
  ed. Paulus, 2016, p. 290)}

Passou-se muito tempo desde o sétimo dia da Criação, dia sagrado para o
Nascimento da arte. Mas o pensamento, a mais temível tortura terrena a
que depois foi submetido Shakespeare, um gênio fixado na terra e
rejeitado pelos Céus --- pois quem é forte em desvendar mentes humanas é
fraco em desvendar as ideias divinas ---, o pensamento atormentava o
homem, que por isso fora expulso do Éden e condenado ao trabalho eterno.
Quanto à arte, um Dom sagrado do Senhor, o homem aprendeu a dirigi-la
contra Aquele de quem a recebera. E a primeira maldição, pronunciada no
monte Ebal conforme os Mandamentos de Moisés, foi:

--- Maldito seja aquele que esculpir ou fundir um ídolo, uma indecência
perante o Senhor, obra das mãos de um artista, e que o guardаr em lugar
secreto!\footnote{Deuteronômio 27: 15.}

Dilacerado por seus desejos e por sua vergonha, o homem, concebido da
Árvore do conhecimento do Bem e do Mal, continuava sem conhecer seus
limites e não tinha medo. Criava ídolos já não de forma clandestina, mas
aberta, elevava ao pedestal pecadores como ele, seus semelhantes... Em
vão, como uma voz no deserto, clamava o grande profeta sofredor
Jeremias:

--- A língua deles foi polida pelo artista e eles mesmos cobertos de
ouro, mas são de mentira e não podem falar. {[}...{]} E não podem se
tornar nada além do que os artistas querem que eles sejam. {[}...{]} Ao
ver uma multidão chegando de todos os lados para adorá-los, tu deves
dizer mentalmente: ``É a Ti que se deve adorar, Senhor...''.\footnote{Baruc,
  Carta de Jeremias 7, 45, 5.}

No entanto, lançaram-se contra Jeremias por sua profecia, conforme a
tradição: bateram nele e o colocaram no porão do escriba Jônatas,
transformado em calabouço popular.\footnote{Jeremias 37: 15.} Quando
Jeremias estava a ponto de morrer, o rei apiedou-se dele e o transferiu
para o calabouço real, no pátio da guarda, onde lhe davam pão.

O rei deu uma ordem a Ebed-Melec, o Etíope: ``Leva trinta homens contigo
e resgata o profeta Jeremias, antes que ele morra''. Ebed-Melec pegou
uns trapos velhos e inúteis e uns retalhos velhos e inúteis e, com a
ajuda de cordas, baixou-os no poço de Jeremias. E Ebed-Melec disse a
Jeremias: ``Coloca estes velhos trapos e retalhos que te foram jogados
sob as axilas e a corda por baixo deles''. E assim fez Jeremias. Puxaram
Jeremias pelas cordas e o tiraram do calabouço, e ele ficou no pátio da
guarda.\footnote{Jeremias 38:10--12. Após ter ficado no pátio da guarda,
  por ordem de Zedequias (Matanias), último rei de Judá, com Israel
  cercada pelas tropas de Nabucodonosor (632 a.C--562a.C), rei da
  Babilônia, Jeremias ainda foi jogado num poço.}

Assim padeceu o grande profeta judeu que previu, entre os irmãos da
tribo de Dã, a chegada do Anticristo e criou a legendáriа doutrina da
não resistência ao ímpio pela maldade e pela violência, ensinamento que,
depois de sete séculos, foi apropriado pelo mundo. Pois todo profeta
prediz contra o rei e contra o povo e por eles é perseguido e punido.
Como o Senhor não poderia, com uma única pena, castigar muitos pecadores
sem que alguns justos perecessem se a vida é um Manuscrito de Deus?
Mesmo о criador terreno, a não ser que trabalhasse no estilo realista
socialista, não conseguiria suprimir o mal deixando o bem a salvo;
poderia apenas, como Gógol,\footnote{Nikolai Gógol (1809--1852), em
  crise nervosa, queimou a segunda parte de \emph{Almas mortas} duas
  vezes: em 1845 e, depois de refeita, em 1852, morrendo algumas semanas
  depois. A obra, portanto, permanenceu inacabada.} lançar o manuscrito
inteiro ao fogo, exterminando-o por completo. Pois bem, o Senhor, que
ainda na época de Noé rejeitou o castigo coletivo, criou, contra as
punições dos reis e dos povos, quatro grandes flagelos. Ei-los segundo o
relato do profeta do exílio, Ezequiel.

O primeiro flagelo é a espada, o segundo a fome, o terceiro o animal
feroz, simbolizado pela luxúria, o quarto a doença, a peste...\footnote{Ezequiel
  14:21. ``Ainda que eu envie a Jerusalém meus quatro castigos
  terríveis, a saber, a espada, a fome, os animais ferozes e a peste
  {[}...{]} sobrará nela um resto que conseguirá escapar {[}...{]}''.
  (\emph{Bíblia de Jerusalém,} ed. Paulus, 2016, p. 1498).}

Às vezes os flagelos acontecem juntos, às vezes separados, às vezes um
se fortalece, às vezes outro... Mas nesse ano, quando o que fora predito
pelo profeta sofredor Ezequiel se realizou e na terra apareceu Dã, da
tribo de Dã, a Áspide, o Anticristo, criado para julgar e amaldiçoar, e
não para abençoar, intensificou-se especialmente o segundo flagelo do
Senhor: a fome. Cumpriu-se assim o que fora predito pelo profeta
Ezequiel:

--- Enviarei as flechas ferozes da fome que vos causarão ruína; para
vossa destruição, aumentarei a fome entre vós e arrasarei o trigo que
vos alimenta...\footnote{Ezequiel 5:16.}

Nesse momento, Dã da tribo de Dã, o Anticristo, veio à terra... Isso
aconteceu em 1933,\footnote{1933 foi o ano da implantação da
  coletivização na URSS, quando as propriedades dos camponeses foram
  confiscadas e agrupadas em cooperativas (colcozes). Houve também a
  apreensão de animais e de cereais. Na Ucrânia, instalou-se o que ficou
  conhecido como \emph{golodomor} (em tradução literal, ``morte pela
  fome'').} no outono, perto da cidade de Dimítrov, na região de
Khárkov... Lá se deu o início da primeira parábola. Pois, quando chegam
os flagelos do Senhor, destinos humanos triviais transformam-se em
parábolas proféticas.

\textbf{Parábola do irmão perdido}

--- A colheita passou, o verão terminou, e nós não estamos
salvos\footnote{Jeremias 8:20.} --- assim falava o profeta Jeremias, em
um dia nublado como esse, vislumbrando os campos desertos da terra
prometida que, no crepúsculo de outono, estavam tão desabitados e
desolados como o céu escuro e ameaçador que pairava sobre eles. ---
Olhei para a terra e ela estava devastada e vazia, olhei para os céus e
não havia luz neles.\footnote{Jeremias 4:23.}

Realmente, da janela de uma antigа taberna, agora a casa de chá do
colcoz ``O Lavrador Vermelho'', via-se aquela mesma terra e aquele mesmo
céu que dilaceraram o coração cheio de compaixão do profeta judeu, o
coração de um pessimista-humanista, um cantor aflito de salmos.

É preciso notar , de passagem, que, se durante os mais de dois mil anos
da atual civilização quase não se alterou o caráter do otimista, não lhe
diminuindo o entusiasmo aéreo e superficial nem lhe aumentando a
sabedoria, o caráter do pessimista modificou-se por completo... Perdendo
o lirismo, ele adquiriu uma inspiração filosófica e um desprezo soberbo
pela vida... Porém, de todas as pessoas reunidas na casa de chá ``O
Lavrador Vermelho'' apenas uma tinha consciência disso, e não passava de
um adolescente, quase um garoto, que visivelmente não pertencia ao povo
local, de modo que os outros fregueses, de início, espiavam-no sem
parar. O garoto estava sentado distante dos outros, num dos lugares mais
incômodos da taberna, a uma mesa perto da janela. Vestia-se à moda
citadina e tinha aparência de um judeu, no entanto, como nesse ano de
coletivização e de má colheita vinham muitos delegados de fora e, entre
eles, muitos judeus, o adolescente acabou por se tornar familiar, e logo
passaram a ignorá-lo. Além disso, como ventava muito na janela,
parcialmente fechada com uma folha de compensado, nenhum dos fregueses
assíduos ocupava a mesa perto dela.

Nessa noite, os fregueses da casa de chá eram da parte mais abastada da
vila, para os tempos de então: os tratoristas recordistas,\footnote{No
  original, \emph{udárniki} (de \emph{udar,} ``golpe'',
  ``choque'')\emph{,} trabalhadores soviéticos que alcançavam os
  melhores resultados em suas tarefas, que batiam recordes de produção.
  Para estimular os operários a produzir, havia propagandas, competições
  e até gratificações em dinheiro.} que se reuniram depois de um
encontro regional, por ocasião do qual trouxeram arenques e pãezinhos
doces ao bufê e sementes de girassol e balas à casa de chá. Por isso
desde manhã os pedintes haviam dado de importunar os tratoristas. Se os
pedintes viessem somente do vilarejo, Chagaro-Petróvskoie, não seria um
grande problema. Mas eles vinham de todos os lados: de
Kom-Kuznetsóvskoie, do povoado de Lípki e dos sítios...

--- Senhor! Jesus Cristo... Filho de Deus...

Esse refrão, proferido ora por uma voz sonora de criança, ora por um
murmúrio enrolado de velho, acompanha a tradicional má colheita e a fome
da Rússia desde o início dos tempos. Na época de Boris Godunóv\footnote{Boris
  Godunóv (1551--1605), regente e depois tsar de Rússia.} e em momentos
mais tardios descritos por Lev Tolstói e Korolenko,\footnote{Vladímir
  Korolenko (1853--1921), escritor e jornalista, conhecido como a
  ``consciência da Rússia''.} pais e mães e todos os trabalhadores
devastados e famintos acabavam sustentados por suas crianças e por seus
velhos, vivendo do nome de Jesus. Korolenko chamou uma vez essa
mendicância da Rússia de grandiosa força popular. No entanto, agora
temor e inquietação se reuniram à má colheita e à fome, e essa força, a
última força na luta contra a desgraça, começou a se esgotar. A Igreja,
por seus pecados, virou pó, e fazia tempo que Jeremias, com tristeza no
coração, havia falado sobre o povo sem pastor:

--- São filhos néscios, não têm inteligência; são mestres em fazer o
mal, mas não sabem fazer o bem.\footnote{Jeremias 4:22.}

Antes nem todos davam esmola por ter bom coração, mas por medo do
pecado. Agora todos os pecados divinos foram revogados pelo novo poder
e, nas igrejas, onde, havia pouco, as bocas indiferentes dos sacerdotes
transformaram verdades vivas em frivolidades, sentia-se cheiro de porão
úmido, de álcool vindo da palha apodrecida, e de batata mal guardada.
Jesus Cristo, da tribo de Judá, foi abolido e sua imagem substituída em
todos os lugares: foi retirada das paredes dos espaços públicos, sendo
raspadа e coberta. Mas mendigava-se como antes, em nome de Cristo,
porque, para os mendigos, nada mais foi inventado. Desde tempos
imemoriais, os miseráveis, por estarem no patamar mais baixo da
sociedade, podem usar para seu sustento algo mais elevado, que aja sobre
a insensibilidade de seus irmãos. Quem seria capaz de mendigar em nome
do Conselho dos Comissários do Povo sem ser considerado um provocador,
passível de punição pelo GPU?\footnote{GPU, acrônimo de Direção Política
  de Estado (\emph{Gossudárstvennoie Politítcheskoie Upravlênie}). Um
  dos órgãos de segurança da URSS, antecessor do KGB.} Por isso o nome
de Cristo foi conservado pela mendicância como um anacronismo, à
semelhança de algumas marcas de cigarro pré-revolucionárias.

Assim, quando em um fim de tarde soou o refrão habitual na casa de chá:
``Senhor! Jesus Cristo... Filho de Deus'', poucos levantaram a cabeça
das rodas de conversa, dos copos de chá de cenoura e das balas de goma,
ou do verdadeiro festim que rumorejava em volta da mesa do chefe da
brigada. Lá havia uma garrafa de álcool diluído e, ao lado do arenque,
pratos com fatias de um toucinho bem rosado...

Um pouco antes, haviam dado esmola a dois jovens irmãos, que cantaram e
dançaram a ciganinha, depois a um velho, e ainda a uma mulher segurando
um bebê... A miséria é inoportuna, não tem tato nem consciência, e sua
vontade é arrancar o máximo para si, passando para trás o irmão
miserável...

Visivelmente, a garota que entrou na casa de chá não queria saber se as
pessoas estavam cansadas no fim do dia, se o que estavam comendo e
bebendo havia sido obtido à custa de trabalho árduo, da sorte ou de
privilégios, ou se os miseráveis as aborreciam, como mosquitos sugando o
sangue de um cavalo de carga.

Em geral, há algo de atrevido e exigente na mendicância das crianças, ao
contrário da mendicância dos adultos e, especialmente, dos velhos. Em
primeiro lugar, a criança que mendiga raramente chora em seu esforço de
apiedar e, quando o faz, isso soa claramente falso, evidenciando que foi
ensinada a fazê-lo e que não se trata de choro espontâneo. Em segundo
lugar, agradece a esmola sem prazer ou, com frequência, sequer agradece,
pega-a como se tivesse recebido o que lhe pertence, como se todos em
volta lhe devessem algo ou fossem seus pais. Além disso, na casa de chá
não havia mulheres, e os homens dariam esmola com mais vontade se o
mendigo não lhes causasse pena, mas alegria, como os dois irmãos, ao
dançarem a ciganinha, generosamente fizeram. Mas a garotinha, ao que
parece, não fazia tempo que mendigava: não divertia o público, apenas
andava por entre as mesas, pronunciando o nome de Cristo de maneira
monótona e com uma voz sonora, como uma cantiga infantil. Ela tinha um
rosto típico de camponesa, tranquilo; nos olhos cinzentos havia algo
como que entre a estupidez e a bondade, mas nos lábios roliços um quê de
mulher, o que não poderia ser compreendido por ela mesma, mas por um
olhar alheio e experiente. Rostos assim geralmente ficam cheios e fartos
por pouco, por um pedaço de pão ou uma fatiazinha de toucinho, pois, na
certa, fazia tempo que não viam nem migalhas disso. Essas migalhas
surgiam fartamente sobre a mesa do chefe da brigada, mas dessa mesa rica
a menina fora expulsa, e, nas outras mesas, mais pobres, ninguém lhe
dera atenção, sequer lhe ofereceram uma bala ou um punhado de sementes
de girassol. Como é sabido, havia motivos para isso: a população vivia
em dificuldades, estava cansada de mendigos e não tinha mais medo do
pecado. Contornando todas as mesas, a garota dirigiu-se para a mais
distante, onde estava sentado o jovem citadino, o que parecia judeu. Mas
de repente ela hesitou. É preciso notar que nenhum dos mendigos que
havia passado na casa de chá nessa noite se aproximou dessa mesa, por
temer, certamente, o rapaz da cidade. A garota logo reconheceu nele um
forasteiro, mas não foi por esse motivo que parou, indecisa. Os
fregueses locais não tinham dado nada e ela e, nesse instante, ela
decidira se aproximar do forasteiro, na esperança de ganhar algo. Mas
seu olhar a deteve, de relance, como se o brilho do fogo interestelar
saísse dos olhos escuros do rapaz. Na realidade, ela não sabia que era o
olhar da Áspide, do Anticristo, vaticinado pelo profeta.

Não, não era o Anticristo que desperta ataques histéricos em artistas
cristãos e sobre quem os filósofos propagam discursos, não era o
Anticristo inimigo de Cristo, nem o Anticristo adulado pelos místicos
modernistas, que o chamavam de Criador e o colocavam acima de Deus, mas
era o Anticristo que, ao lado de seu Irmão, cumpre a vontade divina...
Um foi enviado para a Maldição e para o Julgamento, outro para a Bênção
e para o Amor... Um veio do monte da Maldição, Ebal, outro do monte da
Bênção, Garizim... Apenas por um instante, feito um relâmpago, Dã da
tribo de Dã, profetizado por Jeremias, deixou-se levar por seus
sentimentos, mas de repente o ambiente na casa de chá tornou-se pesado,
o burburinho cessou e todos, incluindo o influente chefe da brigada dos
tratoristas, encolheram as cabeças entre os ombros, de maneira
involuntária e inconsciente, o que só acontece quando sentimos passar
algo pesado e lancinante, anunciando a morte...

O motivo do descontrole de Dã era a saudade que sentia de casa, que
estava fresca em sua memória como um túmulo recém-cavado. A noite
chuvosa, tão frequente no outono de Khárkov, aumentou sua tristeza, que
chegou ao auge quando ele se viu entre rostos estranhos e distantes de
seu coração, mas que se alegravam e sorriam uns aos outros, tornando-se
a gota d'água para a saudade ardente do forasteiro... Durante toda a
noite, Dã, o Anticristo, impressionável como todas as crianças judias,
esforçara-se por encontrar para seus olhos, olhos sábios e maldosos de
Áspide, um objeto tranquilizador, se não para alegrar a alma, ao menos
para fazê-la descansar. Quando ele olhava para o interior da casa de
chá, por toda parte via as cabeças sombrias dos apóstatas: nas faces
desanimadas não havia nem sombra de lirismo, nas insolentes nem sombra
de grandeza e nas bondosas nem sombra de sabedoria. Quando ele olhava
para o lado de fora, surgia aquela desesperança russa, provinciana e
outonal, com álamos molhados dos dois lados da rua, latidos de cães, e
duas ou três luzinhas piscando ao longe, e contra isso nada funcionaria,
nem gritos nem lamentos, apenas um bom copo de aguardente de beterraba.
Mas essa receita eslava era inútil para o filho de Jacó, que, ao cair no
esquecimento, sentia algo semelhante à morte. Е а morte, enaltecida por
tantas religiões e filosofias orientais, repugnava seu povo, seja uma
morte física, seja uma morte na contemplação budista... ``Pois na morte
não há lembrança de Ti; e no túmulo quem Te louvará? Volta-te, Senhor,
liberta a minha alma. Salva-me com Tua graça.''\footnote{Salmos 6:5, 4.}
Assim foi dito no salmo seis. A morte priva o homem da possibilidade de
exercer seu dever: amar ao Senhor em plena consciência. No nirvana
budista ele não ama ao Senhor, mas a si mesmo... Nenhum enviado do Céu
que vagueia por caminhos terrenos pode escapar daquilo que é próprio do
homem. Dã se lembrava desse sermão, anotado com as sentenças da Lei de
Moisés de seus \emph{tefilins},\footnote{Duas caixinhas quadradas de
  couro em que se deposita um pergaminho com quatro parágrafos da Torá:
  \emph{Shemá Israel}, \emph{Vehayá Im} \emph{Shamoa}, \emph{Cadêsh Li}
  e \emph{Vehayá Ki Yeviachá}. As caixinhas são fixadas no corpo por
  meio de tiras pretas de couro.} fixados em seus punhos. Mas ali, em
suas primeiras horas em Khárkov, tudo o que é humano ainda era estranho
a Dã, por isso ele voltou o olhar para dentro de si e viu sua cidade,
iluminada pelo sol do mês de Aviv.\footnote{Conforme calendário judaico,
  o mês de Aviv é o mês da primavera, o primeiro do ano.}

A porta das Ovelhas, a porta dos Peixes, a porta da Fonte, perto do
reservatório de Siloé, em frente ao jardim do Rei, junto à
escadaria...\footnote{Escada que desce da Cidade de Davi. Após a grande
  destruição de Jerusalém (586 a.C.), durante o reinado de Zedequias,
  pelas tropas de Nabucodonosor, os habitantes que restaram na cidade
  reconstruíram suas muralhas com suas doze portas (Neemias 3): Porta
  das Ovelhas, dos Peixes, Velha, do Vale, do Esterco (Monturo), da
  Fonte, do Cárcere, das Águas, dos Cavalos, do Oriente, da Atribuição,
  de Efraim.} A torre dos Fornos... A subida do Arsenal, na Esquina,
perto do túmulo de Davi. A cisterna construída perto da casa do
sacerdote Eliasib. A casa do Rei, no alto, perto do pátio do cárcere,
onde padeceu o profeta Jeremias. O muro de Ofel. A porta dos Cavalos, em
frente à morada dos comerciantes. A porta das Águas, na praça dos
comerciantes, onde, do alto de um estrado de madeira, o grande escriba
Esdras lia desde o amanhecer até o meio-dia ao seu povo, que andava
cabisbaixo desde o jugo da Babilônia. Ele lia o livro da Lei de Moisés,
e o povo era todo ouvidos. Esdras, da tribo de Levi, lia ao povo, e os
sacerdotes davam sentido ao que ele lia. Dã sabia que Esdras vivenciara
a maior felicidade para um profeta: uma rara submissão do povo ao bem.
``E Esdras abriu o Livro diante dos olhos do povo inteiro, porque ele se
achava acima de todos. Quando ele o abriu, o povo todo se
levantou.''\footnote{Neemias 8:5.}

Dã lembrava que, ao se iniciar no que é elevado, ao ouvir as palavras da
Lei, o povo inteiro estava em pé, com lágrimas de felicidade nos olhos.
O mesmo povo que, alguns séculos antes, queimara os sermões de Jeremias
e que, alguns séculos depois, rejeitou o seu rei, Jesus, da tribo de
Judá. Dã sabia que seu Irmão, Jesus, desejava o quinhão de sucesso que
tivera Esdras, sonhava subir ao estrado de madeira ao amanhecer, no meio
da praça dos comerciantes, defronte a porta das Águas, e ver nos olhos
do povo lágrimas de felicidade e de arrependimento. Pois ele amava sua
gente com a mesma paixão que o grande escriba Esdras sentia pelo seu
povo de testa de cobre, como marca de teimosia, e veias de ferro na
nuca, como marca de desobediência ao Senhor. Amava tanto seu povo que
chegava a perder a elegância no discurso. Pois Ele, Jesus, Irmão de Dã,
dizia viver para seus filhos maldosos, e não para os cães bondosos dos
outros. Essa doutrina foi exposta de forma clara, embora bastante
sucinta e incompleta, pelo evangelista Mateus, mas os pregadores
cristãos --- a começar por Saulo, da tribo de Benjamim, depois chamado
Paulo,\footnote{Após sua conversão para o cristianismo, Saulo de Tarso
  torna-se também Paulo, um dos apóstolos de Jesus Cristo:
  ``Então~Saulo, que também se chamava Paulo,~repleto do Espírito Santo,
  fixando nele os olhos, disse...'' (Atos dos Apóstolos 13: 9).
  (\emph{Bíblia de Jerusalém}, ed. Paulus, 2016, p. 1925).} o primeiro
apóstata da terra --- acharam um jeito de não observá-la... Seu Irmão Dã
vivia e lutava para seu povo e morreu pelas mãos dos que cooperaram com
os ocupantes romanos e que hoje seriam chamados ``colaboracionistas''.
Assim como os irmãos oprimidos não entenderam o amor de Jesus por eles,
os opressores não entenderam seu ódio. A história do romano Pilatos, ao
tentar ajudar Cristo, repete a história de Nebuzaradã, o capitão da
guarda do rei da Babilônia que salvara Jeremias do calabouço em que
este, por seu derrotismo, fora colocado por seus irmãos. Pois tanto
Jeremias quanto Jesus indicaram o caminho da não resistência ao mal, o
que só parece idealista àqueles que não compreendem as bases do
pensamento judeu: extrema praticidade na existência e máxima metafísica
quanto ao Sublime. O caminho da não resistência ao mal diante de um
ímpio poderoso é possível, no entanto, com uma ressalva, apontada por
Jeremias. Em princípio, ela assim se resume: que o ímpio lhe tire tudo,
mas você também deve tomar, como butim, sua própria alma do ímpio... O
importante é conservar a sua alma como se fosse uma presa, pois o ímpio,
cedo ou tarde, perderá a alma dele, mas o amor que você retribuirá à
maldade do ímpio não terá proveito para ele. Você mesmo é que tirará
proveito. Aqui está ele, o pensamento prático judeu sobre a não
resistência ao mal pela violência... Mas, diante da face atual do ímpio,
criada pelo progresso da civilização, torna-se cada vez menos viável
cumprir a ressalva do profeta Jeremias, com a qual contava o Irmão de
Dã, Jesus, da tribo de Judá, do monte da Bênção, Garizim...

Como Dã foi longe em seus pensamentos e visões, longe da noite chuvosa
de outono da vila de Chagoro-Petróvskoie, do distrito de Dimítrov, na
região de Khárkov, longe do momento em que a garota mendiga tentava ir
até ele na esperança de conseguir uma esmola. Nos primeiros instantes,
quando ele lhe dirigiu um olhar do Além ainda não de todo arrefecido,
ela se assustou de tal maneira que sentiu vontade de gritar, mas já não
tinha forças. Enquanto suas forças voltavam, Dã lhe estendeu um naco de
pão que tirou de sua bolsa de pastor de couro rústico. Era um pedaço de
pão impuro do exílio, legado pelo Senhor através do profeta do exílio
Ezequiel, feito com uma mistura de trigo e de cevada, com favas e com
lentilhas. Pelos pecados dos homens, o Senhor ordenou que se assasse o
pão impuro em excremento humano, mas o profeta Ezequiel recebeu o
direito de assá-lo em esterco de vaca...

Tanto a esmola quanto quem a oferecia assustavam a garota, mas ela
estava faminta e pegou o pão impuro do forasteiro. Um burburinho abafado
invadiu a casa de chá. A companhia ficou ofendida. Certa expressão
antiga, quase esquecida, surgiu inicialmente nos rostos mais bondosos,
passou para os mais desanimados e, numa indignação peculiar, tocou nos
rostos insolentes. Elаs, pessoas de sua terra, de seu sangue,
recusaram-se a dar algo à garota mendiga, no entanto, um forasteiro, um
judeu da cidade, estendeu a ela um naco de pão. O camponês que estava
mais próximo, com um copo de chá de cenoura na mão --- um homem magro,
ainda jovem, mas sem os dentes da frente, de modo que precisava molhar
as cascas de pão na água quente, para, em vez de mastigá-las, chupá-las,
o que era mais apropriado e mais econômico ---, esse camponês desdentado
foi o primeiro a estender algo a garota, uma casca de pão amolecida na
água; depois outro sujeito, mais afastado, lhe deu duas balas de goma;
alguém despejou um punhado de sementes; e, finalmente, da mesa mais
rica, o chefe da brigada dos tratoristas, a ``excelência'' em pessoa,
chamou a garota com um aceno.

---Vá, bobinha --- cochichou o camponês desdentado ---, não fique
acanhada... Petró Semiónovitch está generoso. Peça um pouco de
toucinho...

De fato, mal a garota se aproximou da mesa, o chefe da brigada, Petró
Semiónovitch, solenemente e sob os olhos de todos, entregou-lhe um
pedaço de toucinho sobre um papel de jornal, como se entregasse uma
recompensa a algum trabalhador recordista --- um corte de dois metros de
tecido ou umas botas de cano alto...

--- É assim --- disse Petró Semiónovitch ---, mas você pede ajuda a
forasteiros... O forasteiro pode estar do lado inimigo, pode ser um
\emph{kulák}\footnote{Camponês abastado.} ou um comparsa de um
\emph{kulák}... Isso ainda precisamos decidir...

A essa altura, Petró Semiónovitch estava bêbado e isso o instigava a
fazer diversos pronunciamentos políticos. Sem coragem de responder e
assustada pela segunda vez num curto intervalo de tempo, mesmo que por
motivos diferentes, a garota pegou o toucinho em silêncio e começou a
enrolá-lo no jornal.

--- Por que não come, criança? --- perguntou Petró Semiónovitch, que de
repente mudou de humor e derramou lágrimas. --- Para quem está guardando
esse naquinho? Por acaso você tem filhos?

--- Meu irmão Vássia\footnote{Apelido de Vassíli.} está me esperando nos
degraus da entrada --- timidamente disse a garota.

--- Existe um irmão Vássia --- disse Petró Semiónovitch ---, isso é bom.
E como você se chama?

--- Maria --- disse a garota.

--- Maria, por que seu irmão Vássia manda você pedir esmola, enquanto
ele mesmo se refresca na entrada?

--- Ele ainda é pequeno... Tem medo.

--- Por que ele teria medo? --- ofendeu-se Petró Semiónovitch. --- As
pessoas daqui não são animais... É o nosso povo... A vila... Outra coisa
são as pessoas de fora... É delas que deve ter medo, caso não tenham um
mandato... Você é daqui, pelo visto, se o seu pai a deixa pedir esmola
numa hora dessas da noite...

--- Meu pai morreu no ano passado --- disse Maria.

--- E como ele se chamava? --- perguntou Petró Semiónovitch.

--- Não sei.

--- Que quer dizer com isso? --- surpreendeu-se Petró Semiónovitch. ---
E a sua mãe, como se chama?

--- Não sei --- respondeu Maria --- Mãe, só mãe.

--- Ora essa --- disse Petró Semiónovitch, limpando com o grande dedo
indicador os cantos da boca, à moda ucraniana ---, filhinha, alguém lhe
ensinou algo errado...

--- Deixe-a ir, Petró... --- disse o moreninho sentado à sua direita.

--- Não, espere, Stepan --- disse Petró Semiónovitch ---, algo não está
cheirando bem... E o seu sobrenome, como seria?

--- Não sei --- disse Maria, quase chorando.

--- Corra --- cochichou, quase inaudível, o camponês desdentado.

Mas Petró Semiónovitch, que se agitou e no ato voltou ao normal,
percebeu o cochichador.

--- Eu vou cochichar também --- disse, pegando a garota pela mão ---,
está querendo morar na Sibéria? Sei que nos sítios se escondem muitas
famílias de \emph{kulakes,} com seus comparsas, para não serem mandadas
à Sibéria... E você é do sítio --- disse aproximando de Maria seu rosto
medonho, com uma cicatriz em forma de espada adquirida na guerra civil.

--- Do sítio --- quase morta de susto, respondeu Maria ---, de Lugovoi.

--- Agora você está falando coisa com coisa --- disse Petró
Semiónovitch, acalmando-se um pouco. --- Continue falando como se deve.

--- Titio --- disse Maria ---, eu não sei o meu sobrenome, também não
sei o nome do meu pai e da minha mãe, porque nossos pais nunca se
ocuparam com a gente, eles não tinham tempo, estavam sempre trabalhando
no colcoz, e, agora que meu pai morreu, minha mãe está fazendo de tudo,
em casa e na horta, é preciso arrumar, arar, semear e cuidar dos outros
trabalhos, mas ela não nos ensinou nada. Eu tenho um irmão mais velho,
Nikolai, e uma irmã mais velha, Chura, \footnote{Chura é apelido de
  Aleksandra.} e os mais novos, Vássia e Jórik, mas Jórik ainda está no
berço.

--- Bravo --- disse Petró Semiónovitch ---, agora você não está se
fazendo de boba. E como chamam vocês? Eu, por exemplo, era chamado na
infância por toda a vizinhança de ``filho de Semión''... Olha o filho do
Semión passando... E vocês?

--- Nós somos os ``filhos da cidadã'' --- disse Maria.

--- Mas como devo entender esse ``cidadã''? Em Dimítrov ou em Khárkov é
que são ``cidadãos''. Aqui são camponeses... Por que vocês são chamados
de ``filhos da cidadã''? A mãe de vocês por acaso é da cidade?

--- Não --- disse Maria, baixando a cabeça.

--- Você está mentindo --- disse Petró Semiónovitch, irritado ---, está
mentindo e não está olhando nos meus olhos --- e sua fala subitamente
perdeu o sotaque e os termos ucranianos, ficou seca, russa, protocolar.
--- Se vocês não são da cidade, por que são chamados de ``filhos da
cidadã''?

--- Chamam porque chamam --- de novo o moreninho, sentado do lado
direito do chefe da brigada, tentou se colocar. --- Será que não conhece
os apelidos dos aldeões, Petró?

--- Fique quieto, defensor... Está querendo bancar o advogado? Mas você
não é judeu para ser advogado... Então, continue --- dirigiu-se à Maria.

--- Fale, menina, não tenha medo --- disse-lhe o moreninho.

--- No ano passado, nosso pai morreu, foi um ano de fome.

--- Isso eu já ouvi --- disse Petró Semiónovitch. --- Continue...
