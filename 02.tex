II

А vida repete а vida, о destino imita o destino, como о dia repete o dia
e a noite imita а noite... O que é a existência senão repetição e
imitação? O dia substitui a noite, a noite substitui o dia. A primavera
imita a primavera, o outono imita o outono, e a base de toda imitação,
ou seja, a existência, é uma ordem racional. Esse é o classicismo de
Deus. O destino de Isaac imita o destino de Abraão, o destino de Jacó o
de Isaac. Tudo o que é mais sublime, que vive através da razão divina,
se repete e vive da imitação, assim como tudo o que é mais terreno е
vive através do instinto divino. O classicismo é a imitação do Senhor
pela razão ou do mundo do Senhor pelo instinto. O profeta imita o
Senhor; o povo, o mundo do Senhor. Mas, quanto mais se desenvolver a
sociedade, mais inovações haverá e menos classicismo. No começo, aparece
o dogma. O classicismo morre, desfigurado por adoradores decadentes. O
inovador, que não possui forças para lutar contra o classicismo vivo,
lança-se sobre o cadáver e celebra a vitória... E então, conforme a
profecia do profeta Jeremias, que destruiu o jugo de madeira, sobrevém
um jugo de ferro. Assim surge o profeta-inovador, que deseja viver pelo
instinto, e o povo-inovador, que deseja viver pela razão. O
profeta-inovador, desejoso do instinto, cria o materialismo idealista, a
utopia social eclética e o idealismo místico-materialista; o
povo-inovador, desejoso da razão, cria o homem-deus e a idolatria; nos
seus melhores momentos, o homem torna-se ateu e, nos seus piores,
idólatra... Cada um almeja criar o que lhe é próprio e dizer algo de
único. No entanto, os patriarcas não buscavam o que era conveniente a
eles, mas a Deus, e os profetas não falavam de si mesmos, mas de Deus...
O pequeno povo pastoril era tão maldoso quanto os outros povos, grandes
e pequenos, próximos ou afastados de si no espaço e no tempo. Ele só se
distinguia dos outros por seus patriarcas e profetas, e por meio deles
foi escohido pelo Senhor. E assim disse o profeta Jeremias:

--- Teus caminhos e tuas ações é que te causaram isso. Foi tua maldade
que fez a amargura atingir teu coração.\footnote{Jeremias 4:18.}

E disse o profeta Isaías:

--- Nossas iniquidades nos arrastam como vento {[}...{]}\footnote{Isaías
  64:6.}

No entanto, os homens depositam muitas esperanças na inovação e contam
com a variedade de destinos. Então a imitação, da qual desviaram a
felicidade que dependia deles, caiu sobre eles como uma desgraça que
deles não dependia.

\textbf{\\
Parábola dos suplícios dos ímpios}

Na cidade de Rjév, na região de Kalínin, em 1940, vivia uma menina de
nome Ánnuchka.\footnote{Apelido de Anna.} Sua mãe também era chamada
Ánnuchka. E a menina sabia seu sobrenome --- Emeliánova. Ela tinha um
irmão chamado Ivan, que todos, sem saber por quê, conheciam por
Mítia,\footnote{Mítia é diminutivo de Dmítri, enquanto Vánia de Ivan.} e
ainda um irmãozinho, Vova,\footnote{Apelido de Vladímir.} de dois anos.
Ánnuchka, porém, não tinha pai, que fora morto na Guerra da
Finlândia,\footnote{A Guerra Civil Finlandesa, de janeiro a maio de
  1918, foi um conflito entre social-democratas, apoiados pela Rússsia,
  e conservadores, apoiados pela Alemanha e Suécia. Com o fim do
  conflito, a Rússia, enfraquecida com a Primeira Guerra Mundial e em
  meio à implantação de um novo regime, perdeu o domínio sobre a
  Finlândia.} já que Rjév é uma cidade do Norte, e muitos homens do
norte foram enviados à guerra. Ánnuchka havia nascido nessa região, mas
não em Rjév, e sim no distrito de Zubtsóv, na aldeia de Nefiédovo.
Ánnuchka lembrava que, quando morava em Nefiédovo, de manhãzinha, no
verão, com o sol acalentando a aldeia, ela gostava de sair da cama só de
camisola, sonolenta, e acomodar-se na terra, ao lado da isbá, para
terminar de embalar seu sono. No entanto, agora o endereço de Ánnuchka
era: cidade de Rjév, 3º setor, barracão nº 3, quarto nº 9. Com tal
endereço, não era possível se acomodar, ao lado da isbá, sob o sol da
manhã. O barracão em nada parecia uma isbá. Havia um cheiro ruim, que
não era de lenha forte, mas de argamassa e de tábuas podres. A terra em
frente não era macia, mas seca e áspera, e as poças, que demoravam a
secar, ensopavam pedaços de jornal, tijolos quebrados e trapos
ensebados. Do aeródromo, onde a mãe de Ánnuchka tinha um emprego na
construção, vinham sem parar ruídos e estrondos, como se muitos tratores
se deslocassem de uma vez. Mas fazia tempo que Ánnuchka sabia que era o
barulho dos aviões, e só às vezes imaginava, como nos primeiros dias,
que tratores rumorejavam. Sua mãe levava Ivan-Mítia ao jardim de
infância e deixava o caçula, Vova, aos cuidados de Ánnuchka, que por
isso não gostava dele.

A isbá da aldeia de Nefiédovo era melhor que o barracão da cidade de
Rjév, mas a cidade era mais alegre que a aldeia. No verão, um circo se
apresentava na praça da feira, e mesmo sem ingresso era possível se
divertir, e no inverno Ánnuchka costumava vestir botas vermelhas de
feltro, compradas numa loja da cidade. No entanto, o episódio que marcou
o destino de Ánnuchka não se deu não no inverno, quando ela vestia suas
botas vermelhas favoritas, mas no verão, quando o circo se achava na
praça da feira. Fazia tanto calor que mesmo as poças em frente ao seu
barracão, que nunca secavam, desapareceram, e delas só restou um pouco
de lama espalhada. Embora sua mãe tivesse tirado os trapos das inúmeras
frestas do barracão, usados para tapar o vento gelado do inverno,
continuava muito abafado, e Vova chorava sem parar, mordia Ánnuchka e se
recusava a comer semolina, cuspindo o mingau nos pés. Ánnuchka, que
sabia que o circo já estava instalado na praça, onde se podia ouvir
música, ficou zangada com Vova, que a obrigava a ficar plantada num
barracão abafado; assim, quando o irmão deu-lhe uma mordida
particularmenre forte, ela o beliscou. Ele começou a chorar com mais
força ainda, de modo que Chura, do quarto nº 12, foi dar uma espiada no
quarto nº 9. Ela trouxe uma bacia de água quente, lavou o rosto de Vova,
as mãozinhas e os pezinhos, lambuzados de mingau, e ele parou de chorar
e dormiu. Depois Chura saiu, e Ánnuchka de novo se viu sozinha com Vova,
adormecido, no barracão abafado. Então ela resolveu, enquanto Vova
dormia, dar um pulo na praça da feira, onde estava o circo. Lá tudo era
muito bonito e muito alegre. Ánnuchka andava por todo lado, olhava para
tudo e ria, mesmo que ninguém a fizesse rir, até que de repente uma
mulher de chapéu-panamá branco lhe disse:

--- Menina, por que está rindo? Rir sem motivo é sinal de tolice.

Ánnuchka ria porque ficar ali, na frente do circo, no meio de uma
multidão elegante, que ouvia música, era melhor do que ficar no barracão
abafado com Vova; porém, ela não se pôs a dar explicações, simplesmente
se afastou e continuou a rir. De repente escureceu e começou a
chuviscar. Todos se alvoroçaram, dizendo: ``Um temporal, um temporal!
Olhem, que nuvem preta!''. Realmente, dali, da praça de feira, via-se
uma nuvem preta avançando: árvores imóveis começaram a tremer, a cúpula
de lona do circo itinerante estalou inquietantemente e a música cessou.
Então Ánnuchka correu para casa. Mal percorreu algumas ruas, caiu uma
chuva forte, raios cintilaram de cima a baixo, um trovão ressoou no céu,
depois um segundo, depois um terceiro, mas Ánnuchka não conseguia se
acostumar com isso e se assustava a cada estrondo. Num minuto, Ánnuchka
ficou tão molhada que seu vestido grudou ao corpo, a respiração ficou
ofegante da corrida, mas ela não podia se abrigar na entrada de alguma
casa nem sob a sacada onde se aglomeravam pessoas encharcadas e alegres,
já que precisava voltar depressa ao seu barracão, na periferia da
cidade, onde Vova estava sozinho --- se até batidas de portas o
assustavam (por isso sua mãe proibia Ánnuchka e Mítia de baterem as
portas), agora ele deveria estar apavorado.

Perto dos barracões, onde havia pouco as poças secaram pelo calor, a
água corria como numa torrente --- alcançava o tornozelo de Ánnuchka e,
em alguns lugares, chegava até seus joelhos. Umedecida, a porta empenou
e, quando Ánnunchka a abriu, a muito custo, com a chave que retirou de
debaixo de uma tábua do piso, água jorrou do quarto, indo em direção ao
corredor... Ela se assustou e gritou:

--- Vova!

Mas Vova não estava em sua cama. Ánnuchka correu pelo quarto,
chapinhando na água, chamando por Vova. Depois notou que a janela estava
aberta e, achando que o irmão tivesse saído por ali, gritou por ela:

--- Vova, Vova! --- pois tinha medo de ser castigada por sua mãe por ter
deixado Vova sair pela janela.

Depois ela olhou sob a cama e encontrou Vova lá, deitado com o rosto
voltado para baixo. Ánnuchka entendeu que Vova tinha caído da cama e
rolado pelo chão. Vova estava molhado e frio, com o rostinho tão
enrugado que era como se estivesse chorando, mas sem emitir som e, por
mais que Ánnuchka o chamasse, ele continuava sem se mexer. Então
Ánnuchka entendeu que Vova estava morto. Nesse momento, ela ficou
completamente apavorada. Não lamentava Vova, a quem não amava, mas temia
que sua mãe, ao voltar do trabalho, lhe aplicasse um castigo severo.
Movida por tais pensamentos, Ánnuchka caiu em desespero e desejou estar
morta como Vova, para que sua mãe não pudesse castigá-la ou gritar com
ela. Mas Ánnuchka não sabia como morrer, por isso ficou simplesmente
sentada no chão, agarrando a cabeça e chorando baixinho, para que nenhum
vizinho entrasse no quarto e descobrisse que Vova havia morrido por
causa dela.

Quando, à noite, sua mãe voltou do trabalho trazendo Mítia do jardim de
infância, a primeira coisa que notou foi Ánnuchka sentada no chão com os
olhos fechados e os ouvidos tapados por suas mãozinhas, para não ver nem
ouvir nada.

--- O que você tem, filhinha? --- gritou a mãe, assustada e, nesse
momento, viu Vova morto sobre a cama.

Ela gritou como nunca havia gritado, е seu rosto e sua voz ficaram
irreconhecíveis. Os vizinhos acorreram no mesmo instante; um foi à casa
do administrador chamar uma ambulância, outro tentou fazer respiração
artificial em Vova, segurando suas mãozinhas e seus pezinhos, mas alguém
disse:

--- É inútil, ele já está morto.

Mítia, o irmão de Ánnuchka, olhava para tudo de soslaio e não chorava,
pois era um menino calmo e ponderado. Mas sua mãe, que Ánnuchka temia
por sua raiva habitual, agora estava fora de si e tornou-se para a filha
mais assustadora do que qualquer animal selvagem da floresta. Ela se
lançou contra Ánnuchka e gritou terrivelmente, batendo nela não com a
palma da mão, mas com o punho, como nunca havia batido... Quando uma mãe
ou um pai batem num filho, mesmo com raiva, sempre pensam na dor que
causarão à criança, e o golpe, por mais doloroso, não é aplicado com
indiferença. Mas naquele instante a mãe batia em Ánnuchka com
indiferença, como se golpeasse um inimigo, e Ánnuchka sentiu seus olhos
escurecerem. Alguém bate assim num filho somente quando está por demais
magoado ou por maldade, pois a mágoa e a maldade, em essência, são
plantas diferentes de uma mesma raiz... Ela queria bater mais, mas foi
contida.

Chura levou Ánnuchka e Mítia para seu quarto, deu um caramelo a cada um
e colocou uma compressa na testa de Ánnuchka. Ela passou a noite com tia
Chura. No dia seguinte, enterraram Vova. Trouxeram um pequeno caixão de
criança e colocaram duas moedas de cinco copeques sobre seus
olhos.\footnote{Rito, seguido por algumas culturas e originado da
  mitologia grega, no qual se deve colocar uma moeda embaixo da língua
  do morto ou duas nos olhos, para que a alma possa pagar seu tributo ao
  barqueiro Caronte pela travessia ao mundo dos mortos.} Ánnuchka queria
ir ao cemitério, mas Chura não deixou, e a menina viu pela janela sua
mãe, que, coberta por um lenço preto e sem mais chorar, andava atrás do
caixãozinho de Vova аcompanhada por Mítia.

Ánnuchka passou também o dia seguinte com Chura e lá almoçou --- uma
sopa deliciosa de cogumelos e batatas assadas com leite cozido. À noite
sua mãe foi vê-la, e agora já não chorava com raiva, mas com doçura, e
voltou a se parecer com quem ela era. Ela beijou Ánnuchka intensamente e
levou-a dali, acariciando-a e apertando-a contra o peito com tanta força
que o ponderado Mítia disse:

--- Mamãe, cuidado, assim vai sufocar Anka.

Desde então, a mãe mudou seu comportamento com Ánnuchka: raramente a
xingava e nunca mais bateu nela. E Ánnuchka, do fundo do coração, se
alegrava por Vova ter morrido. Agora, nas horas vagas, passeava pelas
ruas da cidade ou ia até o aeródromo onde sua mãe trabalhava e por isso
a deixavam entrar. Em geral, ela preferia se relacionar com adultos,
pois não gostava de crianças. Ánnuchka sentia prazer quando se apiedavam
dela, e as crianças, criaturas impiedosas, nunca têm pena de ninguém. Os
garotos da vizinhança a provocavam, assim como os da escola, por isso
sua mãe tentou transferi-la para outra, mas lá também implicaram com
ela; tentou enviá-la no verão ao acampamento dos
\emph{pioneiros},\footnote{Fundada sob princípios comunistas, a
  organização de \emph{pioneiros} reunia crianças da URSS desde a escola
  primária. Em algumas de suas atividades, os \emph{pioneiros} lembravam
  os escoteiros e o símbolo principal de sua vestimenta era um lenço
  vermelho usado como gravata. Depois disso, muitas crianças tomavam
  parte no Komsomol, a organização da juventude comunista.} não o da
empresa onde trabalhava, mas o da cooperativa\footnote{No orginal,
  \emph{kombinat,} grupo de empresas soviéticas trabalhando num mesmo
  setor em busca de melhores resultados.} das empresas de leite, mas
Ánnuchka acabou fugindo de lá, porque não conseguia acordar quando
sentia vontade de fazer xixi. Com seu irmão Mítia ela vivia
amigavelmente, ele a consolava quando ela sofria nas mãos dos outros
garotos, no entanto jamais a defendia. Apenas se aproximava com calma e
dizia:

--- Vamos para casa, Ánnuchka --- e estendia-lhe a mão.

Аssim irmão e irmã iam para casa de mãos dadas.

Desde setembro, Mítia também passou a ir à escola, mas lá não o
provocavam, apesar de todos saberem que ele era o irmão da
Anka-mijona... Mas, em vez de Ivan, como estava escrito na lista da
classe, conforme seus documentos, as crianças só o chamavam de Mítia, a
ponto de a professora também chamá-lo algumas vezes por esse apelido...

Em todo caso, mesmo que Ánnuchka não tivesse se habituado às
provocações, conformou-se, pois era possível conviver com elas,
principalmente por Rjév ser uma cidade grande, com espaço suficiente
para se manter longe dos provocadores maldosos. Além disso, aos poucos
diminuíram as implicâncias, pois em sua classe apareceu um menino que
ceceava, e todos passaram a provocá-lo. Até Ánnuchka o provocava. Assim,
depois da morte de Vova, a vida de Ánnuchka não ia nada mal, até que uma
nova desgraça sucedeu. Essa desgraça não aconteceu no verão, quando o
circo se achava na praça da feira, mas no inverno, quando Ánnuchka
vestia suas botas vermelhas favoritas.

Certo dia, quando Ánnuchka esquentava sozinha almôndegas num fogareiro a
querosene --- ela estudava no turno da tarde, Mítia estava na escola e
sua mãe no trabalho ---, a porta se abriu, sem ninguém bater, e entraram
dois desconhecidos.

--- Menina, você está sozinha? --- perguntou um homem que calçava botas
brancas de feltro revestidas de couro.

--- Sim --- disse Ánnuchka.

--- Então sente-se nessa cadeira e fique quietinha --- disse o outro
homem, que vestia uma peliça curta de cor preta.

Ánnuchka sentou-se e os homens começaram rapidamente a tirar todas as
coisas do guarda-roupa e a colocá-las em malas. Abriram todas as gavetas
e olharam o criado-mudo, andando na frente de Ánnuchka como se ela não
existisse. Depois foram embora, levando, além das malas, uma pequena
máquina de costura.

A mãe de Ánnuchka, quando conseguia uma carona na obra, vinha almoçar em
casa, como nesse dia. Ela chegou e viu: tudo escancarado, o guarda-roupa
vazio, o lugar da máquina de costura sem ela, e Ánnuchka sentada na
cadeira. A mãe de novo se pôs a gritar e de novo os vizinhos acorreram
de imediato, como quando Vova morrera.

--- Fomos roubados! --- gritava a mãe. --- Levaram tudo! Até o terno de
Kólia que eu guardei de recordação... Um terno de lã grossa que ele só
usou duas vezes --- e a mãe se desfez em lágrimas.

O vizinho do quarto nº 11 disse:

--- Eu ouvi gente passando, mas escutei Ánnuchka mexendo com o fogareiro
e achei que fossem parentes.

--- E por que você não gritou? --- Chura perguntou à Ánnuchka.

--- Fiquei com medo de me baterem --- disse Ánnuchka.

--- E por que você não gritou quando eles saíram com as malas? ---
perguntou o vizinho do quarto nº 11.

--- Pensei que estivessem escondidos atrás da porta e tive medo de me
baterem se eu gritasse... --- disse Ánnuchka.

Então, pela primeira vez em longo tempo, sua mãe lhe bateu, mas não com
o punho, como quando Vova morrera, mas com a palma da mão e com piedade;
por mais que usasse de força, era de um jeito maternal. Nesse ínterim,
apareceu o administrador e disse:

--- Tapas não vão ajudar nesse caso. Menina, você seria capaz de
reconhecer os ladrões?

--- Sim --- respondeu Ánnuchka ---, um vestia uma peliça curta de cor
preta e o outro botas brancas.

--- Vamos enfileirar todos os homens dos barracões... --- disse o
administrador. --- Talvez fossem alguns dos novos contratados... Entre
eles há um número colossal de \emph{deskulakizados}...\footnote{A
  \emph{deskulakização} (\emph{raskulátchivanie}), por meio da
  coletivização, foi uma política soviética de repressão aos
  \emph{kulakes}, camponeses enriquecidos, ou aos que eram assim
  considerados.}

Enfileiraram todos os homens dos barracões num terreno baldio coberto de
neve, Ánnuchka saiu, olhou ao redor e ficou aterrorizada. Ao lado dela
estavam sua mãe, o administrador e dois policiais. Andaram ao longo da
fila --- todos olhavam para Ánnuchka com pavor, da mesma forma que ela
olhava para eles. Passaram por todos uma vez e Ánnuchka não reconheceu
ninguém. Havia rostos conhecidos e rostos desconhecidos, mas nem sinal
dos ladrões.

--- Não faz mal --- disse o administrador ---, não é possível distinguir
de primeira.

Passaram pela fila mais uma vez. De novo olhavam para Ánnuchka com pavor
e ela estava mais apavorada que eles. De tão assustada ela não conseguia
distinguir ninguém, todos os rostos se pareciam e mesmo os rostos
conhecidos lhe pareciam desconhecidos.

--- Não faz mal --- disse o administrador ---, vamos de novo... Talvez
ele esteja assustando você com o olhar.

Com efeito, Ánnuchka tremia como se estivesse febril, sem saber para
quem apontar... Já havia molhado a calcinha de susto e o frio era de
congelar, mas ela continuava sem saber quem indicar. Enfim apontou para
o terceiro homem da fila, do lado esquerdo.

--- É este --- disse ela.

--- Menina --- gritou o homem ---, eu sou de Zubtsóv... Meu sobrenome é
Potchiválin... Tenho sete filhos...

--- Se é de Zubtsóv, isso lhe dá o direito de roubar a viúva de um herói
da Guerra da Finlândia? --- disse o administrador e deu-lhe um soco nos
dentes.

Imediatamente sangue começou a jorrar e, ao ver isso, Ánnuchka caiu no
choro.

--- Está bem --- disse o administrador ---, podem levar a menina. Ele
mesmo entregará seu cúmplice.

A mãe levou Ánnuchka para o barracão e já não а repreendia nem lhe
batia, mas tornou-se carinhosa, como depois do enterro de Vova. Após
alguns dias, o administrador passou no quarto nº 9 e disse:

--- Anna Alekséievna, ainda não conseguimos reaver as suas coisas, mas
tenho algo a dizer que irá alegrá-la... Ainda será esclarecido se esse
patife roubou ou não, mas foi inegavelmente provado que, em 1934, em
Zubtsóv, ele incendiou o trigo do colcoz. Considerando a sua ajuda para
desmascará-lo e também o fato de ser viúva de um herói da Guerra da
Finlândia, com dois filhos, de ter sofrido recentemente a perda de seu
filho mais novo e do prejuízo que teve com o roubo, decidimos
conceder-lhe uma habitação e um emprego nas redondezas. Dirija-se ao
armazém nº 40 para preencher as formalidades.

O armazém nº 40 ficava na cidade e seria possível trabalhar num local
quente e coberto. A mãe se alegrou.

--- Agradeço ao camarada Stálin por sua preocupação... --- disse ela.
--- Porque eu estou sozinha com as crianças... O menorzinho morreu... E
agora nos roubaram tudo...

No começo, sua alegria transformou-se em lágrimas, depois ela se pôs a
rir entre as lágrimas, pois nunca imaginou que chegaria a ver o dia de
sua saída do barracão.

Eles lhe deram uma moradia na periferia de Rjév, na outra extremidade da
cidade --- não perto do aeródromo, mas do cemitério. Antigamente, o
prédio onde passaram a morar era a igreja do cemitério, mas, pouco antes
de terem se mudado para a cidade, a igreja fora fechada e seu endereço
tornou-se: Rua do Trabalho, nº 61. As reformas ali foram feitas às
pressas, para que entregassem mais rapidamente os apartamentos à
população necessitada --- nas paredes mal caiadas ainda eram visíveis as
faces de santos, e onde ficava o criado-mudo, sobre o qual se pendurava
um alto-falante, havia um afresco mal feito da crucificação de Cristo
que a mãe cobriu com jornais e, sobre eles, pendurou o retrato de
Stálin. Mas as paredes grossas da igreja eram úmidas, e os jornais
descolaram e enrugaram, de modo que o busto do Cristo ortodoxo ficou ao
lado do busto de Stálin, e era possível imaginar que fossem companheiros
de luta.

A igreja fora fechada, assim como o sacerdote preso, porque constataram
que, no primeiro domingo de Quaresma, sob o pretexto de uma festa
ortodoxa, adoradores de ícones organizaram ali um comício
antissoviético. Supostamente havia aparecido um ícone --- da Nossa
Senhora de Rjév --- que não era obra de mãos humanas e, conforme
informações do departamento de saúde da cidade, não apenas tocaram nele,
mas também rasparam sua tinta sobre a comida e a bebida, o que
favorecera o aumento de casos de infecção. Sem demora, o escritório
responsável pelas reformas, tendo dificuldades em aprontar os
apartamentos, fez um planejamento que se verificou bastante modesto ---
retirada da iconóstase, demolição do altar e outras obras
insignificantes... Alguns meses mais tarde, os primeiros
\emph{stakhanovistas}\footnote{O termo \emph{stakhanovista}
  (\emph{stakhónovets} em russo) relaciona-se com o mineiro Aleksei
  Stakhánov (1905\emph{--}1977), que bateu um recorde de extração de
  carvão em 1935 na cidade de Írmino (Ucrânia), e designava, nos anos
  1930 e 1940, os trabalhadores soviéticos que sobressaíam no trabalho,
  assim como o termo \emph{udárnik} (aqui traduzido como
  ``recordista'').} se mudaram para a antiga igreja, agora o prédio nº
61 da Rua do Trabalho, perto do cemitério. As paredes eram úmidas,
cheiravam a mofo no verão e ficavam cobertas de geada no inverno, além
de transpirarem devido às chaminés improvisadas, que soltavam muita
fumaça; mesmo assim essas paredes protegiam mais as pessoas do frio
intenso e do vento do que as tábuas rebocadas dos barracões.

Ánnuchka, a mãe de Ánnuchka, gostava dali, e a menina também gostava,
mas Ivan-Mítia não expressou sua opinião sobre a antiga igreja em
comparação ao barracão, porque era de caráter reservado.

Não foi possível reaver as coisas roubadas, no entanto eles conseguiram
se arranjar e também adquiriram algumas coisas novas, pois a mãe agora
era uma funcionária com responsabilidades do armazém nº 40, onde ganhava
mais do que na construção do aeródromo.

Mal tinham se abastecido de alguns pertences, incluindo um sobretudo de
inverno forrado de algodão para Ánnuchka, apareceu outro homem em casa,
declarando que queria vistoriar os afrescos e o lugar onde antigamente
ficavam o altar e a iconóstase. De novo, Ánnuchka estava sozinha e, de
novo, teve medo de apanhar, então se sentou, silenciosa e tristemente,
numa cadeira, apesar de não ter sido forçada a isso pelo homem.

Esse homem era Dã, a Áspide, o Anticristo. Os anos na Terra o haviam
amadurecido, e ele aprendera a conversar com as pessoas sem sentir
aversão, o que não é acessível aos anjos celestes, mas apenas aos
profetas, mesmo assim nem a todos nem para sempre. Dã sabia que amar ao
homem significa superar sua aversão por ele; no entanto, mesmo os
grandes profetas, em momentos de fraqueza, eram incapazes de dissimular
essa aversão. Assim aconteceu a Moisés no intervalo entre as primeiras e
as segundas Tábuas da Lei, no momento em que ele quebrou as primeiras,
aflito com a necessidade de entregar seu elevado coração a criaturas tão
baixas, que preferiam os caldeirões de carne da escravidão egípcia ao
maná celestial do Sinai livre. O mesmo aconteceu ao Irmão de Dã, Jesus
da tribo de Judá, que sentia aversão crescente pelos apóstolos, por essa
ralé clerical, escolhidos não por desejo dele, mas por necessidade, e
que eram incapazes de penetrar sinceramente no intento ousado do
Impostor de salvar seu povo --- tão ímpio como os outros povos ---, e,
ao salvá-lo, realizariam o desígnio divino... Assim se deu também com
Eliseu, que, devido à ofensa dos homens, decidiu tornar-se um profeta, e
o pediu com insolência ao profeta Elias:

--- Que o espírito que há em ti seja duplicado em mim.\footnote{2 Reis
  2:9.}

Ao que Elias respondeu:

--- O que pediste é algo muito difícil. Se tu me vires quando eu for
tirado de ti, assim tu ficarás; mas se tu não me vires, nada acontecerá
{[}...{]}\footnote{2 Reis 2:10.}

O que se desenrolou depois com o profeta inspirou Iazykóv, poeta russo
da época de Púchkin, e tanto a grandeza dessa passagem bíblica como a
grandeza da inspiração do jovem Iazykóv foram assinalаdas por Gógol em
\emph{Passagens escolhidas entre minhas correspondências com
amigos}.\footnote{Antologia de Nikolai Gógol (1809\emph{--}1852)
  publicada em 1847.} Gógol escreveu que Iazykóv ali se superara,
tocando em algo sublime. Com efeito, a mão de Iazykóv adquirira uma
potência puramente puchkiniana.

\emph{Quando, bramindo e flamejando,}

\emph{O profeta se elevou ao céu,}

\emph{Um fogo poderoso invadiu}

\emph{A alma viva de Eliseu. }

\emph{Assim, alegre, o gênio estremece, }

\emph{Sente a sua grandeza,}

\emph{Quando diante dele ressoa e resplandece}

\emph{O voo de outro gênio.}\footnote{Trata-se do poema ``O gênio'',
  escrito por Nikolai Iazykóv (1803\emph{--}1846) em 1825.}

Em Eliseu entrou o espírito de Elias, que, ``flamejando, se elevou ao
céu''. Eliseu\footnote{O ciclo de Eliseu é narrado no Segundo Reis.
  ``Eliseu {[}...{]} estava arando quando foi chamado por Elias.
  Acompanhou-o até o momento de sua ascensão, dele recebendo o manto e
  uma `dupla porção' {[}...{]} de seu espírito profético''.
  (\emph{Dicionário bíblico,} ed\emph{.} Paulus, 1983, p. 274)} se
dirigiu de Jericó a Betel já não como um homem calvo desprezado pelas
pessoas, mas como um profeta. Os adultos passaram a ter medo de rir e de
zombar dele, mas às crianças faltava juízo tanto para dissimular a
própria crueldade como para temer a própria maldade. Por isso na revolta
humana, no elemento humano e no totalitarismo humano há sempre um jogo
infantil, e uma sociedade infantil é sempre uma sociedade totalitária. O
Senhor não dá preferência nem aos grandes nem aos pequenos: todos são
iguais perante o Senhor, que castiga a crueldade e a maldade das
crianças, mas o faz quando elas atigem a idade adulta e о castigo
torna-se especialmente pesado. Ao percorrer a estrada de Betel, Eliseu
não tomou consciência da profecia e não superou a aversão pelos homens
cruéis que ainda estavam na primeira infância. ``Enquanto ele andava
pela estrada, as crianças saíram da cidade e zombaram dele, dizendo:
`Anda, careca! Anda, careca!'. Ele se virou, olhou para elas e as
amaldiçoou em nome do Senhor. E duas ursas saíram da floresta e
despedaçaram quarenta e duas dentre as crianças.''\footnote{2 Reis 2:23,
  24.}

O profeta Isaías disse:

--- Se o ímpio não sofrer um castigo, ele não aprenderá a
justiça.\footnote{Baseado em Isaías 26:10: ``Se o ímpio é favorecido,
  ele não aprende a justiça. {[}...{]}''.}

E o sábio rei Salomão lhe respondeu:

--- A justiça que morre castiga os ímpios que vivem...

O Senhor raramente destrói um ímpio perante a face dа justiça, com mais
frequência destrói a justiça perante a face de um ímpio; então os ímpios
estrangulam uns aos outros. Ao matar as crianças cruéis, Eliseu não
soube castigar os ímpios, pois eles devem ser castigados na idade
adulta, quando o apetite para a vida está maduro. A culpa de tudo são os
momentos de fraqueza da alma, quando até para um profeta torna-se
impossível dissimular sua aversão pelo homem e retardar as punições por
seus pecados.

Isso aconteceu também com Dã, a Áspide, o Anticristo, nas ruas de Rjév.
Muitas vezes, durante sua vida terrena, em Khárkov, Kertch e Rjév, Dã
ouviu pelas costas palavras raivosas, às vezes sussurradas, às vezes
pronunciadas em voz alta por gargantas desenfreadas pela embriaguez. No
início, pensava que essas pessoas suspeitassem que ele era o Anticristo,
enviado para a Maldição. Depois, supôs odiarem a tribo de Dã, por terem
descoberto, através das profecias do profeta Jeremias, que o Anticristo
estava predestinado a deixá-la. Mas, por fim, entendeu que odiavam
igualmente as doze tribos de Israel. E Rúben,\footnote{Pela tradição
  bíblica, o antigo Israel era estruturado por uma confederação de dozes
  tribos, dos doze filhos de Jacó: Rúben, Simeão, Levi, Judá, Dã,
  Neftali, Gad, Aser, Issacar, Zabulon, José e Benjamim. Efraim e
  Manassés, filhos de José, deram origem a duas tribos, no lugar de Levi
  e de José.} o primogênito de Jacó, e Simeão, e Levi, que deu origem ao
grande profeta Moisés, e todos os sacerdotes levitas,\footnote{Membros
  da tribo sacerdotal de Levi.} e Judá, o fundador do reinado do
salmista Davi, e o sábio Salomão, e Jesus da tribo de Judá, cujas
imagens pagãs eram cultuadas nas igrejas, e Efraim e Manassés, os filhos
de José, o Belo, e Benjamim, que deu origem ao profeta-sofredor
Jeremias, e Zabulon, e Issacar, e Gad, e Aser, e Neftali... Todas as
doze tribos eram igualmente odiadas. Então Dã, o Anticristo, entendeu
que os ímpios receberiam o castigo completo apenas na maturidade, quando
compreendereriam o valor do mundo divino e, se nada compreendessem até o
túmulo, o castigo de Deus os alcançaria após a morte... No entanto,
tanto Cristo como Anticristo, em momentos de fraqueza, agem às vezes
contra o desígnio do Senhor que os enviou, executando a vontade divina
prematuramente...

Certa vez, andando por uma rua de Rjév, Dã passou por alguém vestindo um
sobretudo cor de ferrugem, desabotoado e pendendo como um saco. Tudo о
que tinha botões estava desabotoado: o paletó, o colete de tricô е a
camisa; a camiseta azul de baixo não tinha botões, por isso não podia
estar desabotoada, em compensação, estava rasgada. Esse homem tinha um
rosto trivial, mas cada um de seus traços se tornava singular, pois sua
trivialidade chegava ao extremo, à alegoria. Os cabelos eram
castanho-claros e levemente grisalhos, mas desgrenhados e eriçados, a
magreza das bochechas era acentuada por duas rugas compridas e pela
barba grisalha por fazer, os olhos nórdicos eram aguados e o nariz
tipicamente eslavo, com inúmeras veias vermelhas, e os lábios de um
formato comum se aderiam por uma crosta de saliva seca e de muco, o que
faria qualquer um involuntariamente estremecer ao pensar na mulher que
tivesse de beijá-los. Quando Dã passou pelo homem, este fitou-o na face
como se o conhecesse. O ódio desfigurou-lhe o rosto sujo, descerrou-lhe
os lábios descarnados, grudados por muco e saliva, e, além do fedor do
ventre descuidado, ele emitiu através dos dentes amarelos, como através
de uma peneira podre, atrás de Dã:

--- Credo, um \emph{jid}, que ódio... \emph{Jid}...

O homem russo simples não pronuncia essa palavra a todo momento, mas
apenas em casos extremos. Costuma pronunciar ``\emph{jid}'' com gosto,
como se mordesse uma maçã suculenta e crocante. A palavra ``judeu''
também é útil para limpar a garganta que ficou rouca de ódio ou exprimiu
muita alegria. Mesmo assim, não se pode compará-la à ``\emph{jid}''...
Não há na palavra ``\emph{judeu}''\footnote{Há três termos em russo para
  se referir aos judeus: \emph{ievrei,} aqui traduzido como ``judeu'',
  \emph{iudei,} vertido por ``hebreu'', e o pejorativo \emph{jid,}
  apenas transliterado.} aquele quê de malícia, de originalidade, que
distingue um cálice de vodca de uma caneca de \emph{kvás}. É agradável
tomar um pouco de \emph{kvás} em um dia quente, mas somente como algo
acessório, não como item principal... Já o pensador intelectual russo
utiliza frequentemente o termo ``\emph{jid}'' na função de adjetivo,
para caracterizar fenômenos e acontecimentos. Na tradição intelectual,
não se ouve tanto ``\emph{jid}'' como ``de \emph{jid}'', pronunciado em
duas notas sonoras. O intelectual pronuncia ``uma ideia de \emph{jid}''
como se tomasse um cálice de vodca de sorva acompanhado por uma perdiz e
limpasse os lábios rubros e amadurecidos com um guardanapo engomado,
produzindo um estalo.

Mas fazia tempo que o homem que topou com Dã limpava os lábios
ressequidos da aguardente com a manga suja e ensebada, pois estava no
limite. E em sua insensatez ele dizia:

--- Credo, um \emph{jid}, que ódio... \emph{Jid}...

Então Dã, contrário ao desígnio divino, não suportou, assim como
sucedera ao profeta Eliseu ao castigar de forma prematura, ou seja,
frágil, as crianças cruéis e ímpias no caminho de Jericó a Betel. Como
Jeremias profetizou, Dã colocou na frente do homem um obstáculo. As
calçadas ruins de Rjév e a boa vodca de trigo da safra de 1941 o
ajudaram nisso. O homem caiu, mas não de frente, quebrando a testa e o
nariz, nem de lado, quebrando o braço --- ele caiu de costas, batendo a
nuca numa pedra do calçamento, e morreu, sem acarretar uma redução
significativa à numerosa e ramificada tribo eslava. O homem não disse
mais nenhuma palavra, ``\emph{jid}'' foi a sua última, e com ela na boca
apresentou-se imediatamente ao Senhor, que, sem nada perguntar, enviou-o
logo para o caldeirão quente de breu, onde o trataram com descortesia е
bateram com ganchos em suas costelas emagrecidas pela revolução e pelos
planos quinquenais. Já aqui, no mundo terreno, seus compatriotas,
pesarosos, juntaram-se em torno do coitado, tentando, antes da chegada
do socorro médico socialista gratuito, limpar a nuca ensanguentada dele
com a água trazida em um tambor vazio de leite por uma camponesa que
voltava da feira. Talvez algum dos compatriotas já tivesse ouvido aquele
bêbado gritar ``\emph{jid''} para qualquer transeunte judeu --- grande
coisa! ---, mas como distinguir o Rabinóvitch do armarinho do
Anticristo, enviado pelo Senhor para a Maldição? São todos filhos do
mesmo pai, apesar de terem mães diferentes, por isso cada um deles tem a
mesma origem, mas não o mesmo fim.

Dois dias depois, enterraram o homem, e o Anticristo foi assistir ao
funeral. Ánnuchka também foi ver, pois morava perto do cemitério e todo
dia esperava um pouco de música. O homem era chamado, neste mundo, de
Pávlik,\footnote{Diminutivo de Pável (Paulo em russo).} como o apóstolo
da tribo de Benjamim, o primeiro convertido do mundo. É verdade que, no
início, quando o apóstolo perseguia cristãos, ele tinha o nome de Saulo,
mas depois começaram a chamá-lo Paulo, do que ele se orgulhava muito,
assim como de sua nacionalidade romana, e foi também dos cristãos o mais
fervoroso, apesar de nunca ter visto Cristo em vida. Mas o homem de Rjév
chamava-se Pávlik desde seu nascimento. Houve um momento em que, por
insistência do padrinho, ele quase foi Vássia, mas, no fim, acabaram por
chamá-lo Pávlik.

O funeral de quem devia ser Vássia e acabou sendo Pávlik foi acompanhado
pela orquestra do clube dos ferroviários, porque Pávlik, neste mundo,
trabalhava nas oficinas ferroviárias de Rjév, possuindo o título de
proletário hereditário e, mais tarde, de alcoólatra incurável. Assim que
adquiriu seu último título, começou a cantar em público a conhecida
\emph{tchastuchka}\footnote{Tipo de canção folclórica russa,
  frequentemente de caráter irônico ou satírico.} russa: ``Bata nos
\emph{jides}, salve a Rússia'',\footnote{A expressão, que incentivava os
  \emph{pogroms}, passou a circular durante a Guerra Civil
  (1918\emph{--}1921). Hoje, é punível por lei. Alguns afirmam que ela é
  de autoria de Nestor Makhnó (1888\emph{--}1934), líder de um movimento
  anarquista camponês no sul da Ucrânia.} que soa melhor quando cantada
por um tenor. E Pávlik era justamente um tenor.

Apesar de ser considerada até hoje de origem popular, essa
\emph{tchastuchka}, como muitas canções populares, teve um dia um autor.
Esse autor foi precisamente Márkov Segundo,\footnote{Nikolai Márkov
  (Márkov Segundo) (1866\emph{--}1945), político e escritor de origem
  nobre nascido no território atual da Ucrânia. Depois da Revolução de
  1917, uniu-se ao Movimento Branco.} deputado de Kursk da Duma do
Estado. Porém, como tantas canções que viraram na boca do povo falação,
havia muito que ela perdera a autoria concreta, tendo sobrevivido à
prova do tempo. E essa era a \emph{tchastuchka} que Pávlik entoava com
voz de tenor.

Pávlik era chamado para ir ao comitê do sindicato das fábricas, sendo
repreendido por reproduzir costumes do antigo regime. Ele começou a
faltar no trabalho cada vez mais. A esposa choramingava:

--- Você morrerá na sarjeta, e ninguém irá ajudá-lo.

--- Ora --- dizia Pávlik, com um gesto de desaprovação ---, depois de
morto, podem até fazer um belo salame comigo...

Mas, quando Pávlik morreu em um acidente infeliz, muita gente foi ao seu
funeral. Com coroas de flores. Carregaram o caixão para a extremidade do
cemitério onde os túmulos com cruzes eram menos numerosos do que com
estrelas. E sobre o túmulo de Pávlik não colocaram uma cruz, mas uma
estrelinha, para que ele continuasse, no outro mundo, sob poder
soviético.

O povo proletário das oficinas ferroviárias não sabia o que sabia Dã, а
Áspide, o Anticristo. No outro mundo, Pávlik foi parar num caldeirão de
breu apolítico, e sua última palavra, ``\emph{jid}'', grudou em seus
lábios com breu, cortando-lhe a boca com suas pontas afiadas. E os
outros pecadores do caldeirão, que também sofriam torturas eternas,
passaram a odiar Pávlik por seu grito torturante de porco, entoado com
voz de tenor: \emph{jid}. Nem por um segundo a dor cessava, nem por um
segundo o grito torturante de Pávlik silenciava. Mas aqui embaixo, onde
o céu é como os olhos de um eslavo do Norte, o corpo de Pávlik jazia
serenamente num caixão vermelho.

Era o início da primavera do ano de 1941 depois do nascimento do Irmão
de Dã, Jesus, da tribo de Judá. Na região de Khárkov ou mesmo em Kursk,
em um dia ensolarado, a neve começava a derreter, mas em Rjév o inverno
não se movia. A neve repousava sobre os túmulos, firme e inerte, os
galhos das árvores do cemitério estavam mortos, e da boca dos que
choravam saía vapor. Dã olhou ao redor, olhou para o rosto do morto e
para os rostos dos vivos e se lembrou de um dos primeiros mandamentos de
Moisés:

``Se alguém surpreender um ladrão em pleno roubo e feri-lo até a morte,
não será culpado de verter seu sangue. Mas, se o sol já tiver se
levantado, o sangue lhe será imputado''.\footnote{Êxodo 22:1, 2 pela
  \emph{Bíblia de Jerusalém}; 2, 3 por outras consultadas.}

Esse foi um dos inúmeros mandamentos bíblicos apresentados de modo
intencionalmente vago. O estilo bíblico evita o excesso de clareza, pois
a clareza em excesso é um \emph{slogan}. Alguns mandamentos exigem um
grande esforço de interpretação e outros um insignificante, como esse.
No entanto, não existe mandamento que possa ser digerido sem esforço
algum. Aqui está sua interpretação: o ladrão surpreendido à luz do dia,
diz o mandamento, tem direito ao perdão, mas a união do ladrão com a
noite não dá direito à piedade.

Dã olhou ao redor e viu que o sol raiava, mas as pessoas em volta tinham
rostos noturnos. E ele entendeu: o próprio sangue que verteram lhes era
imputado...

Nesse momento, no meio da multidão do cemitério, o Anticristo avistou
uma menina de olhar vivo, em nada parecida com a Maria de Khárkov, com
quem, perto de Kertch, ele fora exposto ao terceiro flagelo do Senhor, o
animal-adultério... Apesar de não se parecer com Maria, a menina o fazia
se lembrar dela, e Anticristo começou a observá-la. Seguiu Ánnuchka até
a igreja do cemitério e viu que fora transformada em moradia. Então
pediu para olhar o lugar onde ficavam o altar e os afrescos.

Esses afrescos lhe provocaram aversão, pois infringiam o que havia de
mais sagrado --- o segundo mandamento do profeta Moisés. Como hebreu,
ele sabia que no símbolo de Deus se ocultava a negação de Deus. Que essa
negação havia começado na época da perseguição dos cristãos, nas
catacumbas, nos afrescos que retratavam um monge de Alexandria
descarnado sob a alcunha de Jesus Cristo, da tribo de Judá, vaticinado
pelo profeta Isaías. No entanto, o próprio nome Judá era maldito entre
eles, porque não lhe eram apenas hostis, mas também estranhos, e o
incompreensível tem sempre um sentido único, mecanicamente decorado,
pronunciado por bocas sem juízo, assim como aves falantes pronunciam
palavras humanas... Judá foi amaldiçoado, mas também teriam duvidado de
Jesus Cristo se não vissem continuamente sua imagem, criada por eles
mesmos.

``Procurem a imagem do Cristo em suas palavras, anotadas no Evangelho''
--- aconselharam os Pais da Igreja mais sensatos aos que dele duvidavam.
Porém, os criadores da religião, estranhos a uma percepção de mundo
nacional, só poderiam acreditar, do fundo do coração, no que lhes era
estranho vendo-o com os próprios olhos. Dã, a Áspide, o Anticristo,
sabia para onde levava a fé quando seu objeto fosse visto com os olhos.

Assim como na igreja do cemitério de Rjév, em toda parte era possível
cobrir os velhos ícones dos velhos ídolos com jornais e sobre estes
pendurar novos ícones de novos ídolos. Pois acreditam no que está diante
dos olhos e duvidam do que não veem; como diz o provérbio: ``Longe dos
olhos, longe do coração''. E, quanto mais uma mesma coisa se fixar nos
olhos, mais se acreditará nela. Não era à toa que, em todos os lugares,
sob os olhos de todos, surgia o retrato de um banhista assírio
rechonchudo de bigode, que viera substituir o monge de Alexandria
descarnado. Ali também, naquele quarto, ao lado da imagem do monge de
Alexandria, coberta com um jornal, estava pendurada a imagem do assírio
bigodudo... Mas a fé espiritual na Existência não pode ser coberta com
jornais, nem substituída por um banhista assírio, da mesma maneira que
não pôde ser substituída por um bezerro de ouro\footnote{Do Êxodo
  32:1\emph{--}8: enquanto o povo se afligia com a demora de Moisés, que
  subira ao Monte Sinai para receber os mandamentos de Deus, Aarão
  construiu um bezerro de ouro com os brincos das mulheres e disse ao
  povo que adorasse o novo deus. Em outro episódio (1 Reis 12:28-32),
  Jeroboão fez dois bezerros de ouro para que o povo os cultuasse.} no
deserto dо Sinai.

Assim pensava Dã, a Áspide, o Anticristo, enquanto Ánnuchka, ali
sentada, assustada, esperava que ele abrisse o guarda-roupa e começasse
a pegar os novos pertences que compraram, incluindo seu sobretudo
novinho, forrado de algodão. No entanto, mesmo apavorada, Ánnuchka
olhava furtivamente para ele, pois pensava que, quando enfileirassem
todos os homens depois do roubo e a conduzissem pela fileira, ela
poderia reconhecer o ladrão sem hesitar. De repente Ánnuchka viu pela
janela sua mãe, que voltava para casa por um atalho em frente ao
cemitério levando Mítia pela mão. O rosto de sua mãe estava tristе,
certamente tinha ido ver o túmulo de Vova, pois agora moravam ao lado e
podiam visitá-lo todos os dias. Ánnuchka se alegrou ao ver sua mãe e,
superando o medo, saltou da cadeira e correu ao seu encontro gritando:

--- Um ladrão, um ladrão em casa!

A mãe também começou a gritar, instigada pela amarga experiência do
roubo passado. Por sorte, as pessoas que moravam na igreja eram mais
conscientes do que as do barracão, porque ali, conforme os privilégios
trabalhistas, alojavam os melhores trabalhadores. Eles se juntaram a
tempo de socorrer a desgraça alheia. Por perto não havia nenhum policial
armado, em compensação, um dos \emph{stakhanovistas} havia sido premiado
por um feito heroico com uma espingarda de caça e a levou. Dã nem teve
tempo de cair em si, e uma multidão densa obstruiu a saída dessa parte
da igreja transformada em moradia com tabiques de madeira. Olhavam para
Dã com um ódio alegre, como se costuma olhar para inimigos fracos. Esse
olhar era precisamente o de um antissemita em seu melhor momento, quando
ele pronuncia ``\emph{jid}'' como se desse uma bela mordida numa maçã
madura.

--- Fomos roubados faz pouco tempo e agora tentam de novo --- lamentava
a mãe. --- Sou grata à minha filha, que soube manter a cabeça no
lugar...

--- Dizem que eles só roubam no comércio, mas de resto são honestos ---
falou um dos presentes.

--- É preciso enfiá-lo num envelope e colar uns selos por trás --- disse
o \emph{stakhanovista} premiado com uma espingarda de caça, mantida em
riste.

Queriam se aproximar de Dã, o Anticristo, como um dia tinham se
aproximado de seu Irmão, Jesus, da tribo de Judá. Pois eram as mesmas
pessoas, e Dã, o Anticristo, sabia disso, enquanto elаs não sabiam nada
sobre si mesmas. Como Dã não fora enviado para a Benção, mas para a
Maldição, não para o bem delas, mas contra elas, ninguém podia colocar a
mão nele. De repente, a multidão foi dividida em duas, o amigo foi
separado do amigo ao lado, o marido da mulher, Ánnuchka da mãe... Quando
se reuniram de novo, o Anticristo não estava mais no recinto, já se
achava longe da Rua do Trabalho, embora dentro dos limites da cidade de
Rjév. Depois muitos se puseram a falar. Uns diziam que o bandido estava
com uma faca nas mãos, outros com uma máuser, e alguns chegaram a dizer
que ele segurava o rifle de cano serrado que era usado pelos
\emph{kulakes}. No entanto, como nada do quarto havia desaparecido, o
caso foi esquecido no ato, principalmente por todos estarem sem jeito
entre si depois do ocorrido na captura. Quanto a Dã, a Áspide, o
Anticristo, após deixar a igreja profanada por imagens pagãs, antigas e
recentes, ele foi parar na extremidade oposta de Rjév, perto dos
barracões, onde havia pouco Ánnuchka morava, não longe do aeródromo.

Anoitecia, mas nos campos não havia o silêncio vespertino que
caracteriza o pôr do sol no inverno. O sol se punha em meio ao barulho e
ronco dos motores dos aviões, em meio à vibração do ar gélido. E Dã
voltou a ver a espada que vira pela primeira vez perto de Kertch, a qual
cortava nuvens encharcadas de sangue sobre um mar sangrento. Dessa vez,
o cabo da espada se apoiava no sol vespertino, enquanto o gume
desaparecia atrás dos telhados cobertos de neve da extremidade oeste de
Rjév, e a neve era escarlate como o sangue das artérias. E Dã, a Áspide,
o Anticristo, ouvia a palavra dita pelo Senhor através do profeta do
exílio, Ezequiel:

--- Ai da cidade sanguinária! Que desgraça, na caldeira formou uma
crosta de ferrugem e a ferrugem não sai! Tira pedaço por pedaço, sem
escolher por sorteio. Pois sangue está no meio dela. Ele o pôs sobre uma
rocha calva, e não o verteu sobre a terra, para que ele não se cobrisse
de poeira. Para suscitar a fúria e efetuar a vingança, deixei seu sangue
sobre a rocha calva, de modo que ele não se ocultasse. Por essa razão,
assim dizia o Senhor Deus: ai da cidade sanguinária! Eu também farei uma
grande fogueira!\footnote{Ezequiel 24:6\emph{--}9.}

Depois dessas palavras, o sol se pôs, e a visão da espada e do sangue
desapareceu. Dã, a Áspide, o Anticristo caminhava por uma rua da
periferia de Rjév, iluminada por escassos lampiões --- crepitando sobre
a neve seca e congelada, ele passou diante da luz pacata da noite que
saía das janelas das casas e desapareceu onde começava a cerca da
cooperativa das empresas de leite recém-construído. A essa hora eram
raros os transeuntes na periferia de Rjév, e muito tempo se passou antes
de surgir alguém, que vestia umа \emph{telogreika}\footnote{Casaco
  acolchoado, simples e prático, para inverno rigoroso usado pelo
  Exército Vermelho na Segunda Guerra Mundial. Muito difundida por
  campos de prisioneiros, a \emph{telogreika} deixou de ser uniforme
  militar na década de 1960.} e botas de feltro enfiadas em enormes
galochas.

No entanto, a visão de Dã não se realizou de imediato, mas apenas quando
Ánnuchka já fazia tempo que não vestia suas botas vermelhas favoritas e
aguardava o retorno iminente do circo. Inesperadamente, Ánnuchka passou
a ouvir nas conversas dos adultos:

--- A guerra, a guerra... Os alemães, os alemães...\footnote{A Alemanha
  invadiu a URSS em 22 de junho de 1941. De 1942 a 1943, ocorreram as
  Batalhas de Rjév, perto de Moscou, acarrentado uma enorme perda ao
  Exército Vermelho. Depois de muitas ofensivas, os nazistas abandonaram
  a região.}

Mas para Ánnuchka, no início, nada mudou, e sua mãe também disse a uma
vizinha:

--- Para mim não fará grande diferença, já mataram meu Kólia na Guerra
da Finlândia.

Durante o mês de junho não aconteceu nenhuma mudança. A não ser o circo,
que não veio. Em julho, porém, as coisas começaram a mudar. Certo dia,
sua mãe voltou muito preocupada do armazém nº 40 e disse:

--- Crianças, vamos empacotar tudo. Vamos ter que nos refugiar na aldeia
de Klechniovo, que fica a sete quilômetros daqui.

Empacotaram seus pertences de qualquer jeito, incluindo as botas
vermelhas de feltro e o sobretudo forrado de algodão de Ánnunchka, caso
precisassem passar o inverno em Klechniovo. Trancaram o quarto com
cadeado. Levaram um dia inteiro para chegar, andando sob forte calor.
Não pararam mais de duas vezes para descansar e comer.

--- Crianças, precisamos nos apressar --- disse a mãe --- para nos
instalarmos melhor, antes que outros venham.

Chegaram a Klechniovo à noite e foram acomodados em uma escola, mas
Ánnuchka notou que ali havia muitas pessoas e que ninguém estava
contente com a presença deles...

Viviam em Klechniovo como se viajassem de trem, vigiando suas trouxas,
e, quando as reservas de comida acabaram, começaram a passar fome. Por
isso Ánnuchka e Mítia se alegraram com as palavras de sua mãe:

--- Vamos voltar para casa, em Rjév. Setembro está chegando, e vocês
devem ir à escola.

Voltaram mais rápido para Rjév do que dela partiram, cansaram-se menos,
e, encontrando a casa intacta, alegraram-se: ``agora tudo será melhor''.

Realmente, estar em casa era melhor do que na aldeia de Klechniovo,
apesar da guerra. A mãe voltou a trabalhar no armazém nº 40, e a comida
tornou-se mais abundante. Certamente, não era como antes da guerra, mas
ainda assim comiam melhor.

Certa noite, num dos últimos dias de agosto, a mãe disse:

--- Crianças, amanhã vocês vão para a escola. Vamos arrumar os livros
nas mochilas, para que não precisem procurá-los de manhã e não se
atrasem na primeira aula.

Mal se puseram a arrumar os livros, começou a estrondear em algum lugar.
A última vez que trovejara desse jeito fora durante o forte temporal que
causara a morte do pequeno Vova. Ánnuchka se assustou e sua mãe também.
Ela pegou o filho pelo braço.

--- Vamos correr para a horta --- disse --- е deitar no meio dos
canteiros.

Como no cemitério havia um terreno baldio, as autoridades permitiam aos
\emph{stakhanovistas}, os moradores da antiga igreja do cemitério,
manter pequenas hortas para auxiliar seu sustento. Ánnuchka reparou que
alguns \emph{stakhanovistas}, que não tinham conseguido fugir a tempo,
também estavam deitados na horta, entre os canteiros. De repente, um
estrondo soou bem perto, no cemitério, depois um segundo. Uma fumaça
branca se espalhou, cheirando a omelete queimada. Ánnuchka pôs-se a
chorar, mas o \emph{stakhanovista} premiado com uma espingarda de caça
acalmou-a:

--- Não é nada --- disse ---, não tenha medo, menina... O poder
soviético ainda está vivo.

Depois do bombardeio, Ánnucka retornou para casa com sua mãe e Mítia, e
não conseguiu dormir a noite toda. Carros e carroças passavam, ouviam-se
conversas e, até de manhãzinha, o poder soviético se manteve. Depois de
amanhecer, o poder alemão se estabeleceu.

--- Crianças, fiquem em casa --- disse a mãe ---, não saiam para a rua.

No entanto, o poder alemão não esperou que Ánnuchka e Mítia saíssem, ele
mesmo foi à casa deles, produzindo batidas de pés no corredor que nada
tinham de russo, e atrás do tabique de madeira logo se ergueu um
tumulto, logo qualquer resistência foi superada com facilidade, pois a
força estava do lado alemão. Ánnuchka sentiu tanto, mas tanto medo que
ficou até curiosa, espiando o corredor. Ela não tinha vivido muito
tempo, porém vira mais de uma vez pessoas apanharem, pois vivia em um
país onde esse tipo de agressão era recorrente. De fato ela via com mais
frequência espancamentos sem sangue, a ponto de sangrar só havia
presenciado duas vezes. O administrador do barracão batera até sangrar
no homem apontado por Ánnuchka como ladrão, e depois dois garotos
brigaram na frente dela, derramando sangue... Ánnuchka também conhecia a
dor de uma palmada e mesmo de um soco, como sua mãe lhe dera quando ela
descuidara de Vova e ele morrera --- ela ainda se lembrava disso... No
entanto, Ánnuchka nunca poderia imaginar que era possível se bater num
homem como os alemães bateram no \emph{stakhanovista} que um dia fora
premiado por seu feito heroico pelo poder soviético. Quanto a bater não
a ponto de sangrar, agora isso nem se cogitava. Era como se carregassem
uma bacia cheia de sangue (como donas de casa carregam, após lavar
roupa, bacias com água ensaboada) e tropeçassem na escuridão, derramando
o sangue no chão. E os alemães sentiam mais aversão a cada surra, já sem
o arrebatamento inicial, pois suas botas se sujavam de sangue. Andavam
pelo corredor em volta dos corpos estendidos como se andassem na lama do
outono ou da primavera, saltando de um monte a outro. Então um alemão
não vestido à moda russa disse alguma coisa a um policial usando um
terninho de algodão comprado numa loja de departamentos de Rjév. Este
escancarou a porta, atrás da qual estava Ánnuchka, e gritou para a mãe:

--- Ei, prostituta de Stálin, saia daí...

Ánnuchka começou a chorar de imediato, agarrando-se a sua mãe, assim
como Mítka, então o policial, que inesperadamente deixou transparecer a
tradicional bondade eslava, disse à mãe:

--- Não tenha medo, ninguém tocará em você. É preciso tirar daqui o
comissário, pois ele está coberto de sangue, e os senhores alemães estão
enojados.

A mãe e uma vizinha carregaram o \emph{stakhanovista}, cuja esposa e
filhos tinham sido evacuados, enquanto ele fora retido por estar
atrasado no envio de um equipamento da fábrica... Inicialmente os
alemães mandaram levá-lo para a carroça, mas no meio do caminho mudaram
de ideia, ordenando que o levassem para o cemitério. O policial com o
terninho de algodão de segunda conduziu a transferência do
\emph{stakhanovista} pisoteado por botas alemãs.

--- Mulheres, quanto mais longe o levarem --- disse o policial ---,
melhor será para vocês... Para não feder na frente dе casa.

As duas mulheres passaram com o \emph{stakhanovista} em frente à cerca
construída antes da revolução, diante de cruzes miseráveis e do pequeno
túmulo de Vova, onde havia uma lápide. Elas levaram o corpo até a área
dos túmulos soviéticos com estrelas e pararam perto do túmulo ainda
fresco no qual jazia Pávlik, que fora morto pelo Anticristo com a
palavra ``\emph{jid}'' na boca.

--- Joguem aí --- disse o policial do terninho comprado numa loja de
departamentos, armado com um fuzil-baioneta pré-revolucionário, a
baioneta triangular celebrada em canções russas.

A mãe de Ánnuchka e sua vizinha não jogaram o \emph{stakhanovista}, mas
colocaram-no cuidadosamente sobre a grama do cemitério, apoiando sua
cabeça no túmulo de Pávlik, como se este fosse um travesseiro.

--- Agora vão embora --- disse o policial.

Mal a mãe e a vizinha se viraram, ouviram atrás de si um breve ``eh'',
com o qual os camponeses costumam rachar lenha, e algo como um soluço...
Olhando para baixo, elas aceleraram o passo, no entanto o policial as
alcançou rapidamente, limpando a baioneta suja de sangue com um punhado
de grama.

--- Dão poucos cartuchos --- queixou-se ele, com simplicidade ---, o
fuzil é russo, um troféu de guerra, os cartuchos também, mas são
difíceis de achar --- e, percebendo que as mulheres não respondiam,
acrescentou, bravo: --- Hoje tudo deve ser lavado e varrido. Os alemães
ficarão alojados na casa de vocês, está claro?

Assim começou a vida sob poder alemão. Os alemães se sucediam uns aos
outros sem parar. Uns eram cruéis, outros mais piedosos. Geralmente
apareciam no fim da tarde para passar a noite. Os cruéis enxotavam de
casa a mãe, Ánnuchka e Mítia a pontapés e os piedosos sem pontapés. No
começo, a família dormia na rua, apesar de as noites de Rjév serem frias
em setembro. Pelo menos ainda não chovia, e quando começasse? A mãe
tentava bater nas portas vizinhas, pedia que os abrigassem, mas todos
tinham medo, pensando que fossem judeus, os quais eram procurados pelos
alemães. Nem quando a mãe exibia Mítia pela janela, mostrando que eram
russos, os deixavam entrar, pois poderiam ser da família de um comunista
ou de um \emph{partizan}...\footnote{Membro de uma tropa clandestina de
  resistência.} Finalmente, acharam uma velha bondosa que os abrigasse,
e, desde então, toda noite, assim que os alemães chegavam e os
enxotavam, eles iam dormir na casa da velhinha, levando até roupa de
cama e travesseiros. De manhã os alemães iam embora, e os três voltavam
para casa e não a reconheciam... O fedor alemão de ervilha era
inigualável... Mesmo quando o frio tornou-se intenso, era necessário
escancarar as janelas. O dia inteiro a mãe lavava e arrumava a casa, e
Ánnuchka a ajudava, enquanto Mítia trazia água, e depois os alemães
vinham dormir outra vez... É preciso notar que, além de tudo, a mãe
temia que descobrissem o retrato de Stálin, que ela envolvera
cuidadosamente numa velha camisa do falecido marido, Kólia, e enterrara
no cemitério, entre os túmulos soviéticos mais afastados. Contudo,
ninguém descobriu nada nem se interessou por isso, e a mãe se acalmou.
Ela arrancou os jornais da parede, deixando vísiveis as antigos afrescos
da igreja, pois ouvira dizer que os alemães respeitavam Deus. A bem da
verdade, uma vez, durante uma esbórnia especialmente intensa, sob efeito
de \emph{schnaps-vodca}, os alemães desenharam com carvão por cima das
faces dos santos e, na testa do Cristo, pregado na cruz, fizeram uma
estrela de seis pontas e escreveram: \emph{Jüdisches Schwein}'' ---
``porco judeu''... A mãe ficou com medo de apagar a inscrição e ordenou
a Ánnuchka e Mítia que não tocassem nela.

Eles viviam famintos e se alimentavam sabe-se lá como. Às vezes a mãe
trazia beterrabas, cenouras ou batatas. Um dia, Mítia fez amizade com um
garoto na rua, que lhe disse:

--- Sabe onde ficavam os quartéis militares? Agora muitos dos nossos
estão ali, atrás do arame farpado. Vamos pedir pão a eles.

Ánnuchka disse:

--- Não vá, Mítia, é perigoso, os alemães vão bater em você e podem até
matá-lo.

Mítia foi e voltou inteiro, mas sem pão.

--- Pedimos pão a eles --- disse ---, e eles pediram a nós.

Justamente nesse dia a mãe também não trouxe nada.

``O que vamos comer?'' pensava Ánnuchka.

A essa altura, os alemães, como sempre, chegaram para dormir, porque já
tinha anoitecido. A mãe vestiu Mítia e a si mesma e, quando começou a
abotoar o sobretudo forrado de algodão de Ánnuchka, um alemão disse:

--- \emph{Nein}, \emph{nein}... Não, não... Fiquem mais, como vocês
dizem.

A mãe ficou desnorteada, mas o alemão sorriu e tirou uma fotografia do
bolso.

--- \emph{Kinder} --- disse ---, meu bebezinho... \emph{Zwei}... Também
dois... Eu falo um pouco russo.

Depois pegou duas torradas e deu uma à Ánnuchka e outra a Mítia. Pegou
uma terceira e deu à mãe. O alemão gostou especialmente de Ánnuchka.

--- \emph{Gut, gut} --- disse ---, você deve aprender alemão... Eu serei
professor...

O alemão não foi embora na manhã seguinte, e a mãe se alegrou. Morou na
casa deles quase uma semana, e a mãe se afeiçoou a ele, Ánnuchka também,
somente Mítia ficou alerta. O sujeito se chamava Hans, e, pela primeira
vez em muitos meses, ganhavam ora um pedaço de pão, ora um pedaço de
toucinho, ora um pouco de caldo concentrado de ervilha. Esse alemão
nunca escarrava nem assoava o nariz no chão, e comia polidamente. Assim
que terminava de comer, tirava do bolso um carretel, arrancava um pedaço
de linha e começava a limpar os dentes dos restos de carne e de ervilha.
Ao concluir sua higiene, dava um ou dois arrotos e chamava Ánnuchka para
estudar alemão. Ánnuchka assimilou rapidamente muitas palavras e
aprendeu a contar: \emph{eins, zwei, drei}.

--- \emph{Brot} --- dizia o alemão ---, ``pão''... \emph{Anna mit
Grossvater gehen spazieren}\ldots{} ``Anna está passeando com seu avô''.

Ele notou a estrela de seis pontas desenhada sobre a testa de Cristo e a
inscrição \emph{Jüdisches Schwein}.

--- \emph{Jüdisches Schwein} --- ele disse e começou a rir ---, ``porco
judeu''.

--- \emph{Jüdisches Schwein} --- Ánnuchka repetia agilmente. ---
\emph{Anna mit Grossvater gehen spazieren}\ldots{} \emph{Eins, zwei,
drei}\ldots{}

No fim da semana, no entanto, Hans ficou triste e, numa manhã, abotoou
seu capote militar, pegou seu fuzil-metralhadora, colocou seu capacete e
se transformou num alemão comum, de modo que Ánnuchka até se assustou.

--- А guerra, a guerra --- disse ele tristemente à mãe. --- Rjév é ruim,
Colônia é bom --- e suspirou.

Então ele percebeu que Ánnuchka o fitava assustada, como se ele já não
fosse o alegre e bondoso tio Hans que lhe dava toucinho e a ensinava a
falar alemão, mas um alemão comum que a enxotava de casa a pontapés.
Então Hans sorriu e piscou a ela, apontando para a estrela de seis
pontas esboçada no meio da testa de Cristo e para a inscrição de carvão
que atravessava sua face:

--- \emph{Jüdisches Schwein} --- disse.

--- \emph{Jüdisches Schwein} --- repetiu Ánnuchka ---, ``porco judeu''.
\emph{Anna mit Grossvater gehen spazieren}\ldots{} \emph{Haus,}
``casa''; \emph{Vogel,} ``pássaro''; \emph{Katz}, ``gato''; \emph{Hund,}
``cachorro''.

--- \emph{Gut, gut} --- Hans riu, acariciou mais uma vez a cabeça de
Ánnuchka, curvou-se à mãe e saiu, porque na rua já o chamavam e caçoavam
dele.

À noite vieram outros alemães para pernoitar, e entre eles havia um
parecido com Hans. A mãe pediu em voz baixa à Ánnuchka que falasse com o
alemão na língua dele, como Hans lhe ensinara, pois na semana anterior,
enquanto Hans morava com eles, sentiam-se protegidos e comiam os restos
da comida alemã.

--- \emph{Jüdisches Schwein} --- disse Ánnuchka. --- \emph{Anna mit
Grossvater gehen spazieren}\ldots{} \emph{Haus,} ``casa''; \emph{Vogel,}
``pássaro''\ldots{}

O alemão começou a rir e, como Hans, disse:

--- \emph{Gut, gut}...

Sem demora, a mãe, para ganhar mais simpatia do alemão, trouxe uma bacia
de água quente e uma toalha limpa para que ele se lavasse e se
enxugasse. O alemão se lavou e se enxugou, depois olhou para a mãe e
subitamente a pegou pela saia, abaixo do ventre. A mãe deu um grito,
assustada, e depois outro, porque Mítia golpeou com a cabeça o flanco do
alemão, fazendo-o cambalear. Ánnuchka ficou apavorada, pois sabia como
os alemães batiam. No entanto, antes que o alemão batesse em Mítia, a
mãe mesma deu uma palmada no filho, mas não na cabeça dele, o ponto
mirado pelo alemão, e sim no traseiro. Ela batia em Mítia e, ao mesmo
tempo, afastava-o do alemão enfurecido. Por isso o alemão não machucou
Mítia, apenas os botou para fora de casa, como faziam os alemães antes
de Hans.

Foram novamente à casa da velhinha caridosa, mas não conseguiram dormir,
com medo de que viessem atrás de Mítia. De manhã, a mãe disse:

--- Crianças, fiquem aqui enquanto vou para casa; esperarei que os
alemães saiam e pegarei o que puder... Iremos até a aldeia de Agárkovo,
eu tenho uma prima lá, quem sabe conseguimos um lugar para ficar.

A mãe foi para casa pedindo a Deus que os alemães saíssem, pois, sem a
presença do poder soviético, não havia a quem pedir ajuda além de Deus.
Seu pedido foi atendido: os alemães saíram, entraram num caminhão e
desapareceram. A mãe imediatamente irrompeu no quarto. Evidentemente,
tudo estava quebrado, revirado, molhado, mas no meio da cama se achava a
toalha limpa que ela havia dado ao alemão. A mãe pegou a toalha, mas ela
estava pesada --- continha uma porção consistente e saudável de bosta
ariana, por meio da qual, ao lado das medições do crânio, seria possível
determinar a raça ariana. Impossível confundi-la com bosta eslava, muito
menos com bosta judia. No entanto, o alemão não tinha envolvido sua
bosta numa toalha russa em prol da análise da pureza de sua raça, mas
pelo humor alemão, vigoroso como um prato de salsicha, um humor que se
distinguia, na opinião dele, da ironia judia, ressequida como uma tigela
de galinha. Somente os mais talentosos dos eslavos seriam capazes de
sentir o espírito alemão. A mãe de Ánnuchka, também chamada Ánnuchka,
não pertencia aos melhores elementos de sua raça, não se sentia uma
ariana e, à diferença de um célebre escritor russo do século XIX, não
almejava a unidade ariana dos Urais ao Reno. Ela era movida por seus
pequenos interesses e, assim, pegava as coisas que estavam ao alcance da
mão...

Logo depois, a mãe, Ánnuchka e Mítia se arrastaram por um campo nevado
rumo à aldeia de Agárkovo. Eles não andavam, mas literalmente se
arrastavam, pois carregavam todas as suas coisas. No entanto, não foram
direto a Agárkovo, mas pararam em Klechniovo, e, de novo, ninguém se
mostrou contente com a vinda deles. Deixaram que pernoitassem ali, mas
não lhes ofereceram comida, pois não tinham nada. De manhã, os três
seguiram viagem e ainda pararam na aldeia de Grigórievka. Lá a mãe
esmolou e conseguiu um pouco de batata cozida estragada pelo frio. Não
os deixaram entrar na isbá, por medo do tifo, e levaram as batatas, em
um jormal, para o quintal. Somente na tarde seguinte, a mãe, Ánnuchka e
Mítia chegaram a Agárkovo. Era uma aldeia pequena, não tinha mais de dez
casas; em compensação, era tranquila e os alemães só tinham estado ali
uma vez, de passagem.

A prima da mãe, apesar de não ter ficado muito contente com a chegada
deles, hospedou-os e deu-lhes comida. E assim começou a vida na aldeia
de Agárkovo. Passaram-se o inverno e a primavera, e no verão, em agosto,
tropas soviéticas libertaram a aldeia. Foi uma grande alegria. A pequena
aldeia estava abarrotada de soldados soviéticos, que se alojavam e
pernoitavam nas isbás.

Nosso soldado também fede, mas o fedor dele é familiar, não hostil.
Devemos ainda lembrar que os russos e os demais habitantes da Rússia
comem mais cereais e fermentados do que carne. Por isso seu fedor,
apesar de intenso, não é corrosivo. Já entre os alemães, cuja base da
alimentação é ervilha com toucinho, o fedor é calórico e persistente...

Mas aconteceu uma desgraça: mal o exército soviético libertou a aldeia
de Agárkovo, Mítia ficou doente. A mãe o colocou numa telega que passava
e o levou aos militares do setor sanitário; ela contou que era viúva de
um soldado morto na Guerra da Finlândia e se apiedaram dela, deixando
que Mítia se tratasse lá. Passados alguns dias, Mítia começou a se
recuperar e até aparecia nos degraus da entrada do hospital para se
encontrar com sua mãe e Ánnuchka e lhes dar pão, já que ele recebia à
vontade.

--- Comam --- dizia ele ---, senão vão acabar morrendo...

De novo pareciam felizes, de novo essa felicidade estava permeada de
desgraça. Uma noite, inúmeros aviões alemães lançaram-se sobre a aldeia
de Agárkovo, e de manhã nada havia sobrado dela. Os que conseguiram se
salvar foram à floresta levando tudo o que podiam. A floresta ficava a
três quilômetros da aldeia e agora as tropas soviéticas estavam
instaladas lá. Só que os habitantes ficavam separados dos militares e
Ánnuchka, sua mãe e Mítia separados dos habitantes, que os consideravam
de fora.

Acolheram-se num abrigo subterrâneo, sob uma colina perto de um riacho.
Mítia ficava deitado sobre uma cama macia, feita pela mãe de todas as
coisas que trouxeram, para que ele pudesse se recuperar. No abrigo
penduraram a gaiola com um passarinho que Ánnuchka achara na rua durante
o bombardeio. Sem se importar com os tiros em volta, os gritos e o choro
de crianças, o passarinho punha-se a cantar assim que o sol se
levantava. Ánnuchka adorava o passarinho, sua mãe também, mas Mítia o
amava mais que todos. Ele tentava lhe dar pedacinhos de grama, sementes
de girassol, água fresca... Um dia, Ánnuchka e a mãe ceifavam centeio
nas redondezas e Mítia, deitado no abrigo, ouvia o passarinho cantar
quando, de repente, caiu uma bomba, depois outra, bem perto do abrigo.
Fumaça se levantou, mas a mãe não esperou que o vento a dispersasse, e
correu, em meio à fumaça, para o abrigo, seguida de Ánnuchka. Então elas
viram Mítia, ileso, se arrastando para fora. Parecia que um arado tinha
passado por cima do lugar, e as árvores ao redor estavam queimadas.
Depois viram a gaiola no chão, com o passarinho morto dentro...
Lamentaram ao lembrar como ele cantava, mas que fazer? Mítia disse:

--- Senti algo voando na minha direção, entrei no abrigo, me enfiei num
canto e pensei que tudo iria desabar...

Pouco depois, apareceu uma carroça militar e os levou para longe, dentro
da floresta. Ali Mítia curou-se totalmente; em compensação, logo
Ánnuchka e sua mãe adoeceram... Eles viviam numa choupana de galhos de
abeto, em mau estado, e não havia ninguém para consertá-la. No primeiro
dia em que ficou doente, a mãe carregou, à medida que parava em pé, o
máximo de galhos que pôde com Mítia, para que a choupana ficasse seca
quando começassem as chuvas. Ánnuchka não podia ajudar, sua cabeça ficou
quente e pesada, assim como seus pés e suas mãos, e ela não conseguia se
levantar... Assim, mãe e filha ficaram acamadas por alguns dias. Mítia
ajudava como podia: trazia água, limpava espigas de centeio, descascava
sementes de girassol...

Certa manhã, viram chegar uma carroça do setor sanitário com a cruz
vermelha. Duas militares andavam entre os civis aplicando-lhes vacinas,
enquanto os padioleiros levavam os doentes para o veículo. Levaram
Ánnuchka e sua mãe, mas não Mítia.

--- Ele não está doente --- disseram.

Assim que a carregaram, a mãe disse a Mítia:

--- Filhinho, não vá embora daqui, fique com os outros. Logo voltarei
para casa, para a choupana...

Ánnuchka ainda ouviu essas palavras da mãe, depois não ouviu mais nada.
Quando Ánnuchka voltou a si, viu-se num grande quarto, deitada sobre uma
maca. E logo começou a gritar e a chamar pela mãe. Alguém lhe disse:

--- Não grite, sua mãe está deitada ao seu lado.

--- Virem-me de lado, quero ver minha mãe.

Ánnuchka ouviu suas próprias palavras, mas não ouviu mais nada até se
perceber deitada no chão, forrado de palha, ao lado de homens e
mulheres, deitados e espremidos, que ela desconhecia, e um homem se
apoiava firmemente nela, todo azul, com a boca aberta... Ánnuchka pôs-se
a gritar, sem pronunciar palavras inteligíveis. Alguém disse:

--- Enfermeiro, tire os mortos daqui, faz tempo que pedimos...

E de novo Ánnuchka perdeu a consciência. Quando novamente voltou a si,
estava deitava do mesmo jeito, no mesmo quarto, mas não no chão, e sim
numa cama. Ela desatou no choro e chorou até ver sua mãe, que estava
deitada perto da parede oposta. Cada vez que Ánnuchka caía em si, só se
acalmava e parava de chorar ao avistar sua mãe. No entanto, uma vez
Ánnuchka viu colocarem sua mãe numa maca e a levarem dali. A menina se
desfez em lágrimas e lhe explicaram:

--- Sua mãe está sendo transferida para outro quarto... Aqui só ficam os
doentes de tifo, não de disenteria...

--- Onde estou? --- perguntou Ánnuchka.

--- Num hospital --- explicaram.

--- Que aldeia é essa?

--- Não é uma aldeia, mas uma cidade --- disseram ---, chama-se
Pogoriéloie Gorodische.

Ánnuchka escutou esse nome e com ele dormiu ou desmaiou, era-lhe
impossível discernir. Voltou a si quando a colocavam numa maca.

--- Para onde estão me levando? --- perguntou Ánnuchka.

--- Você será transferida para outro hospital --- disse o auxiliar de
enfermagem ---, aqui perto. São dezoito quilômetros.

E carregaram Ánnuchka passando pelo quarto onde sua mãe estava. Ao
vê-la, a menina começou a chorar e a implorar:

--- Coloquem-me com minha mãe...

A mãe respondeu:

--- Não tenha medo, filhinha, logo irei buscar você.

Levaram Ánnuchka.

Ánnuchka ficou muito tempo internada no novo hospital, mas pouco se
lembrava desse período. Lembrava-se somente do dia em que lhe deram
alta. Era outono e, na sombra, já havia geada. Ánnuchka vestia seu
sobretudo de inverno, forrado de algodão, mas estava descalça. Para
aquecer os pés descobertos, era necessário andar rápido, mas para isso
ela não tinha forças. Caminhando pela rua, Ánnuchka se avizinhou de um
garoto.

--- Para onde você vai?

--- Para Pogoriéloie Gorodische --- respondeu ele. --- Eu sou de lá.

Ánnuchka se alegrou.

--- Quero ir com você, preciso ir lá.

--- Vamos --- disse o garoto ---, eu conheço o caminho... Até a floresta
são seis quilômetros e da floresta à cidade mais doze...

Caminharam o dia inteiro e chegaram à floresta, a seis quilômetros dali.
A estrada que a atravessava era feita de troncos, que estavam cobertos
de lama suja e fria. Ánnuchka pisou com os pés descalços na lama fria
sobre os troncos e pensou: ``Não vou conseguir''. No entanto, continuou
andando. ``Consigo chegar até aquela árvore abatida, mas não posso ir
adiante.'' Mas, ao chegar até a árvore, continuou em frente. Embora
andasse, ela compreendia: ``Um pouco mais e meu corpo vai enrijecer de
vez, mesmo com o sobretudo de inverno, e os pés parecem que não são
meus, não entendo como ainda me levam''.

De repente ela ouviu uma carroça se aproximar. Ao notar que Ánnuchka
estava descalça, o condutor parou os cavalos, desceu e a fez sentar. Ele
não pôde dar um lugar ao garoto que a acompanhava, porque a carroça
estava entulhada de caixas, mas o escoltou. Assim, à noite, chegaram a
Pogoriéloie Gorodische.

Lá Ánnuchka aproximou-se de uma patrulha militar, que lhe indicou o
caminho para o hospital. Ao chegar, ela perguntou às pessoas que
encontrou:

--- Procuro Emeliánova... Eu sou a filha dela...

Uma mulher disse à outra:

--- Emeliánova está mal...

No entanto, Ánnuchka não quis entender que sua mãe estava mal, só
entendeu que estava viva. Entrou no quarto e viu sua mãe no mesmo lugar
em que a havia deixado, usando o mesmo casaco e o mesmo xale... Ánnuchka
se aproximou e não a reconheceu. De longe a havia reconhecido, mas, de
perto, não. Era como se fosse ela e ao mesmo tempo não fosse. A mãe,
porém, reconheceu a filha de imediato e disse:

--- Não consegui buscar você, filhinha, como prometi, mas logo irei...

A enfermeira disse:

--- Menina, vá para a Casa do Camponês, ali poderá passar a noite.

A patrulha militar indicou o caminho da Casa do Camponês à Ánnuchka,
que, ao chegar lá, conseguiu permissão para pernoitar. Ela estava tão
cansada que logo adormeceu, no chão perto da estufa. Quando acordou, já
tinha amanhecido. Ao seu lado, postava-se um soldado, que lhe perguntou:

--- De onde você vem, menina?

--- Da aldeia de Agárkovo --- respondeu Ánnuchka.

--- Então vá até o comandante\footnote{Na Rússia, alguns administradores
  são chamados ``comandantes''.} --- disse o soldado ---, ele dará um
papel para que você consiga carona em qualquer carro.

E o soldado deu um pedaço de pão à Ánnuchka. Ela o comeu e foi até o
local indicado. Entrou na casa e se dirigiu aos militares --- ela não
tinha medo deles, porque, quando morava perto do aeródromo, em Rjév,
acostumara-se a vê-los. Ánnuchka aproximou-se e algum chefe lhe deu um
papel para que conseguisse transporte gratuito em qualquer carro. Então
ela voltou para o hospital, onde lhe disseram:

--- Emeliánova está melhor.

Ánnuchka mostrou o papel à sua mãe e esta lhe disse:

--- Você, filhinha, é muito inteligente... Agora vá para casa, na
floresta, pois Mítia está sozinho... Eu logo vou melhorar e também irei
até o comandante pegar um papel igual, e voltarei...

Ánnuchka foi até a estrada, mas por longo tempo nenhum carro parou,
então ela mostrou seu papel a uns guardas de trânsito e eles a colocaram
em um veículo. Ela chegou à aldeia e encontrou o lugar na floresta onde
os aldeões se instalaram. Viu a choupana de abeto totalmente desfolhada
e suas coisas jogadas no chão, todas molhadas, e ninguém se aproximava.

--- As coisas de vocês estão contaminadas com tifo --- explicaram ---,
ninguém precisa vigiá-las; os piolhos é que vigiam.

--- E onde está meu irmão? --- perguntou Ánnuchka.

--- Seu irmão chorou por três dias --- disseram ---, depois foi até a
casa dos militares.

Assim, Ánnuchka não encontrou seu irmão.

Entretanto, no inverno, todos voltaram aos seus abrigos subterrâneos,
situados nas redondezas da aldeia destruída de Agárkovo. E ela foi ao
abrigo da prima de sua mãe, que, mesmo sem vontade, a acolheu. Ánnuchka
pensava: ``Quando mamãe voltar, vai me achar mais rapidamente aqui''.
Mas, um dia, a prima lhe disse:

--- Sua mãe morreu...

``Como ela pode dizer isso,'' pensava Ánnuchka, ``se aqui não há nem
correio nem telefone?'' Mesmo assim, Ánnuchka resolveu partir, achou a
estrada que levava a Pogoriéloie Gorodische e se foi.

Quando chegou ao hospital, não a deixaram entrar --- era demasiado cedo.
Ánnuchka sentou-se na entrada, enrodilhou-se como uma rosca sob o
friozinho da manhã e esperou. Uma enfermeira lhe deu esperanças:

--- Emeliánova --- disse ---, acho que vi uma.

E, remexendo numa gaveta com documentos, encontrou um papel:

--- Sua mãe morreu no dia 7 de outubro de 1942.

E já era dia 13 de outubro... Ánnuchka voltou para casa, na floresta, de
mãos vazias... A floresta já estava coberta de neve e não havia nenhum
civil. Atordoada, Ánnuchka esquecera que todos da aldeia haviam ido para
os abrigos subterrâneos. Ela vagou muito tempo pela mata, sem gritar nem
pedir ajuda, caminhando calmamente, sem palavras. Um soldado a encontrou
e a levou até os abrigos. Ánnuchka se acomodou de qualquer jeito ---
estava muito apertado ali, havia duas ou três famílias em cada abrigo
--- e dormiu, muito cansada e amargurada. De manhã, foi despertada por
conversas ao redor e saiu: um frio gélido, neve, vento. No entanto,
Ánnuchka agora vestia as botas de sua mãe. Apesar de muito grandes, as
botas esquentavam os pés se eles fossem envolvidos em trapos. Ela viu
por perto uma carroça militar estacionada recolhendo todos os
habitantes. Alguém disse:

--- Vão pegar o trem em Pogoriéloie Gorodische, estão todos evacuando,
porque os alemães avançam de novo.

Recolheram também Ánnuchka, levaram-na a Pogoriéloie Gorodische e a
colocaram num trem. Ela não sabia quanto tempo havia viajado, perdeu-se
em devaneios, lamentando a morte de sua mãe. De repente, como em um
sonho, começou um bombardeio. Em volta tudo eram chamas e tiros. O povo
corria sem saber para onde. E Ánnuchka também fugiu... A noite estava
iluminada como o dia por causa dos incêndios, e seria fácil achar um
caminho se as paragens fossem familiares. No entanto, elas não eram, e
em toda parte Ánnuchka só via o desconhecido. Ela entrou correndo numa
casa que estava totalmente intacta, mas não tinha teto. Lá havia uma
estufa, também intacta, com um ícone em cima. Então ela saiu a toda e,
passando por uma estrada, chegou a uma grande casa repleta de mulheres.
É agradável se andar sozinho por lugares familiares, mas em lugares
estranhos o melhor é ser conduzido. Uma das mulheres conduziu Ánnuchka a
algum canto. Era manhã e tudo estava em silêncio, somente a neve caía.
Um homem apareceu e assustou Ánnuchka, porque ele mantinha sua mão
direita sempre fechada. Depois ela soube que era Kuzmin, o diretor do
orfanato, um ferido de guerra --- os dedos de sua mão direita haviam
sido retorcidos por uma explosão, permanecendo o tempo todo retraídos na
palma. Kuzmin pegou Ánnuchka pela mão com sua mão esquerda e a levou
para um recinto quente onde se apinhavam meninos e meninas vestidos
iguais, com os trajes do orfanato. Muitos meninos, especialmente os mais
novos, usavam vestidos como meninas, porque não havia roupas suficientes
para eles. Assim que Ánnuchka viu as crianças, soube que iriam caçoar
dela, pois todas olhavam para ela com olhos risonhos, como em Rjév,
antes da guerra.

Cada orfanato, assim como cada família, possui suas regras. Nesse
orfanato, havia sido estabelecido, tempos atrás, que todos eram
provocadores e se esforçavam por ser alegrar. Para Ánnuchka foi
inventado rapidamente um apelido, ``chorona'', porque às vezes,
encolhida num canto, ela chorava por sua mãe e por Mítia... Um dia, uma
menina moreninha de nome Sulamita a seguiu furtivamente e inventou o
apelido, depois do qual a vida de Ánnuchka nunca mais teve sossego.

Havia um motivo para essa menina ter se esforçado tanto em inventar um
nome ofensivo para a novata: antes da chegada de Ánnuchka, adultos e
crianças, para provocar, chamavam Sulamita de ``judia''. No início, ela
era a ``moscovita de nariz empinado'', já que vinha de Moscou, depois
virou a ``judia'', pois ela não pronunciava direito a letra ``r''. Assim
que se perdeu dos pais, Mita, ou Sulamita, foi parar em um orfanato onde
ninguém a conhecia por ``judia'', mas nesse logo começaram a ofendê-la.
Certamente, Kuzmin não a provocava --- ele era novo ali e considerado um
estranho. As crianças não o respeitavam, mas gostavam da antiga
diretora, que agora era educadora, tia Kátetchka, também ferida de
guerra, toda encurvada... As crianças a consideravam uma mãe, pois ela
era animada. Quando Sulamita, enfurecida pelas provocações, gritava,
entre lágrimas, que fugiria dali para ir atrás de sua mãe, tia Kátetchka
respondia com um sorrisinho:

--- Aonde você vai? Se seus pais estivessem vivos, eles a encontrariam.
Os judeus não abandonam suas crianças...

E Sulamita compreendia que não tinha para onde ir. As crianças não
gostavam dela também por outro motivo: enquanto andava, ela sempre
estava à procura de alguma coisa no chão e não raro a achava, ora uma
maçã, ora uma moedinha, em troca da qual lhe davam algo para comer na
cozinha, e uma vez até encontrou um soldadinho de chumbo.

--- Essa judia é sortuda --- diziam ---, sempre acha alguma coisa.

Havia uma menina, bem branquinha, Gláchenka, que a própria mãe havia
levado para o orfanato. Gláchenka não queria ficar lá de jeito nenhum,
apesar de terem lhe dado uma bela maçã. Ela chorara e rasgara o vestido
da mãe. Então a levaram para uma sala e começaram a tocar piano. A
menina deixou-se levar pela música e, nesse meio-tempo, sua mãe foi
embora.

A bem da verdade, Gláchenka tentou fazer amizade com Sulamita, mas essa
não queria saber da outra. Gláchenka abraçava e beijava Sulamita,
dizendo:

--- Eu quero ser sua irmã... Por que você não quer brincar comigo? Nós
duas somos órfãs...

Sulamita respondia:

--- Minha mãe nunca me abandonaria. Ela é muito boa, tem cabelo
encaracolado, usa chapéu de palha e outros também. Em Moscou, ela
distribuía balas igualmente entre as crianças do jardim de infância. E
eu a amo muito, apesar de a chamarem de ``\emph{madame}'', porque ela
tem cabelo encaracolado, pinta os lábios e usa chapéus...

--- A minha mãe é má --- dizia Gláchenka e chorava.

Somente Gláchenka e Kuzmin não chamavam Sulamita de ``judia''. Mas ela
não gostava de Gláchenka e tinha medo de Kuzmin, como todas as outras
crianças. Por isso Sulamita se alegrou quando trouxeram Ánnuchka ao
orfanato. Ela seguiu a novata e a apelidou de ``chorona''. Desde então,
as provocações à Sulamita diminuíram, porque passaram a caçoar
principalmente de Ánnuchka. Mas, um dia, os maiorais entre as crianças,
alegres e maldosos, foram provocar, como de costume, uma vizinha chamada
Fiokla.

Fiokla, uma velhinha seca e brava, morava sozinha numa pequena casa
perto do orfanato e, desde tempos imemoriais, talvez desde antes da
guerra, os maiorais zombavam dela.

--- \emph{Sviokla}! --- gritavam. --- Vovó \emph{Sviokla}...\footnote{\emph{Sviokla,}
  ``beterraba''. Fiokla é apelido de Fiénia.}

Em resposta, o cachorrinho ruivo de Fiokla latia, bravo, enquanto ela
mesma proferia xingamentos e ameaças, o que só aumentava a alegria da
molecada. Dessa vez, para agradar os maiorais, Sulamita também quis ir
provocar Fiokla.

--- Não vá --- pediu Gláchenka.

Mas Sulamita foi, e Ánnuchka também. Para agradar, Sulamita se aproximou
correndo da cerca, atrás da qual o cachorro ruivo quase tremia de tanta
raiva, e gritou:

--- Vovó \emph{Sviokla}!

Então a velhinha brava meteu-se para fora e, vendo Sulamita perto da
cerca, disse:

--- E você é uma judia, uma \emph{jidovka}...\footnote{Forma feminina de
  \emph{jid}, termo pejorativo para se referir aos judeus.}

E as crianças maldosas pararam de rir de Fiokla e voltaram a caçoar de
Sulamita. E Ánnuchka, que Sulamita um dia furtivamente seguira, disse:

--- \emph{Jüdisches Schwein}, em alemão, quer dizer ``porco judeu''.

--- Então você sabe falar alemão? --- perguntou Kóstia, a quem cada
criança dava uma fatia de pão da própria ração para não apanhar.

--- Eu sei --- respondeu Ánnuchka, querendo lhe agradar. --- \emph{Anna
mit Grossvater gehen spazieren}... ``Anna está passeando com seu
avô''...

--- Fascista, fascista! --- gritou Kóstia. --- Alemã, alemã!

E as crianças começaram a gritar:

--- Alemã, alemã! Fascista, fascista!

Desde então, passaram a chamar, com empenho especial, Sulamita de
``judia'' e Ánnuchka de ``alemã'' e ``fascista'', e graças a isso elas
começaram a se odiar.

Enquanto isso, Kuzmin, que havia viajado, voltou preocupado.

--- Os alemães estão por perto --- disse ele ---, eu organizei a vinda
de carros, está na hora de nos prepararmos para a evacuação.

Passou um dia, passou outro, mas não veio carro nenhum, e já se podia
ouvir nitididamente o bombardeio. Haviam atingido a estação, mas o
orfanato ainda estava em segurança. Kuzmin chamou Kátetchka e disse:

--- Não podemos esperar mais, vamos partir a pé... Traga as listas das
crianças, que eu vou destruí-las, pois os alemães estão atrás de
crianças judias...

Kátetchka disse:

--- O que é isso? Todos devem sofrer por causa de uma judia? Se destruir
as listas, depois será impossível localizar as crianças...

Kuzmin retrucou:

--- É uma ordem!

Kátetchka disse:

--- Aqui não é o exército nem o \emph{front} para dar ordens.

Então Kuzmin bateu na mesa com a mão que nunca se abria, e Kátetchka
trouxe as listas.

Kuzmin mandou enfileirar as crianças aos pares e de mãos dadas. Calhou
que Ánnuchka e Sulamita ficaram lado a lado, mas ambas temiam
desobedecer a Kuzmin e não reclamaram. As crianças rumavam à estação
quando, de repente, viram ao longe carros se aproximarem de lá.

--- São carros alemães --- disse Kuzmin ---, eu os conheço do
\emph{front}... Vamos mudar o itinerário, vamos para as aldeias mais
afastadas.

Andaram muito tempo. Kuzmin e Kátetchka carregavam nos braços as
crianças menores. Carregavam ora uma, ora outra, e assim chegaram à vila
de Brussiány.

Os habitantes das aldeias vizinhas costumavam ir à vila de Brussiány por
causa da feira. E esse dia era justamente o da feira. Kuzmin ficou
contente, certificou-se que ali não havia alemães, alinhou as crianças
na praça da feira, entre as carroças, e disse:

--- Camaradas camponeses... Aqui diante dos senhores se acham irmãos e
irmãs do orfanato. Peço que cada um fique com aquele que mais lhe
agradar, senão as crianças não vão sobreviver.

Os camponeses se aproximaram e começaram a examinar e a escolher as
crianças. Primeiro foram escolhidas as mais fortes e espertas, porque
poderiam ajudar em casa e também ser utilizadas no trabalho. Depois,
tendo sobrado as menores e as mais fracas, foram selecionadas conforme o
gosto de cada um. Quando escolheram Gláchenka, esta implorou à aldeã que
levasse também Sulamita. No entanto, a mulher percebeu que Sulamita era
judia e se recusou. Gláchenka desfez-se em lágrimas, abraçou Sulamita e
disse que nunca a esqueceria. Mas Sulamita não prestava atenção nela,
tinha outra preocupação em mente: quem iria levá-la? Quase todas as
crianças haviam sido escolhidas, sobraram apenas Sulamita, Ánnuchka e um
menininho franzino, que ficaram com Kuzmin, porque Kátetchka,
tranquilizada depois de ter colocado seus peraltas favoritos em boas
mãos, arranjou para si mesma um trabalho com um velho camponês. Também,
quase não restavam boas mãos, em volta agora havia apenas maltrapilhos,
talvez desabrigados, que só tinham ficado lá por curiosidade. De
repente, Ánnucka viu uma possível mãe adotiva se aproximar, com a qual
qualquer órfão teria sonhado. Roupas limpas, olhos bondosos, e um lenço
camponês cuidadosamente atado à cabeça. Nem uma mãe de verdade ganharia
dessa. Ánnuchka pensava: ``Ela está vindo atrás de mim. Ela não pegaria
o menino, ele é franzino e sem graça, nem Sulamita, que é judia''. A
mulher aproximou-se, olhou para as crianças, tirou subitamente do
pescoço uma pequena cruz de cobre e colocou-a no pescoço de Sulamita. E
a menina abraçou a mãe bondosa que a escolheu.

--- Mamãe --- disse ---, obrigada por me escolher como filha...

Ánnuchka sentiu um aperto no coração, de ciúme e tristeza. A doença
havia lhe tirado o amor de sua mãe verdadeira, mas quem lhe tirou o amor
da mãe adotiva foi a judia que, no orfanato, lhe inventara um apelido
ofensivo depois de vê-la, às escondidas, chorar pela mãe morta. Ánnuchka
sentia um grande peso no coração, mas quem, na contrariedade, conserva o
senso prático da infância é capaz de cometer grandes maldades. E
Ánnuchka desejou a morte de Sulamita, para que a mãe adotiva bondosa
ficasse com ela.

Mas será que ela desejou algo difícil? Seria difícil providenciar a
morte de uma menina judia em 1942, sob domínio alemão? Bastou Ánnuchka
desejá-lo do fundo do coração e autoridades alemãs apareceram na praça
da feira da vila de Brussiány. Entre elas Ánnuchka reconheceu tio Hans,
o homem lhe dava pão e caldo concentrado de ervilha --- os eslavos ainda
não estavam sujeitos ao extermínio completo.

--- Tio Hans --- gritou ela, alegre ---, \emph{Anna mit Grossvater gehen
spazieren}... \emph{Vogel,} ``pássaro''; \emph{Hund,} ``cachorro''...

Hans reconheceu em Ánnuchka a menina em cuja casa morara em Rjév e em
Sulamita uma judia, que, conforme as últimas regras de vida alemãs neste
planeta, não deveria viver em parte alguma. A máquina nacional alemã
trabalhava com precisão e distinção. Transferiram Kuzmin para um campo
de prisioneiros; golpearam com coronhas o rosto da camponesa, que se
cobriu de sangue; retiraram Sulamita da zona destinada à raça eslava da
vila de Brussiány, mataram-na e jogaram seu corpo numa vala; e colocaram
Ánnuchka num vagão de carga, para que ela aprendesse na Alemanha a
cultura alemã e o trabalho alemão.

Dã, a Áspide, o Anticristo viu tudo isso --- fazia tempo que ele andava
pela terra devastada, revestida de sangue, e nesse dia se achava na
praça de Brussiány. Ele viu sangue fresco e ossos secos do ano anterior.
Em dois anos o cabelo de Anticristo, o jovem judeu da tribo de Dã, havia
embranquecido. Ele não fora enviado como executor, mas somente como
testemunha do Senhor...

Ele andava entre resignados e revoltados, entre os que suspiravam pela
vida da qual eram expulsos e os que tiveram a sorte de esquecer a vida
antes de morrer. Mas, certa vez, perto de Minsk, ele se viu ao lado de
um sujeito da tribo de Efraim, pois Dã sabia a que tribos as pessoas
pertenciam, mesmo que elas mesmas ignorassem. E esse homem, sábio e
filósofo, caminhando para o túmulo, disse com pudor e afobação:

--- Nosso povo deveria ter ido embora há muito tempo; somos como uma
visita abusada que ficou tempo demasiado na casa dos outros povos e
agora é posta para fora à força e com humilhação... Nós, judeus, somos
um povo ruim e eu sinto nojo de mim mesmo...

Dã, a Áspide, o Anticristo, olhou à sua volta e realmente não viu muitos
rostos de justos no seu povo, que caminhava para o túmulo... Alguém
cometia adultério, outro ofendia um órfão, esse era sovina e engolia
vivo seu próximo, aquele filosofava de modo indecente, esse rezava com
falsidade, essa traía, aquele abjurava... E Dã, a Áspide, o Anticristo,
disse:

--- Quem está nos expulsando e de onde? Por acaso o Senhor nos expulsa
do Éden? Por acaso os santos anjos nos expulsam do céu? Não, nós,
pecadores, somos expulsos por pecadores decaídos de um mundo decaído...
Olhemos em volta. O adultério não seria um pecado deste mundo? E a
traição? A filosofia indecente? A reza dissimulada? Em disputas entre
irmãos, matamos nossos próprios profetas, de Jeremias a Jesus, mas será
que isso é algo raro neste mundo? Quantos libelos de sangue, quantas
lendas maldosas podem ser criadas sobre outras nações que destruíram
seus próprios justos em disputas entre irmãos? E quе culpa em particular
nos deve ser imposta? Por que antes de sermos colocados porta afora
deste mundo decaído, mas habitável, eles nos arrancam tudo e ficam com o
que temos de melhor? Quando estivermos em outro mundo, iremos adquirir
novamente um destino histórico e outros bens.

O Senhor respondeu a seu enviado, o Anticristo, nesse dia de outono,
perto de Minsk, à beira de uma trincheira antitanque coberta pelo sangue
das doze tribos de Israel:

--- Uma culpa em particular vos é imputada, é a única culpa verdadeira
neste mundo decaído, mas habitável; é ela que vos distingue dos outros
povos e é por ela que vós sofreis o castigo, pois é a única culpa que
vos distingue... A única culpa verdadeira... E o nome dela é Fraqueza...
Somente disso vós sois culpados para com os outros povos, somente nisso
consiste vosso pecado contra Mim. Enquanto essa culpa em particular
existir, Eu perdoarei todos os vossos pecados. Quando vos redimirdes
dessa terrível culpa, Eu vos cobrarei os outros pecados. Já aos vossos
perseguidores, através dos quais vos castigo, cobrarei sete vezes mais,
farei com que paguem até o fim, pois a punição do Senhor jamais se
realiza através dos justos, mas sempre através dos ímpios.

E Dã, a Áspide, o Anticristo, disse a seu povo, através do profeta
Jeremias:

--- Não tenhas medo, meu servo Jacó {[}...{]} --- disse o Senhor. --- Eu
não te destruirei, apenas te castigarei. Não te deixarei
impune.\footnote{Jeremias 30:10, 11.}

Depois disso, Dã retornou a Rjév, aonde tinha sido enviado para
encontrar Ánnuchka, a mártir ímpia, após ter encontrado Maria, a jovem e
bondosa prostituta que tivera, num hospital dа prisão, o primeiro filho
de Anticristo, chamado por ela de Vássia em homenagem ao irmão que
perdera... Não encontrando Ánnuchka em Rjév, ele dirigiu-se à Brussiány,
onde ela causou a morte de Sulamita, \footnote{Sulamita, ``a mais bela
  jovem de todo o Israel'', aparece no \emph{Cantigo dos cânticos,} no
  Velho Testamento.} da tribo de Manassés, que não estava fadada a ser
incluída no Resto,\footnote{O termo bíblico \emph{o resto} ``{[}...{]}
  significa o Israel que sobrevive depois da conquista {[}...{]}'' e
  também adquiriu, nos livros proféticos, sentido messiânico, e Israel
  seria ``o único a quem as promessas da restauração messiânica são
  dirigidas''. (\emph{Dicionário bíblico}, Paulus, 1983, p. 794)} a
deixar um descendente...

Foi dito pelo profeta Isaías: ``Pois, ainda que teu povo, Israel, seja
tão numeroso como a areia do mar, apenas um Resto dele voltará; а
destruição está decidida, transbordará justiça''.\footnote{Isaías 10:22.}

A justiça, que transbordava dessa destruição, se devia a terrível culpa
do povo no mundo: a Fraqueza. Já aqueles que destruíam, repetindo sete
vezes a soberba assíria, diziam:

--- Fiz isso com a força de minha mão e com minha sabedoria, pois sou
inteligente. Mostro os limites dos povos e dilapido seus tesouros.

Anticristo respondeu a si mesmo, através do profeta Isaías:

--- Por acaso o machado irá se louvar diante de quem o ergue? Ou a serra
se vangloriar diante de quem a maneja?\footnote{Isaías 10:15.}

Então Dã, o Anticristo, falou:

--- Por vossa terrível impiedade, o Senhor vos escolheu como um
instrumento de punição contra seu próprio povo, por culpa dele. Há povos
ímpios e há uma terra ímpia. Ao partirem, os povos ímpios levam sua
maldição, e a terra se purifica. Mas uma terra maldita é imóvel, e tudo
o que provém dela é amaldiçoado pela eternidade. Do povo da terra
amaldiçoada não haverá nem resto nem descendente, como acontecera à
Babilônia, sete vezes menos pecadora. Foi dito no livro do profeta
Jeremias: ``Jeremias escreveu em um livro todas as desgraças que
deveriam se voltar contra Babilônia, todas essas palavras se referiam à
Babilônia''.\footnote{Jeremias 51:60.}

O Senhor envia o Cristo aos povos pecadores para a Bênção e o Anticristo
para a Maldição, enquanto os grandes profetas são enviados para
interpretar os trabalhos do Senhor, mas pisar em terra impura não é
permitido nem a Cristo nem a Anticristo, nem mesmo aos eleitos entre os
profetas. Por essa razão, Jeremias não levou pessoalmente o livro da
Maldição à Babilônia, mas o entregou aos que eram conduzidos à
escravidão. ``E Jeremias disse a Saraías: `Quando tu chegares à
Babilônia, atenção, lê todas estas palavras e diz: Senhor, tu falaste
deste lugar que irias destruir até que não restasse nem homem nem
animal, tornando-o um deserto eterno. E, quando tu acabares a leitura
deste livro, prenderás uma pedra nele e o lançarás em meio ao Eufrates,
dizendo: Babilônia afundará e não se levantará mais, graças ao mal que
lançarei sobre ela, e todos eles irão desfalecer'''.\footnote{Jeremias
  51:61\emph{--}64. O episódio da queda da Babilônia (ao norte da
  Assíria), profetizado em detalhes em várias passagens bíblicas,
  aparece descrito em Isaías 47: 1-15 e no Apocalipse 18.}

Dã, a Áspide, o Anticristo, sabia que precisava amaldiçoar, mas аpenas o
Senhor sabia como e quando a Maldição aconteceria. Porém, para realizar
o ritual da Maldição, Dã precisava de um pecador que, em seu sofrimento,
fosse conduzido à escravidão, pois a Anticristo, o enviado do Senhor,
assim como a Cristo, não era permitido pisar em terra impura. Dã sabia
que em seu povo havia muitos ímpios e muitos pecadores; no entanto, os
soberanos dе terras impuras, que resolveram distribuir benesses terrenas
no lugar do Senhor, consideravam a escravidão demasiado apetitosa para
um judeu, pois um escravo pode dormir em um estábulo e comer detritos,
conseguindo levar sua vida. Isso contradizia as instruções de Martin
Bormann,\footnote{Martin Bormann (1900\emph{--}1945), alto oficial da
  Alemanha nazista, secretário pessoal de Adolph Hitler.} um apreciador
da raça ariana, um dos deuses supremos do paganismo nazialemão: ``Os
eslavos, neste mundo, serão escravos dos arianos; enquanto os judeus,
que são animais, não têm direito de existir''. Por isso o Anticristo
precisou achar ímpios sofredores em outras nações, aos quais as benesses
da escravidão alemã eram acessíveis.

Dã, a Áspide, o Anticristo, saiu da vila de Brussiány e foi até a
estação onde eslavos eram enviados, em vagões de carga, para a Alemanha,
para se iniciarem na cultura alemã e no trabalho alemão.

Era um típico dia do norte, ``frio gélido e sol'',\footnote{Primeiro
  verso do poema ``Manhã de inverno'' (\emph{Zímneie utro,} 1829), de
  Púchkin.} como escreveu Púchkin; um dia magnífico, brilhante. Qualquer
um que ainda duvidasse que a natureza fosse desprovida de alma se
convenceria, nesse dia, de que a natureza é uma esposa bonita, mas
infiel. Na alegria e na sorte, ela está pronta para esbanjar suas
carícias e suas formosuras, mas, na desgraça, abandona-o de imediato,
juntando-se aos assassinos perto dos túmulos sangrentos, olhando
impassível para os corpos frios que havia pouco ela agradava com as
hortaliças de seu mato, com o cheiro picante de suas folhas outonais e
com o ar de seus abetos nevados... Como um butim dos que foram
destruídos, aos assassinos cabem a beleza, a generosidade, as carícias e
os deleites propiciados pela natureza, mas não lhes cabe o Senhor. Por
isso Abraão, o Patriarca, adorava apenas o Senhor, mas não as estrelas,
que conduzem para o pântano do fatalismo, nem o sol, que desperta a
beleza material, nem a lua, que desperta a beleza mística, nem a
juventude efêmera das plantas, nem a velhice eterna das pedras, nem o
céu infinito, nem a água indiferente. À noite, o Senhor disse a Abraão
por meio de uma visão:

--- Não temas, Abraão, Eu sou o teu escudo, e tua recompensa será enorme
{[}...{]}\footnote{Gênesis 15:1.}

Desde então, Abraão acreditou no Senhor, mas não na natureza do Senhor,
como acreditavam os pagãos, pois é sabido que o Senhor está na natureza,
mas não é a natureza. Como o homem, a natureza é dominada pela soberba;
como o homem, às vezes ela se insurge contra seu Pai, mostrando-se ímpia
tanto em sua perversidade como em sua beleza...

Assim, a natureza perto da vila de Brussiány, nesse momento, se mostou
ímpia diante do corpo castigado da judia Sulamita, da tribo de Manassés,
que não estava fadada a receber uma semente em seu ventre ainda quente,
a ser incluída no Resto, a deixar um descendente... Ao longe, o sol puro
e gelado no céu fresco do Norte caía calmamente sobre a floresta nevada
e cintilante, num esplendor indescritível, e, se os raios dе sol do Sul,
especialmente no verão, são palpáveis devido ao calor que contêm, ou
seja, não completamente puros, os raios do Norte são extraordinariamente
leves e puros, quase irreais. Essa pureza gélida não seria a origem da
angústia glacial e silenciosa das paixões nórdicas?... Eis que, em meio
a esse esplendor gélido e à luz impoderável do sol, o corpo de Sulamita,
da tribo de Manassés, jazia numa vala, congelado com o próprio sangue,
sangue que fora trazido às suas veias ao longo de muitas gerações, desde
o próprio Abraão, que fizera aliança com o Senhor. Anticristo não se
deteve muito sobre o corpo de Sulamita, da tribo de Manassés, pois ela
ainda estava quente, sua memória estava fresca --- a camponesa bondosa,
agora deitada numa estufa, com o rosto deformado por coronhas alemãs,
ainda se lembrava claramente dela, chorando e lamentando, e, em um vagão
de carga, também se lembrava dela Ánnuchka, que, movida por suas
contradições infantis, havia desejado a morte dela, mas não pensava nela
com a tristeza da camponesa, e sim com pavor, como, nos primeiros
tempos, pensara em seu irmão Vova, que morrera em Rjév devido a um
temporal.

Dã, o Anticristo, conhecia o grande mandamento bíblico: deixem que os
mortos enterrem os mortos. Enquanto a memória do morto estiver fresca,
ainda não de todo esquecida, ainda palpável, enquanto os outros mortos
não a sepultarem decentemente, é permitido apenas rememorar o morto, mas
não falar dele, pois ele ainda é humano, não divino.

O Anticristo passou pelo corpo de Sulamita com calma e tristeza, como se
passasse diante do caixão de um estranho, de alguém distante. Afastou-se
da vila de Brussiány, rumando para onde o dia frio e ensolarado do norte
russo, mostrando-se ímpio, se amotinava contra os sentimentos do Senhor.
E o Anticristo viu muitos ossos humanos. Eram das pessoas que foram
assassinadas ali no ano anterior, nos jazigos de granito, as quais os
outros mortos tiveram tempo de enterrar. Nos campos, porém, também não
se encontravam poucos ossos, pois se acumulavam de vários lugares: Rjév,
Pogoriéloie Gorodische e Zubtsóv, e eram trazidos em vagões-plataforma
--- através das ferrovias de bitola estreita instaladas antes da guerra
--- da estação até os jazigos de granito. Para fuzilarem muitas pessoas
por vez, juntavam migalhas de todа parte; mesmo assim, isso ainda não se
comparava aos grandes fuzilamentos do Sul... No entanto, os fuzilamentos
do Norte, juntando pessoas de toda parte, eram implacavelmente
categóricos е escrupulosos. A essa altura, já havia sido editada a
circular alemã secreta sobre o trabalho insatisfatório dos
\emph{Einsatzgruppen}:\footnote{\emph{Einsatzgruppen,} ``grupos de
  intervenção'': unidades encarregadas do assassinato dos opositores do
  Reich. Em alemão russificado no original.} ``Os numerosos fuzilamentos
dos judeus não provocariam em si objeções se, durante sua preparação e
execução, não fossem cometidas falhas técnicas. Alguns, por exemplo,
abandonavam os corpos, insepultos, no próprio lugar de fuzilamento''.
Era a circular nº 25, datada de 25 de julho de 1942. Agora era inverno
de 1942; no entanto, as infrações e as falhas técnicas ainda não tinham
sido eliminadas. Foi justamente uma falha técnica alemã como essa que
surgiu diante de Dã, o Anticristo, num campo perto da vila de Brussiány.

O Anticristo olhou em volta e, de repente, sentiu em seu ombro a mão do
Senhor, e lhe aconteceu o mesmo que acontecera ao profeta do exílio
Ezequiel: Dã, como Ezequiel, conversou com o Senhor.

``A mão do Senhor estava sobre mim e me conduziu em espírito,
colocando-me no meio dе um vale repleto de ossos. E me fez andar ao
redor deles, e havia muitos ossos sobre a superfície do vale, e estavam
completamente secos. Ele me disse: Filho do homem! Esses ossos irão
reviver? Eu disse: Senhor Deus, Tu sabes disso. E Ele me disse: Anuncia
uma profecia sobre esses ossos, dizendo a eles: `Ossos secos! Escutai a
palavra do Senhor'. Assim dizia o Senhor Deus aos ossos: `Eu farei o
espírito entrar em vós e vós revivereis. Eu vos cobrirei de veias,
criarei carne em vós e vos revestirei de pele, e farei o espírito entrar
em vós, e vós revivereis e sabereis que Eu sou o Senhor.''\footnote{Ezequiel
  37: 1\emph{--}5.}

O Anticristo viu, num campo coberto de neve, os ossos se aproximarem, e
cada osso, embora lançado longe um do outro, achou seu par, e um ruído
se elevou, então veias surgiram sobre os ossos, a carne cresceu neles, e
a pele os cobriu, e tudo ganhou a forma de uma multidão de mortos
recém-enterrados, que se postavam sob o alegre sol do Norte como ídolos
tristes. É sabido que, quando um morto enfurecido aparece e deseja
zombar dos vivos, a primeira coisa que faz é dançar, pois a dança dos
mortos é especialmente temida pelos vivos. Mas ali havia algo de
diferente, já que esses mortos mártires eram tristes e se postavam
imóveis, assim como imóvel jazia, perto de uma vala, o corpo castigado,
congelado com o próprio sangue, da judia Sulamita, que, insepulta,
infringia as instruções sanitárias alemãs.

Então o Senhor disse a Dã, o Anticristo, através do profeta Ezequiel:

--- Anuncia uma profecia ao espírito, filho do homem, anuncia: ``Assim
diz o Senhor Deus: Espírito, vem dos quatro ventos e sopra sobre esses
mortos, e eles irão reviver''.\footnote{Ezequiel 37:9.}

E, como o profeta Ezequiel, Dã, a Áspide, o Anticristo, pronunciou a
profecia e os mortos renasceram, e eram muitos, um enorme batalhão. E o
Senhor disse a Dã, o Anticristo, por meio do profeta Ezequiel:

--- Esses ossos são toda a casa de Israel. Eis o que eles dizem:
``Nossos ossos secaram, nossa esperança morreu: fomos cortados pela
raiz''. Por isso, anuncia a profecia a eles: ``Assim diz o Senhor Deus:
Eu abrirei vossos túmulos e vos farei sair, povo Meu, e vos conduzirei à
terra de Israel. E vós sabereis que Eu sou o Senhor quando Eu abrir
vossos túmulos e vos fizer sair {[}...{]}''.\footnote{Ezequiel 37:
  11\emph{--}13.}

Depois disso, o Senhor tirou a mão do ombro de Dã, а Áspide, o
Anticristo, e os mortos mártires renascidos desfizeram-se novamente em
ossos sobre o campo nevado. Começou a anoitecer, a floresta escureceu e
o esplendor gélido do dia apagou-se. O Anticristo entendeu que isso era
um sinal. Ele precisava correr até a estação, antes que conduzissem à
escravidão, à terra impura, a mártir ímpia Ánnuchka, por cujas mãos ele
deveria enviar a Мaldição a essa mesma terra. Foi para isso que o
Anticristo havia sido enviado pelo Senhor a Rjév, para encontrar
Ánnuchka, a jovem mártir, depois de ter passado por Kertch, onde
encontrara Maria, a bondosa prostituta...

Quando o Anticristo chegou à estação, era noite, estava escuro, e os
lampiões, conforme as condições dos tempos de guerra, iluminavam
palidamente. Seguindo o ruído dos lamentos, o Anticristo achou, entre
tantos vagões, os vagões de carga dos escravos, embora o choro soasse de
modo abafado, pois as portas já tinham sido trancadas --- o trem estava
prestes a partir... Dã passou silenciosamente ao lado do vagão diante do
qual se aglomeravam os alemães que conduziriam os eslavos à escravidão.
Ele não andava silenciosamente por ter medo de ser morto pelos alemães,
pois isso era impossível, mas porque fazia tempo que ele era dominado
pelo desejo de matar um alemão. Na verdade, queria matar todos eles,
para deleitar-se inteiramente, no entanto isso seria uma felicidade
completa, e o Anticristo sabia que neste mundo não existe felicidade
completa. Então ele sonhava com uma pequena alegria --- matar pelo menos
um alemão. Mas o enviado do Senhor não pode determinar Seus desígnios. O
Anticristo sabia que o Senhor não tinha aprovado o profeta Eliseu, que
punira com a morte crianças ímpias e maldosas. O enviado do Senhor tem o
dever de executar apenas o que lhe é devido. Era por isso que o
Anticristo andava silenciosamente diante dos homens cuja morte desejava.

O Anticristo olhou para dentro de um dos vagões, cuja porta os alemães
precisaram abrir, e ele estava apinhado de pessoas, principalmente
mulheres, mas também adolescentes, jovens... Quando os alemães abriram a
porta, todos começaram a se empurrar para tomar um pouco de ar, e
Ánnuchka estava lá, de pé, espremida por corpos estranhos. O Anticristo
tirou de sua bolsa de pastor o pão impuro do exílio, legado pelo profeta
Ezequiel, e começou a distribuí-lo entre os eslavos escravos. Ele deu um
pedaço de pão à Ánnuchka, embrulhado com um rolo de papel, e disse:

--- Come o pão, mas esconde o papel na tua roupa. Quando tu chegares à
terra impura, lerás o que está escrito no papel, depois o prenderás numa
pedra e o lançarás ao rio dessa terra impura.

Ánnuchka olhou para quem lhe oferecia pão e, de repente, reconheceu nele
o homem que, antes da guerra, em Rjév, aparecera em sua casa, uma antiga
igreja, com a intenção de roubar. Ela se assustou e quis chamar o alemão
que havia se afastado para fazer suas necessidades, pois não havia outra
autoridade além dele. No entanto, Ánnuchka não teve tempo, pois uma
mulher que estava ao lado com uma criança nos braços disse de súbito ao
Anticristo:

--- Bom homem, pegue minha criança, porque eu estou morrendo de fome e
seu pão não durará muito tempo... Minha filha morrerá perante meus
olhos...

A mulher estendeu ao Anticristo uma criança envolvida em um cobertor
vermelho de algodão; percebendo-se em braços estranhos, o bebê começou a
chorar furiosamente. Então a desordem, até então despercebida, ficou
evidente. Essa desordem não era nada para os alemães, mas o que eles
viram era inconcebível para sua mente nórdica: iluminado pelas lanternas
das patrulhas alemãs, bem no centro da disposição militar, achava-se um
judeu com a barba por fazer, ainda em vida, respirando o ar gelado e
segurando uma criança, a qual, se conseguisse crescer e se enfiar por
uma fresta, se disfarçaria com a máscara de outra nação, e quem
conseguiria achá-la para exterminá-la?... Conforme a doutrina alemã ---
o organizado cérebro alemão sempre acredita em um materialismo idealista
--- e conforme a doutrina da separação das raças, eles não poderiam
supor que, nos braços de um judeu, estivesse uma criança eslava. O
ímpeto de caçador dos alemães uniu-se à indignação de senhorios
asseados. E surgiu uma alegria recíproca. Alegres, os alemães correram
por todos os lados com o intuito de matar o judeu: pela caixa d'água,
pela estação, pelos vagões vizinhos. Alegre, Dã, a Áspide, o Anticristo,
percebeu o quadro que se formara e pensou: ``Tenho nos braços uma
criança que é mortal e que o Senhor não me impediu de pegar. Por isso,
Ele me perdoará se eu antecipar, em parte, Suas intenções, como me
perdoou quando eu coloquei um obstáculo na frente de Pávlik, o
proletário de Rjév''.

Os alemães começaram a cair, agarrando a barriga, apertando as palmas
das mãos geladas contra os lábios mordidos devido a uma dor súbita,
soltando imundícies sangrentas pelas duas extremidades. Como se
estivesse sob o fogo de metralhadoras, todo o destacamento da guarda
tombou na plataforma coberta de neve, sobre sua própria diarreia. Após a
resolução da questão judia nas trincheiras antitanques de Minsk, após os
ossos secos encobertos pela neve perto de Brussiány, olhando para os
rostos genuinamente alemães, azuis por sufocamente, deformados por
asfixia, Dã, a Áspide, o Anticristo, entendeu o que era a felicidade
terrena...

Mais tarde, autoridades alemãs determinaram que o envenenamento do
destacamento dera-se pelo uso de conservas de má qualidade, e, para
completar, morreu um intendente militar alemão. Dessa maneira, a
quantidade total dos dolicocéfalos ficou ainda menor.

Como é de conhecimento geral, a dolicocefalia --- alongamento do crânio
--- constitui, conforme a doutrina alemã, um indício germânico. Já
Ánnuchka era uma típica braquicéfala, com o crânio redondo, eslavo, e,
por essa razão, ela agora cuidava de porcos num planalto xistoso da
região de Renânia do Norte-Vestfália... Seu patrão era um típico
dolicocéfalo, com o crânio germânico, o que, na opinião dele, mesmo
entre alemães, era um acontecimento raro, sendo um privilégio da zona
rural, pois nas cidades havia forte mistura de elementos morenos:
eslavos do Оeste, romanos, e, honestamente, até espécimes judeus, o que
criava um problema picante, já que o próprio \emph{Führer} --- psiu! ---
tinha cabelos pretos.

Muito mais tarde, no período do pós-guerra, o patrão dolicocéfalo de
Ánnuchka passou a afirmar que sempre fora antinazista e anti-hitlerista,
pois na cúpula do partido nazista predominavam braquicéfalos, de crânio
redondo, e o próprio Hitler não tinha um crânio puramente germânico,
além dos cabelos pretos. No entanto, na época em que Ánnuchka trabalhava
para esse senhorio, ele escondia sua revolta interior da Gestapo e se
esforçava para abastecer a mesa nacional alemã de pratos variados e
suculentos, incluindo patas de porco com repolho azedo... A criação de
porcos e o cultivo de repolho são tarefas árduas, e Ánnuchka, não
habituada ao trabalho alemão, sobre o qual lhe falara o bondoso tio
Hans, ficava extenuada, principalmente porque, se às vezes comia sobras
de repolho no almoço, nunca comia carne. E os outros braquicéfalos
também se esgotavam do trabalho alemão, sem conseguirem restabelecer
suas forças com as refeições alemãs.

Contudo, a região em que acomodaram a escravidão era bonita. As colinas,
que se elevavam suavemente, entremeavam-se com vales, entre os quais os
rios formavam uma sequência de sinuosidades graciosas. Em muitos
lugares, a superfície da terra que deveria ser amaldiçoada estava quase
totalmente coberta por florestas de folhas, onde cantavam pаssarinhos;
por jardins frutíferos, onde pendiam maçãs, peras e ameixas coradas; e
por campos de vinhas, de trigo e de cevada. Tudo isso exigia cuidados,
mas não havia dolicocéfalos habilidosos o suficiente, pois eles tinham
sido encarregados por \emph{Führer}, de cabelos pretos, de estabelecer
na terra de Deus a ordem alemã. Por isso, no período de colheita dos
frutos, para lá enviavam braquicéfalos preguiçosos e assustados. Em sua
maioria, eram jovens que desabrochavam em meio à escravidão e, mesmo com
a alimentação pobre, eram tomados de desejos, especialmente entre as
árvores frutíferas aromáticas.

Um dia, Ánnuchka carregava uma pesada cesta trançada com a ajuda de um
braquicéfalo de Kursk. O rapaz agradava a Ánnuchka. Ele tinha o nariz
arrebitado e os olhos cinzentos e assobiava alegres cançonetas alemãs.
Ao rir das cançonetas, Ánnuchka dava a entender que ele a agradava.
Quando voltavam pelo jardim, com a cesta vazia, depois de descarregarem
as maçãs no armazém, o braquicéfalo de Kursk, de olhos cinzentos, levou
Ánnuchka para uns arbustos e de súbito a agarrou, com força, respirando
pesadamente, como se carregasse de novo uma cesta cheia de maçãs;
derrubou Ánnuchka na grama, desuniu os joelhos dela com os seus e, com
seus lábios, tapou a boca dela. Assim, Ánnuchka repetiu o destino de
Maria, violada perto da cidade de Izium, na região de Khárkov, em 1933.
No entanto, em seguida tudo se deu de outra forma para Ánnuchka e para
seu violador. Ela foi violentada de dia e, à noite, queixou-se ao seu
patrão dolicocéfalo. O dolicocéfalo, que às vezes lia Goethe, não
gostava, como ele mesmo dizia, ``das mistificações de jovens'', por ele
ser um homem semiparalisado e sentir aversão por semelhantes distrações.
Dessa maneira, mandou castigar de forma exemplar o braquicéfalo de nariz
arrebitado, que foi espancado na delegacia de polícia. Mas, como um dos
policias calçava botas de couro de porco extremamente pesadas, com
ferraduras de ferro nas solas, o braquicéfalo de Kursk apanhou mais do
que exigia a justiça e morreu. Então o senhorio dolicocéfalo, que, como
sabemos, entretinha-se com Goethe, foi invadido por dúvidas, sobretudo
porque a relação com a mão de obra se mostrava complicada e, em geral, o
ano de 1944 não estava fácil para a agricultura alemã. Ele lamentou o
bom trabalhador que havia sido o braquicéfalo de Kursk e, zangado com
Ánnuchka, que о induzira a cometer uma injustiça, começou a castigá-la:
enviava-a aos trabalhos mais pesados do chiqueiro, mandava espancá-la
por qualquer falha, dava-lhe uma ração ainda pior que a ração de fome
reservada aos braquicéfalos, acusava-a de ser uma devassa. Por estar
totalmente em poder dele, Ánnuchka, um mês depois do ocorrido, no outono
de 1944, tinha a mesma aparência dos prisioneiros de guerra russos que
trabalhavam na extração de turfa e eram enterrados, perto do trabalho,
em solo pantanoso, onde, aliás, enterravam todos os braquicéfalos que
morriam ou eram assassinados. Ánnuchka sabia que para lá também fora
levado o jovem de nariz arrebitado e olhos cinzentos que a violentara
atrás de uns arbustos.

Certa vez, Ánnuchka estava deitada sobre seus trapos depois de um dia
especialmente árduo --- ela estava febril e, nesse estado, era-lhe
difícil carregar, apertando contra o ventre, a pesada tina com a comida
dos porcos, por isso tinha se esgotado completamente. Todos dormiam,
apenas os porcos grunhiam de vez em quando atrás do tabique, mas
Ánnuchka não conseguia se aquecer. Abraçou com as mãos os joelhos
ossudos e pressionou-os contra o ventre dolorido para se esquentar;
nessa posição, sentiu suas entranhas e se lembrou do rapaz de Kursk.

Assim, depois dо primeirо flagelo do Senhor --- a espada ---, depois do
segundo --- a fome ---, depois do quarto --- a doença ---, caiu sobre
ela o terceiro --- o animal selvagem, a luxúria, o adultério ---, o
único flagelo do qual ela havia sido poupada. E ele chegou em momento
inesperado e inoportuno. Ánnuchka se lembrou do rapaz de Kursk, ou
sonhou com ele, mas ele tinha outra aparência e surgiu na época em que a
mãe dela estava viva e na presença de Ivan-Mítia. Era como se o rapaz de
olhos cinzentos de Kursk a acompanhasse por todos os lugares. Na aldeia
de Nefiédovo, ele sentava perto dela, ao lado da isbá, sob o sol
sonolento e terno da manhã, enquanto Ánnuchka cochilava de camisola,
sentindo prazer... No quarto nº 9 do barracão nº 3 do 3º setor da cidade
de Rjév, o rapazinho também se achava ao lado dela, divertindo-se com o
jogo das pedrinhas\footnote{Antigo jogo, no Brasil conhecido como Jogos
  das pedrinhas ou Cinco Marias, no qual os partipantes lançam as peças
  (às vezes de pano) no ar e tentam pegá-las com uma mão. O que mais
  peças pegar é o vencedor.} em companhia de Ivan, o irmão de Ánnucka,
apelidado de Mítia... No edifício nº 61 da Rua do Trabalho, a antiga
igreja que fora destinada como moradia аos \emph{stakhanovistas}, o
jovem de Kursk, de nariz arrebitado, também morava com Ánnuchka e juntos
passeavam pelo cemitério onde Vova, seu irmãozinho, fora enterrado. Só
que as árvores do cemitério pareciam maiores e mais bem cuidadas, como
num jardim alemão. Havia muitas árvores aromáticas e vinhas, mas também
arbustos com frutinhas silvestres, como os que cresciam na floresta
perto de Nefiédovo. Ánnuchka e o rapaz de Kursk foram colher frutinhas e
pararam atrás de uns arbustos e, de súbito, ele a agarrou e derrubou-a,
sem esforço, pois ela mesma se entregou, abraçando fortemente os joelhos
e pressionando-os contra o ventre, e tudo se tornou quente e
agradável... Mas, de repente, Ánnuchka ouviu: ``Sua mãe, Emeliánova,
morreu no dia 7 de outubro de 1942...''. Então começou a chover, um
temporal. Ánnuchka se esqueceu da felicidade vivida com o jovem de Kursk
e correu a toda a força para que não enterrassem sua mãe sem a presença
dela. Ela chegou ofegante até os barracões, mas havia tanta água que não
foi possível passar, e o caixão com o corpo de sua mãe estava no pátio,
sob a chuva. Ánnuchka viu os antigos vizinhos --- ela se lembrava de
todos --- se aproximarem do caixão para erguê-lo e levá-lo até o
cemitério. Ela gritou:

--- Eu estou aqui... Eu sou a filha de Emeliánova...

Mas eles não ouviam sua voz vinda de longe e a menina não conseguia
transpor a água. Quando os vizinhos se inclinavam para levantar o
caixão, a mãe de Ánnuchka inesperadamente se sentou e disse:

--- Esperem, eu quero dizer uma coisa...

Ánnuchka ouviu claramente essas palavras, mas não conseguiu entender o
que sua mãe disse em seguida, pois a água a impedia de se aproximar e
fazia muito barulho... Os vizinhos, pessoas estranhas, a ouviam,
enquanto sua própria filha não. Ánnuchka correu em direção à água, que
alcançava sua cintura e depois seu pescoço, e ninguém se pôs a
ajudá-la... Mesmo assim, ela conseguiu chegar até o caixão, mas sua mãe
já tinha acabado de falar e jazia, inerte, como antes. Os vizinhos
levantaram o caixão e o carregaram. Ánnnuchka caiu no choro e com esse
choro despertou no chiqueiro alemão, perto do tabique de madeira onde
grunhiam porcos...

A chuva rumorejava no telhado de telhas, mas não havia correntes de ar
--- o chiqueiro alemão se diferencia do chiqueiro russo por ser mais
limpo e por melhor proteger contra o frio. Só que Ánnuchka não tremia do
frio exterior, mas do frio interior, não por causa do vento, mas da
febre. Durante o sonho, Ánnuchka chorara a plenos pulmões, porque estava
em casa e ninguém poderia proibi-la de chorar, mas no mundo real ela
chorava baixinho, porque vivia sob a escravidão alemã. Esse choro era o
pranto divino, vindo do coração, com o qual às vezes o Senhor recompensa
os insensatos, como se dera com a jovem prostituta Maria, em 1933, em um
campo perto da estação de Andréievka. Através do pranto divino, Maria,
naquele instante, se elevara --- sem ajuda de palavras, ela lera o
sermão do Senhor e, sem o uso da razão, compreendera o que o profeta
Isaías, por meio da inteligência, havia recebido:

--- Como uma mãe que consola Eu vos consolarei {[}...{]} Vós vereis, e
vosso coração se alegrará e vossos ossos florescerão como ervas frescas
{[}...{]}.\footnote{Isaías 66:13, 14.}

Ao escutar, sem palavras, esse mesmo sermão e ao compreendê-lo sem
razão, Ánnuchka saiu do chiqueiro alemão aquecido е, sob a chuva, entrou
num atalho da terra impura que deveria amaldiçoar. Enquanto ela
caminhava, parou de chover, e a terra impura, convencida de sua
eternidade, deleitou-se com a lua alemã, cuja visão fazia alemães sem
coração derramarem lágrimas de ternura...

As colinas monótonas alemãs erguiam-se aqui e ali, com magma vulcânico
grudado a elas, entre pastagens frias e úmidas... Na parte nordeste do
rio se estendia uma floresta densa. O próprio rio se achava no meio de
um vale pitoresco, entre margens rochosas, onde dormiam, sob seus
telhados de telhas, verdadeiros povoаdos medievais... Tudo isso seria
amaldiçoado pelo Senhor através de seu enviado, Dã, a Áspide, o
Anticristo, e a execução da maldição estava reservada a Ánnuchka
Emeliánova, de Rjév, a mártir ímpia que fora conduzida à escravidão.
Ánnuchka aproximou-se da margem do rio, sentou-se numa pedra coberta de
musgo e tirou do rolo o papel que o Anticristo lhe dera e do qual ela
havia se lembrado agora. Ali havia inscrições em duas línguas: uma
desconhecida e incompreensível, como os rastros dos passarinhos na neve
ou na areia, e outra familiar, que lhe fora ensinada na escola. Por mais
que a lua germânica tentasse se esconder nas nuvens de chuva, forças
celestes a obrigaram a iluminar Ánnuchka, que, sob a resistente luz
alemã, começou a ler, sílaba por sílaba. Embora, na escravidão, ela
tivesse começado a esquecer o alfabeto, conseguiu decifrar a Maldição
dos profetas bíblicos, agora lançada contra essa terra impura e contra
esse povo ímpio. Com as maldições, os profetas preveniam seu povo do
pecado. Mas sete vezes será amaldiçoadо aquele que usar da própria
maldade para castigar o pecado. Pois, para executar sua cólera, o Senhor
escolhe sempre os tipos mais terríveis.

``Voltarei minha face contra vós e vós sereis abatidos por vossos
inimigos {[}...{]} e vós fugireis sem que ninguém vos persiga. {[}...{]}
Eu vos farei o céu de ferro e a terra de cobre. {[}...{]} vos reduzirei
tanto que vossos caminhos ficarão desertos. {[}...{]} Em vão se
consumirá vossa força; vossa terra não dará vegetação e suas árvores não
darão frutos. {[}...{]} Eu suprimirei o pão que sustenta o homem; dez
mulheres cozerão vosso pão em um só forno, vós comereis, mas não vos
fartareis. {[}...{]} Enviarei temor ao coração dos que restarem entre
vós, e o farfalhar de folhas inquietas os fará fugir.''\footnote{Levítico
  26:17, 19, 20, 22, 26, 36.}

Eis a Maldição do profeta do exílio Ezequiel: ``Eu, o Senhor, disse que
assim será e assim farei. Não voltarei atrás, não pouparei nem
concederei perdão. Eu te julgarei conforme teus caminhos e tuas ações, e
já não te encontrarei pela eternidade''.\footnote{Ezequiel 24:14.} Eis a
Maldição do profeta Isaías, depois reiterada no Apocalipse de João:
``Teu céu se enrolará sobre ti como um pergaminho''.\footnote{Isaías 34:
  4. Apocalipse 6: 14.} E, em cólera, o primeiro profeta bíblico, o
pastor de Técua Amós, ao olhar para a terra impura, anunciou a Maldição
e depois a anotou em um manuscrito, legado pelo profeta Jeremias: ``Eu
odeio, desprezo vossas festas {[}...{]} Afasta de mim o barulho de teus
cantos''. E, no fim, o profeta Amós acrescentou: ``Que o julgamento
corra como a água e a justiça como uma torrente...''.\footnote{Amós
  5:21, 23, 24.}

Com isso, Ánnuchka Emeliánova concluiu a leitura do manuscrito da
Maldição. Saraías, o camareiro-mor do rei, \footnote{O sacerdote judeu
  Saraías, a quem Jeremias (51:61\emph{--}64) se dirigiu ao falar da
  queda de Babilônia (``E, quando tu acabares a leitura deste livro,
  prenderá uma pedra nele e o lançarás em meio ao Eufrates, dizendo:
  Babilônia afundará e não se levantará mais, graças ao mal que lançarei
  sobre ela, e todos eles irão desfalecer' {[}...{]}''), aparece como
  camareiro-mor do rei Nabucodonosor II.} terminara de ler o manuscrito
da Maldição sobre a Babilônia antes do amanhecer, e Ánnuchka, da mesma
forma, terminou sua leitura quando era hora de voltar ao chiqueiro
alemão e carregar as pesadas tinas com a comida dos porcos, para que não
apanhasse por atraso e negligência. Assim, ela achou rapidamente uma
pedra, arrancou um retalho de seu vestido, atou com ele a pedra ao
manuscrito e lançou-o às águas do rio alemão.

O ódio, como sentimento permanente, torna a alma ressequida, mas o ódio
contínuo ao povo alemão, a tudo o que se refere a ele, deveria, desse
momento em diante, virar um traço do povo do Senhor, como uma
advertência aos outros inimigos históricos menos hábeis. Se as gerações
atuais e próximas levam consigo, ao morrerem, essa antipatia, pelo menos
uma desconfiança deve permanecer para sempre, uma desconfiança nacional
e razoável que, de algum modo, acaba transformando esse ódio em
sentimento permanente, uma forma inútil, grosseira e desajeitada de uma
nação se defender. O humanismo místico-nacional nazista endeusava o
homem nórdico e utilizava-o como medida para todas as coisas. A escada
hierárquica racial conduzia do homem nórdico para baixo, e no degrau
inferior estava o judeu desumanizado, separado do humanismo. E isso é
natural. Os judeus, como indivíduos, são tão cruéis quanto o resto da
humanidade. Mas, como formação histórica ou fenômeno bíblico, seu povo
está próximo de Deus, e o homem, em essência, odeia Deus e por isso
odeia os judeus, assim como muitos judeus odeiam a si mesmos e a seu
destino bíblico. Essa ideia é tão inestimável que vale repeti-la em
outras palavras. Certamente, o judeu enquanto homem é tão ruim como
todos os homens, mas o judeu enquanto judeu, conforme a Bíblia, é uma
parte do povo de Deus; para acreditar em Deus, o homem, que é Dele
inimigo, precisa superar sua própria natureza humana, amaldiçoada por
Deus, mas, como poucos o conseguem, o ódio contra os judeus torna-se
algo absolutamente natural. Quanto mais longe um povo se acha de Deus,
no atual estágio histórico, maior é seu ódio e mais naturalmente o
antissemitismo se transforma num indício nacional. O próprio destino
secular do povo judeu mostra ao homem que ele não é o dono da terra, mas
somente um operário de Deus e um errante. Graças a isso, os povos,
especialmente os grandes e fortes, que se imaginam donos do vinhedo
divino e rejeitam a parábola do Evangelho sobre os vinhateiros, odeiam o
povo judeu, o qual, por seu próprio destino, zomba continuamente, embora
muitas vezes de modo inconsciente, da pretensão dos homens de serem
proprietários do vinhedo.\footnote{Trata-se da ``Parábola dos
  vinhateiros homicidas'' (Mateus 21: 33\emph{--}46), na qual se narra a
  história de vinhateiros que arrendaram a terra de um proprietário de
  um vinhedo. Quando o proprietário mandou seus servos receberem os
  frutos, os vinhateiros ``espancaram um, mataram outro e apedrejaram o
  terceiro'' (\emph{Bíblia de Jerusalém,} ed. Paulus, 2016, p. 1742).
  Quando o dono mandou seu filho, os vinhateiros o assassinaram.} Assim
como, na parábola do Evangelho, trabalhadores negligentes mataram
repetidamente os enviados do Senhor --- que os lembravam das obrigações
para com o verdadeiro Dono do vinhedo ---, tentou-se muitas vezes, no
decorrer dos séculos, resolver a questão judia. Mas o alemão fez disso a
base de sua ideia de governo em um momento crucial de seu destino
histórico, em nome do cumprimento de seu dever histórico para com a
humanidade. Pois, como foi dito, a maioria dos homens odeia Deus, seja
de forma secreta ou aberta. Os homens O odeiam porque Ele é forte,
enquanto eles são fracos, porque Ele é imortal, enquanto eles são
efêmeros. Em suas preces, mais imploram que enaltecem; em seus mitos,
celebram titãs como Prometeu, um inimigo de Deus e um mártir da
humanidade que se sacrificou por ela. Como poucos amam a Deus, o alemão,
em sua solução científica categórica da questão judia, se pronunciou em
nome da maioria dos homens, os quais almejavam, segundo a parábola do
Evangelho, serem donos do vinhedo de Deus, e não Seus operários...

Assim que o Senhor soube que Ánnuchka Emeliánova, de Rjév, cumpriu a
Мaldição, três dias antes de ela morrer de febre, chamou Anticristo, seu
enviado, e disse:

--- Tu irás para a cidade de Bor, às margens do Volga, e ficarás lá
enquanto fores necessário...

--- Senhor --- respondeu o Anticristo ---, eu não estou mais sozinho...
Uma criança eslava está comigo, uma menina que salvei em pedido de sua
mãe... Sua mãe não está mais entre os vivos, ela morreu num vagão de
carga a caminho da escravidão alemã...

--- Vai com a criança --- disse o Senhor.

Assim Dã, а Áspide, o Anticristo, dirigiu-se com a criança eslava para a
cidade de Bor, às margens do Volga. O Anticristo deu à sua filha adotiva
o nome de Rute, como a moabita que se unira ao povo dele perto de Belém,
sem saber que, na aldeia dela, a menina era chamada por um nome grego:
Pelágia... Mesmo ao Anticristo nem tudo é concedido saber. Ele também
não sabia o que o esperava dessa vez. O Senhor o ocultou... Ele só sabia
que, em Bor, nas redondezas rio Volga, morava certa Vera Kopóssova com
duas filhas, Tássia, a mais velha, e Ústia.\footnote{Tássia é apelido de
  Taíssia e Ústia de Ustínia.} O marido de Vera, Andrei, estava, então,
no \emph{front}, no entanto ele logo regressaria para sua família,
porque o primeiro flagelo do Senhor --- a espada --- havia terminado.
Embora o mundo decaído fosse merecedor dessa punição, o Senhor
compreendia que o homem não a suportaria por muito tempo. O segundo
flagelo --- a fome --- o homem tem mais condições de suportar; ao quarto
--- a doença --- ele se adapta ainda mais habilmente; e ao terceiro
flagelo --- o animal selvagem, a luxúria --- ele está completamente
acostumado...

Sabendo disso, o Senhor enviou uma recompensa a Ánnuchka, a mártir
ímpia, antes de sua morte, um prazer em troca da Maldição que ela havia
realizado: um sonho feliz, depois do qual Ánnuchka não mais voltou à sua
existência terrível e não divina. Nesse sonho feliz, o rapaz de Kursk
voltou a agarrar Ánnuchka, derrubou-a sobre a terra quente, perto da
isbá, na aldeia de Nefiédovo, onde ela nascera, e fez com ela por bem o
que havia feito de assalto, com violência, na escravidão alemã.

Quando, ao amanhecer, os outros trabalhadores braquicéfalos ouviram os
gemidos de agonia de Ánnuchka, eles se aproximaram e viram em seu rosto
a expressão feliz e irrefreada da paixão que só se manifesta no clímax
nupcial. Casos como esse são conhecidos pela medicina --- assim às vezes
se morre de febre nos anos de mocidade, quando o corpo extenuado carrega
paixões ainda não consumidas.

3

Para obrigar o homem a realizar o primordial, é preciso de desvario.
Para que o trivial exista, deve-se aspirar a algo grandioso. Para que o
homem compreenda o grandioso, é necessário que esse seja rebaixado... Na
cidade de Bor, na região de Górki, antiga província de Níjny-Nóvgorod,
cumpriu-se um dos flagelos do Senhor: o animal selvagem, a luxúria... Em
Bor morava a família Kopóssov: o pai, Andrei Kopóssov, a mãe, Vera
Kopóssova, e suas filhas, Tássia e Ústia. E vejamos em que parábola suas
vidas se transformaram...

\textbf{Parábola do adultério}

Era 1948, quando tudo já havia passado: os pesados sofrimentos
militares, as alegrias esfuziantes do pós-guerra. A impressão de que
tudo isso ficara para trás trazia algo de sério e decadente tanto аos
sentimentos como aos semblantes. Mesmo as esperanças no futuro tinham um
ar decadente; todos então buscavam recuperar a vida antes da guerra,
pois as destruições militares faziam com que eles depositassem no
modesto passado os sonhos de um futuro próspero. Enquanto, em planos
estatais, a aspiração de reencontrar o passado no futuro foi apontada de
forma direta, nas almas dos homens, certamente, essa ideia não foi
planejada com tanta clareza, embora ela existisse e as cativasse, pois a
alma de uma pessoa não é uma fábrica destruída em que a produção de
tratores reduziu em comparação ao que era antes da guerra.

Por um lado, durante a guerra, Bor ficou na retaguarda, embora de forma
parcial, não sofrendo destruições e perdas de civis; por outro, a cidade
foi completamente exposta aos quatro flagelos do Senhor. Não eram raros
os avisos de morte, não era pequena a fome, não eram poucas as doenças
nem os casos de adultérios de mulheres abandonadas e da juventude
empedernida. A audácia eslava também teve seu papel. Algumas pessoas não
davam mais importância às recriminações dos outros ou de sua própria
consciência, como se ignorassem um cachorro impertinente:

--- Ah, a guerra apagará tudo...

No entanto, Vera Kopóssova esperou a volta de seu marido e foi fiel a
ele. Ela trabalhava numa fábrica de costura, fazendo \emph{telogreikas}
e calças de algodão para soldados, e, com seus ganhos e o soldo do
marido, criava as filhas, Tássia e Ústia... Acontecia às vezes de
assediarem Vera. Uma vez ela foi abordada até por Pávlov, а quem mesmo
esposas fiéis lançavam olhares furtivos, sem falar das mulheres que
tinham decidido gozar dos prazeres da vida e não mais se privarem deles.
``Uma vez, dez vezes, tanto faz... A guerra irá apagar tudo... Lá eles
também não perdem tempo...''

Pávlov era um ferido de guerra, mas sem mutilações aparentes: as pernas
e os braços estavam inteiros e os ferimentos eram escondidos sob a
roupa. Tinha um rosto bonito, adornado de olhos azuis, e um bigode fino
sedutor. Vera via mulheres se atirarem a ele na rua, atraindo-o para
suas casas... E Pávlov, com seus modos de marinheiro, não rejeitava
nenhuma: nem а menina ingênua que seduziu, nem a viúva quarentona por
quem tinha sido seduzido... Pávlov aproximou-se de Vera por sua beleza
--- nem a guerra conseguiu envelhecê-la --- e não o fez de mãos
abanando, como de hábito, mas cercando-a de presentes --- um lençо de
seda e duas latas de carne de porco guisada, as quais ele havia ganhado,
por sinal, da viúva quarentona, uma funcionária do setor popular de
alimentação.

--- Pegue --- disse ---, é para você... Para se lembrar do amigo nas
horas vagas...

Isso aconteceu à noite, na Rua Derjávin, não longe da casa nº 2, onde
Vera morava. Se mesmo antes da guerra os lampiões mal iluminavam a rua,
agora a escuridão era completa. A escuridão, como se sabe, excita o
homem, e Pávlov queria aproveitar, principalmente porque era verão e,
nas redondezas, havia um terreno baldio coberto de grama onde de dia
pastavam cabras do subúrbio.

Em tempos de guerra, não era moda dar um bofetão em assediadores
persistentes, e Vera deu um soco no nariz de Pávlov, o que saiu de
maneira pouco feminina, e talvez a ausência de feminilidade tivesse
tirado dele a vontade de repetir a tentativa. Ele apenas insultou Vera
grosseiramente, chamando-a de ``puta'', pressionou o lenço contra o
nariz e foi embora, empregando sua força viril acumulada na viúva
quarentona, que, para seus 23 anos, era até mais interessante. E Vera
dirigiu-se à casa nº 2, onde Tássia, alegre, entregou-lhe a tão esperada
carta triangular de soldado\footnote{Comum entre soldados russos na
  Segunda Guerra, a carta triangular podia ser enviada sem selo. Além de
  indicar tratar-se de missiva militar, a forma triangular facilitava a
  leitura dos censores, já que o envelope era formado pela própria carta
  e apenas a ponta da base do triângulo era colada.} --- um bilhete de
Andrei... Radiante, Vera esqueceu o infeliz incidente. Tássia crescia e
ficava cada vez mais parecida com a mãe, e Vera começou a ajeitar os
cabelos da filha numa trança, assim como trançava os seus. Ústia ainda
era pequena. Porém, no outono de 1945, quando Andrei Kopóssov voltou da
guerra, condecorado com ordens e medalhas, Ústia foi a seu encontro com
os próprios pezinhos, e não mais nos braços da mãe ou da irmã.

Andrei Kopóssov encontrou tudo a salvo. A esposa estava saudável e sem
mudanças, as filhas estavam saudáveis e com mudanças agradáveis, e mesmo
sua bancada de carpinteiro de madeira, em um canto do quarto maior,
estava intacta, encobrindo lascas de madeira de antes da guerra que Vera
não varrera de propósito, para ter uma recordação do pai de suas filhas.
Andrei lembrava que Tássia gostava de brincar com essas aparas, e agora
ele via Ústia, a filha que não conhecia, brincar com elas também. Andrei
derramou lágrimas de alegria. Que prazer ainda poderia desejar um
soldado que havia passado quatro anos na guerra? Assim, com alegria,
transcorreu o fim de 1945, alegria que continuou em 1946, embora neste
ano tivesse começado a fome, que em 1947 se acentuou, então todos
passaram a sonhar com a fartura de antes da guerra que substituíra os
anos de fome da coletivização. Quanto mais o tempo avançava, mais
desejavam o passado... Em 1948, as preocupações com a fome diminuíram um
pouco, mas também as últimas alegrias do pós-guerra, e os sentimentos e
os semblantes ganharam aquele ar decadente... A vida ficou mais
tranquila, a vida ficou mais aborrecida.\footnote{Alusão a uma conhecida
  frase de Stálin pronunciada em 17 de novembro de 1935 num encontro dos
  \emph{stakhanovistas}: ``A vida ficou melhor, a vida ficou mais
  alegre'' (\emph{Jizn stalo lutche, jizn stalo vesseleie}). O poeta V.
  Lébedev Kumátch escreveu uma canção assim chamada em 1936.} Na pista
de dança do jardim da cidade tocavam valsas líricas patrióticas, e os
acordeões alemães, troféus de guerra, deixaram de se ocupar com os
foxtrotes do Ocidente. A juventude, como antes da guerra, se distraía
com o jogo de prendas,\footnote{Numa variação do jogo, cada participante
  deve colocar um objeto pessoal (um relógio, um lenço, uma pulseira,
  etc.) em uma sacola ou um chapéu. Um dos objetos é sorteado e um
  desafio é proposto ao seu dono para que ele possa recuperar a prenda
  que deixou como garantia.} mas sem beijos. E, mesmo a bebedeira, que,
desde tempos imemoriais, conforme a tradição eslava, acontecia
livremente nas ruas, agora era algo para ser feito em casa.

A essa altura, Tássia Kopóssova estava quase na idade de casar, exibindo
toda a beleza dе sua mãe de antes da guerra, e a trança castanho-clara e
volumosa de Tássia nada devia aos cabelos trançados, dourados e
perfumados de Vera, e vice-versa. Mãe e filha desabrochavam: uma em sua
força femínea, outra em sua inocência virginal. Andrei Kopóssov, olhando
para a esposa, amava mais a filha e, olhando para a filha, sentia-se
mais atraído pela esposa, cujo corpo foi guardado para ele durante a
guerra.

No entanto, nesse ponto se inicia a parábola graças à qual o Anticristo
apareceu em Bor, na região de Górki, seguindo as ordens do Senhor. O
terceiro flagelo do Senhor está em toda parte, pois nem mesmo Ele é
livre para revogá-lо, como pode fazê-lo com o primeiro e mais terrível
flagelo --- a espada ---, com о segundо --- a fome ---, ou com о quartо
--- a doença. О terceirо flagelo do Senhor --- o adultério --- segue o
homem como uma sombra, e somente lhe tirando o objeto de desejo é
possível livrá-lo dela. Porém, se o terceiro flagelo é onipresente,
nesta parábola ele foi levado à cena principal...

Durante a guerra, Vera se guardou para o marido, mas, depois da volta
dele, bastou um ano de convívio para ela deixar de amá-lo... Talvez
fosse aquela impressão decadente, о sentimento coletivo e tácito de que
tudo ficara para trás --- o bom e o ruim. Mesmo um grande homem é um
escravo de seu tempo, e Vera Kopóssova era apenas uma mulher simples,
muito bonita no passado e ainda bonita nessa época, só que agora um
transeunte não a seguia obrigatoriamente com o olhar como antes, a menos
que esse transeunte fosse Andrei Kopóssov. Mas Vera deixou de amá-lo.
Ela sentia aversão por seu marido, uma aversão puramente feminina que
ela não podia dividir com ninguém; seria até vergonhoso para uma mulher
decente falar sobre isso...

Como antes da guerra, Andrei trabalhava no comitê do partido da cidade
como carpinteiro, e nas noites e nos dias de folga ficava em sua bancada
moldando utensílios de madeira --- tinas, barrilzinhos para óleo,
batedeiras para manteiga, colheres, saleiros... Em casa pairava o cheiro
agradável de lascas frescas de madeira, com as quais suas duas filhas
brincavam --- a caçula, Ústia, e a mais velha, Tássia, mesmo que esta
estivesse quase em idade de casar. Elas amavam o pai e o chamavam de
\emph{tiátia}, como ele havia ensinado, pois era assim que falavam na
terra dele... Logo que produzia uma considerável quantidade de peças de
madeira, Andrei vendia-as na feira local ou mesmo em Górki. De lá trazia
farinha e outros mantimentos. Vera sempre o acompanhava na ida e o
recebia na volta, preparava pratos apetitosos, mantinha a casa limpa e
arrumada. Porém, à noite, assim que se deitava ao lado dele, ela sentia
que não poderia aceitá-lo, nem que a torturassem... Quando as
intimidades conjugais começavam, era como se ela fosse violada. Prazer
feminino nem se cogitava, mas que pelo menos não fosse tão repugnante...
Vera ficava deitada, indiferente, até que Andrei satisfizesse sua força
viril e dormisse. Logo que ele adormecia, ela passava para o leito de
tijolos das filhas, junto à estufa. Era um leito largo o suficiente para
três pessoas... Andrei percebia a aversão de sua mulher, embora nunca
tivesse mencionado uma palavra a respeito. Mas, nesse assunto, palavras
não são necessárias... Ele começou a tratar Vera com grosseria, depois
passou a bater nela. A primeira vez que o fez foi ao voltar de Górki,
sem farinha nem mantimentos, contudo visivelmente bêbado.

--- Pessoas decentes me contaram! --- gritava. --- Você, puta, aqui,
durante a guerra, se meteu com Pávlov...

E, sem se intimidar com a presença das filhas, Andrei usou de palavras
abertamente obscenas para descrever o que a esposa teria feito com
Pávlov durante a guerra. Depois disso, ele ficou fora de si, como os
camponeses que, saídos de suas aldeias, se enfurecem na cidade e em seus
subúrbios.

Na aldeia, especialmente nos velhos tempos, o camponês dava mostras de
sua fúria de maneira diferente: espancava o que era vivo com crueldade,
mas preservava seus bens materiais, pois o que tem vida pode se
regenerar, ao contrário dos objetos... Andrei, porém, esbravejava à
maneira de um camponês suburbano. Arrastava Vera pela trança, quebrava a
louça, dava golpes no leito junto à estufa com seu machado de
carpinteiro... Houve uma vez em que ele saiu correndo atrás de Ústia,
deixando-a terrivelmente assustada:

--- Essa aqui é de Pávlov --- gritava ---, eu vou matá-la!...

Desde então, assim que Andrei se enfurecia, Vera pegava as filhas e ia
passar a noite na casa de algum dos vizinhos. Havia uma família de
ucranianos --- os Morozenko da Rua Derjávin, nº 8 --- que as acudia com
mais frequência. A bem da verdade, Andrei não tocava na sua bancada de
carpinteiro, à custa da qual comprava pão e vodca --- diante de seu meio
de trabalho ele se continha, assim como diante de sua filha mais velha,
Tássia. Assim, a menina parou de passar a noite fora com a mãe quando o
pai perdia a cabeça, ficando com ele para acalmá-lo.

--- Deite-se, \emph{tiátia} --- dizia ela ---, tome um pouco de salmoura
de pepino, vai se sentir aliviado.\footnote{Método usadо na Rússia para
  curar ressaca.}

O desordeiro russo é sempre capaz de cair no choro depois de concluir
seu assunto principal, depois de maltratar ou mesmo de matar alguém.
Então seu coração logo se tranquiliza, e ele se porta como uma
criancinha --- ``tenham misericódia, boa gente...'' ---, e todos se
apiedam. Um célebre escritor russo via nisso, em geral, uma qualidade
nacional valiosa.\footnote{Provável alusão a Fiódor Dostoiévski, com
  quem o narrador continuamente dialoga. Em ``Vlás'' (\emph{Diário de um
  escritor (1873): meia carta de um sujeito}, Hedra, 2016), Dostoiévski
  compara os bêbados alemães e os russos: ``O russo bêbado gosta de
  beber por desgosto e de chorar. Quando cai na farra, não celebra,
  apenas provoca desordens. {[}...{]} Com atrevimento, ele na certa dá
  provas de que é praticamente um general, ralha amargamente se não
  acreditam e, para que acreditem, no fim das contas, chama sempre por
  socorro. Mas, se ele é tão desordeiro, se chama por socorro, é porque,
  no íntimo de sua alma bêbada, está possivelmente convencido de que não
  é nenhum general {[}...{]} Não está satisfeito consigo mesmo; um
  sentimento de censura cresce em seu peito, e ele vinga-se disso nos
  que estão à sua volta {[}...{]}''.} No entanto, Andrei, na presença de
sua filha mais velha, mesmo livre de sua ira, podia entrar em estado de
enternecimento:

--- Você é do meu sangue, foi por sua causa que eu voltei da guerra, e
não por sua mãe miserável. Graças a você, na Polônia, uma mina me pegou
só de raspão --- ele dizia enquanto trançava e destrançava os cabelos da
filha, chorando e beijando sua trança. --- Era assim que sua mãe
trançava os cabelos quando nos casamos...

Mas, quando Vera estava presente, Tássia não conseguia conter o pai
bêbado. Logo que ele via a esposa, ficava furioso. E não gostava de
Ústia.

--- Ela não é do meu sangue! --- gritava. --- Outro homem a pôs no
mundo...

``Oh, Deus,'' pensava Vera, ``se pelo menos ele arranjasse uma mulher
por aí... Eu daria um jeito de aguentar essa tortura pelas crianças,
desde que ele não me tocasse.'' Com esperança, Vera ficava de ouvidos
atentos às conversas dos vizinhos. Por mais que eles não vissem Andrei
com bons olhos, Vera nem uma vez ouviu qualquer alusão à devassidão
dele, apesar de já não viverem como marido e mulher fazia tempo. Quanto
à devassidão de Vera, havia rumores de que ela teria tido algo com
Pávlov, mas de Andrei diziam apenas que ele bebia, batia na esposa e
maltratava as crianças.

Com o tempo, todos se acostumaram a essa situação. Andrei se habituou à
crença de que sua mulher era uma devassa, Vera ao fato de que seu marido
era um bêbado e um desordeiro e os vizinhos à ideia de que a família
Kopóssov era infeliz e desregrada. Vera até conhecia os sinais dos
acessos de Andrei, sabia quando ele ficaria furioso e quando se
acalmaria. Ele se enfurecia às vésperas da lua cheia e dava uma trégua
com a chegada da lua nova. Desde que começou essa vida infernal, Vera
passou a recorrer a Deus, embora não fosse à igreja, e implorava que
seus dias de folga caíssem na lua nova. A essa altura, Andrei, ао levar
seus produtos de madeira para Górki, gastava tudo o que conseguia da
venda em bebida, lá mesmo, na companhia de conhecidos, demorando um ou
dois dias para regressar para casa. Ele voltava sombrio, no entanto mais
calmo. E, se ele desatinava logo depois, sua fúria não chegava ao
extremo. Tentava bater em Vera, mas não assustava Ústia nem destruía as
coisas de casa. Vera agora só tinha um prazer, além das filhas,
naturalmente. Ela morava em uma bela vizinhança e amava sua terra, a
cidade de Bor... Ali havia bons lugares para pescar, para apanhar
cogumelos, para colher frutas silvestres... Mesmo com sua desgraça
pessoal, ela tinha motivos de alegria. Vera notou que Tássia, nos
últimos tempos, começou a olhar para ela com desaprovação, apegando-se
ao pai; em compensação, Ústia, que Andrei não amava e não deixava mais
brincar com as aparas perto de sua bancada, não desgrudava da mãe. Vera
continuava a trabalhar na fábrica de costura, só que agora não fazia
casacos acolchoados militares, mas jaquetas de algodão sem
personalidade, nas cores azul e cinza, para uso geral. Nos dias de
folga, ia com Ústia para a floresta. Quantos prazeres variados havia!
Podia-se sentir o gosto do prazer, ouvi-lo, apreciá-lo... O ar da
floresta embriagava Vera e ela queria que Ústia, sua filha favorita,
também se embriagasse desse alento. Não à toa a cidade era chamada de
Bor, que, em eslavo, significa ``floresta''.\footnote{Um dos
  significados de \emph{bor} é \emph{floresta de pinheiros situada em
  região seca}.} ``Tássia me recrimina,'' pensava Vera, ``ela é mesmo
filha de seu pai, agora Ústia é minha única família...''. E Vera temia
que Andrei, em um acesso de bebedeira, fizesse alguma coisa contra
Ústia, como ele costumava ameaçar...

Num domingo, no começo do inverno, quando a floresta estava
especialmente perfumada, Vera decidiu levar Ústia para respirar um pouco
de ar puro, mas não a encontrou. Chamou várias vezes, mas em vão... Ela
irrompeu em casa; Andrei trabalhava em sua bancada, sombrio, mas não
estava bêbado. Tássia sentava-se perto dele, recolhendo as aparas de
madeira.

--- Vocês viram Ústia? --- Vera perguntou, preocupada.

--- Não vimos a sua Ústia --- respondeu Andrei, sombrio. --- Eu não sou
seu empregado para correr atrás dos seus pecados e vigiá-los.

Mas Tássia disse:

--- Ela foi para a casa da velha Tchesnokova.

--- Que Tchesnokova? --- Vera continuava aflita.

--- Aquela com quem vivem os judeus --- Andrei sorriu com ar maldoso
---, então talvez ela não seja filha de Pávlov, mas do judeu...

Com efeito, Vera lembrou que a casa nº 30 era da velha Tchesnokova e,
parece, em um dos quartos viviam dois judeus, pai e filha...

Em Bor, na região de Górki, na Rua Derjávin, assim como em outras ruas,
cidades e regiões (antigamente chamadas províncias), sentam-se em bancos
de terra em volta de isbás, em tamboretes na frente de pequenas casas ou
nas entradas de altos edifícios as sentinelas da nação, as raízes
sinuosas do povo: velhas espadaúdas, inteiramente cobertas com xales de
lã, que geraram filhos de ossos fortes. Mulheres com maçãs do rosto
salientes asiáticas, narizes curtos e arrebitados, olhos incolores sem
vestígio de ternura maternal, embora sua eterna vigilância seja também
uma ocupação sentimenal... ``Nós, nós somos russos,'' dizem não com
palavras, mas com as maçãs dо rosto salientes e os narizes arrebitados,
``e vocês?''

Foi por meio delas que todos souberam que na Rua Derjávin (o grande
poeta russo que abençoara Púchkin), na casa da velha Tchesnokova, uma
\emph{velha crente},\footnote{No original, s\emph{tarovierka.} Ainda
  encontrados na Rússia, os \emph{velhos crentes} (\emph{staroviery} ou
  \emph{staroobriádtsy}) romperam com a Igreja Ortodoxa Russa devido às
  reformas do patriarca Níkon (1645\emph{--}1676) --- que unificou duas
  práticas religiosas (grega e moscovita) --- e conservaram as antigas
  liturgias.} moravam dois judeus: um homem, com cerca de trinta anos, e
sua filha, de oito anos. Se a origem judia da filha não podia ser
imediatamente identificada, sendo preciso observar com atenção, a do pai
saltava à vista. Vera também ouvira falar disso, no entanto, entre
tantas desgraças, não dera importância ao fato. Agora só pensava em
Ústia: ``Ela vai ver no que dá sair por aí sem permissão, como se já não
falassem mal o bastante de nossa família''.

A velha Tchesnokova vivia sozinha numa pequena casa, depois de perder
dois filhos no \emph{front} e o marido. Corriam rumores de que ela era
uma \emph{velha crente} ou uma \emph{subbótnitsa}.\footnote{O
  \emph{subbótnik} (\emph{subbótnitsa} é a forma feminina) pertencia a
  um movimento religioso, surgido na Rússia no reinado de Catarina II,
  que mesclava o cristianismo e o judaísmo, guardando os sábados (daí o
  nome da seita --- de \emph{subbota,} ``sábado'').} Às vezes elas se
encontravam na rua, mas não se cumprimentavam. Vera chegou à casa nº 30
da Rua Derjávin e bateu na porta. A velha abriu.

--- Мinha Ústia está com vocês? --- perguntou Vera, brava, como se a
velha fosse culpada de аlgo. Tchesnokova, ao contrário, respondeu em tom
amigável:

--- Está conosco, querida, está conosco... Está ouvindo o gramofone.
Entre...

--- Por que eu deveria entrar? --- replicou Vera. --- Chame Ústia, está
na hora de ir para casa --- e, sem conseguir se conter, disse: --- Achou
uma amiguinha. Como se não houvesse russas o suficiente para ela...

--- E o que ela tem de ruim? --- disse Tchesnokova. --- Rute é uma
menina educada, respeita os mais velhos, е seu pai não bebe...

E, de forma inesperada para si mesma, Vera teve vontade de ver os judeus
com quem sua Ústia passou a se distrair. Ela sacudiu a neve dе sua
peliça curta.

--- Está bem --- disse largando sua peliça na antessala.

Vera entrou no quarto em que o gramofone tocava e viu Ústia sentada à
mesa, ao lado de uma menina branquinha que ninguém jamais tomaria por
uma judia se não soubesse disso antes. O pai da menina era evidentemente
um judeu, sem tirar nem pôr, no entanto havia algo de incomum nele. Em
Bor era raro se encontrarem judeus, embora na região de Górki houvesse
muitos. Ao ver sua mãe, Ústia levantou-se de um salto e disse:

--- Essa é a minha mãe... E essa é Ruthina, minha amiga... E esse é o
\emph{tiátia} dela...

Vera olhou mais uma vez para o ``\emph{tiátia}'' de Ruthina e não pôde
identificar o que tornava esse judeu tão incomum... Ao observá-lo, ela
sentia um medo inexplicável, que, ao se acentuar, enchia seu coração de
deleite...

Realmente, nessa época, Dã, а Áspide, o Anticristo, já era um homem
feito --- seus traços bíblicos estavam completamente definidos. Mesmo
com os cabelos precocemente embranquecidos, graças ao que fora obrigado
a ver e a executar, Anticristo atingiu, nesse momento, o auge de sua
masculinidade em seu caminho terreno. E que Deus livre qualquer mulher
de saber o que é a virilidade do Anticristo... Não, não é uma
devassidão, é algo recluso, sufocado. Não é o Satanás dentro dele quem
seduz. Na força viril do Anticristo se manifesta a força divina, como
acontece com os fenômenos da natureza --- eis o que Vera viu e sentiu,
mas não compreendeu racionalmente, e a força não compreendida por meio
razão é especialmente temível. Graças a esse temor feminino, Vera foi
tomada por uma agitação incômoda.

--- Que música é essa? --- disse. --- Eu não entendo o que ela diz.

--- É um disco judeu --- respondeu Dã, a Áspide, o Anticristo.

--- Ora essa --- disse Vera e deu um riso afobado, como uma camponesa
bêbada na feira. --- E não poderiam colocar um disco russo, já que não
aprendi a língua dos judeus?

--- Podemos colocar um russo também --- respondeu Dã, a Áspide, o
Anticristo, e virou-se para sua filha: --- Rute, traga as
\emph{tchastuchkas} da cômoda.

De repente Rute, que era na realidade Pelágia, embora isso não fosse de
conhecimento dela nem do Anticristo, mudou de expressão: seu olhar
bondoso de camponesa --- ela era natural da aldeia de Brussiány, perto
de Rjév --- revelou uma paixão verdadeiramente sulina, seca, ao alcance
somente das meninas que amadurecem precocemente.

--- Quero que Ústia vá embora --- disse Ruthina ---, eu não vou mais
brincar com ela.

Então a velha Tchesnokova ficou agitada e começou a repreendê-la:

--- Menina sem-modos, por que envergonha seu pai na frente das pessoas?

E seu pai, o Anticristo, também se manifestou, mas sem gritar, olhando
calmamente a filha nos olhos:

--- O que você tem, Rute? --- perguntou ele, pois a conhecia como uma
menina meiga, delicada, bondosa; era como se tivessem trocado sua filha
por outra.

Porém, em vez de responder, Rute deu as costas e foi para o quarto
vizinho.

--- Grande coisa --- disse Ústia ---, uma chata... Eu também não vou
mais brincar com ela. Vamos, mãezinha...

Desorientada, Vera saiu da casa da velha Tchesnokova acompanhada da
filha e sentiu que, além dos velhos infortúnios, havia um novo a
caminho.

E na família do Anticristo muita coisa também mudou após essa visita
inesperada. É preciso notar que o Anticristo amava sua filha adotiva
como só poderia amar aquele que aprendeu a sentir amor eterno por seu
Criador. Por isso os judeus amam tanto as suas crianças, apesar de
muitas vezes não se darem conta do motivo, pois o amor pelo Criador,
para o povo de Abraão, não é tanto uma questão de religião, mas um
instinto nacional. O homem tem uma relação complexa com seus próprios
instintos, com frequência baseados no incompreensível, às vezes do ponto
de vista científico-filosófico, ou numa negação, sem dúvida, impotente.
Dessa maneira, entre os inúmeros homens que renegam o Senhor, os judeus
parecem especialmente afetados; e, entre os ateus de talento, poucos são
judeus --- a maioria destes se acha na espirituosa e fútil sátira
francesa. Via de regra, o judeu ateu ou é privado de talento ou é um
inconsequente. No entanto, mesmo os judeus que renegam o Senhor vivem
com Ele no cotidiano, e o grande instinto nacional do amor, transmitido
através do Senhor, se revela no amor religioso que as mães e os pais
judeus sentem por seus filhos. Que falar, então, do enviado do Senhor, o
Anticristo, além de tudo um homem solitário? Ele amaria qualquer
criança, dando-lhe o pouco que restava do seu amor, dirigido ao Senhor.
No entanto, ele amava a filha ainda mais, cedendo-lhe até uma parte do
amor que reservava ao Senhor, pois um pai sensato sempre ama a uma filha
com mais ardor do que a um filho. Rute-Pelágia, certamente, amava seu
pai da mesma maneira, e o amor filial por ele em nada diminuíra após a
visita inesperada daquela mulher, embora a menina tivesse ficado mais
nervosa e pensativa. Agora Rute mudava de humor rapidamente.

Certa vez, ela voltou da escola alegre e agitada.

--- Pai --- disse ao Anticristo ---, a rua está tão encantadora hoje...
Coberta de neve!

Realmente, grandes flocos de neve caíam pesadamente, mas com suavidade,
porque não havia vento. Rute pegou um prato fundo e foi correndo ao
pátio apanhar um pouco de neve. Depois voltou, colocou o prato molhado
sobre a mesa e disse de modo repentino:

--- Pai, onde você me achou?

Ao longo de sua vida conjunta, Rute nunca fizera uma pergunta como essa
a ele, mas agora o fez. Bem, todo pai pode ouvir semelhante indagação de
um filho, embora, para algumas crianças, isso não seja um segredo,
especialmente para uma menina da idade de Rute.

--- Um dia --- Dã, a Áspide, o Anticristo, respondeu a sua filha ---,
fazia um frio intenso, um vento muito forte soprava. E eu ouvi alguém
chorar. Saí de casa, mas não vi ninguém. Depois, ouvi o choro de novo.
Olhei para o alto e lá estava você, em cima de uma árvore...

Rute sorriu com certa tristeza, sentou-se perto do pai, encostando-se
nele, e disse baixinho:

--- Essa mulher que veio aqui é minha mãe...

--- Não diga bobagens, Rute --- disse o Anticristo ---, sua mãe morreu
num trem alemão... Еssa mulher é a mãe de Ústia.

--- Não --- respondeu Rute ---, eu a observei. Os olhos dela são iguais
aos meus, e os cabelos também... Mas não tenha medo, pai, eu amo apenas
você, mas a ela eu odeio...

--- Isso também não é bom --- disse o Anticristo ---, por que a odeia?

--- Ela olhava para você com maldade --- disse Rute ---, mas antes ela
era bondosa. Lembro como ela fazia manteiga, chacoalhava a garrafa de
leite, que tampava com um travesseiro...

Desde então, o Anticristo passou a olhar para a filha com preocupação,
esforçando-se em estar sempre por perto. E ela também se empenhava
nisso... Agora, para a alegria de ambos, ele a levava para a escola e a
buscava, e eles iam juntos a todos os lugares.

Quanto à Vera, desde sua visita, o que lhe restava de alegria
desapareceu de vez. Antes, Vera empregava todos os seus esforços e
pensamentos em evitar as noites de intimidade com seu marido, já que de
dia ela aprendera a despistá-lo. Agora, toda a sua paixão concentrava-se
na ideia de se entregar ao judeu, de gastar sua energia acumulada nele,
pois ela sabia que ainda tinha muita força como mulher. Мesmo após dois
partos, seu ventre, como antes, era firme como o de uma menina, e Vera
também não perdera a doçura, que seu marido, Andrei Kopóssov, cada vez
mais seco, tentava furiosamente acessar. Mas, nessa época, Andrei batia
nela com menos frequência --- pelo jeito se cansara ---, e, à medida que
se afastava da esposa, mais próximo ficava de sua primogênita, Tássia:
trazia-lhe presentinhos da feira e, de noite, quando não travalhava em
seu canto junto à bancada, gostava de trançar e de destrançar os cabelos
dela. A vida dos Kopóssov ficou menos violenta, porém não menos dura e
torturante... Quando Vera não trabalhava, saía andando a esmo, pois
tinha dificuldade em ficar parada; o que ela mais temia era o ócio do
corpo, pois era quando começava o tormento maior. Improvisava uma cama
no chão, perto da estufa, e se debatia até as três ou quatro horas da
madrugada, adormecendo por um breve período pouco antes de amanhecer.

Em uma noite especialmente penosa, no começo da primavera, perto da lua
cheia, Vera decidiu ir até a casa da velha Tchesnokova, mas precisava de
um pretexto. De manhã, arrumando Ústia para ir à escola, Vera disse:

--- Filhinha, você poderia passar na casa da Ruthina depois da aula? Vou
pegar você lá.

--- Era o que faltava --- disse Ústia. --- Eu não brinco mais com a
Rute. Serguéievna disse que eles são judeus e têm muito dinheiro.

Serguéievna era uma velha de maçãs do rosto salientes e nariz arrebitado
que fazia a sentinela na Rua Derjávin, junto à casa nº 17, advertindo a
todos, por meio de sua feição: ``Nós somos russos, e vocês?''

--- Você dá ouvido a Serguéievna --- disse Vera, nervosa ---, ela é uma
velha. O melhor seria ouvir o que ensinam na aula.

--- Na escola também falam a mesma coisa de Rute --- respondeu Ústia
---, que ela tem muito dinheiro e seu pai é um cosmopolita.

Então Tássia se intrometeu na conversa:

--- \emph{Tiátia} não deixa a gente ir lá.

--- Ah, vocês são desprezíveis --- disse Vera, enfurecida ---, sempre
\emph{tiátia} e \emph{tiátia}... Sua mãe para vocês não vale nada...
Quem criou vocês durante a guerra, quem as alimentou?...

--- E \emph{tiátia} nos defendeu --- disse Tássia ---, ele tem três
ferimentos e condecorações.

--- Mesmo se tivesse dez ferimentos --- disse Vera, com raiva ---, isso
não lhe daria o direito de me humilhar, de me bater, de beber...

--- Ele bebe de tristeza --- disse Tássia ---, porque ama você. Aliás,
não devemos ter essa conversa na frente de Ústia... Ústia, vá para a
escola. Também está na nossa hora, mamãe.

Vera havia conseguido um lugar para Tássia na fábrica como aprendiz na
oficina de costura. Logo que Ústia foi para a escola, Vera disse à
Tássia:

--- Por que você me envergonha na frente dessa criança? Seu pai tirou
você de mim, agora vocês querem me tirar Ústia. De repente Ústia virou a
queridinha, mas antes ela não era do sangue dele... Colocada no mundo
por outro homem... Pávlov...

--- Eu já disse --- respondeu Tássia ---, \emph{tiátia} fala essas
coisas de tristeza. Ele ama você, mamãe.

--- Ora essa --- Vera ficou ainda mais zangada ---, você mal aprendeu a
assoar o nariz e já quer me explicar a vida? Você ainda é minha filha e
deve me obedecer. Será que tratar os vizinhos dessa maneira é uma
atitude humana? Será que você é como a velha Serguéievna?... Foi isso
que lhe ensinaram na escola?... Na escola falam sobre a amizade entre as
nações... Por acaso nossos vizinhos têm culpa de serem judeus? Será que
eles são judeus por escolha?... Ponha a mão na consciência. Se você e
seu pai não deixam Ústia ir, você mesma fará uma visita a eles. Pedirá a
Tchesnokova um desenho para um bordado. Reparei que Tchesnokova tem
almofadinhas com belos motivos no seu divã...

--- Está bem --- disse Tássia ---, se é o que você quer, mamãe, eu vou.
Só não mande Ústia para lá. Ela ainda é uma criança.

Depois do trabalho, mãe e filha dirigiram-se à casa nº 30 da Rua
Derjávin, onde morava Tchesnokova. Vera bateu no portão, e Tássia ficou
distante, permanecendo o tempo todo assim, o que era visível tanto em
sua fisionomia como em seu comportamento. Em sua paixão febril, Vera, ao
pensar que veria aquele que a fazia passar os dias e as noites
suspirando, ficou ruidosa e agitada. Tássia, sempre distante, estava em
silêncio. Assim que viu o judeu, o inquilino dе Tchesnokova, Vera quase
perdeu os sentidos, mal se mantinha sobre as pernas, mas conteve-se e,
em vez de pedir a Tchesnokova um desenho para bordar na almofada, pôs-se
a falar sem cerimôniа, feito uma mulher vulgar, como se não tivesse se
guardado durante a guerra, ainda jovem, vivendo somente das notícias do
marido no \emph{front} e das filhas, renunciando a todas as outras
alegrias:

--- Boa tarde... Eu e minha filha viemos ouvir o gramofone, ou vão nos
mandar embora? --- e riu sem motivo, como riem as mulheres vulgares.

--- Sentem-se --- disse o Anticristo ---, agora mesmo Rute trará as
\emph{tchastuchkas} da cômoda.

Rute foi até a cômoda pegar as \emph{tchastuchkas} e subitamente
empalideceu. E a velha Tchesnokova, que espiava por uma fresta de sua
porta, deu um suspiro forte, dizendo:

--- Oh, nada de bom sairá disso, que Deus nos proteja --- e fez o sinal
da cruz sem juntar as pontas dos dedos como se pegasse uma pitada de
sal, mas unindo dois dedos inteiros, como as pessoas faziam.\footnote{Na
  Igreja Ortodoxa, o sinal da cruz (da direita para a esquerda) é feito
  com as pontas dos dedos indicador, médio e polegar grudadas, por isso
  a referência à pitada de sal, e os outros dois dedos unidos entre si e
  encostados na palma da mão. Entre os \emph{velhos crentes}, o sinal da
  cruz (da esquerda para a direita, como na Igreja Católica) é feito com
  os dedos indicador e médios unidos e estendidos e as pontas dos dedos
  anelar e mínimo, dobrados, cobertas pelo polegar.}

Nesse meio-tempo, Vera tirou do bolso um lenço de cambraia, limpou a
cadeira e disse a Tássia, sempre distante:

--- Sente-se, Tássia, eu tirei a poeira da cadeira, porque você está de
vestido novo --- e novamente ela se alegrou por si só, dando risadinhas.

Tássia não a contrariava, temendo novos embaraços, e se sentou na
cadeira, apenas ruborizada pelos disparates na conduta de Vera. E, ao
enrubescer, a beleza de Tássia, ainda suave, não extenuada pela vida
como a de sua mãe, revelou-se em plenitude. Dã, а Áspide, o Anticristo,
notou essa beleza delicada, e seu coração bateu tão estranhamente que
ele até ficou surpreso com sua condição. Pois, sendo um enviado do
Senhor, ele conhecia somente o amor divino e amava Rute com esse mesmo
amor, como um pai ama a uma filha ou um irmão ama a uma irmã. Mas o amor
dos homens Dã, a Áspide, o Anticristo, ainda não havia provado, embora
fosse, evidentemente, conhecedor da verdade: tudo o que existe de bom
nos homens vem de Deus е foi rebaixado para ser compreendido por eles...
Somente os pecados são perfeitamente ajustados aos homens. Dessa
maneira, o amor humano é também um rebaixamento do amor divino. Se o
amor divino vem da eternidade --- pleno, sereno, forte e inabalável ---,
o amor humano vem do instante --- fugaz, infiel, surpreendente e
chamativo.

Tássia olhou para Dã, a Áspide, o Anticristo, notando sua face bíblica,
e também sentiu seu coração palpitar, mas não se surpreendeu, embora
também fosse a primeira vez que algo semelhante lhe acontecia. A clareza
no amor é um traço peculiar da ingenuidade pueril... Assim se achavam
eles: o Anticristo estava inquieto e surpreso com sua condição, Tássia
inquieta e nada surpresa com sua condição; Rute numa palidez que nãо era
própria de uma criança; a velha Tchesnokova sentada em um banquinho em
seu quarto, junto à fresta da porta, suspirando e fazendo o sinal da
cruz à maneira dos \emph{velhos crentes}; o gramofone festejava e gania
\emph{tchastuchkas} de Vorónej, e Vera, no ritmo, também dava ganidos e
batia palmas. De repente, Vera saltou da cadeira, com o rosto tão
vermelho quanto o de Tássia --- não de constrangimento como a filha, mas
de excitação ---, e começou a dançar à moda russa, como num casamento,
batendo os saltinhos com frequência, como se um saco de ervilhas se
espalhasse pelo chão, estendendo para o lado os braços, que pareciam
dizer: Eis as nossas vastidões... As estepes, as florestas, os rios...
Ainda não estiveram na Sibéria? Espaços infinitos, ilimitados... E tudo
isso foi povoado pela mulher russa. Mas, para povoar essa vastidão, é
preciso conhecer bem seu ofício. Há duas situações em que a mulher
precisa contar com sua experiência: quando o povo é continuamente
exterminado e necessita de reforço e quando o povo vive em espaços muito
amplos que devem ser povoados. Nesses casos, exige-se da mulher
habilidade, uma habilidade doce, fecunda, melíflua, pois a salvação do
povo está na audácia feminina...

Claro que isso não foi dito ou pensado, no entanto transparecia nessa
dança obstinada que a mulher russa é capaz de realizar. Com um grito
apaixonado e despudorado, que lembrava o gemido de uma mulher no auge do
prazer carnal, abrindo largamente os braços, como se precisasse se
refrescar do calor de um leito junto a uma estufa acesa, Vera se
deslocava sob o som impetuoso das \emph{tchastuchkas} de Vorónej; e
inesperadamente ela se encostou em Dã, a Áspide, o Anticristo, e, com os
seios rijos, apesar de ter amamentado duas filhas, doloridos e
excitados, cravou-se nele.

--- Acompanhe-me na dança, Dã Iákovlevitch.\footnote{О patronímico
  \emph{Iákovlevitch} significa \emph{filho de Iákov} (Jacó).}

De repente Rute, que era Pelágia, a filha adotiva do Anticristo, com o
rosto privado da última gota de sangue, gritou feito uma aldeã,
histericamente, e caiu sem sentidos. No mesmo instante, a velha
Tchesnokova saiu correndo de seu quarto, parou o gramofone e estendeu
uma caneca de água ao Anticristo, que, assustado, inclinou-se sobre a
filha.

--- Vamos para casa, mamãe --- disse Tássia, em voz baixa.

Perplexa com o acontecido e agitada pela dança, Vera estava em pé,
ofegante, e, com a respiração entrecortada, expressou-se de forma
genuinamente russa:

--- Fiz algo de errado? Será que devo me desculpar?

--- Não é necessário --- disse Tássia ---, agora eles não estão
preocupados conosco. Vamos, mamãe.

Sem se despedirem, Vera e Tássia saíram da casa cuja paz haviam
perturbado e, ao caminharem, cada uma perdia-se nos próprios
pensamentos. Nas primeiras noites primaveris, perto da lua cheia,
pensamos e respiramos profundamente. A neve derretendo, as árvores
gestantes... A Rua Derjávin matizava-se de verde com as folhas que os
ramos concebiam, a floresta ao lado mostrava sua importância, e, sob as
sombras das árvores, as velhas faziam a vigilância vestidas em novos
unifomes: xales brancos e roupões de algodão. Nessa época, no princípio
da primavera, os trajes de inverno ainda não tinham sido trocados: os
mais pobres usavam casacos forrados de algodão e os mais ricos pelerines
com gola de pele de raposa. Ao ouvirem passos na escuridão, as
sentinelas da nação perscrutavam com os olhos, sussurravam, emitiam seu
sinal silenciosamente, apenas por meio de sua feição: ``Nós somos
russos, e vocês?'' Não são os Kopóssov passando? Que família leviana...
Ele é um bêbado e um desordeiro, ela é uma indecente; e as crianças, o
que as coitadas podem aprender com eles? Vejam, Vera e a filha ficam até
tão tarde na rua... De onde elas estão vindo? Por acaso não é da casa nº
30, da Tchesnokova, moram os judeus?

Assim, sob o olhar vigilante de velhas silentes, mãe e filha chegaram à
casa nº 2, no finzinho da rua. Andrei não estava bêbado, mas já tinha
tomado alguns goles, e queria dar pelo menos duas surras em Vera por ela
ter voltado tarde; no entanto, ao ver Tássia, ele se conteve e somente
fuzilou a esposa com o olhar.

Vera preparou o jantar, mas ela mesma não comeu, indo se deitar no leito
junto à estufa; assim, quem colocou Ústia na cama foi Tássia,
contrariando o costume da casa, já que era sempre Vera quem levava sua
filha favorita para dormir. Vera estava tão cansada e sentia tanta
indiferença para com a vida que a cercava que dormiu no mesmo instante,
contrariando as expectativas de uma terrível noite de insônia.

Desde então, Vera percebeu em Tássia uma mudança que qualquer mulher,
qualquer mãe, notaria sem dificuldade. No começo, essa ideia atingiu o
coração de Vera como uma punhalada, mas, depois de refletir, ela achou
essa transformação bastante oportuna. Pois, na loucura desmedida, a
mulher sempre se torna astuta e calculista. Desde os tempos do Éden, a
mulher é irrefreável em sua loucura. Não por acaso o primogênito de Eva
foi Caim. Não por acaso Eva atraiu-se irresistivelmente pela sedução da
serpente, assim como não por acaso o Senhor lhe disse:

--- Мultiplicarei a dor da tua gravidez; na dor tu colocarás teus filhos
no mundo; tu serás atraída para o teu marido, e ele te dominará
{[}...{]}\footnote{Gênesis 3:16.}

Já a Adão disse:

--- Porque escutaste a voz de tua mulher e comeste da árvore que te
proibi, dizendo: ``Não comerás dela'', tua terra será maldita; com dor
tu te alimentarás dessa terra todos os dias de tua vida. {[}...{]} Tu
comerás o pão com o suor de teu rosto, até que tu retornes à terra da
qual foste tirado, pois tu és pó e ao pó retornarás {[}...{]}\footnote{Gênesis
  3:17, 19.}

Assim a loucura e a impetuosidade da mulher formaram a base da vida
humana, no momento em que o homem pecador fora expulso do paraíso, sendo
condenado ao trabalho eterno... Quando o homem passou do pão divino para
seu próprio pão, estava acompanhado por sua mulher, Eva, nome que,
traduzido da Bíblia, significa ``vida''...\footnote{O nome Eva, do
  hebraico \emph{hawwá,} significa ``a que tem vida'', ``vivente'',
  ``repleta de vida''.} Se na origem da história humana, desde a saída
do Éden, a loucura e a \emph{vida} definiam igualmente a mulher, será
que existe algo capaz de deter seu desejo sem limites? Eis por que o
terceiro flagelo do Senhor, o adultério, é tão intenso e
incontrolável... Nesse flagelo, a mulher é o carrasco, ainda que ela
mesma venha a sucumbir.

Vera Kopóssova compreendeu que só poderia atingir seu objetivo através
do amor de sua filha pelo judeu, diante da qual ele era impotente, pois
também a amava. Vera o compreendeu e guardou, com astúcia, seu
irresistível desvario para o momento certo...

Nesse meio-tempo, a primavera que perfurmava as margens do Volga chegou
ao fim e o verão jovial despontou --- a floresta floriu, começou a
temporada de frutas silvestres. A astuta mulher percebeu que, nos
últimos tempos, sua filha Tássia passou a receber as carícias de seu
sombrio pai com reserva, embora continuasse permitindo que ele trançasse
e destrançasse seus cabelos --- como na juventude fazia com a esposa.
``Chegou a hora,'' pensou Vera.

--- Tássia --- disse ---, vá no domingo ao pico --- assim chamavam o
topo de um barranco coberto pela floresta ---, as framboesas estão
maduras. Seu pai precisa de uma infusão de framboesas frescas para sua
ferida no peito. Eu mesma iria, mas estou ocupada na oficina, preciso
compensar os dias de trabalho que perdi na primavera, quando Ústia ficou
doente. Vá, Tássia, não haverá outro dia para colher; vão pegar tudo e
não sobrará nada para nós.

--- Está bem --- disse Tássia---, eu vou.

Apesar de ter idade para casar, ela era obediente no dia a dia, embora,
num caso excepcional, se pressentisse algo de errado, pudesse contrariar
seus pais. No entanto, ali não parecia haver nada de errado: sua mãe a
havia mandado à floresta para colher framboesas ao seu \emph{tiátia},
ferido no \emph{front}. Na verdade, Tássia até se alegrou --- quem sabe
o amor entre sua mãe e seu pai voltasse...

--- Está bem, eu vou...

``Agora, daqui por diante, eu não posso errar,'' pensava a astuta mãe em
desvario. E, esquecendo о pudor, ela dirigiu-se à casa nº 30 da Rua
Derjávin, onde havia se portado de forma indecente... Dessa vez, a velha
Tchesnokova recebeu-a sem gentileza.

--- O que quer aqui? --- perguntou Tchesnokova, parada na soleira, sem
deixá-la entrar.

No entanto, Vera notou ao lado o objeto de sua paixão, o judeu, que
escolhia framboesas com a filha na entrada de casa.

--- Dona Tchesnokova --- disse Vera ---, vejo que vocês foram à floresta
colher framboesas... Por acaso foram ao pico? Eu preciso urgentemente
delas, porque farei ao meu marido, que trouxe uma ferida da guerra, uma
infusão de framboesas frescas.

--- Então vá até o pico --- respondeu Tchesnokova ---, há frutas que não
acabam mais... Um ano de boa colheita.

--- Mas aí está a desgraça --- respondeu Vera ---, como estou ocupada,
pois trabalho aos domingos, tive que mandar minha filha Tássia sozinha
para lá. O lugar é afastado, e ela é muito jovem. Tem medo de ir sozinha
e eu tenho medo por ela. Será que algum de vocês não irá ao pico?

--- Não --- disse Tchesnokova ---, nós já fomos. Olhe, já estamos
escolhendo as frutinhas. E do que tem medo? O último urso foi visto lá
faz uns três anos. Os ursos foram tão perseguidos que se embrenharam na
floresta, longe das pessoas.

--- Os ursos podem ter fugido --- respondeu Vera ---, mas homens cheios
de más intenções estão à solta. Para uma moça, um homem assim é pior do
que um urso. Alguém pode ir atrás dela. Deus me livre, mas Pávlov pode
assediá-la...

Pávlov continuava uma figura marcada na cidade de Bor, e com ele, feito
o diabo, as mães assustavam suas jovens filhas que desejavam explorar
lugares distantes.

--- Cuidado, Pávlov vai pegar você.

Mas numa coisa Pávlov havia mudado: se durante a guerra ele não
desprezava nenhuma mulher, agora olhava somente para as mais jovens e,
conforme diziam, até para meninas de nove ou dez anos de idade, que são
mais cheiinhas, robustas e vistosas, pois o próprio Pávlov estava perto
dos trinta... No entanto, tudo lhe era relevado, porque seus
companheiros do \emph{front}, que agora ocupavam cargos importantes,
sempre o tiravam de apuros. Eram os rumores. Mas também havia queixas
contra o desordeiro. Certa vez correu a notícia de que Pávlov fora
flagrado praticando um crime e tinha sido preso... Dois ou três dias
depois, viram-no andando de novo pela rua principal, próximo do cinema,
e também no parque da cidade, perto da pista de dança, vestido em um
casaco de marinheiro, bêbado, forte, bonito, embora um pouco inchado,
assediando as moças, puxando brigas... Revoltados, pais e mães enviaram
uma carta ao jornal local, \emph{Pravda de Bor}. E no jornal tiveram que
ponderar. De um lado, era necessário responder às reivindicações dos
trabalhadores; do outro, não podiam ofender os protetores de Pávlov.
Então, o jornal recorreu a um método já comprovado: a literatura, de
cuja existência nada se espera, pois ela não se ocupa de fatos
concretos, mas de fenômenos gerais do país e, às vezes, do mundo. E a
melhor forma para tal generalização é a poesia. Convenientemente acharam
um versejador, consoantes às lições de Marx: ``a procura cria Rafaéis''.
Certamente, esse poeta não era um Rafael; em compensação, ele era
natural dali e tinha crescido na família de um simples operário da casa
das caldeiras a gás do hospital central de Bor. A mãe, de profissão, era
contadora. Esse poeta, da família Sómov, tinha nacionalidade russa e
sonhava estudar no Instituto de Literatura de Moscou, mas, por enquanto,
se aprimorava sozinho em duas direções --- lírica e satírica. A sátira,
diga-se de passagem, era o que mais o atraía. Assim ele caçoava dе seu
sobrenome, vindo de um peixe, e aproveitava para mencionar outros
análogos. Além de Sómov, dizia ele, temos à vontade Erchóv, Piskarióv,
Kárpov, Ókunev e Schúkin, mas Stiérliadev e Sevriúgov\footnote{Sómov vem
  de \emph{som,} ``bagre''; Erchóv de \emph{iorch,} variedade de perca;
  Piskarióv (sobrenome de personagem de Gógol em \emph{Avenida Niévski})
  de \emph{peskar,} ``gobião''; Kárpov de \emph{karp,} ``carpa''; Ókunev
  de \emph{ókun,} ``perca''; Schúkin de \emph{schuka,} ``lúcio";
  Stiérliadev de \emph{stierliad}, ``acipênser'' (tipo de esturjão); e
  Sevriugov de \emph{sevriuga,} ``esturjão'', os dos últimos são peixes
  muito caros e difíceis de achar.} não se encontram, custam os olhos da
cara... Com essa biografia e com essa orientação criativa, Sómov, nesse
momento, serviu como uma luva ao \emph{Pravda de Bor,} satisfazendo a
demanda... Em seu poema, Sómov, em primeiro lugar, mudou o local da ação
--- transferiu-a de Bor para Moscou, aonde ele mesmo desejava ir havia
tempos. Em segundo, trocou Pávlov por Prókhorov e Stepan por Ivan.
Seguindo os passos do fabulista grego Esopo, Sómov escreveu algo como
uma fábula satírica, que começava assim:

\emph{Certo Ivan Prókhorov podia ser visto}

\emph{Entre os inválidos moscovitas,}

\emph{Este Ivan era um belo tipo, }

\emph{Mas indigno das feridas,}

\emph{Conquistadas no cruel combate na defesa dos soviéticos.}

\emph{Mas sobre isso depois falaremos, }

\emph{Аgora escutem...}

Adiante ele enumerava, em versos, todas as ações indecorosas cometidas
por Pávlov.

O primeiro golpe contra Sómov foi dado por Pávlov, que não se deixara
enganar com a linguagem de Esopo. Do segundo golpe Sómov escapou pulando
a cerca do parque, perto da pista de dança. No entanto, o terceiro
golpe, vindo do comitê local dо Agitprop,\footnote{Agitprop, abreviação
  de \emph{agitátsia i propaganda} (``agitação e propaganda''), seção
  responsável pela propaganda de movimentos revolucionários e da
  ideologia soviética. Foi criado em 1920 e renomeado diversas vezes.}
foi indefensável, sobretudo porque Sómov contava conseguir uma carta de
recomendação do comitê para ajudá-lo a passar no concurso do Instituto
de Literatura de Moscou. Ele ouvira dizer que o concurso era formado
principalmente por judeus, mas, se o candidato fosse russo e tivesse uma
carta de recomendação, todos os direitos estariam a seu favor...

O Agitprop resolveu acusá-lo de nada menos que servilismo e de tentativa
de caluniar os defensores heroicos da pátria que derramaram seu sangue
na guerra... No jornal, portas começaram a bater em polvorosa,
produzindo correntes de ar. Alguns sofreram apenas um leve susto com uma
advertência em sua ficha pessoal, outros foram totalmente privados da
possibilidade de participar da construção cultural futura do país. O
\emph{Pravda de Bor} publicou uma carta de um grupo de veteranos de
guerra --- ``Contra a calúnia em versos de certo Sómov'' --- escrita por
Vladímir (Vilner),\footnote{Vilner, nome de origem judia.} um
funcionário do Agitprop. Dessa maneira, Pávlov, de tão bem defendido, se
tornou um completo insolente, e na cidade de Bor todos tinham medo de
deixar suas jovens filhas andarem sozinhas.

E elas tinham vontade de passear, pois as noites de verão de Bor eram
tão sedutoras que, sem elas, o coração jovem se enchia de melancolia. As
ruas verdes, o ar da floresta misturando-se ao do rio em um elixir
incomparável, as valsas soando da pista de dança, executadas pela
orquestra de sopros da cooperativa das empresas de peixe, e, no alto, o
céu sem Deus dos astrônomos resplandecendo, sob o qual se vive mais
tranquilamente, porque ele alegra a todos, e não obriga a nada... Com
dezessete anos, bastaria respirar a plenos pulmões, sonhar com o amor e
olhar para a lua e para as estrelas... Se não fosse pela presença de
Pávlov... Para uma moça, encontrar Pávlov à noite era uma ameaça
terrível...

Uma noite, Tássia encontrou Pávlov perto do lugar onde, durante a
guerra, ele, ainda jovem, tentara assediar a também jovem Vera, e quase
na mesma hora. Claro que não passou de uma coincidência. Pávlov
agarrou-a sem palavras, em silêncio, e foi um milagre que Tássia tenha
conseguido se soltar, indo correndo para casa, com a blusa rasgada, toda
trêmula, atirando-se ao colo da mãe. Nesse dia, Andrei estava viajando,
tinha ido vender suas peças de madeira em Górki. Ele só voltou dois dias
depois, relativamente calmo; por sorte, era o período de lua nova. Vera
lhe disse:

--- Você sai para beber com Pávlov, diz que ele é um veterano de guerra,
seu amigo, pois saiba que anteontem ele tentou agarrar a sua filha.

A face de Andrei escureceu, e ele disse:

--- Na certa, ele achou que a filha puxou a mãe, que tem a mesma
fraqueza --- e saiu de casa, embora já estivesse tarde.

Após meia hora, ele voltou e disse à Tássia:

--- Não tenha medo, filhinha, pode andar tranquila, ele não vai mais
tocar em você. Eu não sei escrever versos, mas posso fazer seus olhos
saltarem.

E realmente Pávlov não se aproximou mais de Tássia, somente fitava-a de
longe. No entanto, Vera não tinha plena convicção de que sua filha
estava em segurança, pois sabia que o lado viril de Pávlov ficava
atiçado quando ele bebia... Diziam que ele andava pela floresta com uma
espingarda, como se estivesse caçando. Em todo caso, não era apenas
contra Pávlov que as moças deveriam se precaver... Fazia anos que Vera
amava e protegia sua filha. Mas devia ser forte a loucura que a dominou,
já que ela usou a própria filha para seus propósitos... Essa mulher
tinha tudo engenhosamente planejado quando fora à casa de Tchesnokova
dizer que sua filha iria colher sozinha framboesas no pico, onde um
riacho corria de um barranco, que ela iria por volta das sete da manhã,
ou até antes, pois era quando havia menos gente e mais framboesas.

Logo que amanheceu, Tássia se levantou, comeu apressadamente, pegou as
cestas e foi para a floresta, seguida de sua mãe, que havia inventado a
história de estar ocupada aos domingos na oficina. Vera embrenhou-se
cuidadosamente num arbusto, pensando com inquietação: ``Será que Dã
Iákovlevitch virá?''. Ela havia se informado de muitas coisas sobre ele.
Soube que ele viera de algum lugar de Rjév, era viúvo --- a esposa
morrera durante a guerra ---, e agora trabalhava como vigia noturno na
cooperativa das empresas de peixe, profissão rara para um judeu, o que
possivelmente o tornava o mais tolo de seus irmãos, sempre bem
colocados. Vera não sabia que, desde que a filha adotiva apareceu na
vida do Anticristo, ele não podia mais se alimentar unicamente do pão do
exílio, legado pelo profeta Ezequiel, e que, entre as profissões que dão
sustento aos homens, a de vigia noturno era a mais apropriada para Dã,
porque o isolava das pessoas e, sob o céu noturno, o fazia se lembrar de
seu velho ofício de pastor... Vera havia descoberto muitas coisas sobre
o inquilino judeu da Tchesnokova, mas estava longe de saber de tudo.
Evidentemente, antes de mais nada, ela não sabia que Dã Iákovlevitch era
o Anticristo, o enviado do Senhor... De uma coisa ela estava certa, como
uma mulher magoada e uma mãe amorosa: Dã Iákovlevitch e sua filha Tássia
amavam um ao outro, mas não sabiam como se encontrar e não se decidiam
sobre dar esse passo... Uma mulher que sente paixão por um homem que ama
a filha dela encontra-se numa estranha situação. Ora ela sente a carne
da filha como uma parte da sua e se deleita, ora ela sente nessa carne
seu próprio infortúnio e sofre, chegando a odiá-la, como o homem
involuntariamente odeia e amaldiçoa sua mão, seu pé ou sua cabeça quando
doem muito... Assim, Vera ora se deleitava intimamente por meio da
felicidade de sua filha, ora via nessa felicidade o sucesso que outra
mulher lhe roubava. Vera teria caído numa loucura espiritual, e não
apenas carnal, se a inconsciente astúcia bíblica --- graças à qual Eva
fora amaldiçoada pelo Senhor --- não lhe sugerisse que, no sofrimento, o
melhor é confiar na razão, e não no sentimento --- somente na felicidade
é sensato desfrutar dos sentimentos. E, logo que ela entendeu isso,
transformou-se numa decaída comum, dominada por uma paixão desmedida que
deveria ser satisfeita, sem medir esforços... Então ela tramou um
encontro entre o homem a quem ela desejava e não tinha acesso e a mulher
a quem ele amava, Tássia, sua filha.

A mulher astuta seguia sua filha, e sem demora elas chegaram ao pico, um
lugar selvagem e isolado. O barranco estava coberto por árvores e
arbustos, о riacho sussurava, e havia framboesas a perder de vista. Mas
o judeu não estava lá, não tinha vindo, embora sem dúvida tivesse ouvido
as palavras de Vera. Para que a filha não a notasse, Vera sentou-se num
canto afastado e suspirou. Sem suspeitar de nada, Tássia começou a
colher framboesas. Ela já tinha enchido quase metade de sua cesta
quando, de repente, ouviu galhos estalarem e, em meio a uma clareira, o
judeu apareceu e também carregava uma cesta de vime. Tássia levantou a
cabeça e deixou cair sua cesta, e as frutinhas se esparramaram pelo
chão. Movidos por uma força estranha e hilária para os Céus, os amantes
se viram um nos braços do outro. Dã, а Áspide, o Anticristo, cuja
residência terrena se estendia de Hetalon a Emat, e Tássia Kopóssova, da
cidade de Bor, da região de Górki. Sem palavras, sem lágrimas, sem
suspiros, eles ficaram abraçados, cada um envolvendo firmemente os
braços no que era seu: Anticristo em Tássia, Tássia em Anticristo. Eles
estavam em pé, enlaçados, e Vera deitada, escondida atrás de uns
arbustos, e tudo o que fazia parte dela sofria. No entanto, essa mulher
em desvario conseguiu superar o sofrimento dessa paixão decaída,
protegendo sua razão... Enquanto isso, os dois continuavam abraçados,
inertes, até que Tássia, uma moça frágil, sentiu as mãos e os pés
enfraquecerem nessa imobilidade ardente. Então o Anticristo, que agora
sentia em si todas as sensações de sua amada, disse:

--- Você virá amanhã?

--- Sim --- respondeu Tássia ---, depois do trabalho, às seis da tarde.
A oficina de costura fecha às cinco, mas ainda trocarei de roupa...

E eles se separaram sem se beijarem. O Anticristo foi embora depressa,
pois sabia desaparecer instantaneamente, e Tássia ficou colhendo
framboesas para que sua mãe não suspeitasse de nada. Mas sua mãe, tendo
superado sua fraqueza, estava feliz com o acontecido, pois tudo ocorrera
como o planejado.

Assim começou o amor inabalável de Anticristo e Tássia. Certamente, não
era amor divino, com o qual um irmão ama a uma irmã ou um pai ama a uma
filha, mas também não era amor humano, como de um homem por uma mulher.
No entanto, como o Anticristo não podia amar de outra forma e Tássia,
bem ou mal, amava pela primeira vez, eles não ficaram surpresos com
semelhante amor. Continuaram a se encontrar no mesmo lugar, onde o
barranco coberto pela floresta começava, ao lado do riacho... Assim que
via Dã, Tássia dava alguns passos ao seu encontro, como uma sonâmbula
sob a lua cheia, e, no último passo, as forças a abandonavam, os joelhos
cediam; se desse mais um, ela tombaria sem sentidos, mas o Anticristo
nunca a deixava dar esse passo a mais, que talvez fosse a salvação dela.
Enfraquecida, Tássia não caía no chão, mas no peito dele, e eles ficavam
ali, enlaçados, sem palavras, sem beijos. Toda vez o encontro deles era
idêntico, pois só о amor frívolo precisa de variedade. Tássia recebia
tudo do abraço do Anticristo, enquanto sua pureza casta e sua ternura
ajudavam-no a escapar do flagelo da luxúria, ao qual ele, como tudo o
que é terreno, estava sujeito. Assim, numa floresta perto da cidade de
Bor, na região de Górki, realizou-se o sonho eterno de um terceiro tipo
de amor, nem carnal nem ascético...

Os pais da atual revolução sexual, os habitantes da cidade de Sodoma,
que tentaram violar os anjos, procuravam por essa terceira forma de
amor. Os primeiros maridos de Tamar, os irmãos Her e Onã, procuravam-na
também. Mas Her morreu, e Onã, ao entrar na casa da esposa de seu irmão,
espalhou sua semente pela terra, imortalizando seu nome como uma doença
ou como um capricho. Outras perversões também se originaram da busca
pelo terceiro amor,\footnote{Os temos usados (do grego antigo) para os
  três tipos de amor bíblico são: \emph{Ágape} (amor divino),
  \emph{Eros} (amor erótico) e \emph{Philia} (amor fraternal).} nem
masculino nem feminino, mas é impossível encontrar o terceiro órgão ou
criar um \emph{perpetuum mobile} sexual. Até hoje só se tem notícia de
um caso do terceiro amor, que não é carnal nem ascético, tampouco é o
substituto grego, o platonismo, já que o pecador tem talento para a
substituição, como vemos, por exemplo, no cristianismo grego... Mas ali
não havia substituição. Tássia Kopóssova, da cidade de Bor, experimentou
o terceiro amor no verão de 1949... Ela o encontrou porque não o
procurava... É a tal lei bíblica não formulada: quem não procura
encontra, quem procura perde... No entanto, existe ainda a lei do
materialismo dialético, que não precisa obrigatoriamente ser estudada
pelas teorias de Feuerbach, pois foi exposta com muita clareza nesta
canção soviética: ``Quem é alegre ri, quem quer conquistará, quem
procura sempre achará...''.\footnote{Trata-se de um trecho de
  ``Cançãozinha de Robert'' ou ``Canção do vento alegre''
  (\emph{Piéssenka Roberta} ou \emph{Piésnia o vessiólom vietre}), de
  1936, que aparece no filme \emph{Os filhos do capitão Grant}
  (\emph{Diéti kapitana Granta}), baseado no romance homônimo de Júlio
  Verne. A canção tem letra de Vassíli Lébedev-Kumátch
  (1898\emph{--}1949), poeta considerado o criador das canções
  soviéticas de massa, e música do compositor Isaak Dunaiévki
  (1900\emph{--}1955).}

Stepan Pávlov, que durante toda a sua vida rejeitara o ópio bíblico,
queria conquistar Tássia, por isso a procurava em toda parte e a
encontrou no pico, perto do barranco, abraçada ao judeu... Pávlov andava
pela floresta com sua espingarda, assobiando a canção sobre o vento
alegre, na qual estão expostas as bases da dialética. De repente, ele
viu ao longe: Irmãos, o que está acontecendo ali? Os \emph{jides} se
apoderaram de tudo, incluindo as moças... Ele parou de assobiar,
levantou a espingarda, e só Deus sabe o que passou por sua cabeça nesse
instante... Depois reconsiderou e, consumido pela ofensa, tramou algo
mais engenhosо. ``Ninguém nessa cidade ousou me bater, mas o pai dela,
Andrei Kopóssov, me deu uma surra. Muito bem, se um rapaz russo não lhe
agrada, soldado, você terá como genro um judeu, um rato dа retaguarda.''
Arrastando-se à maneira militar, Pávlov aproximou-se e escutou a hora do
encontro de Tássia com o Anticristo no dia seguinte. Pávlov sabia onde
encontrar Andrei, e o achou sem esforço e sem dialética. Estava no
centro de Bor, na frente do cinema, um pavilhão de compensado, chamado
``Danúbio Azul'', ainda que na tabuleta anunciassem: ``Cerveja, bebidas,
petiscos''. Sabe-se lá o que o Danúbio fazia numa cidade às margens do
Volga. Talvez o pavilhão tivesse sido batizado por algum freguês que
atacara Budapeste ou tomara Viena ou Bucareste. O fato é que o pavilhão
de compensado realmente estava pintado de azul. E quem trabalhava ali
como balconista era Niura, com quem Pávlov havia vivido no passado. Ele
costumava, antes de cair na bebedeira, ralhar com Niura por ela não
encher o copo até a borda ou por enganá-lo na conta. Mas ela não cedia,
porque era uma mulher que atingira a igualdade plena de direitos.

--- Você é uma cadela --- dizia Pávlov alegremente.

--- E você é um miserável --- alegremente respondia Niura.

--- Você é uma ladra...

--- E você é um velho safado...

--- Vou meter o cacete na sua mãe...

--- Meta na sua, sairá mais barato...

Nesse ponto, Pávlov, sob efeito do que acabara de ver e suportar, disse
a Niura:

--- E você é uma judia, uma \emph{jidovka}...

Niura caiu no choro.

--- Mas como judia? Por que, irmãos, ele me ofende deste jeito?

Os fregueses se intrometeram na conversa.

--- Deixe disso, Niura, não leve Pávlov a mal... Não arrume sarna para
se coçar... E você, Stiopa,\footnote{Apelido de Stepan.} venha cá, vamos
beber...

Andrei Kopóssov também estava no pavilhão, mas em outro grupo. Começaram
a beber separadamente, mas terminaram juntos. Quando os grupos se
uniram, Pávlov disse a Andrei Kopóssov:

--- Vamos para fora, tenho um assunto a tratar com você...

--- Muito bem --- disse Andrei.

Os outros companheiros de copo, sabendo da desavença entre eles,
tentaram acalmá-los:

--- Deixem disso, rapazes, vocês são dois veteranos de guerra. Que
acerto de contas pode haver entre irmãos eslavos?...

``Irmãos eslavos'', nessa época, era uma expressão da moda, vinda do
\emph{front}. Pávlov respondeu:

--- Eu não vou bater em Andrei, pois sei que ele quebraria minhas
costelas, e minha conversa com ele será amigável.

Saíram. Postaram-se na frente do pavilhão, fumaram \emph{Trud} --- marca
de cigarros do pós-guerra ---, regaram os alicerces do edifício, e
Pávlov, além disso, aliviou a pressão do intestino duas vezes, em alto e
bom som... Quando Pávlov ia começar a falar, um cachorro da rua se
aproximou, mostrando lealdade, e interrompeu seus pensamentos.

--- Ah, miserável --- gritou Pávlov, acertando-lhe uma pedra. O cachorro
deu um gemido estridente e fugiu ganindo.

--- Então, o que você queria? --- começou Andrei, mas, vendo que o outro
hesitava, deu alguns passos para trás, pois, se Pávlov quisesse dar o
troco pelo golpe que recebera antes, Andrei lhe daria outro com o pé,
bem na barriga.

Pávlov notou o gesto e disse:

--- Você tem raiva da pessoa errada, Andriucha...\footnote{Apelido de
  Andrei.} Eu sou um veterano de guerra, assim como você... Tássia é
filha de um combatente, e minhas intenções para com ela são sérias...
Mas há um judeu que passou toda a guerra na retaguarda e está seduzindo
sua filha...

--- O quê?! Que judeu?! --- gritou Kopóssov.

--- Não desconte em mim --- respondeu Pávlov ---, é aquele judeu que
mora com Tchesnokova, na casa nº 30 da Derjávin.

E ele contou o que vira... Andrei ficou vermelho, depois empalideceu e
gritou somente duas palavras:

--- Eu mato!

--- Não se precipite --- respondeu Pávlov, contente por tê-lo atingido
com mais eficácia do que um soco nos dentes ---, você sempre me olha
torto, Andriucha, mesmo quando bebemos juntos. Dá ouvido a fofocas, acha
que eu tive algo com sua esposa. Não nego, ela até tentou grudar em mim,
mas eu a mandei passear, porque sou fiel à camaradagem do \emph{front}.

Andrei rangeu os dentes.

--- Deixe minha mulher fora disso, a conversa não é sobre ela, mas sobre
minha filha.

--- Mas eu tenho um plano para sua filha --- disse Pávlov. --- Quando
eles se encontrarem amanhã no pico, nós pegaremos os dois em
flagrante... De acordo?

--- Sim --- respondeu Kopóssov ---, agora vamos beber mais um pouco...

Eles tomaram mais algumas. Andrei mergulhou num silêncio sombrio e
apático, depois do qual não se sabe se o homem cairá num sono pesado e
profundo ou se matará alguém. Mas Pávlov, ao contrário, esfuziante, sem
reservas, foi tomado de alegria quando a famosa \emph{tchastuchka}
russa, herdada dе seus avós e bisavós, saiu na ponta da língua. Sua voz
agora, a bem da verdade, era um pouco rouca, e não mais aquela voz
agradável de tenor; em compensação, ele dava gritos do fundo do coração:

--- ``Bata nos \emph{jides}, salve a Rússia...'' Chaim fechou a
vendinha... Abraão e Sara, que casalzinho engraçado... O corajoso
Jacozinho na guerra... Nós os defendemos, e eles crucificaram Cristo,
venderam o poder soviético... Nós nas trincheiras, e eles em suas
vendinhas... Durante a guerra, eu não vi nenhum judeu na linha de
frente... Um judeu foi para o \emph{front}, mas de medo ele se matou...

Pávlov gritava tão alto que a polícia reconheceu sua voz familiar,
pensando que ele, mais uma vez, tivesse puxado briga no ``Danúbio
Azul''. Foram até lá: havia barulho, mas nenhuma briga.

--- Por que essa gritaria, Pávlov?

--- E por que os judeus bebem nosso sangue?

--- Pávlov, não perturbe a ordem --- disse o sargento.

--- E eles, podem perturbar? Querem tirar uma filha do próprio pai, um
veterano de guerra...

--- Quem quer, de que pai?... Se tiver provas, escreva uma queixa
formal... De que pai querem tirar a filha? Do que está falando?

--- Do meu amigo... Que foi para a guerra... Que derramou seu sangue...
--- disse Pávlov, enrolando a língua de tão bêbado.

Então Andrei deu um murro na mesa com tanta força que fez quebrar uma
vasilha cara da balconista Niura.

--- Cale a boca, miserável...

--- Estou quieto --- respondeu Pávlov. --- Está tudo em ordem, sargento,
tudo em ordem...

--- Vão para o inferno --- disse o sargento. --- Resolvam vocês mesmos,
mas sem perturbar a ordem...

E ele saiu. Depois disso, Pávlov, já em silêncio, bebeu mais uma, depois
outra, depois cochilou, apoiando a testa na mesa, mas foi despertado por
uma leve brisa noturna, encostando as costas na parede.

Tudo estava em silêncio, a tranquilidade urbana em plenitude. A cidade
de Bor, à beira do Volga, sabia dormir com doçura. Não importa para que
lado se olhasse, não havia nenhuma janela iluminada, nenhum barulho além
do farfalhar das folhas, nenhum movimento além do cintilar das estrelas
e do ir e vir da lua por entre as fissuras das nuvens de chuva.

Quando Pávlov acordava dessa maneira, em meio a essa tranquilidade,
sentia, nos primeiros minutos, algo insólito, algo que ele não podia
entender. Тinha a impressão de que era novamente um bebê e olhava do
berço por uma janela sombria, ou de que lhe surgia a Palavra, endereçada
somente a ele, pois cada pessoa tem sua própria Palavra e, se ela não a
escuta, a Palavra permanece inutilizada no mundo, ou ainda era como se
ele visse pela primeira vez essa cintilação dе estrelas elevadas e a
tensão incomum produzida por essa visão comprimisse sua testa dura de
marinheiro --- parecia-lhe que algo iria jorrar a qualquer momento, como
um córrego de água pura jorrando de uma enorme pedra cinza prisional,
que era a testa de Pávlov para qualquer pensamento puro. No entanto,
bastava se mexer um pouco, respirar fundo, endireitar os membros
dormentes para voltar às suas necessidades prementes, ou seja, antes de
tudo, ele apalpava suas calças. Se estivessem secas ou apenas um pouco
molhadas devido a uma pequena necessidade, ele ia atrás de Váliuchka,
uma jovem enfermeira, de Tánetchka, técnica do departamento de
manutenção predial da cidade, de Ninka, de Aleksandra Ivánovna, ou de
qualquer outra, a escolha era farta. Mas, se as calças estivessem
molhadas e grudentas de ponta a ponta, devido a uma necessidade maior,
ou seja, quando, após seu sono de bêbado, ele despertava com o traseiro
assado --- o que acontecia especialmente no verão, pois nessa estação
petiscavam frutas (maçãs e as ameixas do Volga) ---, se isso
acontecesse, ele se dirigia somente para um lugar --- para a casa de
Aleksandra Ivánovna, a mesma viúva do setor de comércio alimentício que
um dia o seduzira, um jovem ferido de guerra, inaugurando o rol das
mulheres de Pávlov em Bor. A essa altura, a viúva já se aproximava dos
cinquenta anos, mas estava sempre pronta para receber Pávlov, lavá-lo,
alimentá-lo, ajeitá-lo na cama... Agora era verão e, como durante a
noite ele bebera muito e comera muitas maçãs podres e não lavadas,
vendidas pela imprestável da Niurka, sentiu-se, ao despertar,
completamente impelido a ir à casa de Aleksandra Ivánovna. Lá ele dormiu
o resto da noite e parte do dia, pois, antes da caçada noturna ao judeu,
era preciso estar ``novo em folha''.

Kopóssov e Pávlov maquinaram tudo com esperteza: o primeiro movido pela
amargura, o segundo pelo ódio. Kopóssov saiu um pouco mais cedo do
trabalho, Pávlov um pouco mais cedo da casa de Aleksandra Ivánovna, e
eles não se encontraram no próprio pico, mas no triângulo --- havia
também um lugar com esse nome na floresta, mas ninguém lembrava mais por
que era chamado assim... Pávlov estava bêbado, Kopóssov sóbrio, mas
munido de um machado de carpinteiro bem afiado que metera atrás do cinto
militar, sob o paletó.

--- Eles estão lá --- disse Pávlov em voz baixa ---, no lugar de
costume. Eu já os vi, estão ali abraçados, como sempre...

O eslavo fica calado quando a fúria o tortura, acumulando seu ódio para
o momento decisivo. Kopóssov colocou a mão no machado e andou pelo
atalho na direção indicada. Afastou cuidadosamente um arbusto molhado,
porque chuviscava desde manhã, e de fato viu ao longe sua filha nos
braços do judeu... O eslavo fica calado quando tomado pela fúria, mas,
no momento decisivo, pode dar o grito selvagem de seus ancestrais, com o
qual, na época da grande migração dos povos, saqueavam os Cárpatos,
sonhando se estabelecer no Danúbio, e não no Dnieper... Então Kopóssov,
o pai sofredor, com o machado de carpinteiro na mão, deu exatamente esse
grito inarticulado... Já Pávlov gritou algo mais atual e articulado,
mais precisamente: ``Bata nos \emph{jides}, salve a Rússia''.

Assim que os viu, Tássia estremeceu inteira e, pela primeira vez, chorou
de medo nos braços de seu amado.

--- Quem são eles? --- perguntou o Anticristo a Tássia.

--- Meu \emph{tiátia} e seu amigo Pávlov --- chorando e tremendo,
respondeu Tássia.

--- O que eles querem? --- perguntou o Anticristo, pois com ele às vezes
acontecia o seguinte: em momentos extremos, ele parava de repente de
entender a vida que o cercava e de seu âmago surgia aquela aversão
celestial que sentia pelos homens.

--- Corra --- Tássia disse chorando a Dã ---, em mim \emph{tiátia}
apenas baterá, porque me ama, mas você ele partirá ao meio, porque o
odeia. Corra, \emph{tiátia} está com o machado...

--- Ele não nos tocará com o machado --- disse o Anticristo ---, não nos
tocará com nada além de sua mão.

--- A mão dele também é pesada, pode até mutilar alguém --- disse
Tássia, tremendo de medo ---, mas Pávlov prefere esganar.

Nesse meio-tempo, Kopóssov e Pávlov já corriam, deslizando pela grama
molhada do declive, e se aproximavam. Suas faces raivosas já podiam ser
distinguidas. Em Kopóssov, à raiva misturada ao sofrimento tornava-o bem
pouco atraente. Em Pávlov, ao contrário, à raiva misturada à alegria o
fazia lembrar um eslavófilo satírico, fascinante e espirituoso.

--- Aperte-se contra mim, minha querida --- disse o Anticristo ---,
aperte-se com força e não tenha medo... Eles não vão nos tocar com muita
força.

--- Mas como não?! --- perguntou Tássia, quase perdendo os sentidos. ---
Por que não com muira força se estão com tanto ódio?

--- Porque não terão tempo para isso --- respondeu o Anticristo. ---
Assim que nos tocarem, ambos irão morrer...

Ainda que Tássia já estivesse trêmula, o que ela viu à sua frente, perto
de seu rosto, deixou-a febril e fora de si... Os olhos ardentes e
fulminantes da Áspide sobressaíram em contraste aos traços judeus,
suaves e dóceis de seu amado, inflamando-se com o ódio do Inferno, com o
Flagelo Universal de Deus... Tássia gelou e sentiu medo, e não pelo
homem que amava, que parecia ter sumido, mas por seu pai.

--- Não toque em \emph{tiátia} --- ela suplicava, sem saber a quem se
dirigir ---, não toque em \emph{tiátia}...

--- Pena --- disse o Anticristo ---, então serei obrigado a poupar o
outro também. Pois eles tramaram juntos, e, neste momento, não podem ser
castigados separados... Mas depois cada um terá seu próprio suplício...

Kopóssov e Pávlov não conseguiram parar, assim como quando se desce a
toda a velocidade de uma montanha íngreme, e, como que puxados por um
vento misterioso, passaram correndo pelos amantes abraçados... Então
foram levados por entre os arbustos do barranco, arrastados pelas
encostas lamacentas e escorregadias devido à chuva, e jogados no riacho,
que sussurrava pacificamente entre as pedras... Nessa corrida
involuntária, Kopóssov e Pávlov perderam a capacidade de controlar seu
corpo, seus braços e suas pernas.

--- Eh! --- involuntariamente, Kopóssov bateu seu machado de carpinteiro
numa grande pedra úmida. Era um bom machado, mas o cabo trincou.

E Pávlov, bêbado, sentiu as pedras do riacho nos próprios ossos.

--- Eh... Ui... Filho da mãe... A grama está escorregadia... O judeu
levou vantagem com a chuva da manhã.

Logo que Kopóssov e Pávlov desapareceram no barranco, os traços do
Anticristo se apagaram, instantaneamente, e diante de Tássia surgiu seu
amado.

--- Eu vou para casa --- disse Tássia ---, e você também vá embora...
Avisarei quando e onde vamos nos encontrar, porque aqui não será mais
possível... Não tema por mim, até logo --- e eles se beijaram pela
primeira vez, pois, a partir desse dia, o lado sublime desse amor, que
era do terceiro tipo, ficou para trás, e o amor deles tornou-se humano,
com beijos e desejo de variedade.

Quando Tássia voltou para casa, sua mãe a viu e se alarmou.

--- Mamãe, eu me apaixonei... --- disse Tássia e a abraçou, pressionando
sua face contra a dela, de modo que as duas tranças douradas e
volumosas, da mãe e da filha, se uniram.

--- Por quem? --- perguntou a mãe cuidadosa, mas também a mulher astuta.

--- O vigia noturno da cooperativa das empresas de peixe --- respondeu
Tássia ---, que aluga um quarto na casa da velha Tchesnokova.

--- Mas para que tantos rodeios? --- disse a mãe, dissimulada. --- Não
fui eu mesma que a levei pela primeira vez à casa de Dã Iákovlevitch?

--- Ah, mamãe, como ele é doce... --- a filha deixou escapar de forma
inocente, o que fez sua mãe, apaixonada pelo mesmo homem que ela, ser
tomada pelo ódio e pelo ciúme.

--- E se o seu pai descobrir? --- perguntou Vera, brava, como se ela
mesma não tivesse tramado tudo.

--- O \emph{tiátia} já sabe --- respondeu Tássia.

Vera sobressaltou-se, e dessa vez não foi um susto dissimulado.

--- Desde quando?

--- Ele acabou de descobrir.

--- E o que ele fez, bateu nele?

--- Ele queria...

--- Quer dizer que não conseguiu alcançá-lo?

--- Talvez tenha sido isso --- respondeu a filha de forma evasiva.

No entanto, a estranheza da conversa foi dissipada com um chute na
porta, que escancarou, e na soleira surgiu Andrei Kopóssov, cujo aspecto
fez a pequena Ústia cair no choro... De fato, havia com que se assustar.
A roupa molhada, rasgada pelos galhos, cheia de lama, a boca torta, os
lábios mordidos, os dedos brancos das mãos fechadas de antemão. Sem
palavras, Vera se atirou na frente dele para proteger a filha; sem
palavras, ele bateu na esposa com força, mas como de hábito, portanto
não com força o suficiente, e, sem costume, bateu com força demasiada em
Tássia, que no ato começou a sangrar... Vendo sua filha ensanguentada,
Vera gritou selvagemente --- a infeliz mulher compreendeu o que havia
causado e que era culpada de tudo. No mesmo instante, ela entendeu qual
era o castigo do terceiro flagelo do Senhor, o animal selvagem, o
adultério... E ouviu, talvez sem razão, como um fragor em suas têmporas,
a maldição de Moisés contra o adultério:

``Que o Senhor te lançe a maldição e a imprecação diante do teu povo,
que Ele faça teu colo murchar e teu ventre inchar {[}...{]}''\footnote{Números
  5:21.}

Com os braços erguidos, ela se atirou contra o marido, para defender a
filha, mesmo à custa da própria vida, ou para confessar tudo a ele,
diante das filhas. Mas não havia mais quem defender, nem a quem
confessar... Assim que bateu em Tássia, Andrei enterneceu e começou a
soluçar de forma nada viril, ao passo que, quando batia em Vera, era
sempre com fúria e desprendimento. Andrei deitou-se na cama com o rosto
virado para baixo e Tássia sentou-se ao seu lado, pressionando o lenço
contra o nariz quebrado e apoiando a mão na cabeça do pai. Vera entendeu
que era desnecessária ali, e não apenas superou seu remorso como sentiu
mais vontade de levar até o fim o que planejara, para si e para seu
prazer.

--- Vamos passear, filhinha --- disse à assustada Ústia ---, vamos até a
floresta, respirar um pouco de ar puro.

Quando Tássia e seu pai ficaram a sós, ele disse:

--- Filha, você é a minha única felicidade, será que eu seria capaz de
lhe desejar algum mal?

Tássia respondeu:

--- \emph{Tiátia}, eu sei que não era sua vontade, foi Pávlov quem o
instigou... Ele é um canalha...

--- Concordo --- respondeu Kopóssov ---, Pávlov, na certa, é um canalha,
apesar de ser um veterano de guerra... Mas será que não existia outro
rapaz na cidade para você? Por acaso nossa cidade não é russa?

--- \emph{Tiátia,} não posso viver sem ele --- respondeu Tássia, como
uma moça de dezessete anos ---, sem ele prefiro me jogar no Volga...
\emph{Tiátia,} acredite na filha que ama você.

Andrei Kopóssov ficou um tempo calado e então disse:

--- Você pegou isso da sua mãe depravada, a desgraça está aí... Não é à
toa que você se parece tanto com ela.

Assim terminou a conversa, embora ela tivesse começado com sinceridade e
pudesse ter resolvido muitas coisas... Mas nada foi resolvido. Vera
voltou com Ústia e começou a preparar o jantar, e Andrei foi para sua
bancada moldar os barrilzinhos para óleo, as tinas, as batedeiras para
manteiga e outros utensílios de madeira que pretendia levar para a
próxima feira em Górki.

Vera pretendia executar seu plano durante a viagem de seu marido. Era
difícil acreditar que tal plano se realizaria, mas ela também não podia
aceitar que ele fosse irrealizável.

Uma mulher que rejeita o pudor não deve nutrir grandes paixões, e é na
vida trivial que está sua salvação... Vera não conhecia essa verdade e,
se a conhecesse, não seria capaz de cumpri-la... Por muitos anos, ela
vivera resignada com seus desejos femininos mais íntimos, no começo
insaciados por circunstâncias da guerra e depois em razão da sua própria
loucura... Esse desejo macerava nela como uma bebida forte que, num
gole, nos faz cair no esquecimento... Eis a morte, eis o nascimento, eis
a eternidade...

O homem só é capaz de entender a Eternidade rebaixando ao extremo esse
sentimento divino. E o rebaixamento extremo da Eternidade é o prazer...
Somente através do adultério, da luxúria, uma criatura mortal pode tocar
o Eterno, e o amor recíproco enobrece a insignificância infame do homem
diante de Deus... Uma Ideia Elevada está acima do amor recíproco, no
entanto a existência dela são casos raros e já não completamente
humanos, apesar de nascerem entre os homens... A ideia da salvação da
linhagem humana impeliu as filhas de Ló, após a destruição de Sodoma, a
cometer adultério com seu próprio pai, embebedado por elas.\footnote{Gênesis
  19: 30\emph{--}32: ``Ló subiu de Segor e se estabeleceu na montanha
  com suas duas filhas, porque não ousava continuar em Segor. Ele se
  instalou numa caverna, ele e suas duas filhas. A mais velha disse à
  mais nova: `Nosso pai é idoso e não há homem na terra que venha
  unir-se a nós, segundo o costume de todo o mundo. Vem, façamos nosso
  pai beber vinho e deitemo-nos com ele; assim suscitaremos uma
  descendência de nosso pai.'' (\emph{Bíblia de Jesuralém,} ed. Paulus,
  2016, p. 58)} A Ideia do Nascimento do Messias impeliu Tamar a cometer
adultério com o pai de seu marido, Judá, ao se disfarcar de meretriz e
enganá-lo.\footnote{A figura bíblica de Tamar, que, depois de casada com
  Her e Onã, relacionou-se com o pai deles, Judá, foi retomada mais de
  uma vez pelo autor. ``Vendo-a, Judá tomou-a por uma prostituta, pois
  ela cobrira o rosto. Dirigiu-se a ela no caminho e disse: `Deixa-me ir
  contigo!' {[},,,{]}'' (Gênesis 38: 15, \emph{Bíblia de Jerusalém,} ed.
  Paulis, 2016, p. 84), depois disso Tamar engravidou e teve gêmeos:
  Farés e Zara.} O que impeliu Vera ao adultério com o Anticristo,
amante de sua filha, foi ocultado dessa mãe louca e infeliz. Mas, em seu
desvario, ela era astuta e obstinada... Ela sabia que Dã Iákovlevitch
ficava em casa até o meio-dia, pois ele dormia depois de seu plantão
noturno, ou seja, ela precisava achar um momento em que a velha
Tchesnokova e a filha não estivessem em casa, especialmente a segunda...
Se uma filha que ama seu pai tem ciúme até da própria mãe, imagine das
mulheres de fora... E a filha de Dã Iákovlevitch era um caso à parte:
Ruthina era uma menina nervosa, empalidecia à toa e a ponto de desmaiar.
Mas sua aparência não condizia com esse caráter passional, ela parecia
uma aldeã, eis que fenômeno... Na idade de Rute, Vera era igualzinha a
ela e nada compreendia até completar dezesseis anos, quando se casara.
De fato, logo que se casou, aprendera tudo muito rápido... Quanto à
Rute, a julgar por sua palidez e por seus desmaios, ela não tinha mais
nada a aprender, embora não tivesse nem dez anos... Era astuta,
possivelmente já possuía a astúcia de uma mulher. Mas aí era mais
simples: tratava-se de quem seria a mais astuta...

E, em matéria de astúcia, Vera levava vantagem... Ela esperou que
Tchesnokova e Ruthina fossem à feira, seguiu-as até lá e só então bateu
no portão da velha. Dã Iákovlevitch o abriu.

--- Bom dia --- disse Vera ---, minha filha Tássia não está aí?

--- Não --- disse o Anticristo, embaraçado ---, ela não costuma vir
aqui.

--- Então ela só costuma ir ao pico? --- disse Vera, trancando o portão.

A imagem de um gancho ou de um cadeado trancando um portão por dentro
provoca no ato arrepios a uma mulher ávida... Será que Dã, a Áspide, o
Anticristo, vindo de uma terra em que meretrizes frequentemente
ameaçavam os desígnios dos profetas, não entenderia esses arrepios?...
Ele mesmo havia sido exposto ao terceiro flagelo do Senhor perto de
Kertch, com a jovem prostituta Maria, em 1935.

Então o Anticristo disse a Vera:

--- O que você quer? Eu lhe darei tudo, apenas vá embora daqui...

Vera, uma mulher indômita de coração extenuado, respondeu:

--- Não quero nada além de você... Se você não ficar comigo, enviarei
minha Tássia, por quem você se apaixonou, para longe daqui, e você nunca
mais a verá... Ela não se atreverá a desobedecer à mãe, e o meu marido,
o pai dela, me apoiará.

O Anticristo lhe disse através do profeta Ezequiel:

--- Você não é como uma meretriz, pois despreza presentes. Você é como
uma mulher adúltera que, em lugar do marido, acolhe estranhos. Assim, em
sua devassidão, você é diferente das outras mulheres, pois ninguém a
procura; ao contrário, você mesma é que oferece presentes e descobre sua
nudez aos seus amantes.\footnote{Ezequiel 16:31, 32, 34, 36. A passagem
  é uma adaptação do trecho bíblico.}

Vera respondeu no mesmo tom, abatida por sua luxúria melancólica:

--- Faz tempo que não descubro minha nudez a ninguém, nem mesmo a meu
marido, só a você quero descobrir-me. Em relação ao seu presente, ele
não será adornado de ouro e de prata, ele surgiu do meu sangue e vive do
meu sangue... Seu presente será minha amada filha Tássia...

O Anticristo disse:

--- Você sabia, mulher, que, se o Senhor pune uma simples meretriz como
um pecado comum, do tipo que os homens cometem a rodo, sua entrega à
luxúria terá uma punição especial?... A pena dos adúlteros é a mesma dos
que derramam sangue...

Vera, uma mulher russa que povoou com sua habilidade um continente
enorme e desabitado, respondeu:

--- Eu aceito tudo...

Quando a necessidade faz uma habilidade se desenvolver ao extremo, esta
já não pode se limitar ao que é exigido, mas sai à procura de
possibilidades de manifestar-se а favor de suas próprias necessidades...
Toda habilidade que serve ao outro aspira, no fim das contas, a servir
também a si mesma, a existir por si mesma e a deleitar-se consigo mesma.
É essa a habilidade da mulher. E onde existe habilidade aprumada existe
arte: seja a de um poeta, de um carpinteiro ou de uma mulher... Dã, а
Áspide, o Anticristo olhou para Vera:

--- Você sabe como é o julgamento do tribunal do Senhor? --- perguntou
ele. --- Ele irá abandoná-la a uma fúria sangrenta e ao ciúme.

--- Eu aceito tudo --- Vera limitavou-se a repetir, apoiando as costas
no portão trancado, pois suas pernas mal se seguravam.

Dã, a Áspide, o Anticristo, voltou a fitar Vera e viu diante de si a mãe
ainda jovem de sua amada Tássia, da qual poderia ser separado se não
satisfizesse a paixão da mulher que a carregara no ventre... A situação
não parecia completamente humana, mas tudo se misturava na cabeça do
Anticristo, e ele não sabia se ali havia uma Ideia semelhante à de
Tamar... Ele se lembrou das palavras do profeta Ezequiel: ``Eis o que
todos que falam provérbios podem dizer sobre ti: tal mãe, tal
filha''.\footnote{Ezequiel 16:44.} Lembrou ainda que seu último encontro
com Tássia havia terminado com um beijo, isto é, com o rebaixamento do
que havia de sublime e uniforme entre eles. Talvez, no próximo encontro,
ele e Tássia desejassem mais variedade, então o terceiro flagelo do
Senhor, ao qual a decaída Vera queria se sujeitar, poderia cair também
sobre sua filha pura e doce...

--- Está bem --- disse o Anticristo ---, mas lembre-se do que disse o
Senhor: ``Eu farei tua conduta recair sobre tua cabeça''.\footnote{Ezequiel
  16:43.}

--- Eu aceito tudo --- Vera disse, ou melhor, sussurrou.

No quintal da casa da velha Tchesnokova havia um celeiro, desses que se
encontram com frequência em quintais de regiões não completamente rurais
nem completamente urbanas. Antigamente Tchesnokova mantinha lá uma vaca
e pequenos animais, mas ela tivera que abrir mão de tudo pelo imposto
salgado do pós-guerra que taxava não apenas vacas como qualquer pintinho
de uso particular, para liquidar a tendência para a propriedade privada.
Agora no celeiro, coberto pela palha que sobrara dos animais, guardavam
quinquilharias --- a bicicleta do filho mais velho de Tchesnokova, morto
na guerra, e as ferramentas necessárias para casa...

De forma inesperada para si mesma, Rute, que era Pelágia, a filha
adotiva do Anticristo, disse à velha Tchesnokova que precisava voltar
para casa e ver seu pai e, quando, ao chegar lá, não o encontrou em
parte alguma, não pensou inicialmente em procurá-lo no celeiro, pois
achava que ele estivesse na floresta, perto do pico... Fazia algum tempo
que Rute sabia dos encontros de seu pai com Tássia, a irmã mais velha de
Ústia Kopóssova, mas ela guardava segredo e somente chorava baixinho de
noite. Não encontrando o pai, Rute quis ir até seu quarto para
estirar-se na cama e chorar, pois lá ninguém descobriria sua tristeza.
No entanto, um leve ruído no barracão atraiu sua atenção. Rute
aproximou-se cuidadosamente, olhou por uma fresta e viu o que lhe
pareceu o inferno, o que nem mesmo a todos os adultos é dado conhecer.
Ela viu seu pai com uma aparência terrível, e pernas nuas de mulher
levantadas acima dele, como se o engolissem...

Num canto, sobre a palha, Vera se deitava de costas, para melhor
acomodar seu ventre, saciado pela primeira vez em muito tempo,
respirando avidamente, como se enchesse o peito do ar limpo da montanha.
Não, não era uma respiração comum, não eram as inspirações e as
expirações do ventre que sentem o prazer rotineiro da noite e que
geraram Tássia e Ústia... Eram sopros a plenos pulmões, no alto da
montanha, onde o ar é tão limpo que, um pouco mais alto, se tornaria
insalubre para a vida, pois a vida necessita de lufadas de ar de camadas
mais baixas e elementares; e cada inspiração era incomparável, como se
fosse a primeira, e cada expiração era uma doce lembrança do que havia
acabado de acontecer... Porém, quanto mais profunda a inspiração, mais
curta a expiração, e eis que ela não existe mais e, em seu lugar, há uma
inspiração eterna e profunda, como antes da morte, porque o último
suspiro do homem é apenas inspiração, enquanto a expiração é liberada
pelo cadáver...

Rute, uma menina viva que se achava no inferno, viu as pernas da mulher
caírem como mortas, moles e pesadas, sobre a palha podre. A claridade se
apagou. O crepúsculo do dia nublado se impôs, e Rute, na escuridão do
celeiro, distinguiu apenas vagamente as sombras do pai e de Vera em
movimento e, ao prestar atenção no sussurro deles, escutou um riso
baixinho e alegre de mulher... E, assim, Rute repetiu o destino de
Ánnuchka Emiliánova, a mártir infeliz de Brussiány que fora impelida à
maldade pela felicidade alheia. Como foi dito perto da praça ocupada da
vila de Brussiány: ``Quem, na contrariedade, conserva o senso prático da
infância é capaz de cometer grandes maldades''. No mesmo instante, Rute
soube como se vingar do pai, pelo que ele fizera a ela, sua filha amada,
e de Vera, pelo que esta fizera a ele, seu pai amado. A menina sabia,
por Ústia, onde morava a família Kopóssov: bem perto, na casa nº 2 da
Derjávin... Rute foi correndo até lá e viu Ústia sentada no quintal
separando frutinhas silvestres.

--- Onde está sua irmã Tássia? --- perguntou Rute, que era Pelágia.

--- Não é da sua conta --- respondeu Ústia ---, não brinco mais com
você, você é judia e tem muito dinheiro.

Nesse meio-tempo, Tássia apareceu no quintal e disse à irmã:

--- Quem lhe ensinou essas coisas, você não tem vergonha?

--- Que me importa?! --- disse Ústia. --- Ela não veio atrás de mim, mas
de você.

--- O que aconteceu? --- perguntou Tássia, assustada com a aparência da
menina, que estava muito pálida. --- Alguma coisa com seu \emph{tiátia}?

--- Sim, com \emph{tiátia} --- respondeu Rute. --- Vamos até minha
casa...

Fora de si, Tássia correu atrás de Rute, entrou no quintal e se dirigiu
à casa dela, sem nenhuma cautela, já que havia combinado com Dã de se
encontrarem somente na floresta ou em outro lugar afastado.

--- Não é aqui --- disse Rute, e apontou para o celeiro. --- Olhe pela
fresta e veja o que meu pai e sua mãe estão fazendo...

Desconcertada, Tássia olhou pela fresta e viu o que Rute havia visto.
Pois o Anticristo e Vera compreenderam que essa era sua festa terrena,
que ela não se repetiria, por isso se esforçavam para prolongá-la...

Uma mudança instantânea aconteceu a Tássia. Onde foi parar sua doçura de
menina? Eva, a mãe ancestral, que, em sua paixão desmedida, seduzira
Adão e gerara Caim, amaldiçoada por Deus, manifestou-se em Tássia, para
que com o pecado da inveja o do adultério fosse punido...

Ela saiu correndo do quintal da casa nº 30 da Derjávin e foi até o cais
do rio, onde se acomodou em um banco e ficou à espera de seu pai, que
voltaria nesse dia da feira de Górki. Quanto à Rute, ou Pelágia, ela
correu à floresta e andou longamente na esperança de perder-se,
embrenhou-se na mata até cair, esgotada, sobre uns arbustos, gastando o
resto de suas forças em lágrimas.

Tássia ficou sentada no cais até anoitecer, petrificada e sem
pensamentos, ouvindo com indiferença as conversas ao redor e os gritos
das pequenas gaivotas ávidas do Volga, apelidadas ``macaquinhos''. À
noite, seu pai chegou. Tinha vendido todos os utensílios de madeira por
uma boa soma e, mesmo depois de beber, sobrara dinheiro para a farinha e
o toucinho... Ele viu Tássia e se alegrou.

--- Olá, filhinha... Veio encontrar seu \emph{tiátia}?

--- Sim --- disse Tássia ---, pois agora você é meu pai e minha mãe...
Ela despedaçou meu amor... Eu vi mamãe no celeiro da Tchesnokova, sobre
a palha, e não ouso dizer com quem...

--- Então não diga --- o pai respondeu em voz baixa e ponderada, apenas
curvando-se um pouco mais sob o peso dos produtos que trouxera de Górki,
como se a farinha e o toucinho tivessem se transformado em ferro ---,
não diga nada, filhinha... Vamos para casa...

Em casa, Vera os recebeu de forma inusitadamente alegre, chegou a ser
carinhosa com o marido, o que não acontecia fazia muito tempo.

--- Eu aqueci a estufa --- disse ---, querо preparar panquecas de
trigo-sarraceno...

Na casa dos Kopóssov havia uma estufa que, na Rússia, é conhecida por
``russa'', embora ela possa ser encontrada em outros lugares. Mas na
Rússia muitas coisas são chamadas ``russas'' --- as bétulas são
``russas'', mesmo que cresçam mundo afora, o céu também é ``russo'',
apesar de existir em toda parte. Assim, na casa dos Kopóssov havia uma
estufa russa em que se assa pão, cozinha-se \emph{schi}\footnote{Sopa à
  base de repolho com legumes e carne.} em rescaldo numa caldeira de
ferro e se douram panquecas magníficas... Andrei gostava de panquecas de
trigo-sarraceno, mas fazia tempo que Vera não fazia panquecas, sua
especialidade.

--- Muito bem, mulher --- disse Andrei tirando os produtos que trouxera
como se fossem uma carga pesada ---, achei justamente farinha de trigo e
de trigo-sarraceno, e um belo toucinho... Prepare as panquecas com ele,
à moda russa... Panquecas fritas no toucinho são excelentes...

--- Posso fritar no toucinho --- Vera tentava agradar ao marido de todas
as formas e, ao passar por ele, roçou seu cabelo como que por acaso,
mas, na verdade, seduzia-o.

--- Andriucha, vá se lavar --- disse ela ---, acabou de voltar de
viagem...

--- Eu já me lavei --- disse Andrei ---, e você, Tássia, leve Ústia para
passear enquanto as panquecas ficam prontas. O tempo está bom...

--- É verdade --- disse Vera, agitada ---, vá, filhinha, vá passear com
Ústia...

Sem dizer nada, Tássia pegou Ústia e saiu, e bastou o gancho trancar a
porta por dentro para Vera sentir, pela primeira vez em muito tempo,
desejo por seu marido... Ela se aproximou, sentou-se no banco, ao seu
lado, olhou com ternura para os botões de sua camisa militar, desbotada
de tantas lavagens, escorregou a mão por trás de sua gola, mais perto do
corpo do qual, por seu próprio desvario, se afastara por tanto tempo...
No mesmo instante, Andrei agarrou seu pescoço com uma mão e com a outra
sua perna, como se pega uma galinha antes de matá-la, e a levou até a
estufa.

--- O que há com você?... Por quê?... --- gritou Vera, apavorada.

--- O que há comigo só eu sei, mas por quê --- respondeu Andrei ---,
isso você deve saber...

E bateu a cabeça de Vera na quina da estufa, o que fez de imediato sua
trança dourada matizar-se de sangue. Depois disso, ele tentou meter a
cabeça de Vera na estufa quente --- com uma mão empurrava a cabeça dela
e com а outra enfiava palha. Е а palha incendiou... A essa altura,
bateram na porta... Geralmente, quando alguma vizinha ia pedir pão,
batia na porta várias vezes e, sem resposta, ia embora. Mas essa não foi
e batia com força, fazendo o gancho da porta saltar... Como se dessa vez
ela não viesse por vontade própria, mas enviada por Deus... Com as
batidas na porta, Andrei caiu em si e soltou Vera, que, ensanguentada e
queimada, ergueu o gancho da porta e saiu correndo de casa... Nesse
momento, Tássia e Ústia precipitaram-se na direção dela, ambas se
desfazendo em lágrimas... No meio do caminho, Tássia de repente se
lembrou da voz mansa de seu pai e retornou depressa para casa. Andrei
apareceu na soleira, viu os vizinhos indignados em volta, viu a esposa,
ensanguentada e queimada, abraçada por suas filhas em pranto, então
disse:

--- Entrem para que as pessoas não as vejam nesse estado.

--- Você é um tirano! --- gritavam de todos os lados. --- Por que bate
na sua esposa? Será que não existe justiça para você?...

--- Entrem em casa --- voltou a dizer Andrei ---, não vou mais bater...
Eu não me sinto bem...

Nesse intervalo, alguém trouxe uma toalha molhada à Vera, que a colocou
na cabeça ensanguentada, a dor aliviou um pouco e o sangue parou de
jorrar, coagulando. Ela pegou suas duas filhas e voltou para casa.

--- Dê um pedaço pão com sal --- Andrei disse à Vera ---, quero comer.

Ela lhe deu o pão; ele se sentou no banco e comeu uma bela porção,
metade do naco que havia em casa.

--- Agora dê água --- disse Andrei ---, estou com sede.

Vera deu-lhe uma caneca de madeira cheia de água.

Ele a tomou de um fôlego.

--- Dê mais --- disse.

Ela deu-lhe mais... Andrei virou outra caneca cheia.

--- Agora eu vou dormir --- disse ele, subindo na estufa russa.

Passado algum tempo, Vera e suas filhas ouviram roncos.

--- Vamos dormir também --- disse Vera, deitando-se com as filhas no
leito de tijolos junto à estufa... De repente, começaram a ouvir gemidos
de Andrei.

Há diversos tipos de gemidos. Existe o gemido animado, através do qual o
homem chama a atenção para si mesmo, e existe o gemido indiferente para
com tudo o que é vivo, através do qual o homem diz a si mesmo o que não
poderia dizer de outro modo. O outro modo de dizê-lo seria pronunciar as
palavras do Salmo, desconhecidas por ele, nunca ouvidas nem lidas:

``Estou farto de meus lamentos, minha garganta secou, meus olhos se
fatigam à espera de Deus''.\footnote{Salmos 69:4 pela \emph{Bíblia de
  Jerusalém,} e 69:3 nas outras bíblias consultadas.}

No entanto, há momentos e circunstâncias em que só é possível dizer isso
por meio de gemidos. Mesmo se Andrei Kopóssov estivesse com o Livro dos
Salmos nas mãos, não diria com mais precisão o que disse com seus
gemidos, pois uma série de passagens da Bíblia russa foi traduzida de
maneira desajeitada.\footnote{As primeiras traduções da Bíblia para o
  russo apareceram apenas no século XIX. Antes disso, eram utilizadas
  versões no eslavo eclesiástico. A primeira edição completa da Bíblia
  (\emph{sinodálnyi}) saiu em 1876, sendo, em 1939, atualizada por um
  missionário polonês (B. Götze (1888\emph{--}1962)).} Assim, o Salmo nº
87, versículo 4, necessário para o homem que agonizava, foi assim
vertido: ``Pois minha alma está saturada de desgraças, e minha vida se
aproxima do inferno'', enquanto, pelo original, teríamos: ``Pois minha
alma está saturada de ofensas, e minha vida se aproxima do túmulo''.

A vida e a morte de Andrei Kopóssov, da cidade de Bor, da região de
Górki, antiga província de Níjni-Nóvgorod, é a prova da imprecisão dessa
tradução russa da Bíblia. Entre uma alma ``saturada de desgraças'' e uma
alma ``saturada de ofensas'' há uma grande diferença. Descer ao inferno
por causa de desgraças seria injusto, mas ofensas levam invariavelmente
ao túmulo... Essa é apenas uma das imprecisões da versão russa da
Bíblia. Felizmente, o gemido agônico não exige tradução.

--- É melhor ir ver o que há com seu pai --- disse Vera.

--- Eu não posso, tenho medo --- respondeu Tássia e, subitamente, sentiu
uma pontada no estômago, que fez um calafrio passar por todo o seu
corpo.

Então Vera se levantou, afastou a cortininha e viu o marido deitado de
lado. Os olhos estavam abertos, e seu olhar era incomumente profundo e
ausente.

--- Você não está desconfortável nessa posição, Andrei? --- perguntou
Vera.

Andrei não respondeu, ficou com o mesmo olhar profundo e ignoto, fixo
num canto do quarto, onde a escuridão antes do amanhecer fervilhava...
Vera começou a virar o marido, para que, deitado de costas, o acomodasse
melhor, e, no momento em que o virava, ele morreu. Mas Vera não entendeu
isso de imediato. Nem quando a língua de Andrei, tão grande que não se
sabe como cabia em sua boca, lançou-se para fora, como uma onda, e no
ato se encolheu, como que acionada por uma mola, Vera entendeu o que
ocorrera. Só quando as pernas de Andrei se esticaram sozinhas e seus
olhos se fecharam, ela entendeu e chorou por seu marido morto, sentada à
sua cabeceira.

A pequena Ústia acordou e caiu no choro, não pela morte do pai, da qual
ainda não sabia, mas porque sua mãe estava chorando... Pois, toda vez
que seu pai batia em sua mãe e esta chorava, Ústia caía no choro em
seguida... Já Tássia, nos primeiros minutos após o acontecido, não
conseguiu se aproximar do pai devido às pontadas no estômago, sentindo
calafrios. E ela passou esses minutos no quintal, no frio noturno...

Parecia que essa terrível noite não teria fim, mas seu fim chegou. De
manhã, tudo já havia voltado ao normal. Ústia foi levada para a casa dos
vizinhos, e Vera e Tássia lavaram o corpo de Andrei numa tina para lavar
roupa. Pela primeira vez na vida, Tássia viu o corpo nu de seu pai e
sentiu, além do pesar de filha, um constrangimento desagradável. Já Vera
não via esse corpo nu fazia muito tempo e sentiu, além do pesar de
esposa, repulsa e uma espécie de horror... Quando começaram a vestir
Andrei, não acharam meias decentes, pois ele só usava roupas puídas,
bebendo todo seu dinheiro. Vera foi obrigada a enfiar nos pés do marido
seu único e belo par de meias de seda, que foi cortado para que
parecesse de homem. Porém, vestido em seu terno de festa e colocado no
caixão, Andrei Kopóssov adquiriu para a esposa e a filha a aparência de
um morto familiar, que, conforme as crenças pagãs, deve ser lembrado
apenas pelo lado bom, enquanto tudo o ele que fez de ruim deve ser
esquecido... É o defunto em nome do qual fazem juramentos, cuja sombra
sagrada consola na tristeza, cujo corpo em decomposição por vezes torna
uma esposa mais fiel do que quando vivo e repleto de seiva e força
viril. Vera sabia que agora seria fiel àquele corpo que se decompunha
até desaparecer, e Tássia sabia que seria fiel aos desejos do pai morto,
já que não o fora quando ele estava vivo... Não será com o judeu que ela
perpetuará a linhagem dos Kopóssov... Essa linhagem será russa, da
região do Volga... Será a linhagem de Vesselóv, um motorista de segunda
classe, o filho de Serguéievna, a velha sentinela. E Tássia terá dois
filhos --- Andrei Vesselóv e Varfolomei Vesselóv... Claro que ela ainda
não via tão longe, nem conhecia seu futuro sobrenome, mas sabia que
seria russo.

Quando estavam reunidos em volta do defunto --- pois toda a Rua Derjávin
compareceu à cerimônia, com exceção de Tchesnokova, da casa nº 30, uma
\emph{velha crente} ---, Pávlov apareceu de supetão, bêbado,
naturalmente. Ele se aproximou do caixão, sentou-se ao lado, olhou para
o morto e subitamente pegou sua mão.

--- Andriucha, o que você tem, irmão?... Vamos beber alguma coisa...

O corpo jazia em silêncio, sem se mexer, como uma estátua. Pávlov soltou
a mão, que caiu sobre o peito do morto.

--- Eu vou --- disse Pávlov ---, senão sou capaz de chorar --- disse ele
e saiu.

Nesse meio-tempo, as sentinelas da nação, velhas instaladas nos bancos
da cidade, comentaram:

--- Na casa nº 2, dos Kopóssov, ele mesmo morreu... A esposa indecente
acabou com ele... E, na nº 30, a filha do judeu desapareceu, faz dois
dias que a estão procurando. O judeu perdeu totalmente a cabeça, porque
sua filha, ao que parece, se afogou no Volga...

E Serguéievna acrescentou de sua parte:

--- Tomara que todos eles percam a cabeça e se afoguem no Volga...

O filho de Serguéievna, Serguei Vesselóv, o futuro continuador da
linhagem dos Kopóssov, fato de que ele ainda nem suspeitava, ouvindo as
palavras de sua mãe, riu e disse:

--- Mamãe, se todos eles se afogassem no Volga, os peixes iriam morrer
de tanto fedor... Parece que a judia não se afogou no rio, mas se perdeu
na floresta... Ela foi vista lá pela última vez...

--- Nada mal --- respondeu Serguéievna ---, a floresta também não é
coisa que se despreze... Não se consegue sair de lá sem conhecê-la bem,
e na mata densa, num lugar mais afastado, um urso pode fazê-la em
pedaços ou um homem bem-disposto pode ultrajá-la... Nada mal...

Com efeito, o Anticristo procurava sua filha como um louco havia dois
dias, pois nem a ele era concecido saber tudo, mas somente o que o
Senhor desejava que ele soubesse. Ele não sabia onde Rute estava, mas
sabia por que ela desaparecera e suportava o sofrimento desmedido e
religioso de um pai judeu que ama incondicionalmente sua criança. A
bondosa Tchesnokova afligia-se com ele, mas se afligia à maneira russa,
com um sentimento inconsciente da infinitude do espaço e do povo. Por
mais que se perca, o fim jamais chegará.

--- O que se há de fazer, querido? --- disse. --- Deus dá, Deus tira.

Mas, quando cada alma e cada palmo de terra são considerados, a dor da
perda é imensurável... E, em seu pesar, o pai judeu, o Anticristo, о
enviado de Deus, não queria acreditar no desígnio divino. E disse,
através do profeta Jeremias, o que o justo Jó consagrara em seu destino
e cuja vulgarização é a base do ateísmo:

--- Tu serás justo, Senhor, ainda que eu o questione em juízo, no
entanto falarei contigo sobre a justiça. Por que o caminho dos ímpios é
afortunado e os pérfidos prosperam?\footnote{Jeremias 12:1}

O Senhor respondeu ao Anticristo, que havia perdido sua filha Rute, da
mesma forma que este respondera à Maria, que perdera seu irmão Vássia.
Ele respondeu através do profeta Isaías:

--- Eu me revelei aos que não perguntavam por mim. {[}...{]} ``Aqui
estou! Aqui estou!'', dizia Eu a um povo que não me chamava por Meu
nome...\footnote{Isaías 65:1.}

O Anticristo compreendeu o que já sabia, mas, no infortúnio, havia
esquecido. Quem não escolheu, mas foi escolhido, não pode fazer
perguntas ao Senhor. Deve fazer perguntas a si mesmo e esperar as
respostas do Senhor.

Ele foi de novo à mata densa, de onde fazia pouco que voltara, molhado
pela umidade da floresta... Quanto mais o Anticristo se afastava de
lugares habitados, mais ele era atingido pela tristeza, mais ele ansiava
por solidão, como um animal que se esconde de todos para morrer, pois
essa grave questão deve ser solucionada longe das pequenezas do
cotidiano... O bom é viver entre seus semelhantes, mas morrer longe
deles. O Anticristo entendeu que não fora enviado pelo Senhor a Bor para
amaldiçoar, mas para ser amaldiçoado. Somente o Senhor pode amaldiçoar
sem ser amaldiçoado.

Dã, a Áspide, o Anticristo, sentou-se num cepo podre, coberto de musgo,
e pôs as mãos na cabeça. Enquanto isso, sua filha Rute, que era Pelágia,
estava por perto, a dez minutos de caminhada por entre as árvores
derrubadas pela tempestade e por entre os arbustos espinhosos e
enredados em teias de aranha. Еra o terceiro dia em que ela vagava pela
floresta, alimentando-se de frutas silvestres e de folhas, bebendo em
poças de água e dormindo encostada em troncos de árvores. Ela estava
quase sem voz de tanto gritar e seu vestido se achava em pedaços,
rasgado pelos galhos das árvores... Nesse instante, ao sair para uma
clareira aquecida pelo sol, ela decidiu descansar um pouco, deitou-se e
adormeceu, esgotada. Seu sono era pesado e a levou para longe dali, mas
só ao despertar ela entendeu aonde fora. Assim, dormindo, ela foi
surpreendida por Pávlov, um homem bem-disposto, sempre pronto a ultrajar
meninas na floresta, especialmente se fosse uma menina judia, conforme
as esperanças da velha Serguéievna.

Depois do enterro de Andrei Kopóssov, Pávlov bebera, fora à missa em
memória do morto e chorara, mas não visitara nenhuma mulher, de modo que
havia muita energia acumulada nele... Arrastado para fora da cerimônia
de tão bêbado, ele dirigiu-se à floresta com sua espingarda, levemente
mais sóbrio. Entrou numa brenha onde nunca tinha estado. E, como uma
miragem diante de um homem sedento no deserto, diante de Pávlov apareceu
uma menina dormindo, totalmente indefesa... Ele notou que suas pernas
nuas eram bem constituídas para sua idade e seus seios em crescimento
frescos e firmes. O esgotamento e o medo que Rute experimentara nos dias
e nas noites passados na floresta uniram-se à tranquilidade de seu sono
puro, e a expressão confiante da menina seduzia o homem e o animal
embrenhados na floresta... Com um bramido inarticulado, Pávlov se atirou
contra ela e, quando ele se inclinou, ela abriu os olhos. Se Pávlov
pudesse voltar a si, lembrar-se dos momentos em que ele mesmo acordara
ao pé de uma cerca, na solidão e na tranquilidade, à espera de uma
Palavra, dirigida somente a ele, a qual estava à sua procura pelo
mundo... Mas a Palavra não encontrou Pávlov... Ele até se alegrou com o
despertar da judia, e o violador sentiu uma alegria raivosa diante da
fraqueza de quem ele odiava.

--- Ah, minha pequena Sara, farei um estrago aí na frente! --- gritou
Pávlov, em êxtase. --- Vai ficar dodói... \emph{Azohen vei}\footnote{``Ah,
  que desgraça!'', corruptela do iídiche.}... --- como todo eslavo com
um desejo irrefreado, ele aprendera duas ou três expressões em iídiche,
principalmente tristes, que lhe pareciam especialmente engraçadas e que
sua língua de eslavo de fato reproduzia de forma cômica. ---
\emph{Azohen vei}... --- repetiu Pávlov e de repente sentiu em suas
costas uma respiração quente e úmida...

Eram duas ursas que haviam saído da mata densa, tal como acontecera
perto de Belém, quando duas ursas bíblicas surgiram de uma floresta para
castigar, atendendo ao chamado do profeta Eliseu, as crianças maldosas
que o ofenderam. Mesmo que Pávlov levasse a espingarda no ombro, ela era
fajuta e as ursas estavam muito perto. Seria horrível se elas lhe
quebrassem as costelas e ainda pior se o despedaçassem... E Pávlov caiu
no choro. Não conseguia mexer nem as pernas nem os braços, ficou ali
parado, chorando e suplicando.

--- Eu quero viver --- nem ele sabia a quem dizia isso, se à menina que
queria violar ou às criaturas selvagens e insensatas.

Аs duas ursas espicharam-se para Pávlov e o cheiraram... E ele não
agradou... Elas cuspiram em seu rosto, uma depois da outra, cobrindo o
ex-combatente da marinha de saliva cheia de muco. Então cheiraram Rute,
lamberam suas mãos e foram embora, desaparecendo entre os arbustos. Com
sua partida, Pávlov perdeu o equilíbrio que o pavor havia lhe propiciado
por um instante. Caiu tal como se postava, todo esticado. Assim caem as
pessoas paralisadas... Semiparalisado e privado da faculdade de falar,
ele arrastou-se pela floresta em busca de pessoas, de vida. Às dezenove
horas em ponto do dia seguinte ele chegou à estrada, e, como, por sorte,
ali era fácil encontrar homens russos, o inválido conseguiu se explicar
a alguém, pedindo para ser levado à casa de Aleksandra Ivánovna, a viúva
de cinquenta anos. Pois sua fala voltou aos poucos, enquanto sua
virilidade o deixou para sempre.

Aleksandra Ivánovna, a funcionária do setor de comércio alimentício,
estava pronta para recebê-lo, fosse como fosse, pois ela era a única de
suas mulheres que o amava. Desde então, ela começou a levá-lo todos os
dias, numа cadeira de rodas, para respirar ar fresco, dizendo aos
conhecidos:

--- Suas velhas feridas de guerra reapareceram... Acabaram com Stiopa...

E Rute, através do Sinal revelado na perda da virilidade de Pávlov,
entendeu que era a profetisa Pelágia,\footnote{Há mais de uma versão
  para Pelágia entre os cristãos. Uma delas diz que Pelágia foi uma
  bailarina bela e sedutora, Margarida, nascida em Antioquia. Após ouvir
  um sermão do bispo Nono, ela batizou-se, repartiu seus bens, abandonou
  sua cidade e foi para Jerusalém, onde, disfarçada de homem, viveu até
  o fim da vida. O dia de Santa Pelágia é comemorado em 8 de outubro.}
nascida na vila de Brussiány, perto da cidade de Rjév. Ela se lembrou de
que isso lhe fora dito no sonho interrompido por Pávlov. Da mesma forma
que Eliseu recebera o espírito do profeta Elias, Pelágia recebeu o
espírito de seu pai, o Anticristo. E foi Pávlov quem contribuiu para
isso. Assim, nem Pávlov havia sido criado em vão pelo Senhor.

Pelágia pôs-se a caminho e rapidamente encontrou seu pai, que se
sentava, desanimado, num cepo podre. E disse:

--- Eu estou aqui...

O Anticristo se atirou à filha, viva e intacta, e eles se abraçaram com
alegria.

O profeta Jonas, que passara três dias no ventre de uma
baleia,\footnote{Desobedecendo às ordens divinas, o profeta Jonas
  recusou-se a ir à cidade de Nínive (hoje se situaria no Iraque) para
  amaldiçoá-la, tendo sido castigado por Deus, que ``{[}...{]}
  determinou que surgisse um peixe grande para engolir Jonas''.
  (\emph{Bíblia de Jerusalém,} Jonas 2: 1, p. 1631).} purificou a cidade
de Nínive do pecado com sua maldição. Também por uma maldição,
Anticristo, que havia pecado, se purificou.

E Disse Dã, a Áspide, o Anticristo:

--- Perdoa-me, Senhor.

E sua filha, a profetisa Pelágia, respondeu:

--- O Senhor é a minha força e o meu canto.\footnote{Salmos 118: 14.}

Ela agora sabia quem era seu pai, mas ele não sabia quem era sua filha e
pensava que Rute tivesse aprendido as palavras dos profetas com a
\emph{velha crente} Tchesnokova. Dã, a Áspide, o Anticristo, lhe disse:

--- Rute, minha filha, você cresceu nessas paragens, mas agora nós
devemos deixá-las.

--- Eu não me importo --- disse a profetisa Pelágia ---, onde você
estiver será meu lugar.

O Anticristo se alegrou, porque o Senhor o enviava para a próxima
cidade: Vítebsk, onde, em 29 de setembro de 1949, Kukharienko, Aleksándr
Semiónovitch, nascido em 1912, seria condenado como um inimigo perigoso
do poder soviético, sendo transferido para os campos de trabalho
correcional de Burepolómski. Mas isso já é o preâmbulo da parábola
seguinte.

4

Existe uma questão russa eterna e, pode-se dizer, fundamental: ``Quem
está arruinando a Rússia?''. Assim que um homem russo se faz essa
questão, ele olha para os lados, se não se tratar, claro, de um literato
genuinamente russo. Mas, se for duplamente russo, ou seja, um homem e um
literato russo, ele não olhará ao redor, mas, colocando-se essa questão,
cravará os olhos na toalha de mesa suja de vinho, como se procurasse
nela a resposta a esse velho enigma russo.

Na época de Vladímir, o Cristianizador, o homem russo, um pagão, esteve
na iminência de adotar a fé muçulmana. Na \emph{Rus}\footnote{Principado
  medieval que originou a Rússia, a Bielorrússia e a Ucrânia.} teriam
sido erguidas mesquitas russas, de pedra e de madeira. Mikula
Selianínovitch usaria um turbante e Iaroslavna\footnote{Mikula
  Selianínovitch, \emph{bogatyr,} guerreiro eslavo de grande habilidade
  e força, que aparece em várias narrativas bélicas (\emph{bylinas}) do
  ciclo de Nóvgorod. Símbolo da mulher fiel, Iefrosinia Iaroslavna, pelo
  que se supõe, foi esposa de Ígor Sviatoslávitch (1151\emph{--}1201),
  príncipe de Nóvgorod-Siéverski, e filha de Iarosláv Osmomysl
  (1135\emph{--}1187). Sua figura aparece em ``O pranto de Iaroslavna'',
  um dos trechos mais poéticos de \emph{O canto da campanha de Ígor}
  (\emph{Slovo o polku Ígoreve}).} um xador, e não haveria questões
fatais, tão peculiares ao cristianismo. Mas, no último momento, contra a
vontade da maioria dos nobres e de todo o povo, Vladímir\footnote{Vladímir,
  o Grande ou São Vladímir (960\emph{--}1015), príncipe de Nóvgorod e de
  Kíev, foi responsável pela cristianização da \emph{Rus}, até então um
  país pagão. Vladímir retirou os embaixadores de Corásmia, uma poderosa
  região muçulmana situada na Ásia Central, onde se negociava a adoção
  do islamismo, e adotou a religião cristã.} retirou a delegação de
Corásmia e a enviou para Bizâncio. Assim, por mero acaso, em lugar do
islamismo russo, surgiu o cristianismo russo. No entanto, será que a
geografia da Rússia é tão cristã? Ao leste, dos montes Urais até a
cordilheira de Altai, a Rússia cai na Ásia; ao sul, da Turquia aos
Bálcãs, a Ásia avança sobre a Rússia; e o Volga, uma relíquia nacional,
desemboca também na Ásia...

Eis a imagem da jovem Rússia --- não a nórdica, meditativa, severa... No
leste, rosto redondo e cabelos castanhos e, no sul, pretos; no leste,
olhos claros e estreitos e, no sul, escuros e encobertos por firmes
maçãs dо rosto asiáticas. Faz trezentos ou quatrocentos anos que nesta
Rússia de maçãs dо rosto salientes existe a ideia nacional. Se
remontássemos às origens, antes dessa geografia, quando eslavos do
leste, expulsos das margens do Don, povoaram as redondezas do rio
Dnieper, um negociante e viajante árabe olharia para seus olhos
irriquietos de nômade e diria: ``Se este povo aprender a montar num
cavalo, será o flagelo da humanidade''. Uma fala profética, clara e nada
enigmática. O enigma mais profundo se dá quando não existe enigma algum.
O poço mais profundo é aquele que nem foi cavado. A cultura da Rússia
está ligada à Europa, mas sua civilização à Ásia. Isso é um problema,
mas não um enigma. O problema deve ser resolvido por meio de um árduo
trabalho espiritual, que desviaria a ideia nacional dinâmica. Já o
enigma não precisa de solução, mas é possível refletir sobre ele num
estado de contemplação para o homem russo, como descreveu Gógol em
\emph{Almas mortas}: ``Você não pensa em nada, e os pensamentos caem
sozinhos em sua cabeça''.\footnote{Na verdade, a frase surge em \emph{O
  capote,} conto escrito por Gógol em 1842.} Sem dúvida, foi exatamente
nesse estado que surgiu a questão fatal, ainda não resolvida: ``Quem
está arruinando Rússia?''. Esta questão não foi inventada, caiu sozinha
em sua cabeça...

A bem da verdade, uma resposta foi, ao que parece, encontrada com a
ajuda das mentes habilidosas do povo e da \emph{intelligentsia}
extrema-nacionalista. Quem leva a Rússia à ruína, supostamente, estaria
claro... A resposta veio sozinha e sem esforço... Mas quem ajudaria o
russo nisso? Mais uma questão... O homem russo se familiarizou com essas
questões; desde o início dos tempos, o ortodoxo foi habituado a elas
através de suas aflições e de seus infortúnios.

--- Salve-se --- gritam para ele.

--- Mas como? --- geme ele, cansado, esgotado.

--- É óbvio: bata!

Apesar de cansado, o russo sempre achará forças para bater em alguém.

--- Nesses aqui?

--- Nesses mesmos... E naqueles também...

--- Por esses Deus me perdoará, mas aqueles são dos nossos... É como diz
a canção popular:

\emph{Sai, sai daí, garoto, }

\emph{Vai ver o mundo, vai,}

\emph{Há gente à beça, }

\emph{No meio, tua mãe e teu pai.}

\emph{Diz aí, garoto, }

\emph{Quantos mataste?}

\emph{Dezoito ortodoxos, }

\emph{E duzentos e setenta} jides\emph{.}

\emph{Pelos} jides \emph{serás perdoado, }

\emph{Pelos russos jamais...}\footnote{A versão trazida pelo autor é uma
  mistura de duas antigas canções populares russas: ``O garoto na
  solitária\emph{'',} citada em \emph{A lírica dos blatares}
  (criminosos) (ed. Fênix, 2001)\emph{,} de Fima Jiganets, e
  \emph{``}Vivia em Odessa...'', citada no trabalho de Anatóli
  Gueórguievski (1888\emph{--}1955), \emph{Os russos no Extremo Oriente,
  ensaios folclóricos e dialetológicos,} publicado em Vladisvostok entre
  1926 e 1932.}

--- Você também será perdoado pelos russos... Veja a Rússia, um mar de
russos, um povo incontável. Por mais que se tire água, ele não
diminuirá. A mulher russa trabalhou e povoou esse território. E os que
foram tirados desta imensidão não serão mais vistos...

Realmente, se a juventude russa ou as gerações futuras, ainda não
nascidas, lessem as terríveis memórias de testemunhas oculares, poderiam
pensar: ``Como era terrível a vida russa naquela época... Como as
pessoas podiam viver então?''. Não havia nada de terrível, e as pessoas,
em sua maioria, viviam normalmente. Viviam até com alegria, com fé na
justiça, e o clima russo contribuía para isso. O clima não permitia o
calor extremo, е, no frio intenso, aqueciam-se batendo palmas. Em 1937,
por exemplo, a primavera foi magnífica, tudo floriu prematuramente, e as
pessoas começaram a se recompor dos tormentos da coletivização, e no
verão de 1949 a fome dо pós-guerra já havia passado. A situação ia mal
apenas para uma esmagadora minoria, que podia ser contada pelos dedos,
isso se cada dedo equivalesse a um milhão... Mas а Rússia não é tão
apertada como a Europa. Na Rússia, não é comum se contarem pessoas pelos
dedos. Aqui, desde o início dos tempos, vive-se num mundo de dar inveja.
No entanto, não se pode invejar o russo por tudo. E por que é assim? Eis
o destruidor da Rússia, aquele que se esforça para acabar com ela...

--- E onde está ele?

De novo 75. Voltamos à velha questão: ``Quem está arruinando a
Rússia?''. Olha-se para os lados, para a toalha suja de vinho, com as
mãos apoiadas nas bochechas... Os órgãos de repressão tentam resolver
esse enigma nacional à sua maneira.

E assim entrou no rol dos destruidores da Rússia Aleksándr Semiónovitch
Kukharienko, o encarregado pelo setor de estocagem de grãos da região de
Vítebsk. Ele foi condenado no verão de 1949.

``Isso não é uma questão divina,'' pensou o Senhor, ``aqui, ao que
parece, não há ninguém para ser condenado à maldição divina. Eles mesmos
é que se condenam e entender isso não é difícil, mesmo para a razão
humana limitada.''

Mas o homem sofre de uma doença: quer entender o que não pode, mas o que
pode não quer... Essa doença espiritual do homem pecador vem dо quarto
flagelo.

``Lançarei uma doença aqui, uma doença mortal,'' decidiu o Senhor.
``Essa doença corrói tanto o espírito como a alma e o corpo.''

Assim o Anticristo, o enviado do Senhor, foi iniciado na parábola da
doença do espírito.

\textbf{\\
Parábola da doença do espírito}

Aleksándr Semiónovitch Kukharienko era de nacionalidade bielorrussa, o
terceiro irmão russo... A enumeração das três primeiras posições da
nação é imutável, segundo o grau de importância eslava, mas daí em
diante não existe tanta rigidez. Às vezes em quarto lugar aparece o
georgiano, às vezes o uzbeque, o moldávio ou mesmo o cazaque, às vezes o
georgiano é o sexto, atrás do estoniano, e o cazaque o sétimo, na frente
do moldávio... A partir do quarto lugar, tudo é decidido pelo acaso, mas
os três primeiros lugares eslavos são imutáveis. O bielorrusso é o
terceiro, a partir do russo, e vem logo atrás do ucraniano... Isso não é
ruim, se levarmos em conta que, desde sempre, o bielorrusso vive em
terra infértil... No século XIX, um conhecido difamador da monarquia
absolutista escrevera: ``Nosso mujique de Orlóv chegou a tal ponto que
se tornou tão miserável quanto um bielorrusso...''. Pois o princípio da
igualdade não foi trazido do Ocidente, a ideia de que ele fora criado
por \emph{slogans} da Revolução Francesa é uma ilusão. O princípio da
igualdade está nas raízes da consciência nacional russa. ``Todos bem ou
todos mal, eis a justiça.''

Na cidade de Vítebsk viviam duas importantes famílias de funcionários
públicos: Kukharienko e Iarnutóvski. A família Kukharienko era feliz,
mas a família Iarnutóvski ia de mal a pior. Os Kukharienko, Sacha e
Valiucha,\footnote{Sacha é apelido de Aleksándr e Valiucha (ou Vália) de
  Valentina.} se conheceram nas florestas de \emph{partizans}\footnote{Durante
  a ocupação alemã, formaram-se grupos de resistência armada
  (\emph{partizans}) nas florestas da Bielorrússia. Vítebsk, a nordeste
  da Bielorrússia, foi ocupada por nazistas de 11 de julho de 1941 a 26
  de junho de 1946. Lá criaram um gueto para isolar os judeus da cidade
  e de regiões próximas (em 1939, mais de 20\% da população de Vítebsk
  era formada por judeus).} da Bielorrússia, onde, contrariando o
regulamento e em caráter excepcional, nasceu Nínotchka, enquanto
Míchenka\footnote{Nínotchka é apelido de Nina e Micha (Míchenka) de
  Mikhail.} já nasceu em Vítebsk, após a libertação do país. Logo depois
da guerra, o quadro de administração e de direção da Bielorrússia era
formado sobretudo por \emph{partizans}. Os mais influentes se esforçavam
em nomear para cargos de chefia seus iguais, combatentes que
sobreviveram... Foi assim que Kólia Iarnutóvski, especialista em minas,
assumiu seu cargo de chefia. Casou-se com a secretária da procuradoria
municipal, Svetlana. Eles se casaram por amor, mas não se acertaram.
Mesmo assim trabalhavam arduamente e não se envolviam em nada amoral.
Segundo registros do cartório, tiveram dois filhos. De modo que poderiam
não se dar conta de que estavam privados da felicidade se não fosse pela
feliz família Kukharienko... No fundo, eles nem sabiam em que consistia
a felicidade dos Kukharienko, mas sabiam que eles, Sacha e Valiucha,
eram felizes... E, de fato, por que tanta felicidade? Por que perto da
casa dos Kukharienko cresciam grandes flores amarelas? Por que, nos dias
livres, Sacha Kukharienko gostava de andar de bicicleta vestido numa
camisa de seda laranja, com sua filha Nínotchka acomodada na frente? Por
que, no verão, Valiucha usava uma blusinha branca e uma saia cinza e um
lenço branco amarrado na cabeça e, no inverno, botas cromadas е um
casaquinho com gola felpuda ruivo-acinzentada? Por que os Kukharienko
comiam \emph{gаluchkes}\footnote{Espécie de nhoque ucraniano, feito à
  base de farinha, ovos e creme azedo, cozido em leite ou caldo.} com
colheres de madeira coloridas? Tudo isso Svetlana tentou copiar, até
aprendeu a cozinhar \emph{variénikes}\footnote{Espécie de ravióli
  recheado normalmente de batata, mas também de queijo, cogumelos,
  frutas vermelhas, etc.} bielorrussos de batata melhor do que
Valiuchka. Mas nos Iarnutóvski não havia a felicidade que os Kukharienko
demonstravam às pessoas em volta. Além disso, as duas famílias viviam em
condições materiais idênticas e bastante razoáveis para a Bielorrússia
do pós-guerra, saqueada e queimada. E as duas famílias trabalhavam do
mesmo modo para superar essa destruição.

O bielorrusso, desde tempos antigos, ama sua miserável mãe Bielorrússia,
assim como o ucraniano ama sua mãe Ucrânia, rica e repleta de
\emph{kulakes}, e o russo ama sua grande e espadaúda Progenitora. Mas o
bielorrusso sempre amou com um amor menos marcante, com certa frieza, à
maneira polonesa ou lituana, embora sem o pitoresco polonês... No
nacionalismo bielorrusso não há a paixão ofendida ucraniana, o impulso
briguento russо ou a teatralidade católica polonesa... E isso não é nada
surpreendente. A terra bielorrussa, em sua maior parte, é uma planície
pantanosa, coberta por florestas densas e permeada por rios que causam
enchentes na primavera. O solo é pouco fértil; os pântanos, os charcos,
os alagamentos na primavera e o lamaçal intransitável no outono
dificultavam, em tempos antigos, o convívio da população. A ideia de
unidade, indispensável para o nacionalismo, foi expressa ali sem muita
clareza e, em grande parte, tomada de empréstimo da
\emph{intelligenstia,} de algumas mentes polaco-lituanas; essa ideia não
amadureceu nas entranhas do povo, que por muito tempo preservou em
lugares mais isolados, como na região pantanosa de Pinsk, não uma
consciência nacional, mas tribal. Nem o arrogante civilizador
greco-romano, nem o cruel saqueador mongol mostraram grande interesse
por esses pântanos miseráveis. Em compensação, sofreram a invasão de uma
massa de judeus desalojados, expulsos dos lugares mais abastados da
nação, onde entenderam a lei de Darwin bem antes de ela ser formulada.
Essa peculiar expansão judia, sem facas mas com trouxas, quando o
desalojado chegou à morada do miserável, contribuiu para o aparecimento
dе uma ideia de unidade nacional autêntica, que, graças à tutela
polaco-lituana,\footnote{De 1569 a 1795, o Reino da Polônia e o
  Grão-ducado da Lituânia se uniram, criando a República das duas nações
  ou a Comunidade polaco-lituana. Com um sistema semifederativo e
  semidemocrático, o reino se alastrou pela Bielorrússia e Letônia e por
  partes da Ucrânia, Estônia e Rússia.} rapidamente alcançou padrões
mundiais. O resto do nacionalismo da Rússia Branca\footnote{A Rússia
  Branca é a Bielorrússia (\emph{biélyi, ``}branco''), enquanto a
  Pequena Rússia a Ucrânia (\emph{Malorrossia, mályi,} ``pequeno'').} é
pouco conhecido, mas é improvável que tivesse se desenvolvido seriamente
numa direção antirrussa, proibida. Por isso, na Bielorrússia, prisões
sob acusação de nacionalismo eram bem menos frequentes que na Ucrânia.
No entanto, elas existiam, e foi justamente assim que a feliz família
Kukharienko e a infeliz família Iarnutóvski se arruinaram.

Se Kukharienko, o encarregado pelo setor de estocagem de grãos, por
acaso, ou por vontade de Deus, acabasse na prisão, deveria ser por
crimes na área agrícola. No entanto, foi preso por uma questão cultural.
Um dia, numa vila, ele achara um velho livro de Buratchók-Boguchévitch,
\emph{O pífaro bielorrusso}.\footnote{Frantsisk Boguchévitch
  (pseudônimo: Matsei Bugatchók) (1840-1900) foi um dos pais da nova
  poesia bielorrussa, uma das bases para a construção da ideia nacional
  bielorrussa, a qual começou a se consolidar no fim do século XIX.} No
livro diziam que ``a língua bielorrussa é tão humana e nobre como a
francesa, а alemã ou qualquer outra. Será que só podemos ler e escrever
numa língua estrangeira?''.\footnote{O prefácio de \emph{O pífaro
  bielorrusso,} antologia poética de 1891, é considerado um manifesto do
  nacionalismo bielorrusso.} Kukharienko dirigiu-se com o livro ao
instituto pedagógico local, onde soube pelo catedrático Bogdanóvitch que
Buratchók-Boguchévitch era o fundador da moderna poesia bielorrussa. О
encarregado pela estocagem de grãos da região de Vítebsk descobriu ainda
que, além de Buratchók-Boguchévitch, contribuíra para o renascimento da
cultura bielorrussa Ianka Lutchina,\footnote{Ianka Lutchina (pseudônimo
  de Ivan Neslukhóvski) (1851\emph{--}1897) foi um poeta nascido em
  Minsk cuja lírica mesclava elementos do realismo e do romantismo.
  Escrevia em russo, bielorrusso e polonês. A antologia poética \emph{O
  feixe,} escrita em bielorrusso, foi publicada em 1903.} que, a partir
de 1889, passara a escrever poemas em bielorrusso e publicara a
antologia \emph{O feixe}. O catedrático Bogdanóvitch, que, por
coincidência, era um parente distante do escritor pré-revolucionário
Bogdanóvitch, ocupou-se com prazer do interesse pela ideia nacional
bielorrussa vindo de um importante funcionário público, e ainda um
ex-\emph{partizan,} e lhe pediu para organizar uma exposição.

Aleksándr Semiónovitch Kukharienko realmente era um grande aficionado de
tudo o que era bielorrusso, de modo que pudesse saborear pratos e
canções de seu país. Afinal, a canção e a comida nacionais estão a um
passo da cultura nacional. Só que, em 1949, a questão cultural tornou-se
a mais perigosa, assim como era a questão dos detonadores de minas em
1942. Kukharienko enviou a sugestão de Bogdanóvitch a Iarnutóvski, que
trabalhava justamente nessa perigosa esfera da construção do socialismo,
no Agitprop. Iarnutóvski, que não deixava de se surpreender com a
estranha felicidade da família Kukharienko --- chegara a diminuir a
frequência de suas visitas, a conselho da esposa, Svetlana, a secretária
da procuradoria ---, resolveu consultar as instâncias superiores. Como
resultado da consulta, Bogdanóvitch foi preso. O catedrático tentava
mostrar sob um ângulo supostamente positivo a luta de classes dos
proprierários de terra poloneses contra a Rússia, que considerava a
Bielorrússia como sua conquista cultural... Bogdanóvitch foi preso no
dia 2 de junho e, na manhã do dia 19, na hora do desjejum, vieram atrás
de Kukharienko...

Na véspera, a família Kukharienko estivera na floresta, pois famílias
felizes desfrutam de uma alegria especial não apenas reunidas em casa,
mas também fora de casa, em plena comunhão. Assim, caminharam por uma
trilha da floresta. Sacha levava pela mão sua esposa Valiucha e
Nínotchka seu irmãozinho Míchenka.

A floresta bielorrussa não é como a da região do Volga ou a da Ucrânia.
A floresta para o bielorrusso é como o rio para os habitantes do Volga
ou como o campo para os ucranianos. Por séculos, a floresta alimentou e
vestiu o bielorrusso. A vegetação da floresta, com sua safra de
frutinhas silvestres e cogumelos, não era um complemento para o
bielorruso, mas o pão de cada dia. Desde tempos imemoriais, o
forasteiro, ao passar por esses lugarejos, engasga-se com seu pão
ressequido com arenque e cebola amarga cor de ferrugem... Mas eis as
árvores provedoras bielorrussas... Árvores fortes, sólidas, como paredes
de uma casa que aquece e protege... Eis as clareiras ensolaradas,
cobertas de frutas...

--- Parem, crianças --- disse o pai ---, olhem ali, uma serpente...
Vejam, Nínotchka e Míchenka... Enquanto um bielorrusso não matar uma
serpente, não será um bielorrusso de verdade, assim diz o nosso povo...
Nínotchka, pegue uma pedra, aproxime-se e mate a serpente.

Valiucha se inquietou:

--- Para onde você a está mandando, e se a serpente picar?

--- Como, picar? --- respondeu Sacha. --- Será que uma bielorrussa terá
medo de serpentes? E eu estarei ao lado...

Então o pequeno Micha começou a chorar e disse:

--- Não é preciso matar a serpente, ela também quer viver, ela também
tem filhotes.

--- Ah, meu filhinho --- disse o pai ---, será possível ter pena de uma
serpente? Veja o que ela está fazendo agora. Ela está se aquecendo ao
sol. E quando uma serpente se aquece ao sol, ela o suga. É por isso que,
depois do verão, o sol enfraquece. Agora pense em quantos répteis
existem na terra e em quantos verões despontam na terra. Todo verão, uma
multidão de répteis suga o sol e, se você for um homem, matará a
serpente. Este é o seu dever. Mas, se ainda por cima for um bielorrusso,
não terá o direito de passar ao lado de uma serpente viva. Essa é a
nossa crença nacional.

Ele se inclinou e pegou uma pedra com uma mão e com a outra conduziu
Nina atrás de si, cuidadosamente... Enquanto isso, a serpente se aquecia
sobre a relva da floresta e, em sua alegria, perdeu a astúcia diante de
seu inimigo secular, esquecendo por alguns instantes a
maldição-advertência lançada pelo Senhor no tempo do Éden, o paraíso,
depois de Eva ter sido seduzida:

``Pelo que tu fizeste, serás amaldiçoada diante de todos os animais
domésticos e de todos os animais selvagens; tu andarás sobre teu ventre
e comerás pó pelo resto de tua vida. Criarei inimizade entre ti e tua
mulher e entre tua semente e a dela; ela golpeará tua cabeça e tu
picarás seu calcanhar {[}...{]}''.\footnote{Gênesis 3:14, 15, que assim
  se inicia: ``Então o Senhor Deus disse à serpente {[}...{]}''.}

Nínotchka atirou uma pedra na cabeça da serpente, amolecida na relva
quente, desprovida da astúcia por causa do prazer que sentia; e a pedra
a acertou em cheio, esmagando-lhe cabeça. Ela começou a se debater, pois
achava que a maldição do Senhor não tiraria seu direto de viver, já que
o homem e, especialmente, a mulher também foram amaldiçoados. A
serpente, que havia pouco se deliciava com o sol divino e comum, o qual
todos sugam e enfraquecem, continuava a se debater. No entanto, ela foi
cortada em pedaços com uma pá de sapador pelo pai e pela filha, e Micha,
desfazendo-se em lágrimas, foi levado por Vália para longe desse
espetáculo. A família feliz não viu, porém, que mais duas serpentes, uma
grande e outra pequena, encarava-a por trás de uns arbustos com olhos
gélidos cheios de ódio.

--- Bravo! --- disse o pai e deu um beijo na filha. --- Agora você se
tornou uma verdadeira bielorrussa, porque cumpriu a crença popular ao
matar a serpente com suas próprias mãos.

Foi desse modo singular que o domingo de 18 de junho ficou guardado na
memória de Nínotchka...

No dia 19, perto das nove horas, quando a família Kukharienko estava
comendo \emph{galuchkes} no desjejum com suas colheres coloridas,
apareceram dois sujeitos vestindo sobretudos de couro, apesar da manhã
ensolarada.

--- O senhor está preso...

Tudo isso acontecia não sem medo, mas como se fosse algo corriqueiro.

--- Mostrem o mandato --- disse Kukharienko.

O mais magro, bigodudo e aparentemente mais importante, resmungou e
enfiou a mão no bolso com má vontade, mostrando o mandato... Kukharienko
viu que a lei fora observada, o mandato fora assinado pelo procurador
Vassíli Makárovitch. E, ao ver a assinatura de Vassíli Makárovitch, аo
lado de quem ainda na antevéspera estivera sentado numa reunião, sentiu
um aperto no coração... Nas famílias felizes, os corações estão unidos,
existe entre eles uma ligação invisível. Sacha sentiu um aperto no peito
e Valiucha, até então petrificada, começou a chorar.

--- Não chore, Valiucha --- disse Sacha, beijando sua boca lambuzada do
creme azedo dos \emph{galuchkes} ---, não chore, assustará as crianças.

Mas era tarde. Nínotchka desfez-se em lágrimas, agarrando-se ao pai, e
Míchenka, ao contrário, escondeu-se num canto.

--- Nínotchka --- disse o pai ---, ontem na floresta você matou uma
serpente com as próprias mãos, do que tem medo? Seu pai logo voltará. Eu
vou lhe comprar uma boneca e voltarei.

--- E traga-me da viagem um canivete --- pediu Míchenka.

--- Não --- disse o pai ---, canivetes são afiados, você cortará o
dedinho. Míchenka, eu lhe trarei algo melhor.

Embora Míchenka ainda fosse pequeno, de algum modo entendeu que seu pai
não estava apenas saindo, mas prestes a partir. Ele não havia pedido
simplesmente para trazer algo, mas para trazer algo da viagem. Quanto a
Vália, a esposa amorosa de Sacha, no início ela compreendeu a situação
bem menos do que as crianças, pois, ao adquirir experiência de vida,
aprendera a não compreender o evidente. No entanto, fez tudo o que era
esperado de uma esposa durante a prisão do marido. Ela rapidamente
juntou suas coisas e despediu-se sem gritaria, para não assustar as
crianças; e, ao dirigir-se até o carro em que Sacha entrava, viu um
mundaréu à sua volta, no qual ela era tão pequena, tão insignificante...
Nínotchka também viu tudo isso pela janela --- a bem da verdade, ela não
reparou no mundaréu estranho, mas viu a rua e guardou a imagem de seu
pai partindo, de suas costas...

Nesse mesmo dia, os Iarnutóvski, Kólia e Svieta,\footnote{Apelido de
  Svetlana.} também foram presos, uma hora antes, e seus filhos foram
enviados para o orfanato de Vítebsk... Dessa forma, ficou claro para
Valiucha que até nisso sua família tinha sido mais feliz. Ela não sabia
por quanto tempo desfrutaria dessa felicidade, mas decidiu
aproveitá-la... Vestiu as crianças depressa, fatiou o pão, colocou num
pote de meio litro o mingau de semolina ainda quente, encheu de bombons
uma lancheira usada a tiracolo e disse:

--- Crianças, vamos para a estação.

Eles chegaram à estação.

--- Nínotchka --- disse sua mãe ---, você vai viajar com Míchenka para
Moscou, para a casa da tia Klava.\footnote{Apelido de Klávdia.}

--- E você? --- perguntou Nina.

--- Eu ficarei aqui, perto do seu pai --- respondeu Valiucha. ---
Nínotchka, você já é uma menina crescida, no caminho não conte a ninguém
o que aconteceu ao seu pai, apenas fique de olho em Míchenka.

De repente, Valiucha sentiu sua cabeça girar: ela lembrou que, durante a
ocupação alemã, havia na periferia de Vítebsk um campo de concentração,
de onde as mulheres, através do arame farpado, pediam pão aos
transeuntes ou que eles levassem seus filhos embora. Valiucha e sua
amiga Stássia, depois morta num destacamento, afastaram com esforço o
arame farpado e pegaram um menino de dois anos e dois meninos de seis, e
ainda uma menina de oito... Os alemães começaram a atirar das torres de
vigilância, por isso elas não conseguiram pegar as outras crianças que
as mães, empurrando-se, tentavam lhes entregar... Em perigo, uma mãe
normalmente aperta seu filho contra si, no entanto, às vezes, para
salvá-lo, ela o entrega a alguém, afastando-se, confiando num acaso
perigoso, pois, em situações desumanas, o dia é mais terrível que a
noite, uma rua repleta de pessoas mais temível que uma floresta cheia de
lobos, o familiar mais ameaçador que o desconhecido... O que sentiam
essas mães amorosas, empurrando umas as outras, esforçando-se por
afastar de si suas crianças? Se sentissem nesse momento tristeza e
sofrimento, não seriam capazes de fazer isso... Não, numa situação
desumana, o coração arruína o homem e tudo o que é humano. Somente o
instinto desumano da fêmea, não maternal, pode salvar... Por isso
Valiucha beijou rapidamente Míchenka e Nínotchka e colocou-os num vagão
do trem que partia para Moscou; quando o trem pôs-se em marcha e as
crianças se afastaram, Vália, em vez de amargura, sentiu alegria... Ela
caminhou alegre por algumas ruas e, apenas ao entrar em um parque sujo e
deserto, começou a lastimar. Lá perto ficava um pavilhão onde se
anunciava: ``Cerveja. Bebidas''. Vália entrou e tomou um copo de vodca.

O instinto desumano que a ajudou a mandar tão habilmente suas crianças
amadas para longe também a ajudou a dominar o horror que invadia seu
coração. A vodca não a livrou do horror, mas tornou sua alma pequena,
mais fraca, e almas fracas suportam grandes desgraças com mais
facilidade. Depois de beber, Valiucha foi procurar Kulechóv, no
NKVD\footnote{Acrônimo de \emph{Naródni Komissariat Vnútrennikh Diel}
  (Comissariado do Povo para Assuntos Internos). Substituindo o OGPU, o
  NKVD era responsável pela segurança do Estado e tinha sob controle a
  polícia secreta soviética.} local, pois o conhecera ainda no movimento
dos \emph{partizans}. Lá ela discutiu com alguém na recepção. Depois, ao
andar pela rua, as pessoas a evitaram. Passados três dias, ela foi
presa. Assim sucumbiu uma família feliz.

Em Vítebsk, Sacha Kukharienko ainda tratava o investigador de uma
maneira informal, mas em Minsk começaram a bater nele, a pisoteá-lo, a
esmagar seus dedos com os saltos dos sapatos, e, com a ajuda dessas
infrações da legalidade socialista, revelaram-se os pormenores de seu
nacionalismo bielorrusso e de sua ligação com а Gestapo no período da
guerra. O inquérito foi concluído e em 29 de setembro realizou-se o
julgamento... Enquanto Sacha Kukharienko tentava provar sua inocência,
procurando a verdade e exigindo justiça, tudo lhe era muito difícil, e
ele raramente pensava nos filhos e na esposa. Quando mais relaxado,
esquecendo seus próprios méritos e as injustiças dos outros, tudo ficou
mais fácil, muito mais fácil, e ele já não pensava em nada além da
esposa, Valiucha, e dos filhos, Nínotchka e Míchenka.

Eis o que aconteceu com as crianças... Nina e Micha chegaram bem a
Moscou --- no início comeram pão, mingau de semolina e bombons e depois
compraram chá e bolacha do condutor. Além disso, os passageiros vizinhos
lhes deram embutidos. Assim que Nínotcka se viu sozinha, tornou-se uma
mulher independente, mostrando a mesma desenvoltura de Maria, da vila de
Chagaro-Petróvskoie, da região de Khárkov, que em 1933 havia viajado
sozinha, sem a sua mãe, levando seu irmão Vássia, é verdade que em
outras circunstâncias... Nina contava aos passageiros que eles não
tinham pais fazia tempo e foram criados por uma mulher estranha, mas
agora descobriram uma tia, Klava, em Moscou... Em geral, as crianças
sabem mentir e gostam mais de fazê-lo do que os adultos. Pois cada
mentira é uma espécie de brincadeira. O pequeno Micha também participava
da brincadeira da irmã, e assim eles chegaram a Moscou. Um passageiro
bondoso, um velho moscovita, levou as crianças até o endereço que Vália
Kukharienko anotara em quatro pedaços de papel, em caso de perda, e
colocara na lancheira com os bombons. Nessa lancheira, que Vália
pendurara no ombro de Nínotchka antes de partir, havia um coelhinho
bordado. Vália não avisara Klávdia por telegrama, para que a viagem das
crianças não despertasse atenção e também por saber que Klávdia não
ficaria contente com a chegada delas, de modo que o melhor seria que
tudo acontecesse de forma repentina. Havia muito que ela não trocava
correspondências com a irmã e não gostava de seu marido, de
nacionalidade judia.

Klávdia era muito mais velha do que Vália, fora muito bonita quando
jovem e se casara, ainda antes da guerra, com um crítico de arte
moscovita que conhecera em Ialta. O sobrenome, nome e patronímico desse
crítico: Ívolguin, Aleksei Ióssifovitch. Aleksei, Klávdia e seu filho
Saviéli --- um adolescente doentio, resultado de uma mistura de sangue
malsucedida, pensativo, pоrém mais inclinado para alucinações do que
para reflexões --- moravam num grande apartamento em um dos melhores
lugares de Moscou --- o bulevar Tverskói. O defeito desse apartamento
consistia em se localizar no térreo. Mas isso era apenas metade do
problema, porque em edifícios antigos as janelas eram colocadas bem no
alto, quase na altura do primeiro andar das novas construções, e embaixo
havia um porão onde também morava gente. A desgraça maior, no entanto,
era que o apartamento era comunal e, além da família Ívolguin, que
ocupava três aposentos, o escritório de habitação instalara num dos
quartos menores a zeladoria. Desse modo, apesar de terem somente um
vizinho, precisavam dividir com ele a cozinha, a sala de banhos e o
telefone, o que era constrangedor. Os Ívolguin escreveram inúmeras vezes
a várias instâncias, pediram requerimentos a diversos institutos
culturais em que Aleksei Ióssifovitch colaborava, no entanto, sem
sucesso. Na zeladoria que ficava no apartamento dos Ívolguin morava o
zelador tártaro Akhmét, que afrontava todos ``com um canivete'', do qual
uma vez Aleksei Ióssifovitch se safara trancando-se no toalete. Se ele
se tivesse se trancado na sala de banhos, a coisa terminaria mal. A
porta dela era fraca, podre, e o gancho mal se segurava.

--- Vá procurar Fadéiev\footnote{Aleksándr Fadéiev (1901\emph{--}1956),
  escritor, vencedor do prêmio Stálin (1946) com o romance \emph{A jovem
  guarda} (\emph{Molodaia gvárdia})\emph{.} Ocupou diversos cargos na
  União dos Escritores Soviéticos.} --- dizia Klávdia аo marido, brava
---, só ele poderá nos ajudar a nos livrarmos disso.

--- Como eu poderia me dirigir ao secretário-geral da União dos
Escritores Soviéticos por uma bobagem dessas? --- respondeu Ívolguin,
gesticulando. --- Sem isso, eles já falam de mim...

--- Que falem --- respondeu Klávdia, gesticulando como ele, pois esposas
dе judeus com frequência se tornam plasticamente parecidas com seus
maridos quando o casal vive sozinho, e não com uma grande família
eslava, em que o marido judeu é como um filho adotivo...

--- Mas eu nem o conheço --- disse Ívolguin.

--- Como não conhece? --- respondeu Klávdia. --- No funeral civil de
Mikhoels\footnote{Salomon Mikhoels (1890\emph{--}1948), ator e diretor
  soviético de origem judia, condecorado Artista do Povo da URSS (1949).
  Durante a Segunda Guerra Mundial, foi presidente do Comitê Judaico
  Antifascista, iniciativa do governo soviético, mas, em 1948, foi morto
  por ordem de Stálin com outros membros do comitê.} ele o cumprimentou.

--- Fadéiev cumprimentou todo mundo, estava muito transtornado ---
respondeu Aleksei Ióssifovitch.

--- А mim ele não cumprimentou --- disse Klávdia, enchendo a conversa de
repetições e disparates, com os quais ela poderia vencê-lo.

--- A você não, mas a mim sim --- gritou finalmente Ívolguin, nervoso.

--- Não grite --- gritou Klávdia, também nervosa ---, vocês adoram
gritar.

--- Quem somos ``nós''? --- Ívolguin corou, е não de raiva, mas de
vergonha e indignação, como acontecia cada vez que ouvia a palavra
``judeu'', não importava onde nem por que; era como se fosse flagrado
fazendo algo impróprio, do mesmo jeito que seu filho Saviéli fora
recentemente flagrado por Klávdia no toalete... Saviéli, naquele
instante, ficara igualmente vermelho de vergonha.

A aparência de Ívolguin era ordinária, porém o sobrenome notável, e não
era um pseudônimo, mas o que levava no passaporte --- seu pai, um
intelectual de antes da revolução, um patriota russo, o trocara a
contento, como ele mesmo dizia: ``o gato em iídiche tornou-se o
passarinho em russo''...\footnote{O sobrenome do pai seria Katz,
  ``gato'' em iídiche, enquanto Ívolguin vem de \emph{ívolga,}
  ``papa-figo'', ave da família dos oriolídeos.} Com o sobrenome
Ívolguin, Aleksei teve sorte, somente o patronímico atrapalhava um
pouco. Muitos nem sabiam que Aleksei Ióssifovitch era judeu. No funeral
civil de Mikhoels, onde discursaram Fadéiev, Zúbov e outras
personalidades russas ilustres, Aleksei Ívolguin disse algumas palavras
também. A palavra ``judeu'' não foi mencionada no funeral, e Aleksei
Ióssifovitch só se sobressaltou duas vezes...

No entanto, Akhmét, o zelador, de algum jeito descobriu que o vizinho
que о hostilizava era judeu.

--- \emph{Jid} --- gritava Akhmét, bêbado ---, vou fazê-lo em pedaços...

--- Vá falar com Fadéiev --- dizia Klávdia ---, o tártaro ameaça aleijar
você e Saviéli, ou você não se importa com seu filho? Você ainda não se
interessou em procurar um bom psiquiatra para ele --- e, não se
contendo, fez o marido sentir-se mal: --- Já não é suficiente você ter
presenteado o menino com esse nariz tão comprido?... As crianças o
provocam na rua...

--- E que tenho eu com isso? --- corou Ívolguin, nervoso. --- Olhe, meu
nariz é normal, e meu pai também não tinha um nariz judeu.

--- E quem poderia ter um nariz judeu, eu ou quem sabe meu pai, um
toneleiro dе aldeia? --- disse Klávdia e, vendo que o marido corava como
de hábito, acrescentou: --- Só falta você me acusar de antissemitismo,
pois todos os judeus do nosso instituto sabem que eu não sou antissemita
e que meu marido é judeu.

--- O que o antissemitismo tem com isso? --- disse Aleksei Ióssifovitch.
--- Você sabe que meu ponto de vista a respeito é abrangente.

E, então, nessa noite, ele se calou, não disse mais nada à esposa, pois
essa discussão aconteceu de noite, evidentemente sem a presença de
Saviéli. Aleksei pegou o livro \emph{Obras escolhidas dе pensadores
russos da segunda metade do século XVIII}\footnote{Enciclopédia, em dois
  volumes, organizada e prefaciada pelo filósofo Ivan Iákovlevitch
  Schipánov (1904\emph{--}1983), publicada em 1952 em Moscou pela
  Politizdat (Editora de literatura política da URSS).} e sentou-se em
sua cadeira de balanço predileta, lendo a seguinte frase: ``Lembremos de
que origens modestas descenderam os povos russos primitivos e que
grandeza, glória e poderio eles alcançaram...'', então ele teve
pensamentos agridoces: como seria bom se ele descendesse dos eslavos,
autóctones, ou, em último caso, dos tártaros ou dos iacutos. E que não
judeu seria ele, bom e humano, quanta coisa faria por aqueles que, sem
sorte, nasceram judeus, de pai e mãe, e, o principal, numa condição que
não pode ser mudada. Quando se nasce judeu, isso é definitivo, mesmo
quando se morre como russo. Talvez para seu filho, Saviéli, fosse ainda
pior, mais ultrajante. Faltava-lhe metade, apenas metade... Ah, que
riqueza é ser russo, e como os russos não valorizam isso, eles não amam
a Rússia o bastante... Ele sabia que muitos russos não tinham amor
suficiente pela Rússia... Se ao menos permitissem a ele, Aleksei
Ívolguin, ser um russo, que patriota ele seria... No entanto, ele sabia
também que muitos russos ficavam descontentes quando um judeu amava a
Rússia, sentiam ciúme dessa relação e preferiam um judeu inimigo da
Rússia. E muitos judeus justificavam esses pensamentos ofensivos... Sim,
sim, ele até podia apontar o dedo para alguns... Não apreciavam o pão
russo, não valorizavam a hospitalidade russa... Ingratos!... Ah, como
ele os detestava... Por causa deles, nós também sofremos... Klávdia é
russa... Os bielorrussos também pertecem, efetivamente, à estirpe
russa...

Depois disso, seus pensamentos, como sempre se dá nesses casos,
dispersaram-se, indo para muitas direções, e tornaram-se cansativos,
assim como, depois de passado o fervor inicial, são cansativas todas as
conversas e reflexões sobre o judaísmo. Além disso, apareceu Saviéli,
estranhamente agitado, olhou para seus pais e disse:

--- Brigaram de novo?

E eles sentaram-se para jantar. Aleksei Ióssifovitch pensava que, além
dos pensamentos cansativos e rotineiros sobre o judaísmo, da discussão
igualmente cansativa e rotineira com a esposa e da estranha agitação de
Saviéli, essa noite só seria memorável pela forte chuva... Mas a noite
passou a ser lembrada principalmente pelo desaparecimento de Akhmét... O
zelador еstava sumido fazia dois dias e então souberam pelo guarda da
rua, Efraim Nikoláievitch, que Akhmét fora preso: havia esfaqueado
alguém.

--- Vá imediatamente atrás de um requerimento --- dizia Klávdia, alegre
---, para que não coloquem mais ninguém aqui.

Requerimentos desse tipo precisavam de três assinaturas influentes,
obrigatoriamente eslavas, mas de preferência russas... Que terminassem
em ``ov'', ``in'' ou, no pior dos casos, em ``enko''.\footnote{Os
  sufixos ``ov'' e ``in'' nos sobrenomes são característicos russos,
  enquanto o ``enko'' é muito achado entre ucranianos, sobretudo do
  Leste, e também entre alguns bielorrussos.}

Ívolguin correu para um escritório --- a influente assinatura russa
terminada em ``ov'' havia viajado a trabalho ---, correu para outro ---
a terminada em ``in'' descansava na Crimeia ---, mas no terceiro ele
conseguiu a eslava, não russa, terminada em ``enko''... Ele correu
alegremente para casa, mas Klávdia o recebeu com desgosto.

--- Tarde demais... Pode deixar sua assinatura eslava de molho. Já
meteram alguém aqui... Ainda com uma filha... Akhmét, pelo menos, era
sozinho.

Ívolguin notou que haviam tirado o cadeado da porta do quartinho usado
pelo zelador, e era possível ouvir vozes masculinas e femininas.

--- Quem é? --- perguntou Ívolguin, com os olhos.

--- Vamos embora, idiota --- respondeu Klávdia com os olhos.

Eles passaram para a sala de estar e sentaram-se perto do piano de
cauda, desanimados.

--- Quem é? --- perguntou Aleksei Ióssifovitch, agora em voz alta.

--- Certamente um judeu --- respondeu Klávdia.

--- Como? --- disse Ívolguin. --- Um zelador judeu?... Que piada... ---
e começou a rir.

--- Não há nada de engraçado --- Klávdia sorriu também ---, tudo
dependerá da primeira conversa... Vamos colocá-lo logo no seu lugar...
Aqui, penso eu, será mais fácil... Em último caso, quebro-lhe a cabeça
com uma panela. Ele não vai me espremer em minha própria pátria. Ele
precisa lembrar que está morando na União Soviética...

Aleksei Ióssifovitch sabia que sua esposa, uma funcionária da
contabilidade do Ministério de Construção de Estradas Automotivas, seria
realmente capaz de bater com uma panela em alguém se estivesse certa de
que não levaria uma facada, à maneira tártara, mas fosse apenas intimada
a comparecer аo tribunal, à maneira judia.

--- Não faz mal, eu mostrarei no tribunal quem são eles... Vieram em
bando para Moscou. Metem-se em toda parte, até em cargos de zelador.

--- Não é necessário ir ao tribunal --- disse Ívolguin ---, deixe
comigo, eu os entendo melhor que você. O descaramento judeu teme a
palavra ríspida. Eles estão sempre sussurrando, querem resolver tudo de
mansinho. Mas comigo não vão falar desse jeito. Eu lhes mostrarei que
seus problemas não me interessam --- e foi para o corredor.

Foi lá que aconteceu seu primeiro encontro com Dã, a Áspide, o
Anticristo... Para não cumprimentá-lo e para dizer algo ríspido, o
crítico de arte refletiu, franziu a testa e se deteve, enquanto o
enviado do Senhor, o Anticristo, logo o discerniu, о reconheceu. Quem
estava à sua frente usando chinelos, uma camiseta de redinhas e um
pijama de seda era da tribo de Rúben, o primogênito de Jacó, que fora um
dia muito poderosa, mas fazia tempo que entrara em declínio, e dessa
tribo poucos foram incluídos no Resto e deixaram um descendente... О que
estava parado na frente do Anticristo era o fim de um processo que
começara na escravidão egípcia, quando as crueldades exaustivas do faraó
lutavam contra a tenacidade e a vontade de sobreviver dos filhos de
Jacó. Quanto mais o faraó os extenuava, mais eles se multiplicavam, até
o dia em que, na tribo de Levi, nasceu Moisés...

No entanto, quando Moisés nasceu, o mal já havia se proliferado, pois na
opressão, quando o homem não vive, mas sobrevive, e não tem Deus ao seu
lado, o bem não consegue sobreviver, enquanto o mal consegue e vive,
perto dos caldeirões de carne, a vida que lhe é rotineira.

De Rúben, o primogênito forte e bondoso de Israel, nascera aquele que se
postava de chinelos e de pijama de seda na frente do Anticristo,
lançando olhares impuros e acariciando com as mãos rechonchudas, não
acostumadas ao trabalho, sua barriga redonda, como ela se fosse uma
criança querida. O que estava parado na frente do Anticristo, no
corredor, era um primor de indecência e de maldade. Mas a indecência não
é capaz de criar nada de forma primorosa, não cria uma vilania perfeita
nem um malfeitor perfeito. Por que, então, existe essa profusão de
maldade, tão primorosa e desmedida? Quem a cria? Ela é fruto da
bondade... Só a bondade frutifica, e ela não gera apenas o seu
semelhante, mas também o seu contrário... Toda maldade nasce da bondade,
mesmo que a bondade também gere a si mesma... Por que o Senhor permitira
isso, por que a maldade se reproduzira até entre Seu povo? Eis a questão
irônica dos ateístas e a questão louca dos místicos... Para que o Senhor
precisaria de um Ívolguin, Aleksei Ióssifovitch, quando havia Moisés,
Jeremias, Isaías e Jesus de Nazaré?... A resposta é simples para quem lê
e relê não apenas o acréscimo cristão tardio --- o Evangelho, que não
contém nenhuma palavra independente ---, mas também o poema divino sobre
a criação do mundo, o princípio da Bíblia, sem o qual não é possível
entender nada posterior... Ívolguin existe, porque, depois do Éden, o
homem virou uma criatura amaldiçoada. Ele fora condenado a executar o
trabalho eterno e a fazer parte da história, quando no Éden não havia
nem trabalho nem história. Graças à misericórdia divina, na terra vivem
os profetas e os justos, e da misericórdia provém a bondade, ao passo
que a maldade vem do ser que a produz. Com a compreensão disso, o
profeta bíblico se distingue do humanista meloso... Porém, quando
perscrutou o sorriso maldoso do mujique, um ateu oprimido, o humanista
russo Aleksándr Blok\footnote{Aleksándr Blok (1880\emph{--}1921),
  expoente da poesia russa simbolista.} renegou o humanismo, e isso foi
um grito no deserto, pois o mal já havia se proliferado em demasia... O
humanismo também se multiplicou, infecundo em meio às massas e fecundo
na convivência com o individualismo, a personalidade isolada. No começo,
proliferou-se o humanismo cristão, antibíblico, que depois, num sextо da
superfície da terra,\footnote{Área que equivalia à da URSS.} foi
superado por seu filho ilegítimo, o humanismo materialista, ao qual era
devoto Ívolguin, Aleksei Ióssifovitch, um judeu internacionalista ou,
dizendo-o na língua cristã, simplesmente um convertido, não batizado com
água pura, mas com uma ideologia pura e melodiosa, que, em princípio,
possui a mesma base da bondade que gera a maldade.

--- A borra do chá --- finalmente o crítico de arte judeu achou o que
dizer ao zelador judeu ---, não jogue a borra na banheira --- proferiu
Aleksei Ióssifovitch em voz alta, sem sussurros ---, não temos a
obrigação de limpar sua sujeira e de sua filha.

Assim que Aleksei, da tribo de Rúben, disse sua repreensão de teor
comunal, Dã, da tribo de Dã, lembrou algo sobre Aleksei Ióssifovitch que
este, evidentemente, desconhecia. Ele era um descendente longínquo do
judeu que Moisés, durante a escravidão egípcia, salvara do ataque de um
egípcio, o qual foi espancado e morto. E o judeu, assustado, gritou para
Moisés:

--- Quem te tornou nosso juiz?\footnote{Êxodo 2:14. Ao saber da morte do
  egípcio, o Faraó planejou matar Moisés, que se refugiou na terra de
  Madiã. (\emph{Bíblia de Jerusalém,} Paulus, 2016, p. 105)}

О judeu sabia que, depois ser ridicularizdo pelo egípcio, este o teria
solto e chegaria a tempo ao caldeirão de carne. Mas Moisés, seu defensor
inoportuno, pusera tudo a perder... E com sarcasmo, um traço que se
tornaria peculiar à arte contemporânea, o antigo judeu da escravidão
egípcia exclamou:

--- Quem te tornou nosso chefe?... Queres matar-me como mataste o
egípcio?\footnote{Êxodo 2:14.}

Assim está na tradução russa e imperfeita da Bíblia. Já, no original,
está dito que o judeu ``mostrou os dentes a Moisés''. Eis uma definição
precisa, uma marca clara ... Dos que estavam lá, foi esse judeu quem
mostrou os dentes a Moisés. Com efeito, Aleksei Ióssifovitch olhou para
o Anticristo --- que, devido ao seu caminho terreno, tinha o aspecto
bastante cansado e os cabelos embranquecidos --- e algo de deplorável e
pronvinciano transpareceu no rosto desse zelador judeu. Alguma coisa
mordazmente engraçada passou pela cabeça de Ívolguin, pois, sendo um
crítico especializado em arte russa, ele podia rir plenamente da
tristeza universal contida nos olhos de um judeu, tal como costumava rir
Voltaire, o favorito, а vedete do livre-pensamento humanista russo...

Então Aleksei Ióssifovitch abriu a boca, mostrando os dentes que já
tinham mastigado muito pão russo e muito salame ucraniano... Uma
combinação de coroas de ouro na frente, pontes cromadas dos lados e uma
ossada cor de café aguado nos intervalos... Aqui, pensava Dã, pelo bom
serviço prestado ao povo-senhorio, um judeu zeloso é recompensado com
comida, bebida e ar puro... A maior condecoração não vinha estampada no
peito, mas na boca, entre os dentes...

--- Há-há-há --- proferiu Ívolguin, de modo claro e nítido, sem os
sussurros judeus.

E o Anticristo disse mentalmente, através do profeta Isaías:

--- Quem vós estais ridicularizando? Contra quem vós abris a boca e
mostrais a língua? Será que vós não sois os filhos do crime, a semente
da mentira?\footnote{Isaías 57:4.}

No entanto, a tristeza universal dos judeus, que divertira Voltaire e
que agora divertia Ívolguin, não estava apenas nos olhos do Anticristo,
mas também nos olhos do próprio Ívolguin, ou melhor, em seu aspecto mais
decaído e insignificante...

Pois o que é insignificante é o grandioso rebaixado ao extremo...
Rebaixem ao extremo a grande tristeza universal e ela se transformará
num temor habitual e covarde. Não importava o que Aleksei Ióssifovitch
fizesse, seus olhos, à sua revelia, expressavam sempre o mesmo: ``Tenho
medo, tenho medo...''.

--- Abrão, não tenhas medo\footnote{Gênesis 15:1.} --- disse o Senhor ao
Patriarca.

Essa foi uma das principais condições da aliança do Senhor com Abraão e
da transformação de Abrão em Abraão,\footnote{Abrão era o nome original
  de Abraão, o primeiro patriarca bíblico: ``E não mais te chamarás
  Abrão, mas teu nome será Abraão, pois eu te faço pai de uma multidão
  de nações''. (Gênesis 17:5, \emph{Bíblia de Jerusalém,} 2016, p.
  54\emph{--}55). ``Abraão é explicado aqui pela assonância com
  `\emph{ab hamón,} ``pai de multidão''. (Ibidem, p. 54)} de um
peregrino da Babilônia em Patriarca do povo do Senhor... Mas os que se
multiplicaram no Egito, perto dos caldeirões de carne da escravidão,
começaram a esquecer o Senhor, rompendo justamente essa Aliança.

--- Tenham medo, é preciso ter medo --- dizem ainda hoje ---, a lebre
teme a vida toda e assim se mantém viva...

É o que se ensina aos familiares mais novos, depois de um bom cálice de
aguardente de ginja. E num tratado filosófico, por trás de uma pilha de
ideias brilhantes, de repente se ouve:

--- Tenho medo, tenho medo...

E nos raciocínios de um sábio convertido também surge: ``Tenho medo,
tenho medo''. Assim como na lírica hábil e talentosa do poeta que versa
sobre igrejas e bétulas, esperando que, através do \emph{te-déum}
querido ao coração, dos doces sussurros dos ``jardins de outono
desfolhando'' e da neve natalina artisticamente retratada, o leitor
russo esqueça, ou ao menos perdoe, sua origem judia... Desse modo, eles
romperam a Aliança com o Senhor...

Logo que Ívolguin, Aleksei Ióssifovitch começou a rir, mostrando os
dentes ao Anticristo, o temor daqueles olhos se acentuou. E, através do
profeta Isaías, o Anticristo se dirigiu a todos no gênero feminino, pois
todos vieram da mesma mulher fraca e volúvel, todos são parte de sua
carne:

--- Quem te deixou assustada e atemorizada o ponto de te tornar infiel,
de deixar de te lembrar de mim e de me guardar no teu coração? Não será
porque estou há tempos calado que tu deixaste de me temer?\footnote{Isaías
  57:11.} --- e o Anticristo acrescentou de sua parte: --- Quem teme
muito os homens não teme a Deus...

Entretanto, Aleksei Ióssifovitch, o crítico de arte, através de seu medo
--- o sentimento mais forte e fecundo que possuía ---, de algum modo se
inteirou do que estava acontecendo no corredor do apartamento comunal,
apesar de não compreendê-lo. Ele parou de rir e foi para seu aposento às
pressas, sem dizer mais nada.

--- Eu revelarei a sua verdade --- disse o Anticristo, olhando para as
costas encurvadas e gordas de Aleksei, vindo da um dia gloriosa tribo de
Rúben ---, eu revelarei a sua verdade e os seus atos, e eles não lhe
serão vantajosos...

Assim os vizinhos se separaram, e o corredor ficou vazio.

--- Eu lhe mostrei quem manda aqui --- Aleksei Ióssifovitch disse a
Klávdia, recobrando o ânimo na sala de estar ---, e ele não ousou dar um
pio. Um típico \emph{jid} de província... Por causa de tipos como ele é
que não gostam de nós.

Já o Anticristo, entrando em seu quartinho, sentou-se à mesa com a filha
adotiva, Rute, para tomar chá. Depois do que acontecera na floresta,
perto da cidade de Bor, pai e filha falavam pouco um com o outro, mas
trocavam olhares, olhares igualmente luminosos e sorridentes... Assim
deveria ser a vida comum do enviado do Senhor, um homem já grisalho, e
da jovem profetisa terrena... Às vezes, pai e filha trocavam algumas
palavras, mas logo se calavam novamente. Pois as pessoas conversam sem
parar para se livrarem do penoso sentimento de afastamento e de
estranheza entre suas almas. Quando o pai se calava longamente, a
profetisa Pelágia sabia o que seu silêncio ocultava. Então ela pegava a
Bíblia, que cheirava a vida de velhinhas: da capa gasta emanava um
cheiro adocicado, de canela e bolor, e das folhas engorduradas cheiro de
algo podre. As passagens que mais lhes agradavam estavam sublinhadas ou
anotadas, com um mesmo lápis azul. O livro dos Salmos e as parábolas de
Salomão eram os que tinham mais anotações... Essa Bíblia fora dada à
Ruthina pela Tchesnokova, uma sectária dos \emph{velhos crentes}...

Para um homem que absorveu cultura e acumulou sabedoria como se
acumulasse bens, e não como um dom do Senhor, essas marcas e anotações
não tinham nenhum valor. Já ao homem de intelecto refinado, pleno da
sátira voltairiana, essas notas poderiam provocar risos e reforçar a
convicção da insignificância da simples fé popular... E é assim em se
tratando da fé simples das massas, que só pode ser acessada por meio de
ritos e crendices. Pois a autenticidade na singeleza é ainda mais rara
que na razão. Mas a Bíblia está repleta de extremos raros e divinos. Aos
outros só restava depositar esperança nos ritos e num guia espiritual
honesto e inteligente --- um padre para o povo simples ou algum filósofo
religioso para o meio cultivado. Contudo, a história da religião mostrou
que raramente essas esperanças se realizam: ora falta razão, ora
honestidade. Eis o que fora sublinhado nas parábolas de Salomão, com o
lápis azul, pela \emph{velha crente} Tchesnokova, que mal sabia ler e
escrever: ``O temor do Senhor é fonte de vida, pois afasta as armadilhas
da morte''.\footnote{Provérbios 14:27.} Sobre isso é possível filosofar,
embora o intelecto refinado não considere essa ideia uma grande iguaria.
E adiante: ``Mais vale um prato de verduras com amor que um boi bem
nutrido com ódio...''.\footnote{Provérbios 15:17.} ``Mais vale possuir
pouco com o temor do Senhor do que um grande tesouro com
inquietação''.\footnote{Provérbios 15:16.} O intelecto refinado, em
geral, cairá na risada diante dessa evidência, dessa sabedoria pueril.
Rirá sem compreender que aqui a sabedoria não parte da moral, a qual
foram familiarizados pelos sacerdotes levianos e filósofos-humanistas
melosos, mas parte do sentimento de egoísmo pessoal, ao que de fato
confiam suas ações.

--- Egoísta --- diz Salomão ---, se gostas de ti mesmo, comerás um prato
de verdura com amor, e não carne de boi com ódio.

O filósofo-humanista tenta ensinar a bondade partindo de uma moral
estranha à natureza humana. A Bíblia ensina a bondade partindo do
egoísmo do homem, pois ela não ignora, à maneira dos humanistas, a
verdadeira natureza do homem; mas, à diferença dos fascistas engenhosos,
que se apoiam na maldade e ensinam a maldade, a Bíblia, partindo da
natureza maldosa dos homens, ensina a bondade.

--- O homem mal-intencionado perverte o seu próximo --- a profetisa
Pelágia lia um trecho que havia sido sublinhado pela \emph{velha crente}
---, ele o guia para o mau caminho. Fecha os olhos para inventar uma
impostura, morde os lábios, faz o mal; ele é a fornalha da
maldade...\footnote{Provérbios 16:29, 30.}

Nesse momento, alguém tocou à porta da entrada, mas nem o pai e nem a
filha se mexeram. Pois, quando ambos estavam em casa, não havia ninguém
para visitá-los.

--- Os cabelos brancos são as coroas da glória quando se está no caminho
da justiça --- a profetisa Pelágia lia as parábolas de
Salomão.\footnote{Provérbios 16:31.}

Quem tocou à porta foi o bondoso companheiro de viagem que chegara de
Vítebsk com Nínotchka e Míchenka. Ao tocar a campainha, ele não foi
embora, mas virou a esquina e esperou para ver se as crianças seriam
recebidas. Se por algum motivo não as recebessem, ele as levaria para a
sala da polícia destinada às crianças. No entanto, elas foram recebidas
com um grito de mulher que lhe pareceu de alegria e, satisfeito, com um
sentimento agradável por seu ato caridoso, o homem, que preferiu
manter-se anônimo, se afastou. Quem gritou foi Klávdia, que, ao
reconhecer os filhos de sua irmã Valentina, vindos sem aviso, logo
suspeitou de algo ruim. Não, o bom homem apenas imaginou que o grito
fosse de alegria... Deixando as crianças entrarem, Klávdia pôs-se logo a
perguntar onde estavam seus pais e por que eles haviam vindo sozinhos.
Ívolguin se agitava ao lado e repetia:

--- Klávdia, não se agite, é preciso pensar com calma...

Mas Saviéli, o adolescente doentio e mestiço, deitado no divã, estudava
Nínotchka, dona de olhos cinza, a prima que ele via pela primeira vez.
Com essa recepção, Míchenka caiu no choro e, em seguida, Nínotchka
começou a piscar seus belos olhos de gata, iguais aos de sua mãe.

--- Mais essa --- disse Ívolguin, no qual de algum modo havia se
preservado o instinto do homem judeu, vulnerável às lágrimas infantis.
--- Agora você assustou as crianças. Antes de tudo, é preciso
alimentá-las.

--- Sim, claro --- disse rapidamente Klávdia.

Deram às crianças um pouco de sopa requentada da véspera e macarrão com
almôndega, uma para cada. Enquanto elas comiam, Klávdia e Aleksei
Ióssifovitch, trancados no dormitório, liam a carta que acharam na
lancheira com o coelhinho bordado. Klávdia só leu até a parte que
mencionava a prisão de Sacha.

--- Está claro --- disse e, empalidecendo, deixou cair o papel ---,
antes ela não ligava para mim, mas agora, ao cair em desgraça, quer me
destruir. Ela não quer entender que meu marido é judeu. Nós devemos
ficar longe de qualquer suspeita.

--- O que minha origem tem com isso? --- perguntou Ívolguin,
estremecendo, como sempre acontecia quando mencionavam, em voz alta, sua
essência terrível e vergonhosa.

--- Tudo! --- gritou Klávdia, com raiva. --- E não se faça de
desentendido. Cordeiro de Deus... Quer que aconteça com você o mesmo que
aconteceu com Sherman?

--- O que Sherman tem a ver comigo? --- disse Ívolguin, tentando conter
seu coração agitado. --- Sherman mantinha ligações com parentes na
América --- então ele sentiu o habitual ``tenho medo, tenho medo''
disparar, voar, invadir sua alma. ``Tenho medo, estou apavorado'',
gritava uma alma da um dia gloriosa tribo de Rúben, uma alma entre
tantas outras, e a elas disse o Senhor, através do profeta Ezequiel:

--- E ao chegar às nações para onde se dirigiam, profanaram o meu santo
nome, pois sobre eles falavam: ``Eles são o povo do Senhor, mas partiram
da terra Dele''.\footnote{Ezequiel 36:20.}

``Tenho medo, estou apavorado!'' gritava roucamente a alma de Ívolguin,
arrastada do corpo pelo medo, como se arrasta um prisioneiro à noite de
seu leito.

E Ívolguin sussurrou com a voz rouca:

--- Ouvi dizer que na Bielorrússia há um processo sério, estão julgando
os nacionalistas... Bogdanóvitch e outros...

--- Temos que resolver o que fazer com os filhos de Vália --- disse
Klávdia, com firmeza, mas já sem nervosismo ---, Vália pode ficar
ofendida, mas eu não posso deixá-los aqui. Eu também tenho um filho. E,
materialmente, também seria difícil para nós, mas isso não é o
principal...

--- Está bem --- disse Ívolguin precipitadamente ---, mas agora não é o
momento. Precisamos ir dormir... De manhã decidiremos...

Aleksei Ióssifovitch sabia perfeitamente o que a esposa havia decidido,
embora ele ainda não conhecesse os detalhes; no entanto, temia ouvir em
voz alta o que ela tinha em mente e se esforçava por adiar este
momento... Ele temia tanto as ações torpes como as nobres. Ele temia
tudo e, mesmo quando ousava gritar com pessoas em posição inferior,
amedrontava-se.

Antes de dormir, serviram aos filhos de Vália, Nínotchka e Míchenka, e
ao deles também, Saviéli, um copo de gelatina com pãozinho. Arrumaram a
cama no divã da sala para as crianças, que, cansadas, rapidamente
adormeceram. Saviéli se deitou em seu quarto, com a porta destrancada,
pois, por ordem de Klávdia, a trava havia sido retirada por um
serralheiro. Isso acontecera depois que Saviéli fora flagrado cometendo
seu pecado juvenil --- o mesmo pecado de Onã, o segundo marido de Tamar
---, dessa maneira, Saviéli sentiria que seus pais poderiam entrar a
qualquer momento e flagrá-lo em pecado. No entanto, nessa noite seus
pais tiveram outras preocupações, e ambos se levantaram com os olhos
inchados e saíram para trabalhar sem café da manhã. As crianças de novo
comeram macarrão com almôndega e gelatina e começaram a brincar. O
pequeno Míchenka se enfiou num grande relógio de parede, apanhando o
pêndulo. Saviéli perguntou a Nínotchka:

--- Você sabe fazer ginástica?

--- Como assim? --- estranhou Nínotchka.

--- É muito simples --- disse Saviéli ---, eu vou levantá-la e você fará
diferentes movimentos com os braços. Entendeu?

--- Entendi --- disse Nínotchka ---, eu brincava assim com meu pai em
Vítebsk... Ele me erguia em seus braços bem alto... Ou me levava para
passear em sua bicicleta... E também me ensinou a recitar poemas...
Escute...

Em frente à escola, há uma nova casa,

Na nova casa nós moramos,

Pela escada nós corremos

E contamos os andares:

Um andar, dois andares,

Três, quatro, e estamos lá.\footnote{O poema ``A casa nova'' aparece na
  \emph{Cartilha} (1934, Editora Estatal de Ensino Pedagógico) do
  pedagogo Piótr Afanássiev (1874\emph{--}1944), que introduziu um
  método fônico analítico-sintético de alfabetização, o qual passou a
  ser utilizado pelas cartilhas russas.}

Saviéli lembrava que, quando era criança, ainda sem idade para ir à
escola, gostava de olhar revistas de moda em que havia mulheres bonitas
desenhadas, e ele passava o dedinho sobre suas pernas lisas e lustrosas,
o que era tão agradável como chupar uma balinha... Ele não sabia,
evidentemente, que sua má mistura de sangue é com frequência castigada
pelo quarto flagelo do Senhor, a doença, e pelo terceiro, o animal
feroz... Mesmo assim, embora fosse uma criança, sentava-se
intuitivamente com a revista em algum canto isolado, onde passava o
dedinho sobre aquelas pernas brilhantes, lustrosas e amarelas... Assim,
desde a tenra idade, ele se acostumou a associar o sentimento de prazer
à solidão. De seu cantinho ele observava as meninas do pátio,
afastando-se delas na classe, e sofria, até que um dia, no banheiro da
escola, um menino lhe ensinou o vergonhoso prazer... Ele também gostava
de ir ao circo e a apresentações de ginástica, para ver os homens
levantando as mulheres pelas pernas e pelas coxas. Por isso, ao ficar
sozinho com sua prima --- o pequeno não podia ser levado em conta ---,
ele decidiu, pela primeira vez ensaiar seu próprio número, e seu coração
palpitou como nunca. E ele entendeu --- naturalmente não por meio da
razão, pois ainda era muito tolo, mas por meio de suas mãos --- o que
era um corpo de mulher, diante do qual eram insignificantes todos os
outros prazeres, quе entretinham também Onã, o segundo marido de
Tamar... Ali estava ela, a força de gravidade feminina, terna e úmida,
em nome da qual qualquer insensatez se justifica... Será que ginastas e
artistas de circo sentem isso todos os dias?... Ele ainda não havia
conhecido o tédio de um homem saciado de pratos exuberantes, gansos
corados e carpas assadas com creme azedo... Ele era filho de uma família
moscovita abastada de 1949, mas se alimentava de salsichão e de
almôndega.

Nínotchka também gostava quando Saviéli a erguia, dava gritinhos e
agitava os braços, e Míchenka batia palmas. As crianças ficaram tão
entretidas que nem notaram a chegada dos adultos. Klávdia entrou bem na
hora da pirâmide, mas Saviéli não conseguia fazer Nínotchka acertar a
posição, pois ela sentia cócegas. Finalmente Nínotcka consentiu e,
apesar de seus gritinhos, permitiu que ele encaixasse sua mão bem no
fundo...

--- O que está acontecendo aqui? --- gritou Klávdia, muito pálida, mas
foi uma pergunta retórica, pois ela sabia muito bem o que se passava.
--- Parem já com isso!

--- Nós estamos brincando --- disse Nínotchka, rindo.

Klávdia pegou Saviéli pelo braço, arrastou-o para o dormitório e lhe deu
uma bela bofetada. Em seguida, entrou Aleksei Ióssifovitch, que também
bateu nele, mas não com tanta força, pois era um pai judeu.

--- Mais um motivo para mandá-los embora --- disse Klávdia em voz baixa
---, uma menina estranha em casa irá perverter Saviéli.

--- Sim, sim, eu concordo --- respondeu Ívolguin, e seu coração, com a
covardia habitual, pôs-se a saltar como uma lebre ---, mas antes nós
devemos, sem falta, servir-lhes o almoço... Antes que... --- e ele
hesitou.

Depois do almoço, Klávdia ordenou ao assustado Saviéli:

--- Você ficará em casa... Eu, seu pai e as crianças vamos sair para
resolver uma questão, está claro?

Sentindo-se culpado, Saviéli não teve coragem de lhe desobedecer e se
deitou no divã. O casal Ívolguin e os filhos dos Kukharienko, os
parentes que haviam sofrido repressão, tomaram um trólebus, passaram
para outro, e chegaram à estação Bielorrúski. Na estação, dirigiram-se à
sala destinada a mães e filhos. Acomodando as crianças, Klávdia e
Aleksei Ióssifovitch se afastaram para um canto perto da janela e
começaram a sussurrar. Depois Klávdia partiu, e Aleksei Ióssifovitch
aproximou-se das crianças e sentou-se ao lado, meditativo. Após
refletir, levou Nina para o canto perto da janela, onde havia conversado
com Klávdia, e disse:

--- Você já é uma menina crescida, deve saber que seus pais foram presos
e que não é possível esconder isso. Vivendo conosco, uma hora ou outra,
vocês serão descobertos, porque nós somos seus parentes. Por isso, pegue
Micha e leve-o para a sala de espera comum, sente-se lá e comece a
chorar. Se perguntarem por que você está chorando, responda que sua mãe
os abandonou e não voltou mais. Seu sobrenome agora é Ivanova.

Nínotchka era uma menina aplicada e obedecia aos mais velhos. Pegou
Micha, foi até a sala grande, sentou-se e começou a chorar. Não chorava,
porém, pela mãe inventada que os teria largado, mas por sua mãe
verdadeira, de Vítebsk, e chorava também por seu pai. As pessoas
começaram a se aproximar e a perguntar o que havia acontecido.
Aproximou-se também uma mulher com uma braçadeira vermelha, a vigia de
plantão da estação:

--- Qual é o problema? --- perguntou. --- Por que você está chorando,
menina?

--- Nossa mãe nоs deixou --- disse Nínotchka, como lhe ensinara o tio
Aleksei ---, e não voltou mais.

De repente seu peito foi invadido por uma mágoa profunda, e ela sentiu
pena de si mesma e de Míchenka...

--- A menina está dizendo a verdade --- disse a vigia. --- Eu vi a mãe
com eles na sala para mães e filhos --- evidentemente, ela tinha visto
Klávdia ao lado das crianças e tomou-a por sua mãe. --- Pegue seu
irmãozinho e venha comigo --- disse ela.

Nínotchka pegou Míchenka nos braços e foi atrás da vigia. Quando passava
em frente do telégrafo da estação, ela viu o tio Aleksei, que espiava
por entre as costas das pessoas e olhava para ela com inquietação. E de
repente o tio Aleksei não estava mais lá... Nínotchka seguiu a mulher,
atravessando algumas passagens, depois a plataforma de embarque, depois
uma rua próxima da estação. Míchenka era pesado, e Nínotchka mal se
aguentava em pé, seus braços quase se desprenderam. Mas eis que entraram
em uma casa. A vigia foi embora, e as crianças ficaram sentadas no chão,
num cantinho, por longo tempo. Finalmente, foram chamadas para outro
recinto, onde se sentava um policial. Ele começou a perguntar quem eles
eram e de onde vieram. Nínotchka, lembrando-se dos conselhos do tio
Aleksei, respondeu como ele havia ensinado, e Míchenka, assustado, ficou
calado. Porém, quando uma mulher severa com um pente sobre os cabelos
grisalhos começou a indagar às crianças, elas se desfizeram em lágrimas,
e Nínotchka contou tudo o que ocorrera, inclusive que o sobrenome deles
não era Ivanov, mas Kukharienko... Então lhes serviram um bom almoço...
As crianças ficaram nessa casa por três dias, depois dos quais foram
enviadas de trem para a cidade de Tobólsk.

No início, eles ficaram em um orfanato, a sete quilômetros de Tobólsk,
chamado Makárenko, que ficava num velho mosteiro no meio da floresta. Lá
era tudo muito bom. No verão, eles nadavam nos rios Irtých e Tobol.
Perto do orfanato havia um criadouro de raposas, e as crianças iam
frequentemente vê-las. No entanto, logo ocorreu um incêndio. Diziam que
o orfanato havia sido incendiado pelas monjas, que queriam se vingar do
poder soviético por este ter desapropriado o mosteiro, entregando o
local à educação dos órfãos. Após o incêndio, todas as crianças foram
transferidas a Tobólsk, para o orfanato Krúpskaia, que não era tão bom
como o primeiro. Mas, numa manhã, a diretora mandou chamar Nina e lhe
disse:

--- Kukharienko, amanhã você será transferida.

--- Para onde? --- perguntou Nina.

--- Lá você descobrirá.

--- E meu irmão Micha?

A diretora não respondeu nada. De manhã, Nina se despediu de Micha e foi
levada com outras crianças, em vagões de carga, para muito longe. O
lugar para onde a transferiram era realmente detestável. Não davam
comida suficiente, e os educadores eram bravos. Nas redondezas, havia
enormes \emph{sopkas},\footnote{Montanhas arredondadаs encontradas no
  extremo leste da Rússia, onde muitos campos de prisioneiros
  (\emph{láguer}) foram instalados.} e as crianças, para que não se
afastassem, eram o tempo todo ameaçadas com os ursos. Certa vez, Nina
viu conduzirem um comboio de prisioneiros de guerra ou de detentos. Nina
guardou a lembrança de uma mulher que tinha sido agredida por um
escolta, e sangue escorrera por seu rosto... Desde então, Nina tornou-se
muito arredia e dizia grosserias aos superiores, por isso foi colocada
no porão usado para guardar os barris de repolho azedo dos órfãos.

A educação no orfanato era regulada com severidade e as faltas cometidas
pelas crianças invariavelmente castigadas... Ali não acreditavam em
lágrimas.\footnote{De ``Moscou não acredita em lágrimas'', antiga
  expressão russa (vinda provavelmente da época do Grão Príncipe Ivan
  Kalitá (1288\emph{--}1340 ou 1341), quando as povoações vizinhas
  tinham que pagar altos tributos ao Principado de Moscou) que, em 1979,
  deu nome ao conhecido filme dirigido por Vladímir Menchóv
  (\emph{Moskvá slezam ne viérit}).}

Em geral, desde tempos imemoriais, a Rússia gosta de chorar e de se
lamentar, isso faz parte do caráter nacional russo. No entanto, por
volta de 1952, a vida nacional russa representava, como nunca, a vida de
todo o Estado, atingindo plena integridade e uma ordem de severidade
monástica. A salvação de uma alma jovem e imatura consiste geralmente em
uma percepção frívola da vida. Tal frivolidade acompanha, desde sempre,
a alma russa em momentos difíceis e a salva da ruína. Em 1952, um ano de
aço e das armas, essa frivolidade salvadora foi erradicada de todos os
lugares, até do antissemitismo se erradicou a alegria. Não zombavam mais
dos judeus, não riam mais deles, e o número de piadas sobre eles
diminuiu. Em compensação, surgiu uma profusão de artigos de um rigor
ascético, compilados, literalmente, no extremo da ideologia dominante...
Parecia que, a qualquer momento, a palavra oral iria irromper na
linguagem impressa... À noite numa cançoneta, de manhã no jornal... As
estrofes da conhecida \emph{tchastuchka} ``Bata..., salve...'' eram
cantadas sem aquela alegria impetuosa, mas com severidade, como um hino.
Extenuada e cansada, a alma russa modificou-se por completo, e já não
emanava o cheiro do ezfuziante \emph{pogrom} ortodoxo, mas o do grave
\emph{pogrom} medieval, católico... Era normalmente doce, e não amargo,
o bombom polonês \emph{jid}, chupado por bocas polonesas cheirando a
enxofre e depois passado para outras bocas, que, embora também eslavas,
eram mais largas e menos ossudas.´Como era agradável segurá-lo na boca,
e petiscá-lo com um cálice de vodca era tão bom quanto um pepino
salgado... Numa conversa erudita, o bombom também refrescava
agradavelmente a boca. E sussurrava ao literato genuinamente russo uma
resposta aos eternos enigmas russos... O bombom polonês \emph{jid} era
saboroso\emph{,} mas, em 1952, o ano de aço e das armas, ele se tornou
uma pílula amarga. Queimava as bocas, deformava os rostos.

Só Deus sabe quantos rostos terríveis Aleksei Ióssifovitch Ívolguin já
tinha visto! Sua alma não gritava mais: ``Tenho medo!'', simplesmente
tremia, sem palavras.

--- Ligue para Fadéiev --- Klávdia sussurrava na cama.

--- Para lembrá-lo de seu discurso no funeral civil do nacionalista
burguês judeu Mikhoels? --- mostrou os dentes Aleksei Ióssifovitch,
acuado.

--- Para que lembrá-lo disso? --- dizia Klávdia. --- Você acha que ele
ainda lembra onde encontrou você?

--- Não, não --- dizia Ívolguin. --- Agora o principal é não ser notado.

Contudo, seria difícil não ser notado quando aquela eterna questão
russa, ``Quem está arruinando Rússia?'', se revelava em toda a sua
extensão, inflamando e corroendo o homem russo. Em um casamento russo
movimentado, é fácil desaparecer em meio à euforia geral: finge-se
bêbado e mete-se embaixo de uma mesa; mas, quando o russo faz um balanço
de suas ofensas e sua fala se enche de chiados e sibilos ---
\emph{xxi...}, \emph{jji...} ---, tente passar despercebido... ``Lixo...
Traste... \emph{Chacal}... \emph{J-j-j-jid}...'' Andando pela rua,
sibilam de todas as direções. Nos lugares públicos --- nos escritórios,
nas salas de cinema, nos meios de transporte --- chiam por toda parte...
Aleksei Ióssifovitch passou a sentir receio de andar de bonde, de
trólebus... O antissemitismo em bondes, trólebus e ônibus não era um
fenômeno recente; no entanto, o transporte urbano se transformou num
comício sobre rodas... A liberdade de expressão garantida pela
Constituição, nesse sentido, sempre fora respeitada, mas agora havia
mais oradores enfiados nos trólebus do que no Hyde Park. Mesmo antes, em
épocas mais alegres, Aleksei Ióssifovitch ficava temeroso quando, entre
os passageiros, começavam mexericos sonoros. Um dia, um sujeito
brincalhão entrou no trólebus --- embora raramente, isso acontecia. O
sujeito farejou o ar e disse:

--- Cidadãos, que perfume delicioso de alho... Camaradas, embora ainda
não esteja claro quem exala esse odor agradável, agora ele é nosso, é
coletivo.

Alguns ficaram calados, mas alguns deram risada, e Aleksei Ióssifovitch
baixou os olhos e encolheu a cabeça entre os ombros. Não fora ele quem
comera o alho, mas seu coração parou. ``Agora vão me golpear pelas
costas com uma palavra terrível... Agora vão dizer...'' Mas não
disseram... Passou... E numa outra vez passou também... E ainda numa
outra... No entanto, Aleksei Ióssifovitch esperava... Até que, certo
dia, no trólebus nº 20, que fazia o itinerário da avenida Marx à
Floresta dos Abetos Prateados (uma pomposa denominação eslava), um russo
disse a Aleksei Ióssifovitch, olhando-o nos olhos:

--- Se nós não tivéssemos que escrever receitas em latim, há muito tempo
teríamos esganado todos vocês, \emph{jides}.

E o trólebus, esse coletivo espontâneo, apoiou o orador com um silêncio
que manifestava aprovação. Pois, num coletivo russo, o judeu é uma peça
importante, indispensável, para a sensação dе unidade nacional.

Aleksei Ióssifovitch não saltou do trólebus, mas praticamente despencou
na praça de Púchkin, o gênio russo, onde ficou longo tempo sentado, com
a mão no coração.

Dois dias depois, foi para Leningrado por uma incumbência da revista
\emph{O teatro,} e, durante toda a viagem, um russo de sua cabine
dirigiu-se a ele ``com a maior sinceridade''.

De um modo geral, o antissemitismo no transporte urbano distingue-se
nitidamente do antissemitismo no transporte ferroviário. Dentro da
cidade, as distâncias são curtas, há aperto e uma troca rápida de
personagens, tudo isso conduz ao dinamismo, à gritaria, a fórmulas
prontas, como \emph{slogans}. No trem, acontece o oposto. Tudo é mais
livre, há tempo de sobra para se familiarizar com os companheiros de
viagem. Fazem reflexões ponderadas, ``a bem da verdade'', análises.
Aqui, quando o sujeito não quer gritar, mas raciocinar, observa-se o
primeiro mandamento de um antissemita. O primeiro mandamento consiste em
dizer que ele tem muitos amigos judeus. E em refletir sobre a
fraternidade. Foi exatamente no estilo do antissemitismo sob o embalo do
trem, as batidas das rodas, que Dostoiévski escrevera, em março de 1877,
sua ``Questão judaica''.\footnote{Trata-se de um artigo que saiu, em
  1877, no \emph{Diário de um escritor,} já uma publicação independente
  de Dostoiévski, e não mais uma coluna da revista \emph{O cidadão,}
  fundada pelo príncipe Meschérski (1839\emph{--}1914) e editada pelo
  escritor em 1873 e 1874. No artigo, Dostoiévski se defende de
  acusações de antissemitismo, inclusive do uso do termo \emph{jid} em
  seus textos.}

--- Sim, sim --- aprovava Aleksei Ióssifovitch ---, concordo com o
senhor... Eu sempre fui um internacionalista, e há muito não observo os
preconceitos da minha nação... Até meu sobrenome é internacional,
Ívolguin, e sou casado com uma bielorrussa... E recebo com gratidão o
apelo de Fiódor Mikháilovitch,\footnote{Nome e patronímico de
  Dostoiévski.} ``Viva a fraternidade!''. Concordo com Fiódor
Mikháilovitch quando ele diz que é mais provável um judeu não ser capaz
de entender um russo do que o contrário... Cá entre nós --- acrescentou
com os olhos brilhando, contente por encontrar uma pessoa culta, e não
um baderneiro ---, cá entre nós, eu nunca gostei de mulheres judias...
São desleixadas, nervosas e, em assuntos íntimos, têm aquela avidez
judia... As eslavas são melhores --- e o crítico de arte Aleksei
Ióssifovitch, confiante, deu um estalido com os lábios.

De fato, dizendo qualquer indecência por desejo de agradar seu
interlocutor, Aleksei Ióssifovitch disse uma verdade. Quando um povo cai
em desespero, isso se reflete, antes de tudo, na mulher, pois ela é a
feição nacional do povo. Nos campos de concentração cotidianos, que eram
os assentamentos judeus, nas noites nupciais azedas de primos e primas,
em quartos abafados --- para que correntes de ar não resfriassem os
pulmões tuberculosos ---, a feição das beldades bíblicas, a cada
geração, decaía mais. Mulheres de narizes desproporcionais, coxas
ossudas e barrigas flácidas geravam filhos de ossos estreitos,
encurvados, fracos, doentes... Por isso todos os que, por acaso, nasciam
saudáveis esforçavam-se por fugir de tudo o que é judeu, apesar das
severas proibições dos talmudistas dogmáticos; os corpos sãos fugiam,
salvavam-se dos campos de concentração cotidianos, onde os judeus eram
trancafiados para sua decadência e degeneração... Fugiam também algumas
mulheres bonitas que, seguindo seu instinto biológico, empenhavam-se em
perpetuar não a sua própria estirpe, que perecia, mas uma estirpe
estrangeira e forte. Fugiam também os mais inteligentes, tenazes e
habilidosos... Passavam por qualquer fresta, por qualquer abertura...
Como escreveu Herzen: ``A necessidade fez os \emph{jides} se tornarem
astutos e engenhosos''. Ninguém convocava fóruns internacionais por esse
pretexto, ninguém criava fundos financeiros internacionais humanitários.
Os que pereciam se salvavam sozinhos. Eles fugiam do que é judeu para
preservar a própria humanidade. Mas só compreenderam o preço que pagaram
por isso muito mais tarde, embora ainda hoje nem todos o compreendam.
Ele é muito maior do que pagou Fausto a Mefisto. Eles não venderam sua
alma, mas seu espírito. A alma conserva o homem dentro do homem,
enquanto o espírito conserva Deus dentro do homem. Os que fugiram de si
mesmos salvaram sua alma, mas arruinaram seu espírito...

Assim fugira de um assentamento o avô de Aleksei Ióssifovitch Ívolguin,
com um nome cômico para o ouvido eslavo, Chaim, e o sobrenome Katz...
Katz era um bom sobrenome para se conseguir um ganha-pão na Alemanha,
mas na Rússia era necessário outro... Então Ióssif Katz, filho de Chaim,
comprou de um comissário de polícia o sobrenome Ívolguin. E foi uma
pechincha: cinco rublos de prata. Mas, se estivesse em um distrito
remoto, por um rublo de prata compraria até o sobrenome da sua majestade
imperial, Románov. No entanto, Ióssif Katz, um dentista, comprou seu
sobrenome em São Petersburgo, onde a vida era mais cara. E aceitou o que
lhe deram --- já que tinha que ser Ívolguin... Como depois lhe foi grato
seu filho, Ívolguin, Aleksei Ióssifovitch... Para um judeu da Rússia,
uma herança dessas é melhor do que qualquer capital, do que qualquer
propriedade. Assim, o recém-criado Ióssif Ívolguin era daqueles judeus
que viviam bem, porque sabiam trabalhar com mais habilidade em seu
ofício, e a Rússia, cada vez mais, precisava de hábeis engenheiros,
advogados e outras profissões que eram vistas com suspeita pelos
camponeses e proprietários de terras russos. Esses patriotas russos de
origem judia se reuniam no jornal petersburguês \emph{A
Palavra},\footnote{\emph{A Palavra (Riétch)} foi um jornal diário de
  linha liberal, ligado aos \emph{cadetes} (constitucionais democratas),
  sobre política, economia e literatura que funcionou entre 1906 e 1917.
  Em 1918, o jornal circulou ainda como \emph{Nossa palavra (Nacha
  Riétch), A palavra livre (Svobódnaia riétch), O século} (\emph{Viék}),
  \emph{Nova palavra (Nóvaia Riétch) e Nosso século (Nach Viék).}} que
\emph{O Estandarte Russo}, das Centenas Negras,\footnote{\emph{O
  Estandarte Russo} (\emph{Rússkoie známia})\emph{,} jornal nacionalista
  ortodoxo de Petersburgo, funcionou de 1905 a 1917. Foi editado pelo
  médico Aleksándr Dubróvin (1855\emph{--}1921), um dos líderes da União
  do Povo Russo, a qual estava vinculada ao movimento das \emph{Centenas
  Negras (Tchernossótennyi),} uma organização paramilitar
  ultraconservadora, apoiada pelo governo tsarista, conhecida pelo
  xenofobismo e antissemisitmo.} chamava, não sem razão, de judeu.
Quanto mais \emph{A Palavra} publicava artigos contra o sionismo,
acusando-o de tentar atrair os judeus para o âmbito de um nacionalismo
estreito, e não para a cooperação fraternal com o grande povo russo,
mais esbravejava \emph{O Estandarte Russo}, das Centenas Negras, que
também publicava artigos contra o sionismo, mas estes eram mais
inebriantes e desdenhosos, exigindo a prevenção da tomada de poder da
humanidade pelo \emph{kahal} mundial de judeus... As Centenas Negras
faziam a carpintaria pesada, enquanto os judeus russófilos pequenos
reparos... O doutor Dubróvin, líder da União do Povo Russo, ficava verde
de raiva quando lia \emph{A palavra}... A causa vital de um verdadeiro
russo --- a propaganda antissemita --- foi passada para as mãos de
judeus intelectualizados, e a eles coube o pedaço mais saboroso... Pois
esses tratantes até com o antissemitismo conseguiram lucrar...

Ah, o jornal \emph{A palavra}... Aleksei Ióssifovitch começara sua
carreira de crítico literário precisamente lá, publicando, quando ainda
um jovem jornalista, uma nota descrevendo como, em certo lugarejo,
talmudistas perseguiram um adolescente que havia adotado o cristianismo.
E o comissário de polícia, pelo que diziam, não havia reagido às queixas
do sacerdote, porque fora subornado por ricos doadores da sinagoga. No
entanto, agora Aleksei Ióssifovitch raramente conseguia permissão para
se expressar contra os cosmopolistas nas páginas de jornais, e isso foi
um péssimo sinal. Recentemente havia ocorrido a ele um caso bastante
desagradável. Aleksei Ióssifovitch escrevera um artigo longo analisando
como, por trás dos métodos aparentemente românticos de Mikhoels,
transparecia o nacionalismo judeu pequeno-burguês. O artigo tratava de
um tema atual, mas não fora aceito. De repente, ele viu seu artigo, um
pouco modificado, numa forma mais primitiva, assinado por um conhecido e
influente sobrenome russo terminando em ``ov''. Aleksei Ióssifovitch
ficou desnorteado. No fim das contas, não queria louros, ``cuspo sobre o
bronze pesadíssimo''.\footnote{Trecho do poema ``A plenos pulmões''
  (\emph{V vies golos,} 1929-1930), de Vladímir Maiakóvski. Aqui na
  tradução de Haroldo de Campos: ``{[}...{]} Morre, /meu verso, /como um
  soldado/ anônimo /na lufada do assalto. /Cuspo /sobre o bonze
  pesadíssimo, /cuspo /sobre o mármore viscoso. {[}...{]}''
  (\emph{Maikóvski, poemas.} Perspectiva, 2006, p. 135).} No entanto, em
sua versão original, o artigo seria muito mais proveitoso para a
propaganda patriótica... Sim, o que o doutor Dubróvin sonhara em 1905, o
ano dos \emph{pogroms} ortodoxos, realizou-se em 1952, o ano de aço e
das armas. O judeu era cada vez mais afastado da propaganda patriótica
russa. Até sua habilidade era sacrificada em nome dos princípios. Tempos
terríveis haviam chegado para Aleksei Ióssifovitch. Por toda parte,
jornais recusavam seus serviços, e quem podia saber se amanhã ele não
seria privado de seu ganha-pão na universidade?

--- Ligue para Fadéiev --- Klávdia sussurrava na cama ---, ele ajudará.
Se não fosse o incidente com minha irmã Vália, eu mesmo iria até ele na
qualidade de sua mulher e de uma bielorrussa.

Nessa época, Aleksei Ióssifovitch desenvolveu a conhecida doença de quem
não espera mais nada de bom do mundo exterior. Ele passou a temer mais a
porta de entrada do que um animal selvagem... De repente tocam a
campainha, fazem um escarcéu estranho, com batidas de pés...

Seu vizinho, o zelador, levantava-se cedo. Ao ouvir seus passos no
corredor, Aleksei Ióssifovitch pensava com a cabeça doendo: ``Eis uma
profissão segura para um judeu --- zelador. Nosso vizinho é esperto, eu
jamais teria pensado nisso. Em caso de genocídio, a profissão não o
salvará. Mas, em caso de um extermínio baseado na luta de classes, é das
mais seguras profissões''.

--- Ligue para Fadéiev --- repetia Klávdia, com sua teimosia feminina,
vendo a salvação somente no adultério espiritual.

--- Está bem --- disse Aleksei Ióssifovitch. --- Amanhã eu ligarei.

Se o disse para acalmar a esposa ou se realmente havia se decidido,
ainda não estava claro para ele... Mas, em 1952, o que significava
``amanhã'' para quem trabalhava na esfera mais perigosa da construção do
socialismo, a cultura? Cada ``amanhã'' exigia novos sacrifícios, como se
um cruel ídolo pagão. Mas as vítimas humanas não eram substituídas por
ovelhas, como acontecera a Abraão, que, ao chamado do Anjo, trocara
Isaac por uma ovelha, para ser imolada no fogo em seu lugar.\footnote{Episódio
  narrado em ``O sacrifício de Abraão'' (Gênesis 22). Abraão recebera a
  ordem divina de sacrificar seu filho Isaac, mas, ao estender a mão
  para imolar Isaac, o Anjo lhe apareceu: ``Não estendas a mão contra o
  menino! Não lhe faças nenhum mal! Agora sei que temes a Deus
  {[}...{]}'' (\emph{Bíblia de Jerusalém,} 2016, p. 61). Na frente de
  Abraão surgiu, então, um cordeiro.} A manada humana sacrificável
diminuiu tanto que as vítimas passaram a ser escolhidas entre os mais
bem colocados. Cada artigo sobre as questões da luta ideológica exigia
novas vítimas, assim como cada reunião a portas fechadas e cada
assembleia geral. E o ``amanhã'' de Ívolguin também chegou: deram-lhe
uma punhalada em um seminário que discutia as questões da representação
do inimigo de classe na dramaturgia contemporânea. E do que se
lembraram? Do tempo em que Ívolguin era jovem e desejava chamar a
atenção. E como concentrar a atenção em si se não com uma boa polêmica?
A polêmica, em particular, fora contra os que achavam que o inimigo de
classe só podia ser representado se fosse ridicularizado,
caricaturado... ``Um membro do Komsomol,'' dizia, ``não é capaz de criar
a imagem do inimigo de classe com todas as suas nuances psicológicas.
Certamente, é possível representar o inimigo de um modo ridículo e
caricatural. Fazendo isso, o artista expressa apenas a sua própria
atitude, o ódio contra seu inimigo. Mas isso seria um procedimento
satírico que deveria ser espalhado por toda a obra''.

--- Em outras palavras, Ívolguin pede que, além da caricatura do inimigo
de classe, criem uma atmosfera caricatural da realidade soviética, para
que a impressão artística geral não seja prejudicada. Ele afirma que, ao
lado do Khlestakóv moderno, não deveria existir um caráter soviético
moderno, um caráter positivo e pleno, mas que é preciso de um Prefeito
soviético, de um Skvoznik-Dmukhanóvski\footnote{Khlestakóv e o prefeito
  Skvoznik-Dmukhanóvski são personagens da peça \emph{O Inspetor Geral}
  (1836)\emph{,} de Nikolai Gógol.} qualquer...

``Como está abafado aqui, estou sufocando... Seria possível abrir as
janelas?... Abri-las inteiramente... Tenham piedade... Não espero que me
perdoem, não tenho direito de ter esperanças, apenas tenham piedade de
mim.''

--- Cito: ``Retratar o inimigo de classe tal como ele é, em todo o seu
desenvolvimento filosófico e psicológico, em toda a amplitude de sua
atuação...''. Em outras palavras, com a desculpa da objetividade,
Ívolguin clama que levem ao palco sermões antissoviéticos...

--- Ívolguin... Ívolguin... Ívolguin... Ívolguin...

E de repente alguém disse: ``Katz''...

--- Ívolguin-Katz, assim como seu amado Meyerhold, pertence, com o
perdão da palavra, à plêiade que Lunatchárski chamou de
``\emph{intelligentsia} azeda'', e, apesar dos erros que o próprio
Lunatchárski cometeu depois, nessa questão ele tinha razão...

--- Stanislávski também se rendeu às influências estrangeiras, ao
realismo burguês... No entanto, ele achou forças em si mesmo...

Aleksei Ióssifovitch vivenciou, nesse instante, um estado de espírito
estranho, uma miragem espiritual, uma condição inesperada. Lembrava-se
sempre das palavras do russo do trólebus nº 20, da linha Avenida
Marx\emph{--}Floresta dos Abetos Prateados: ``Se nós não tivéssemos que
escrever receitas em latim, há muito tempo teríamos esganado vocês,
\emph{jides}''. E agora os esganavam... Será que aprenderam a escrever
em latim? Não, meus caros, vocês ainda não sabem o que é o latim. Nosso
latim está no fundo do coração. Enterrado profundamente, como um defunto
querido. E, em cima dele, a sólida terra preta\footnote{Terra preta
  (\emph{tchenoziom}), solo fértil das estepes e das pradarias do sul da
  Rússia.} do povo e a argila estéril da \emph{intelligenstsia}
arrependida.

``Às quatro horas celebraram uma missa solene. Eu estava no paraíso. O
órgão soava. Longas aleias de véus brancos. O repique suave dos sinos de
prata embalados pelas mãos suaves de meninos pálidos. O coro dos anjos.
Lábaros de rendas delicadas e perfumadas. As velas e a luz diurna se
alastrando das janelas. Nuvens de fumaça espiralando dos incensários e o
outono dourado atrás dos vitrais. As estátuas da Madonna; o ruído dos
que faziam suas preces sobre o piso de pedra era tão abafado como o
sussurro das folhas do lado de fora. Fiquei ali longo tempo, até que o
cansaço me obrigou a ir embora.''

Esse era Meyerhold na época da encenação de \emph{A Irmã
Beatriz}.\footnote{Peça de Maurice Maeterlinck (1862\emph{--}1949)
  encenada por Vssiévolod Meyerhold (1874\emph{--}1940) em 1906 no
  Teatro de Arte Dramática fundado pela atriz Vera Komissarjévskaia
  (1864\emph{--}1910) em Petersburgo.} Eis o que é o latim, camarada...

Quando o seminário terminou, todos os que saíram viram Aleksei
Ióssifovitch sentado numa poltrona profunda e macia, num salão em frente
à sala de reunião. Uma luz lateral iluminava-lhe o rosto, um rosto duro
de defunto, branco como mármore. Com o corpo todo estirado, a nuca
apoiada no encosto da poltrona e a cabeça jogada para trás, ele escorava
as mãos, brancas como mármore, esticadas bem para a frente, no elegante
castão, grosso e com um monograma de cobre, de sua bengala nodosa e
coberta de verniz amarelo. Assim ele se sentava, e todos passavam por
ele como se passassem por cima de um cadáver espezinhado. Quando o
Senhor olhou para ele, teve pena de Seu santo nome, que fora tão
desonrado, e disse:

--- Não é por vós que faço isso, mas por Meu Santo Nome, que vós
profanastes entre as nações por onde passastes. Santificarei o meu
grande nome, que profanastes entre as nações, e as nações saberão que Eu
sou o Senhor quando Eu mostrar aos seus olhos minha santidade em vós. Eu
vos tirarei dentre as nações, vos reunirei de todos os países, e vos
conduzirei a vossa terra. Então borrifarei água pura sobre vós e vós
sereis purificados de todas as vossas imundícies, de todos os vossos
ídolos vos purificarei. Eu vos darei um coração novo e um espírito novo,
tirarei de vossa carne o coração de pedra e vos darei um de carne.
{[}...{]} Então vos lembrareis de vossos caminhos ruins e de vossas
ações rancorosas, e vós sentireis repulsa por vós mesmos em virtude de
vossas iniquidades e vilanias. {[}...{]} Os povos que restarem em torno
de vós saberão que Eu, o Senhor, reedifiquei as cidades destruídas e
cultivei o deserto. Eu, o Senhor, o disse e o faço.\footnote{Ezequiel
  36:22\emph{--}26, 31, 36.}

Assim falava o Senhor, olhando para Aleksei Ióssifovitch, espezinhado
por sua insignificância, da tribo de Rúben, e, enquanto Ele falava, a
profetisa Pelágia, sentada num banquinho perto do parapeito da janela,
lia Suas palavras no livro do profeta Ezequiel, na Bíblia que ganhara da
velha Tchesnokova. Seu pai, Dã, a Áspide, o Anticristo, nesse momento,
varria as folhas de outono caídas no pátio, que, umedecidas pela chuva,
haviam grudado na terra. Esse trabalho era difícil e demorado,
prolongava-se até a noite, e sua filha adotiva, a profetisa Pelágia,
pegou uma pá de madeira e saiu para ajudá-lo. Assim eles trabalhavam até
que, entre os inquilinos que passavam, surgiu seu vizinho, Aleksei
Ióssifovitch, que andava como um cego, tateando o caminho com sua
bengala elegante comprada em Sótchi. Ao terminarem o trabalho, pai e
filha foram tomar o chá da tarde, pobre mas feliz, num estado de amor
mútuo. E a família Ívolguin se sentou para seu rico e amargo jantar,
segundo a receita das parábolas de Salomão: carne de boi gordurosa e
frita...

Nervoso e angustiado, Ívolguin-Katz empanturrou-se de carne gordurosa e
foi se deitar. O medo invadiu a família Ívolguin. Mesmo Saviéli, o
adolescente mestiço que, havia muito, não pensava em nada além do corpo
feminino, que sentira através de sua prima Nínotchka, assustou-se com o
estado do pai e disse:

--- Papai e mamãe, eu não vou mais magoar vocês...

No entanto. Klávdia, mergulhada em sua tristeza, gritou para ele:

--- Vá para o seu quarto!

Desse modo, Saviéli retirou-se para seu quarto e, ao se ver sozinho, sem
ninguém o controlar, entregou-se ao pecado. E Klávdia começou sua
habitual conversa sob os lençóis:

--- Ligue para Fadéiev... Depois será tarde.

--- Está bem --- respondeu Ívolguin ---, ligarei amanhã.

Ele dormiu ou desmaiou dе medo. E sonhou que realmente ligava para o
secretário-geral da União dos Escritores Soviéticos, membro do Comitê
Central, deputado do Soviete Supremo. Ele falava com Fadéiev ao
telefone. E o telefone era um cone de jornal, o mesmo usado, em bancas
ou na feira, para embalar alguns tipos de produtos. Claro que não foi
apenas por meio de um cone de jornal que Aleksei Ióssifovitch conseguiu
uma ligação com o camarada Fadéiev. Algo se pendurava em seu ombro, uma
espécie de bolsa, e Aleksei Ióssifovitch sabia que isso era parte do
aparato de ligação direta. Mas ele apenas sentia seu peso, não conseguia
vê-lo nem apalpá-lo. Na realidade, usava somente o cone de papel para
falar, como se fosse um megafone.

--- Salve, camarada Fadéiev --- disse Aleksei Ióssifovitch.

--- Salve, camarada Ívolguin --- soava do cone de papel.

Um peso saiu-lhe do coração. ``Chamou-me de camarada. E não de Katz.''

--- Camarada Fadéiev --- Aleksei Ióssifovitch dizia dentro do cone ---,
hoje, no seminário sobre a representação do inimigo de classe na
dramaturgia, um grupo de pessoas, que não é digno de confiança política,
fez as mais absurdas acusações contra mim... Sim, absurdas, camarada
Fadéiev...

No cone de papel, fez-se um longo silêncio, mas sentia-se que a ligação
não fora cortada: o camarada Fadéiev simplesmente refletia para não dar
uma resposta apressada... Após uma longa pausa, o camarada Fadéiev
respondeu do cone de papel.

--- Por acaso sou pago porque amo meu avô?

Fadéiev não havia refletido à toa, sua resposta era algo filosófica, uma
espécie de parábola. Mas qual era seu sentido?

--- Camarada Fadéiev --- Aleksei Ióssifovitch gritava dentro do cone
---, explique-me...

A ligação ficou mais baixa, e não foi possível espremer mais nada do
cone de jornal. Aleksei Ióssifovitch acordou suando frio.

A noite estava bem avançada, quase amanhecia, o momento mais imóvel de
uma grande cidade insone. A noite se calava e a alvorada ainda não tinha
despertado... A esposa dormia e o quarto de Saviéli, atrás da parede,
estava silencioso. Aleksei Ióssifovitch sentou rapidamente à
escrivaninha e escreveu, sob a luz de um abajur, uma carta breve e clara
carta a Fadéiev... Não omitiu nada... Depois se vestiu, foi para o
corredor na ponta dos pés, tentando não respirar, destrancou a porta e
saiu; tremendo por causa da umidade da alvorada outonal, foi até a
primeira caixa do correio e, logo que despositou a carta, sentiu todos
os seus membros estremecerem; agarrou a fria caixa de metal pública e,
como um bêbado, lastimou entre lágrimas sua vida arruinada... O que o
havia arruinado? Por que se sentia tão ferido? Será que ele era o
primeiro homem arruinado do mundo? Mas ele não vivia por si mesmo, pelo
que ele era, nem decaía por isso. \emph{Ivan e Mária}\footnote{No
  original, \emph{Ivan da Mária,} nome russo para a planta
  \emph{Melampyrum nemorosum}, cuja flor, que dá uma vez por ano, tem
  duas cores distintas, normalmente pétalas amarelas e folhas superiores
  azuis. Há algumas explicações para o nome russo, baseadas em lendas
  populares. Uma delas diz que Ivan e Mária, depois de se apaixonarem e
  de se casarem, descobriram-se irmãos, decidindo se transformar em uma
  flor de cores distintas e contrastantes. Outra versão do conto diz que
  eles foram transformados em flor por castigo divino.} não haviam se
encontrado para conceber Aleksei Ióssifovitch, e foi isso que o
arruinou... Que lástima, que lástima... Ah, se ele fosse fruto de uma
concepção imaculada, e não de Ióssif Cháimovitch... Aleksei Ióssifovitch
encostou a testa no metal frio e indiferente da caixa do correio, que, a
partir desse dia e de forma irrecuperável, o separou da carta para o
camarada Fadéiev, escrita sob o abalo de um sonho estranho. E, através
de seu choro confuso, ele repetiu as maldições do profeta Jeremias,
dirigindo-as a si mesmo: ``Maldito seja o dia em que nasci! O dia em que
minha mãe me gerou não será abençoado! Maldito seja o homem que trouxe a
boa nova a meu pai, dizendo: `Tu teves um filho', causando-lhe grande
alegria. E que esse homem seja como as cidades que o Senhor destruiu sem
piedade, que ele ouça os brados pela manhã e os prantos ao meio-dia. Por
que ele não me matou no ventre materno? Assim minha mãe seria meu caixão
e seu ventre ficaria eternamente grávido''.\footnote{Jeremias
  20:14\emph{--}18.}

Pela primeira vez, através do pranto nacional, Ívolguin-Katz sentiu de
repente sua verdadeira alma; até então, ele somente estremecia e se
acovardava impiamente, à maneira judia, mas ria e chorava impiamente, à
maneira russa. Cada um possui seu choro, seu riso, seu temor... O russo,
em seu temor, é religioso, enquanto o judeu é ateísta. O russo ri
largamente, se esquecendo de tudo, ri como um ébrio, como uma criança,
de modo antirreligioso, e chora do fundo do coração, livremente... Mas o
riso e o choro nacionais judeus não possuem a liberdade ímpia dos
russos... O riso e o choro judeus são voltados para Deus... Ao rir e
chorar, o judeu não se afasta de si, pois, rindo ou chorando, ele ainda
se vê de fora... Seu riso é irônico e seu choro sensato... Apenas no
temor o judeu cai no esquecimento, no ateísmo, violando a promessa feita
por Abraão ao Senhor...

Desde essa alvorada de outono, quando, pela primeira vez, Aleksei
Ióssifovitch chorou à maneira judia, algo sucedeu a sua alma, e ele se
meteu na cama e ficou à espera de sua prisão... O ano de aço e das
armas, chegou ao fim, e começou 1953, um ano incomum e blindado, e sua
prisão ainda não havia ocorrido. ``Não pode ser,'' pensava Aleksei
Ióssifovitch, preocupado, ``serei preso em janeiro, nos primeiros
dias.''

À noite, o planeta Vênus brilhava intensamente. Não seria ela mesma, a
estrela de Belém? Não estaria o Natal ligado a Vênus?

Dã, a Áspide, o Anticristo limpava a neve do pátio e da rua em frente ao
seu edifício, e lembrava como eram frias e estreladas as noites de
dezembro e de janeiro perto de Belém, onde Rute, a moabita, se unira a
Boaz, para perpetuar a tribo de Judá. Na constelação do Sagitário,
cintilava a sensual e radiante Vênus... Desde meados de janeiro, havia
nevado muito e o Anticristo não conseguia limpar a rua sozinho, por isso
sua filha, a profetisa Pelágia, ajudava-o... Nessa época, Vênus já tinha
se deslocado para a constelação do Capricórnio e, no fim do mês, no
degelo, com as ruas escorregadias, Vênus passou para a constelação do
Aquário...

``Vão me prender em fevereiro,'' pensava Aleksei Ióssifovitch, ``nos
primeiros dias de fevereiro os médicos assassinos\footnote{Trata-se de
  um processo contra médicos soviéticos de destaque, quase todos judeus,
  que se deu entre janeiro e março de 1953. Foram acusados de
  conspirarem com um grupo terrorista ligado a organizações sinonistas e
  da inteligência dos EUA para assassinar líderes soviéticos. Com a
  morte de Stálin, o processo foi interrompido, e a nova liderança
  soviética reconheceu que se tratava de uma armação.} de jalecos
brancos confirmaram definitivamente as reflexões ponderadas, sob o
embalo do trem, de Dostoiévski... Não pró, mas contra...''\footnote{Alusão
  ao artigo ``O pró e o contra'' (continuação de ``A Questão Judaica''),
  publicado por Dostoiévski em \emph{Diário de um escritor,} na mesma
  edição de março de 1877.}

Ao longo de fevereiro, as ruas continuaram escorregadias e, em março,
além do regelo, começou a ventar... Agora Vênus, a estrela de Natal,
brilhava na constelação de Áries... Em 2 de março, Aleksei Ióssifovitch,
enfim, foi preso. Levaram-no diretamente de sua cama, onde ele se
deitava coberto por emplastros à base de mostarda, para curar o vento
frio da gripe.

O investigador era um ucraniano de sobrenome Serdiuk. Um sobrenome
militar, cossaco. Era possível encontrar um sargento com esse nome,
assim como um general ou um oficial reformado e literato... No caso,
tratava-se do capitão Serdiuk... Um jovem de Vínnitsa, onde conheciam
muito bem os judeus.

\emph{Vivia ali certo} \emph{Chaim, }

\emph{Por todos adorado...}

Serdiuk elaborava o protocolo tentando se livrar do inoportuno refrão,
como se afugentasse uma mosca. Ele disse:

--- Agora diga, cara de pau: onde você está escondendo o ouro?

E subitamente, Aleksei Ióssifovitch mostrou os dentes, muito assustado:

--- Só faltou me falar ``cara de judeu'', e o senhor ainda se diz um
investigador soviético...

Então Serdiuk passou a tratá-lo por senhor, com cortesia, e disse:

--- Tenha a bondade, coloque-se a par deste material --- e lhe estendeu
uma pasta.

Satisfeito com sua pequena vitória, Aleksei Ióssifovitch soergueu-se
para pegar a pasta, e, nesse instante, Serdiuk acertou-lhe os dentes com
seu punho-martelo de cossaco... Aleksei Ióssifovitch recuou com as
pernas um pouco dobradas, andando de trás para a frente... Ele andou,
andou, andou... O gabinete não era tão grande, mas também não era
pequeno... Ele andou, andou, andou... Daí não foi possível ir adiante, e
ele bateu com a nuca contra a parede...

Assim, o capitão Serdiuk conduziu seu interrogatório de forma
equivocada, e isso lhe foi imputado quando a legalidade foi
restabelecida. Ele foi dispensado dos órgãos, ingressando no instituto
de odontologia, pois ainda era jovem e podia escolher outra carreira,
ainda que fosse parente da anterior. Antes ele quebrava dentes, agora
aprendia a colocá-los de volta. Ou seja, corrigia os erros cometidos no
passado. E Aleksei Ióssifovitch Ívolguin, da tribo de Rúben, morto
durante o interrogatório, finalmente se juntou ao seu povo.

Nesse ano, havia longos pingentes de gelo nos telhados, anunciando longa
primavera. Gansos migratórios voavam alto: sinal de água e de enchentes.
Muita seiva acumulada nas bétulas: um verão chuvoso...\footnote{Provérbios
  russos sobre o tempo.} Entre as águas da primavera, entre as chuvas de
verão, o ano de 1953, incomum e blindado, se desfazia, ia embora. Tudo
embotou, perdeu a solidez e a seriedade. E um camponês rico e gordo, de
rosto redondo, \footnote{Alusão a Nikita Khruschóv.} chegado a ditos
populares, incumbiu-se de explicar à Rússia sua eterna questão. Mas isso
aconteceu um pouco mais tarde. Antes, começou uma fase muito
desinteressante, e a população viveu de maneira desinteressante por
volta de dois anos, de modo que o Anticristo e a profetisa Pelágia não
tinham muito a fazer --- o Senhor não lhes enviou nada de novo... A
profetisa só usou de sua força uma vez, castigando Saviéli, que,
torturado pelo terceiro flagelo do Senhor, a espiava no toalete da sala
de banhos... Por ser um jovem, a profetisa o castigou cruelmente, e ele
foi levado para uma clínica psiquiátrica. Então Klávdia, que havia se
tornado uma mulher solitária, uma viúva inconsolável e uma mãe
sofredora, passou a visitar o zelador... Como todas as criaturas
essencialmente más que sofreram um grande infortúnio, ela não ficou mais
bondosa, porém mais tola. Mas existem diversos tipos des tolices, e a
tolice em uma pessoa maldosa torna-a inquieta. Lágrimas jorram
facilmente, por qualquer motivo, e ela compartilha suas tristezas, de
forma indiscreta, com qualquer um. Assim, subitamente, de um mulher
ranheta е destemida, Klávdia se transformou numa velhinha desamparada,
tola e enfadonha...

Nínotchka Kukharienko, ao visitar sua tia, encontrou-a nesse estado.
Apesar de seus tormentos, Nínotchka havia se transformado numa moça
bonita e forte, mas não muito inteligente, por isso recentemente se
casara de forma irrefletida... Encontrando-se depois de um longo
intervalo, a sobrinha e a tia deram-se bem. Depois Nínotchka contou
sobre o encontro ao Anticristo e à profetisa Pelágia:

--- Nós nos atiramos aos braços uma da outra, gritamos e choramos.

Desde então, volta e meia elas iam tomar chá com a família do zelador Dã
Iákovlevitch. Nínotchka, uma jovem falante, contava:

--- Em 1949, meus pais foram vítimas da repressão e, no mesmo processo,
foi envolvida a família Iarnutóvski. Claro que naquele tempo eu era
muito pequena, mas me lembro de várias coisas, mesmo da época em que
ainda me carregavam no colo.

A essa altura, Klávdia geralmente desatava no choro e dizia:

--- Bravo! Você conseguiu se conduzir na vida, não se levou por um mau
caminho. E como você se parece com a minha irmã Vália...

--- Procurei meus pais por dois anos --- contava Nínotchka ao Anticristo
e à profetisa Pelágia. --- No começo, eu encontrei a mãe do Iarnutóvski,
Vassilina Matvéievna. Ela também dedicou um bom tempo à procura de seus
parentes pela Bielorrússia, mas não solicitou ajuda da União, porque
estava doente e era analfabeta. Mas ela se afligia muito com seu filho
Nikolai. Quem nos ajudou nas buscas foi o ex-ministro de Justiça da
Bielorrússia, o camarada Vetróv.

Nesse ponto, Klávdia voltou a chorar, lembrando-se de seu marido,
Aleksei Ióssifovitch, e da doença grave de seu filho, Saviéli.

--- Vamos embora, tia --- disse Nínotchka ---, você está transtornada.

--- Não, fale, fale. Dã Iákovlevitch é um homem bondoso. Ah, como é
agradável desabafar as mágoas com um homem bom, como é prazeroso, eu sei
por mim mesma.

Nínotchka continuava:

--- Não foi possível encontrar meu pai (ao que parece, ele não estava
mais vivo), nem os Iarnutóvski, mas a minha mãe, Valentina, eu consegui
achar... Porém, ao achá-la, fiquei muito desapontada, pois vi uma mulher
totalmente dominada pela bebida, e foi muito doloroso perceber que,
naquele período tão difícil de sua vida, ela não conseguiu resistir e se
entregou. Porém, ao me encontrar, ela não pôde continuar assim e se
matou, afogou-se no Volga...

Nínotchka se calou e Klávdia aquietou-se, não chorando como de costume,
embora, aparentemente, fosse o momento mais apropriado para isso... O
Anticristo e sua filha, a profetisa Pelágia, também ficaram em silêncio.
``Aqui se manifesta aquele sofrimento que, para os filósofos cristãos, é
a medida de todas as coisas,'' pensava o Anticristo. ``No entanto,
somente o homem bom se torna mais inteligente com o sofrimento; o homem
ruim, sem personalidade, pode apenas atoleimar. É por isso que os
sofrimentos e a tolice são mais difundidos no mundo.''

--- Meu pai, Aleksándr Semiónovitch Kukharienko --- continuava Nínotchka
---, ficou preso no campo de Burepolómski,\footnote{O campo de trabalho
  correcional (\emph{Ispravítelno-Trudovói Láguer)} Burepolómski, em
  Níjni Nóvgorod, era extremamente rigoroso --- para lá eram mandados
  criminosos reincidentes. O campo também é lembrado por ter sediado a
  primeira grande revolta de detentos (1951).} mas ninguém sabe onde ele
foi parar depois. Minha mãe me disse que ele lhe escreveu até o momento
em que ela sonhou com a morte dele.

--- Minha irmã era tão bonita... --- disse Klávdia, pressionando um
lenço contra os olhos.

--- Sim --- disse Nínotchka ---, minha mãe tinha uma constituição
robusta e uma beleza encantadora. No verão, usava uma blusinha branca,
uma saia cinza e um lenço branco na cabeça e, no inverno, botas
cromadas, saia de um delicado xadrez e um casaquinho com gola felpuda
ruivo-acinzentada... Lembro que havia flores amarelas perto de nossa
casa... Às vezes fico triste, especialmente de noite... Mas não é
nada... Pois eu sou motorista, trabalho num caminhão, assim como meu
marido Fiédia\footnote{Apelido de Fiódor.}. Não é à toa que escolhi essa
especialidade. Em caso de guerra, serei a primeira a ir para o
\emph{front}, me sentarei em um tanque e me vingarei dos imperialistas,
por todos nós. Eu entendo que, se não fosse o cerco imperialista, tudo
seria diferente.

Nínotchka tinha vindo por pouco tempo e, no dia seguinte, após essa
conversa noturna, ela deveria voltar para casa, no Extremo Oriente, onde
ela havia crescido, no orfanato.

--- A pátria não se esqueceu de mim, me deu abrigo e educação --- dizia
Nínotchka ---, eu me casei, caí em boas mãos... Meu irmão Míchenka
morreu de febre tifoide em Tobólsk. Eu sou a única sobrevivente da
família Kukharienko. E, de repente, tenho a impressão de que estou
sozinha no mundo, claro, em meu coletivo, grande e unido...

Depois isso, ela foi se deitar acompanhada por sua tia atenciosa, para
que não perdesse o trem na manhã seguinte.

Aristóteles, um contemporâneo dos últimos profetas bíblicos, trezentos
anos antes do nascimento de Cristo e da deturpação desse grande
personagem bíblico, escrevera que sem ação não há tragédia, mas pode
haver tragédia sem personagens. Por exemplo, segundo ele, a maior parte
das novas tragédias não representam personagens, pois a tragédia não é
uma imitação dos homens, mas da ação e da vida, da felicidade e da
infelicidade, e ambas residem no interior da ação.\footnote{Reunião de
  trechos de \emph{Poética} (de traduções russas da obra).}

Depois de 1953, começou na Rússia um período em que, conforme
Aristóteles, a ação histórica continuava, mas as personagens
desapareciam. A tragédia conclui a vida, ou um período de vida, de um
homem e de uma nação, a comédia a reanima. Diante do Anticristo, o
enviado do Senhor, a personagem passou pela coletivização torturante,
pela guerra nefasta e pelas esperanças do pós-guerra; pelo segundo
flagelo do Senhor --- a fome ---, pelo primeiro --- a espada ---, e pelo
terceiro --- o adultério... Mas, ao chegar ao quarto flagelo --- a
doença, a chaga do espírito ---, a personagem já quase havia
desaparecido, em definitivo, tornando-se mais fraca e mais trivial,
embora a força da infelicidade não tivesse diminuído, mas aumentado. No
entanto, se olhássemos de cima, perceberíamos que mesmo antes, tanto na
Rússia como no resto do mundo, houve grandes malfeitores de caráter
desinteressante, ordinário e mesquinho, e grandes mártires de alma
mesquinha. É pouco provável que Púchkin ou Shakespeare se interessassem
pela personalidade de Hitler-Schicklgruber\footnote{O nome de bastismo
  do pai de Adolf Hitler era Alois Schicklgruber (1837\emph{--}1903),
  que mudou seu sobrenome, de origem camponesa, para Hitler (Hiedler)
  (que veio de seu padrasto), considerando-o mais apropriado para sua
  carreira política.} ou de Stálin-Djugachvíli.\footnote{Sobrenome
  verdadeiro de Stálin. Há muitas versões sobre a origem do sobrenome
  ``Stálin'', mas depois se espalhou a ideia, ao que parece para
  fortalecer sua imagem, de que ele teria sido criado de \emph{stal,}
  ``aço'' em russo.} É pouco provável também que as vítimas de suas
crueldades, especialmente no período mais atroz, fossem interessantes
como personagens. A grande tragédia perde a personagem, mas, sem
personagem, não é possível haver uma existência longa. Daí é a comédia
que floresce, então o renascimento acontece através da personagem
cômica. E é exatamente assim. Muitas personagens cômicas aparecem no fim
dos anos cinquenta e sessenta. Como sempre, na comédia, elas surgem em
combinações bizarras, com inclinações estranhas e, com frequência, sem
nenhuma explicação, de maneira muito caótica, pois a comédia é o gênero
mais afastado do Senhor e, portanto, o mais humano.

Ao voltar da clínica psiquiátrica, Saviéli, de um adolescente de
inclinações perversas, virou um jovem sonhador, inofensivo e lírico. E o
caminho natural para ele, evidentemente, era seguir no mais cômico dos
estabelecimentos de ensino que já existiram no mundo: o Instituto de
Literatura da União dos Escritores Soviéticos. Lá ele encontrou pessoas
do Volga, conterrâneos da cidade de Bor, da região de Górki, o jovem
poeta lírico Andrei Kopóssov e o satírico Sómov. E também Vássia
Korobkóv, um rapaz estranho, de biografia misteriosa, passado da idade
de estudar, pelo que diziam um antigo ladrão, de olhos escuros e
aparência oriental, quase judia, mas era um notório antissemita que
gostava de tumultuar. Com esse grupo dava-se também um velho desleixado,
Ilováiski, literato erudito que começou a falar do cristianismo russo
bem antes de conversas desse tipo se tornarem respeitadas pela sociedade
e valorizadas por mulheres implicantes.

Devemos notar que, ao começar uma conversa, Ilováiski mostrava seu
melhor lado, um homem inteligente e um hábil popularizador de ideias.
Mas isso apenas na primeira meia hora de contato com ele. Nesses trinta
minutos, ele habitualmente esbanjava toda a sua grande sabedoria para
depois, o restante do tempo, dizer asneiras sem parar.

Foi o que se deu quando Ilováiski foi apresentado ao Anticristo e à sua
filha adotiva, a profetisa Pelágia. Quem proporcionou esse encontro,
evidentemente, foi Saviéli, que amava Rute-Pelágia fazia tempo, amava em
segredo, como ele, em assuntos do gênero, havia se acostumado a se
deleitar. Ilováiski desalinhava-se à maneira dos debatedores russos, mas
seus olhos, apesar de claros, não eram abertos como o dos russos. Além
disso, ele tinha sido reabilitado e agora bebia todos os dias. Às vezes,
ele parecia interessado em Klávdia, a viúva de Aleksei Ióssifovitch
Ívolguin, mãe de Saviéli. Em todo caso, quando ele aparecia, Klávdia
sempre pintava os lábios e, em lugar do retrato de Stálin, sentado à
escrivaninha de seu gabinete no Krêmlin, ela havia pendurado um ícone de
Jesus Cristo.

Um dia, eles tomavam chá e travavam o habitual e tedioso debate russo
sobre Cristo. Os russos sabem fazer muitas coisas com alegria, inclusive
debater. Mas, quando discutem sobre Cristo, é sempre com um tédio
surpreendente e de forma confusa, embora convincente. Experimente
discutir sobre Cristo com um cristão russo de ideias firmes. Nas
primeiras palavras, tem-se a impressão de que é possível falar
abertamente com ele e fazê-lo mudar de opinião, pois seus argumentos, à
primeira vista, parecem por demais tediosos e ingênuos. No entanto,
conforme a discussão se alonga, você se surpreende com uma estranha
sensação: sente-se mais inteligente do que o russo, mas é o russo que
fala com mais inteligência... Dã, a Áspide, o Anticristo pensava nessas
horas que nem se o seu irmão, Jesus, da tribo de Judá, um sábio fariseu
da casa de Davi, Jesus, o filho adotivo de José, aparecesse em pessoa,
ele conseguiria provar nada sobre si mesmo para um cristão russo como
esse, mesmo que Jesus tivesse provado habilmente suas ideias aos membros
de sua própria seita de fariseus, pois, embora essas pessoas lhe fossem
hostis, de um modo geral compartilhavam de sua visão de mundo e de sua
fé na Lei de Moisés... Aqui, as opiniões de Ilováiski eram aparentamente
iguais às dele, de Cristo, estudadas no Evangelho, mas a visão de mundo
desse cristão russo lhe era totalmente hostil, estranha. Cada palavra
dele se tornava irreconhecível, a ponto de você, o interlocutor, se
sentir impotente diante de suas próprias palavras. Daí surgiu, em
essência, a teoria ateísta de que Deus, depois de ter criado o mundo,
não mais interferiu em suas questões, pois, visto assim, é como se Deus
não mais existisse , apesar ter existido um dia. É na crença na
existência remota de Deus que reside a única diferença entre o
materialismo teológico e o materialismo comum.

Mas o sentido da discussão era especialmente inatingível quando cristãos
russos de ideias firmes começavam a discutir entre si e a dizer a mesma
coisa, mas usando palavras tão distintas que a discussão se tornava
completamente incoerente. Tudo ganhava um ar tão absurdo que parecia
que, a qualquer momento, se desvendaria a incógnita que era inacessível
a discussões sensatas, inteligentes... Sem consciência de si, o
insensato dirá a Palavra... Aquela Palavra que é o ponto principal do
menos judeu dos quatro Evangelhos e o mais apreciado pelo intelectual
decadente russo, o Evangelho de João... E os insensatos são levados
desse Evangelho ao Apocalipse... O Apocalipse de João é igualmente
estimado. Contudo, será o mesmo João? A obra menos judia da literatura
evangélica é o quarto Evangelho, e a mais judia é o Apocalipse, o livro
do ódio e da esperança. É aquele mesmo ódio contra o Império Romano que
também sobrecarregou o coração de Cristo. O Apocalipse mostra com
evidência o que o Evangelho de Mateus sugere com cautela: o ódio dos
construtores do Templo contra os construtores da Torre de Babel, que
representa qualquer império. No entanto, o Evangelho de Mateus --- assim
como os Evangelhos de Marcos e Lucas, mas especialmente o de Mateus ---
foi escrito com João, o pai do Apocalipse (são irmãos de espírito),
enquanto o Evangelho de João foi escrito por um inimigo hábil e
talentoso, do ponto de vista literário, não do espiritual. Foi no quarto
Evangelho que nasceu originalmente a Palavra, mas seu sentido só fica
claro depois. É plástico, à maneira grega, mas se pode sentir a
tentativa de dar uma imagem a Deus, sente-se que esse foi o início da
divisão entre o bíblico e o grego, entre o cristianismo judaico e o
cristianismo pagão. No entanto, acontece justamente o contrário: o
Senhor às vezes dá o sentido ao insensato, mas sem a Palavra, o sentido
através do pranto divino, inarticulado, da mesma forma que, em 1933, a
jovem mártir Maria chorara perto da estação de Andréievka.

Todo o espírito do quarto Evangelho é grego e antibíblico. Mesmo assim,
no cosmos não há baixas alturas. O grande é grandioso inclusive na
decadência, no misticismo e na queda. Somente no insignificante não há
queda ou decadência. O acmeísta Gumilióv escreveu: ``E no Evangelho de
João está dito que a Palavra é Deus...''.\footnote{Do poema ``A
  palavra'', escrito em 1921 por Nikolai Gumilióv (1886\emph{--}1921),
  que, ao lado de Anna Akhmátova (1889\emph{--}1966), sua esposa, e
  Óssip Mandelstam (1891\emph{--}1938), criou o acmeísmo, buscando uma
  linguagem mais apolínea e clara, pautada no mundo sensível e no
  cotidiano.} Claro que não é assim, isso não é o modo bíblico... A
Palavra sempre rebaixa o sentido. Em um diálogo entre Deus e um profeta
se rebaixa o divino, em um diálogo entre um profeta e o povo se rebaixa
o profético. Os profetas sabiam que na Palavra elevada Deus é rebaixado
e na Palavra insignificante Deus está ausente... No entanto, há tempos
não existem profetas e o divino fora inúmeras vezes rebaixado, antes de
se aproximar do povo através da insignificante.

Por isso hoje mesmo a Palavra aleatória, grega e não bíblica do quarto
Evangelho é valorizada. Mesmo uma Palavra humana, que se adianta ao
sentido de Deus...

No auge da discussão russa acalorada sobre Cristo, quando todos --- até
os habitualmente mais tolos, como Klávdia, a viúva de Aleksei
Ióssifovitch --- davam mostras de inteligência (por isso não era
possível se deter sobre algo ou entender alguma coisa), Ilováiski disse,
agarrando com os dedos reumáticos uma xícara de chá de amplo consumo,
branca de borda azul, que cheirava a vodca:

--- Olhem para este cálice --- usou a palavra ``cálice'' em lugar de
``xícara'', pois se achava um conhecedor da Antiguidade ---, olhem para
este cálice... Agora é um simples cálice... Mas vou jogá-lo no chão e
ele se tornará um cálice complexo...

Com efeito, à russa, com modos antiburguses e impiedosos, ele arremessou
no chão o objeto que não lhe pertencia, ouviu-se um estalo, fragmentos
se espalharam e todos se calaram, pois realmente o cálice ordinário
tornou-se complexo. Então Dã, a Áspide, o Anticristo entendeu que
através desse homem insensato o Senhor lhe enviaralhe um Sinal,
permitindo que a Palavra fosse dita antes de seu sentido ser
determinado. Sua filha adotiva, Rute, a profetisa Pelágia, também o
compreendeu.

Assim, desde 1933, desenrolaram-se as quatro parábolas do Senhor, e cada
parábola continha os quatro flagelos do Senhor, revelados através do
profeta Ezequiel. Em cada parábola, um dos flagelos se sobrepôs sobre os
demais, tornando-se o ponto principal. Ora se soprepôs o segundo flagelo
--- a fome ---, ora o primeiro --- a espada ---, ora a terceiro --- o
animal feroz, o adultério ---, ora o quarto --- a doença, a peste. Eis
que entre esses flagelos se concluiu a vida de uma geração, e agora se
faz necessário fazer um balanço por meio de uma quinta parábola. O
profeta Moisés fez um balanço do desígnio de Deus por meio do sangue do
Testamento, despejando-o num cálice para depois aspergir o povo com
ele.\footnote{Passagem descrita aparece no Êxodo 24: 5, 8: ``{[}...{]} e
  imolaram a Ianweh {[}Senhor{]} novilhos como sacrifícios de comunhão.
  {[}...{]} Moisés tomou do sangue e o aspergiu sobre o povo, e disse:
  `Este é sangue da Aliança que Iahweh {[}Senhor{]} fez convosco,
  através de todas essas cláusulas''. (\emph{Bíblia de Jerusalém,}
  Paulus, 2016, p. 137)} Moisés não aspergiu o povo com a água do rio,
mas com o sangue. Mas esse cálice agora havia se quebrado, e sobre isso
irá discorrer a quinta parábola, pela qual o Anticristo foi enviado à
terra...

5

Quando o cristianismo, a criança que se tornou órfã, perdeu sua mãe
judia --- em virtude da eterna rivalidade entre os construtores do
Templo e os da Torre de Babel ---, ele inicialmente caiu nas mãos dos
que sabiam, se não tudo, ao menos muita coisa sobre sua mãe, mas eram
hostis a ela. O tutor grego, pois essa figura era principalmente grega,
representante de uma base espiritual completamente diferente, fez de
tudo para que essa criança não conhecesse a verdade sobre si mesma. Para
isso, ele introduziu a vida em reclusão não como um método criativo
temporário, usado tanto por Moisés quanto por Jesus, mas como uma rotina
monástica permanente, que criou a base ideológica para que a criança
fosse definitivamente tirada de sua mãe judaico-cristã, sendo obrigada a
esquecer a verdadeira imagem materna, suas verdadeiras esperanças, seus
pesares e sofrimentos em meio ao seu povo, que perecia. Na reclusão
monástica, deu-se até um novo aspecto físico a Jesus. Não, não era a
imagem do sábio fariseu que, já na mocidade, surpreendera professores de
cabeleiras brancas, conhecedores da Bíblia, nem a figura daquele que
compreendera o sentido prático e a força da doutrina do profeta Jeremias
sobre a não resistência ao ímpio, do qual, quando enfraquecido, é
possível tomar a própria alma, como um butim. Também não era a aparência
do sábio que havia percebido que a voz do profeta é uma voz que clama no
deserto. O profeta vaticina o futuro, mas o povo só reconhece a razão do
profeta quando o futuro se converte em passado. Por isso o profeta
precisa do mesmo poder que possuía Moisés. Cristo-Rei --- eis quem é
agora o Salvador do povo... Ele sabe como é pesada a cruz do rei dos
judeus... Os mais corajosos e desprendidos são ignorantes, os mais
sensatos e sábios são covardes e interesseiros. Assim se dá sempre que
um povo é oprimido por longo tempo, e isso estava claro para ele, o
conhecedor da Bíblia e dos profetas. Ele se lembrava das palavras de
Moisés, ele sabia que o Salvador e o Patriota deviam possuir também
astúcia, porque o mundo é um covil de lobos. Com pessoas letradas,
exprimia-se com a fala afiada e furiosa do experiente polemista; com
pessoas ignorantes falava por alegorias, pois o caminho da escuridão
passa pelo misticismo, e a confiança do ignorante só pode ser
conquistada se este não compreender nada do que ocorre. Se um ignorante
entende o particular, rejeita o todo que lhe é inacessível. Desse modo,
o milagre se faz necessário tanto no geral como no particular, tanto na
ideia principal do que traz alívio ao homem como em pequenas curas. Para
a cúpula dos colaboradores letrados, acomodados no trono de Moisés, ele
não passava de um impostor irrequieto, o que, a propósito, condiz com a
realidade. Eles o entendiam e por isso o odiavam. Para os ocupadores
romanos, ele era o destruidor da Lei de Moisés, um rival da ideologia
pagã. Eles não o entendiam e por isso tentaram usá-lo como um
colaborador. Dessa sorte, também Jesus repetiu, quase com exatidão, o
destino de seu pai espiritual, o profeta Jeremias, que fora colocado no
calabouço por seu povo amado e de lá retirado pelos assírios, seus
inimigos odiosos. Pois o profeta pode prever e compreender o destino de
um povo, mas é impotente diante de seu próprio destino. Assim, o
Salvador também era impotente diante de seu destino. A verdade estava
nas palavras dos que o ridicularizaram, pregrado numa cruz: ``Salvou os
outros, mas a si mesmo não pode salvar''.\footnote{Marcos 15:31.} Ele
estava surpreendentemente sozinho, não só na cruz, como também antes de
ser colocado nela. Os apóstolos, que ele sempre desprezou interiormente,
perto do fim da vida dele, ficariam anda mais desapontados, procurando
um jeito de livrarem-se dele. Os ignorantes, em contato com uma grande
personalidade, começam a entender o particular, rejeitando o todo que
lhes é inatingível.

Pouco antes da Páscoa, na casa de Simão, o leproso, em Betânia,
fortaleceu-se o conflito direto entre Jesus e os apóstolos. Assim foi
descrito no mais confiável Evangelho, o de Mateus:

``{[}...{]} aproximou-se Dele uma mulher segurando um vaso de alabastro
com um unguento de grande valor que ela verteu na cabeça Dele, com o
corpo recostado na mesa. Vendo isso, seus discípulos se indignaram,
dizendo: `Para que esse desperdício? Pois se poderia vender esse
unguento por uma boa soma e distribuí-la entre os pobres.''\footnote{Mateus
  26:7, 8.}

Aqui, os apóstolos fazem uma clara alusão ao fato de Jesus não observar
seu próprio ensinamento, segundo o qual tudo deve ser repartido com os
pobres. Compreendendo a recriminação deles, Jesus respondeu:

--- Vós sempre tereis os pobres convosco, mas a mim nem sempre
tereis.\footnote{Mateus 26:11.}

Ele se lembrava das palavras de Moisés: ``{[}...{]} não favorecerás o
pobre em seu processo {[}...{]}''.\footnote{Êxodo 23: 2,3.} Ele sabia
que a pobreza é uma doença e uma desgraça, mas não é um mérito... Foi
justamente depois dessa disputa que Judas Iscariotes decidiu entregar
Jesus ao sumo sacerdote. Mas o que, a rigor, significava entregá-lo nas
condições estritas da letra da Lei? Significava provar a culpa dele
perante o tribunal. ``Os sumos sacerdotes, os anciãos e todo o Sinédrio,
o conselho supremo, estavam à procura de falsos testemunhos contra
Jesus, para que ele fosse entregue à morte, mas não os encontraram,
embora lá se achassem muitas testemunhas falsas. Por fim,
apresentaram-se duas, que disseram: --- Esse homem disse: `Posso
derrubar o templo de Deus e reedificá-lo em três dias'.''\footnote{Mateus
  26:59\emph{--}61.} A quem Jesus disse isso? Conforme o Evangelho,
disse-o apenas aos apóstolos, isto é, as duas falsas testemunhas
desconhecidas eram apóstolos. Todo o comportamento posterior de Judas
Iscariotes, que é retratado na literatura cristã e no Evangelho de João
como a encarnação do mal, na realidade nos mostra que esse homem não
passou de um instrumento nas mãos de inimigos mais finórios e perigosos
de Jesus achados entre os apóstolos, que permaneceram no anonimato.
Judas era mais ingênuo e mais sincero, menos hábil em ocultar seus
sentimentos do que os outros, e Jesus, que suspeitava de um complô entre
os apóstolos, apontou Judas simplesmente por ele saltar à vista mais do
que os outros, sem dúvida encobrindo o plano astuto de alguém. Ao
indicar Judas, Jesus também não demonstrava confiança nos demais. No
monte das Oliveiras, Jesus lhes disse: ``Esta noite todos se sentirão
tentados por mim, pois está escrito: `Eu irei ferir o pastor e as
ovelhas do rebanho irão dispersar''.\footnote{Mateus 26:31.}

Na província, na Galileia, Jesus era uma personalidade conhecida, mas na
capital poucos sabiam de sua existência, e quando o ``destacamento de
ouro'' de Jerusalém chegou para prendê-lo, foi preciso que Judas
beijasse um dos doze forasteiros para indicar quem havia
blasfemado.\footnote{``Seu traidor dera-lhes um sinal, dizendo: `É
  aquele que eu beijar; prendei-o'' (Mateus 26:48). (\emph{Bíblia de
  Jerusalém,} Paulus, 2016, p. 1752)} E mais adiante: ``Mas tudo isso
aconteceu para que se cumprissem os escritos dos profetas. Então todos
os discípulos, tendo-o abandonado, fugiram''.\footnote{Mateus 26:56.}

Assim Jesus caiu, vítima não apenas do ódio evidente dos que colaboravam
com os romanos, mas também vítima de uma conspiração interna dos
apóstolos, que induziram Judas e o expuseram. A prova de que Judas
Iscariotes era um homem ingênuo, mediano, mas consciente é sua conduta
após o julgamento: ``Então Judas, vendo que Jesus fora condenado e
tomado pelo remorso, devolveu as trinta moedas de prata aos sumos
sacerdotes e aos anciãos, dizendo: `Eu pequei ao entregar sangue
inocente'. Mas eles lhe disseram: `Isso não nos diz respeito, resolve
sozinho'. E, jogando as moedas de prata no Templo, retirou-se e foi se
enforcar''.\footnote{Mateus 27:3\emph{--}5.} Aqui é possível reconhecer
um homem de caráter honesto, mas insensato, que não se deu conta do
curso dos acontecimentos, abismado com o fato de suas palavras
insensatas terem levado Jesus a ser setenciado à morte. Contudo, Judas
está caracterizado, tanto na literatura como no pensamento cristão, como
a figura canônica do traidor, para que os verdadeiros e sensatos
traidores fossem mantidos em segredo. Ainda hoje esses traidores são
considerados santos apóstolos e em sua honra são erigidos templos de
Deus.

Assim, a calúnia e a mentira apareceram, desde o princípio, no seio dos
apóstolos, sendo ainda reforçadas pelo apóstolo Paulo, da tribo de
Benjamim, que nunca vira Jesus ou ouvira sua Palavra viva e provinha dos
antigos inimigos de seu Ensinamento... Por isso, não é de admirar que,
na reclusão grega, tenham dado até um novo aspecto físico a Jesus,
descarnado, com a carne mortificada, que mais lembrava santo Antônio do
que o filho da casa de Davi.

Mais tarde, na Alta Idade Média, o cristianismo adolescente se achava
nas mãos daqueles que não eram somente hostis à mãe palestina, mas
também ignoravam sua essência. Apenas eventualmente, através da magia
negra, o cristianismo vislumbrava a verdade oculta em si, mas ele se
amedrontava e castigava os mais talentosos por havê-la descoberto.
Conforme amadurecia, o cristianismo caiu nas mãos de pessoas totalmente
estranhas ao judaísmo, já que os gregos eram hostis ao judaísmo, mas não
estranhos a ele. Eis por que muitas coisas simples e praticamente claras
da casa materna tornaram-se complexas, inatingíveis, refletindo a
metafísica impenetrável da casa estrangeira. Pois toda palavra humana,
em outro mundo, transforma-se em cifra. Talvez seja por essa razão que o
dogma fundamental do cristianismo, a não resistência ao mal, , tenha
sido convertido em código metafísico não pelos primeiros cristãos, mas,
o mais provável, pelos povos poderosos da Alta da Idade Média, quando a
hostilidade dos antigos tutores do cristianismo contra sua verdadeira
mãe judia ainda era sentida como uma ação viva, e não como um elemento
mitológico, o que só depois surgiu no seio do cristianismo eslavo; ao
mesmo tempo, na Alta Idade Média, a língua espiritual da Bíblia já tinha
sido perdida, tornando-se incompreensível. Quando as palavras sobre a
não resistência ao mal perderam a essência --- palavras que Jesus, da
tribo de Judá, dirigia ao seu povo amado, teimoso e desobediente, que
era exaurido por uma luta árdua ---, elas viraram o provérbio do filho
de Deus que, ao descer do céu, entabulava conversas no deserto com
monges gregos que mortificaram sua carne. Quando dessas palavras
desapareceram a sabedoria do político e a amargura do patriota, restaram
ensinamentos universais, privados da linguagem nacional, cada vez menos
acessíveis ao coração vivo... Por que isso se deu? Desde seus
primórdios, o cristianismo sempre foi hostil ao judaísmo, mas afirmava
sua fé no mundo por meio de autossacrifício e abnegação. Foi assim que a
afirmação extremada da origem divina e celestial de Cristo levou ao
ateísmo. Será que os ateístas não se ocupam da mesma tarefa ao tentar
provar a origem mitológica, anti-histórica, da figura de Jesus, ao
negá-lo como uma personalidade nacional, um dos líderes do movimento
nacional de Nazaré?

Em tempos remotos, o mercador grego Marcião\footnote{Marcião de Sinope
  (c. 85\emph{--}160), fundador do marcionismo, propôs a separação do
  Antigo e Novo Testamento, que teriam dois deuses distintos, sendo
  acusado depois de heresia.} escreveu um Evangelho em que negava a
relação de Cristo com o Deus judeu da Bíblia. ``O Deus da Bíblia,''
afirmava Marcião, ``é o Deus do mundo material, enquanto o pai de Cristo
é o Deus do mundo espiritual.'' O Concílio ecumênico, então, rejeitou o
Evangelho de Marcião. Ele era visivelmente falso, distorcia
demasiadamente o fato autêntico, cheirava a politeísmo e paganismo. No
entanto, muito mais tarde, o Concílio acrescentou aos três Evangelhos
canônicos um quarto, o de João, que, vale sublinhar, não tinha relação
alguma com o São João, o autor do Apocalipse. Nesse quarto e decadente
Evangelho, de uma forma mais hábil e pitoresca que em Marcião, prova-se,
em essência, a mesma coisa, e Cristo é separado do Deus bíblico... É
interessante notar que, de Evangelho em Evangelho, enfraqueceu-se o
motivo da conspiração dos apóstolos contra Jesus. No mais antigo e
autêntico Evangelho, o de Mateus, o episódio é apresentado
integralmente; no Evangelho de Marcos, ainda é apresentado de um modo
marcante; em Lucas, aparece um tanto enfraquecido; em João, está
completamente ausente. Os episódios mais trágicos que precederam a morte
de Jesus foram descritos de forma muito distinta. Do Evangelho de João
desapareram tanto a conspiração dos apóstolos como a hostilidade entre
os apóstolos e Jesus, e nada se dizia a respeito das duas falsas
testemunhas misteriosas, da calúnia que levou Jesus a ser setenciado à
morte. Quanto a Judas, ele foi representado como um traidor isolado, um
filho de Satanás. Foi omitida a passagem em que ele, mergulhado na
tristeza, abdicou das moedas de prata, e, ao contrário, sublinhou-se sua
cobiça através da caixa de moedas que carregava. A bem da verdade, foi
descrito como Pedro, por fraqueza de cárater, se afastou provisoriamente
de Jesus, mas esse é um fato flagrante. No entanto, o principal, o
complô premeditado por um grupo de apóstolos que estava claro em Mateus,
foi totalmente ocultado em João. Assim, a conspiração dos apóstolos
contra Jesus transformou-se em uma conspiração do cristianismo contra o
Cristo. O claro e simples Cálice de Deus foi quebrado impiedosamente em
complexos fragmentos metafísicos, filosófico-religiosos. Na ``Lenda do
grande inquisidor'', Dostoiévski nos oferece uma imagem de Cristo sem
vida, antinacional, cósmica, celeste, mas a conspiração terrena do
cristianismo contra o Mestre foi descrito de um modo bastante preciso. É
verdade que, na obra de Dostoiévski, o cristianismo foi chamado de
``catolicismo''; no entanto, no mundo cristão, em fragmentos, isso não
passava de um procedimento polêmico natural que poderia tranquilamente
também ser voltar contra a ortodoxia.

Assim, separando-se da Bíblia e da Lei de Moisés, o cristianismo
enveredou pelo caminho natural e lógico do isolamento e do
\emph{raskol}.\footnote{O termo \emph{raskol} significa ``cisão'',
  ``cisma'', ``divisão'', e também alude ao movimento dos \emph{velhos
  crentes} (\emph{staroobriádtchestvo}), que rompeu com a Igreja
  Ortodoxa Russa no século XVII devido às reformas do patriarca Níkon. O
  termo \emph{raskol}, a propósito, surge no sobrenome de Rodion
  Románovitch Raskólnikov, a personagem principal de \emph{Crime e
  castigo} (1866), de Dostoiévski.} A conspiração contra Moisés se
transformou na conspiração contra Cristo. Não é de hoje que os ideólogos
do cristianismo não têm uma ideia espiritual comum e, como não a têm,
procuram um inimigo físico comum, que poderia ajudar a preservar uma
unidade ilusória. Contudo, esse inimigo físico comum foi encontrado há
tempos, ainda na rotina monástica dos primeiros anacoretas gregos. E o
nome dele é prazer. O cristianismo ensina o homem a fugir do campo dos
prazeres, o campo de Satanás, desviando-o do caminho que leva ao Senhor,
enquanto a Bíblia o ensina a atravessar o campo dos prazeres, de
Satanás, indo na direção do Senhor, pois não há outro caminho: o homem
foi amaldiçoado e o Senhor o expulsou do paraíso repleto de alimentos
celestiais, fazendo-o procurar seu próprio alimento espiritual à custa
de muito suor. Se, no campo dos prazeres, um ateísta se esforça por
buscar seu pão espiritual, ele realiza o anseio do Senhor; mas, se um
homem que se considera religioso espera, no campo dos prazeres, que o
pão espiritual caia do céu, ele vai contra o Senhor. O cristianismo, que
governou o mundo por mais de quinze séculos, agora acusa o ateísmo pelas
imperfeições do mundo, embora este tenha tomado o poder há menos de um
século. Esse é o mesmo cristianismo que exerceu o poder sobre o mundo,
apoiando a conspiração secreta dos apóstolos contra o Cristo. Também é o
mesmo cristianismo que passou séculos em ociosidade espiritual,
entregando-se a uma contemplação puramente budista de verdades
metafísicas e substituindo o Ato por discussões enfurecidas sobre o bem
e o mal... É o cristianismo que até hoje cobre de maldições aqueles que,
num ímpeto saudável e sincero, correm para o campo dos prazeres, vão
para onde convém, conforme os desígnios do Senhor. Mas, para sua
infelicidade, os que fugiram dos sermões alienados passaram pelo
perigoso campo do Diabo conduzidos não pelo árduo trabalho espiritual do
Mestre, mas pelos próprios instintos físicos. Em razão disso, eles com
frequência perecem --- no início do caminho, graças à ignorância juvenil
e, se conseguirem transpor o início, graças a descomedimentos senis, que
levam do Cerne fecundo para outra extremidade, onde reina uma sabedoria
perversa e mística. A ruína desses infelizes só provoca um riso maldoso
nos eunucos cristãos em espírito, acomodados еm seu ócio espiritual. De
resto, agora muitos desses eunucos trocaram as vestes religiosas pela
toga laica do professor de filosofia ou mesmo por um paletó de literato.

Eis a verdade: quem conhece a Bíblia conhece tudo o que é acessível ao
homem, que não conhece nem a Bíblia nem a si mesmo... Um exemplo disso é
a própria Rússia... Há mais de quatro séculos erigem na Rússia a Torre
de Babel.\footnote{Período remonta ao reinado (1533\emph{--}1547) de
  Ivan, o Terrível, o primeiro tsar da Rússia.} A Bíblia faz uma
advertência: a torre consumirá toda a energia, talento e paixão, mas não
será concluída, e a força e o talento virarão pó, como acontecera na
Babilônia. Mas o Cálice foi rejeitado e quebrado, verdades claras
tornaram-se fragmentos metafisicamente complexos. Agitaram-se,
construíram. Veio o arquiteto nacional, Dostoiévski, e olhou em volta. A
torre se aproximava do céu no fim do século XIX. ``Ah, o povo russo.
Onde um russo põe os pés vira terra russa. Mas, irmãos, devemos dar a
essa torre a aparência de um Templo. Assim nos distinguiremos do
Ocidente. Teremos tanto uma Torre como um Templo. O império será
poderoso, a religião será poderosa.'' No entanto, os mais hábeis e
abnegados construtores dos andares superiores eram ateístas. Então, os
construtores cristãos se afastaram e agora riem da desgraça das pessoas
que continuam o desafio da Babilônia ao Senhor, desafio por eles mesmos
iniciado, riem da desgraça daqueles que foram ensinados a receber
verdades do céu como se viessem diretamente das mãos do Filho de Deus,
mas na realidade vieram das patas da frente dos monges anacoretas
gregos. E a história já provou como, nesse caso, é fácil destituir o
Habitante do céu e colocar outro no Seu lugar...

Tudo vem diretamente do céu, pois no Evangelho de Mateus (e eles sabem
que esse é o mais autêntico Evangelho, apesar de admirarem e encherem de
elogios o quarto e decadente, em que o talento literário prevalece sobre
o conteúdo espiritual) há os versículos 63 e 64.\footnote{Mateus 26.} Os
cristãos gostam de citá-los como prova irrefutável. E o que há nesses
versículos? Jesus é levado para o tribunal. O sumo sacerdote, o homem
que fez a grande tribo de Levi chegar ao limite de sua humilhação,
faz-lhe uma pergunta:

--- Tu és o Cristo, o filho de Deus?

Jesus lhe responde:

--- Tu mesmo o disseste. Mas também vos digo: de agora em diante, vós
vereis o Filho do Homem sentado à direita do poder e vindo sobre as
nuvens do céu.

Então o sumo sacerdote rasgou as suas vestes e disse:

--- Ele blasfema.

Mas será que o Cristo blasfemou? Ignoremos o fato de que essa passagem,
em geral, é obscura e anti-histórica. Conforme a Lei de Moisés, só quem
desonra Deus blasfema. Mas, aqui, o Cristo não desonrou Deus. Suponhamos
que o sumo sacerdote, ao colaborar com os romanos, tenha infringido a
Lei de Moisés, mas será que Jesus a infringiu? Todo judeu se considerava
o Filho de Deus, pois, desde os tempos de Abraão, o povo era o povo do
Senhor. Todo patriota podia sentir em si uma força messiânica quando seu
povo corria o risco de desaparecer. Além disso, ao ``Messias'' celeste
costumava-se acrescentar o título terreno de ``Rei dos judeus''. Um
título estranho para uma personalidade de outro mundo, metafísica e não
nacional. No tocante à Ascensão do Filho do Homem para as alturas
celestiais, isso de modo algum é uma blasfêmia, pois, nesse caso, se
deveria acusar também o canonicamente reconhecido profeta Elias, que se
elevou ao céu num torvelinho de fogo... Isso não é uma blasfêmia, como
afirma o sumo sacerdote do Evangelho, mas também não é um fenômeno único
que prova a origem celeste, como afirmam os ideólogos do cristianismo,
confiando nos versículos 63 e 64. Isso não é nada mais do que o estado
de espírito genial de uma grande personalidade em um momento extremo. De
modo que, na realidade, os ideólogos cristãos, ao tentarem elevar o
acontecimento, rebaixaram-no, por serem estranhos à história e a
concepção de mundo nacional judia. E não existe outro caminho para o
verdadeiro entendimento da Bíblia e do Evangelho se não através da
história e da concepção de mundo judia. Mas o Cálice foi quebrado.

O Cálice em si não tem complexidade. Em seu primeiro aspecto, ele não
inquieta o espírito, já seu fragmento, em seu primeiro aspecto que é
também o último, inquieta, pois ele possui uma aparência única, acabada,
de alfa a ômega. Quanto menor o fragmento, mais longe ele ficará do
Cálice, mais acabado ele será e mais inquietará o espírito. Mas o que
inquieta com seu primeiro aspecto exige menos tensão espiritual para que
se alcance a profundidade. O fragmento inquieta o espírito
imediatamente, mas o Cálice não: ele é claro. Só que na clareza do
Cálice se oculta um sentido muito mais profundo do que na essência
impenetrável do fragmento. O Cálice é material e prático na existência,
ele insere na existência o material. Trata-se justamente do que os
judeus foram sempre acusados. Os judeus, dizem, introduziram o lado
material no mundo, eles o arruinaram. E os metafísicos russos nacionais
ficam particularmente exaltados com essas afirmações. Sim, o Cálice é
prático e dialético na existência, mas no eterno ele é metafísico; o
fragmento é metafísico e místico na existência e dialético no eterno,
tentando alcançar o inacessível, dando ao finito um sentido dialético;
conferindo às paixões humanas, ao amor e ao ódio humano desmedido um
sentido místico e metafísico, elevado e eterno e, ao mesmo tempo,
buscando atingir, de modo dialético e filosófico, noções eternas
absolutas, como o Céu e Deus. Entre o Cálice e seus fragmentos existe a
mesma diferença que entre a fé e as religiões, entre o sentido e as
concepções, entre a primazia do sentimento íntimo e a primazia do rito
público... Mas o Cálice de Deus foi quebrado, e disso tratará a última e
quinta parábola do Anticristo, o enviado do Senhor.

\textbf{\\
Parábola do cálice quebrado}

Andrei Kopóssov, como acontece às crianças concebidas por uma mãe que
vivera uma paixão intensa, era um jovem de saúde frágil. A rigor, a
saúde de uma criança pode ser fragilizada também por outros motivos, mas
era como se a paixão desmedida e doentia de sua mãe, Vera, tivesse
exaltado para sempre o menino. Ele cresceu nervoso e, ao mesmo tempo,
tímido, com um sorriso amarelo no rosto. Andrei não conheceu seu pai, em
honra do qual recebera seu nome, pois ele havia morrido meses antes de
seu filho nascer, o que é sempre algo ruim para uma criança. Ele não era
amado em família. Suas irmãs, Tássia e Ústia, davam-lhe palmadas; os
filhos de Tássia, Andrei e Varfolomei Vesselóv, brigavam com ele; o
marido de Tássia, Nikolai Vesselóv, ria dele; e a velha sentinela,
Serguéievna, mãe de Vesselóv, fixava nele olhares de desaprovação.
Somente sua mãe, Vera, o amava, no entanto ela mesma se sentia
intimidada em família: quando suas próprias filhas gritavam, ela se
calava, com ar culpado, e não tinha forças para defender seu filho
querido. Por essa razão, a vida de Andrei em sua cidade natal, Bor, da
região de Górki, foi-lhe um peso desde a infância. Rejeitado pelos
homens, ele se lançou aos livros, tornando-se um frequentador assíduo da
biblioteca de Bor. Nessa época, já tinha passado dos dezesseis anos, e
só um milagre o impediria de escrever versos. Mas o milagre não
aconteceu. Seu futuro estava claro. Sómov, o versejador profissional de
\emph{Pravda de Bor}, colocou-o definitivamente no seu verdadeiro
caminho.

--- Envie seus documentos para o Instituto de Literatura. Você é russo,
do Volga, e ainda talentoso, será sem falta admitido.

O próprio Sómov, que poderia ser pai de Kopóssov, já tinha tentado ser
aceito várias vezes, mas sem sucesso. Contudo, estava confiante de que,
dessa vez, teria êxito, pois tinha, finalmente, conseguido uma carta de
recomendação do Agitprop local.

--- Eles têm uma cisma comigo por causa dos meus versos satíricos sobre
o ferido de guerra Ivan Prókhorov --- explicou Sómov ---, esses versos
agora circulam por Moscou de mão em mão... Ah, Moscou... Andriucha, você
não pode imaginar como é a vida literária lá. E a vida sexual também não
é coisa que se despreze, e todas as moças fumam... Também não precisa
ficar vermelho, que adolescente você é...

Os primeiros poemas de Andrei, que foram publicados no \emph{Pravda de
Bor}, começavam assim:

\emph{Do pão um naco, do Volga um trago... }

--- Você tem talento popular --- dizia Sómov ---, agora isso é muito
apreciado... Todos estão incomodados com a literatura judia... ``Do pão
um naco, do Volga um trago'', há um quê de cristianismo russo.

Assim, pela primeira vez, no décimo sexto ano de sua vida, Andrei ouvia
falar do cristianismo russo como de algo importante e sério,
diferentemente da imagem que lhe passavam os jovens no Komsomol e as
velhinhas cômicas nos átrios das igrejas.

Agora, sentado num quarto em Moscou, que teve a sorte de alugar de uma
velhinha moscovita, já que ela passava a maior parte de seu tempo com o
filho casado, agora, lembrando-se daquela antiga conversa e se sentindo
um homem totalmente diferente, que na realidade não era, Andrei, como
era comum a naturezas semelhantes, teve um acesso de vergonha, diante se
si mesmo e por si mesmo.

Realmente, chegando a Moscou, Andrei ficou ainda mais parecido consigo
mesmo, isto é, seus modos de adolescente se acentuaram mais, no entanto,
à diferença de Saviéli, ele não tinha medo e vergonha das moças, porém
se isolava delas tanto quanto das outras pessoas. Não é que fugisse das
pessoas, mas preferia ficar sozinho. Ao se tornar um estudante do
Instituto de Literatura, ele perdeu o gosto por escrever versos, mas
refletia muito sobre arte e aprendeu a tirar disso uma felicidade que o
levava às lágrimas. Também passou a se ocupar da religião, no começo em
discussões tolas em rodas de amigos e, depois, em suas próprias
reflexões. E essas reflexões contínuas e doentias, normalmente mais
pesadas do que condizia à sua idade, revelaram muitas coisas para ele.
Por exemplo, fazia algum tempo que ele suspeitava que a principal ideia
dos humanistas --- não existem povos ruins, todos os povos são bons ---
era insossa como comida de hospital, sem sal e sem carne suculenta. Essa
ideia era tão destituída de talento como a ideia racista da
superioridade de alguns povos sobre outros. Mas a ideia racista, ao
menos, possuía carne --- ainda que fosse suína e suja, era uma carne
saudável de amor por si mesma e de aversão por tudo o que se achava fora
dela. Ele já sabia que a entrada para esse labirinto se dava através das
questões infantis do cristianismo sobre o bem e o mal. Ele sabia também
que a zona pantanosa cristã de questões metafísicas, alheias a Cristo,
havia retirado parcela importante da força espiritual da cultura
ocidental, impedindo-a de se aproximar de verdades bíblicas que são a
base da existência. Às vezes, compreendia isso com tanta clareza que
todos os sofrimentos espirituais dos gênios do passado pareciam-lhe
compreensíveis. Isso tudo ora o intimidava e o assustava, ora o levava
da clareza aos comentadores de verdades evangélicas, conhecidos e
reconhecidos pela juventude moscovita, que era dominada por eles. E ele
voltava a cair no círculo sedutor das discussões cristãs sobre o bem e o
mal, nas quais as pessoas que julgava mais tolas que ele falavam com
mais inteligência e traziam argumentos irrefutáveis. Suas tentativas de
replicar trouxeram-lhe a fama de reacionário, maldoso, quase um homem de
tendências racistas, e quando, uma vez, numa discussão, Vássia Korobkóv,
sujeito nervoso e antissemita notório, gritou para ele: ``Fascista!'',
Andrei entendeu que as verdades evangélicas formadas durante séculos,
tal como eram impostas por autoridades no assunto, realmente não
deixavam a um homem de juízo outro caminho: ou aceitar essas verdades
tal como haviam sido formadas ao longo de quinze séculos, ou virar um
racista. Isso o assustou e ele parou de procurar companhias dadas a
colóquios espirituais e religiosos, deixando para trás a sólida
reputação de reacionário e de misantropo e, segundo expressão de Vássia,
de descendente da décima quinta geração dos fariseus que haviam
rejeitado e crucificado Cristo. Foi nesse momento, ao perder a confiança
em si e cair em desespero, que Andrei topou com uma passagem a respeito
da relação de Moisés com seu povo. Já tinha lhe acontecido muitas vezes
de ouvir sobre as Tábuas quebradas de Moisés e mesmo de ler e reler a
passagem em que Moisés se indignara com seu povo, que havia traído Deus
e quebrado as primeiras Tábuas, e, somente após ser convencido por Deus,
ele escrevera as segundas. Mas ele tinha lido essa passagem sem o
interesse e a tensão espiritual que lhe causavam os outros episódios do
Evangelho.

Então, de repente, numa manhã, perto das onze horas, quando a senhoria
se achava fora e ele estava completamente sozinho, Andrei leu sobre as
Tábuas de Moisés como se ouvisse falar delas pela primeira vez, com
surpresa e entusiasmo, como se, dessa vez, não tivesse colocado, como de
hábito, sua velha bíblia desleixada, comprada de ocasião, sobre a mesa
coberta por uma toalha antiquada, de um modelo usado antes da guerra,
nem folheasse as páginas com marcas de dedos, mas como se subitamente
realizasse uma subida, em busca da verdade, para algum lugar bem alto,
numa montanha, um lugar mais próximo de si mesmo e mais longe da
existência popular comunal.

Os humanistas ensinavam que não existem povos ruins. Uma atitude nobre,
mas que exigia uma violência contra o próprio bom senso. Os racistas
ensinavam que existem povos superiores e inferiores, e que, entre os
superiores, incluíam, além deles mesmos, os seus próximos, ``por
amizade''. Uma atitude nada nobre, mas realista e conforme o espírito do
cotidiano. Já o ensinamento bíblico de Moisés, se ponderado no estado de
espírito em que Andrei se encontrava nessa manhã, dizia que, em geral,
não existem povos bons. Uma ideia que não exigia uma violência contra o
bom senso, nem dava a ninguém uma superioridade inata em matéria de
vilania. Eis um ponto de partida claro e sólido que permite entender
muito da história material e da vida espiritual do homem. A Bíblia não
dizia em absoluto o que afirmavam muitos de seus adeptos, nem continha o
que negavam seus inimigos. Além disso, enquanto a Bíblia dos ortodoxos
se fechava em si mesma com arrogância, sob o ímpeto multifacetado e
enfurecido de ruas de cristãos aficionados por sua ideologia metafísica,
a Bíblia viva mostrava a inverdade e a essência pagã do culto dos
sofrimentos como base da moral, mostrava a substituição do principal
pelo secundário e que o humanismo --- o endeusamento do homem --- e o
racismo --- o endeusamento da raça --- são irmãos temporãos e frágeis,
mas concebidos sob a paixão do culto dos sofrimentos corporais humanos.

Tudo isso Andrei entendeu num instante e tomou nota, sem rasuras, em um
bloco de papel, por cerca de meia hora. Ele sabia que, por enquanto, não
compreenderia mais nada e que logo passaria a duvidar do que havia
compreendido. Dessa maneira, ele não se deixou seduzir por novas
esperanças, fechou rapidamente a Bíblia e guardou a folha com sua
caligrafia --- mas como se fosse de outra mão --- não entre seus papéis,
mas onde guardava seu dinheiro e seus documentos, num bolso secreto da
jaqueta pendurada atrás do armário, uma jaqueta tão velha que qualquer
ladrão a teria desdenhado.

Faltavam oito para o meio-dia quando Andrei terminou sua vida autêntica
e começou a falsa --- ele marcou esse fato com precisão. Ele começou a
vida falsa com a preparação do desjejum. Dirigiu-se à cozinha comunal
coberta de fuligem onde havia mesinhas individuais na mesma quantidade
das famílias de inquilinos, colocou sobre o fogão a frigideira da
senhoria, derramou alguns ovos na gordura endurecida de frituras
anteriores e, atento ao chiar dos ovos, pensou em como poderia
aproveitar seu dia, sem que perdesse e depreciasse o que havia acabado
de descobrir. Se ficasse sozinho, significaria passar um dia cerebral,
orientado para uma direção, concentrado num único ponto, o que
fatalmente o levaria a dúvidas e poderia anular seu achado. Já se ele
encontrasse pessoas com suas ninharias cotidianas, significaria comparar
continuamente sua descoberta secreta com as trivialidades que aconteciam
em volta e, como resultado, deixaria uma má impressão de si e ainda
defrontaria seu pensamento ainda frágil com algo já estabelecido,
palpável e sólido, o que, de novo, reduziria e empalideceria seu achado.
Dessa maneira, o melhor seria passar o dia em companhia de pessoas, no
entanto evitar assuntos cotidianos e, de preferência, discussões
religiosas. Daí se lembrou de que na galeria Tretiakóv\footnote{Galeria
  de arte em Moscou dedicada a artistas russos. Foi fundada em 1856 pelo
  empresário e mecenas Pável Tretiakóv (1832\emph{--}1898).} havia sido
aberta uma exposição de um pintor francês, antigo emigrado russo, que
provocou barulho e gerou rumores extraoficiais. ``Mas que sorte,''
pensou Andrei, ``e ainda visitarei a galeria Tretiakóv, faz tempo que eu
não vou lá. Ligarei para Saviéli e para Sacha Sómov, meu conterrâneo. E
também ligarei para Vássia Korobkóv, para me cercar de pessoas
diferentes. Não ficarei sozinho o dia todo e, entre pessoas diferentes,
haverá menos conversas sinceras, amigáveis e fúteis. Não preciso delas
agora.''

Era verão, início de junho, as aulas no instituto chegavam ao fim, os
exames se aproximavam, por isso, esse dia, conforme as normas
específicas do instituto, era livre, sem aulas. ``Não terei outra
ocasião para visitar a galeria, dizem que a exposição não ficará por
muito tempo,'' pensava Andrei e, tirando a frigideira do fogo, foi até o
telefone comunal que, em plena hora de trabalho, não estava, felizmente,
sendo usado pelos vizinhos. Primeiro, ligou para Saviéli. Respondeu-lhe
uma voz feminina, sua mãe ou a vizinha. Saviéli ainda estava dormindo, e
Andrei ficou ouvindo pelo menos cinco minutos os estalos e os ruídos do
telefone. Finalmente, soou uma batida, ouviram-se vozes afastadas, de
homem e de mulher, e Saviéli, limpando a garganta, tossindo, disse:

--- Desculpe, meu velho, fui me deitar tarde... Bom dia...

Andrei falou da galeria Tretiakóv e da exposição.

--- Claro! --- disse Saviéli, entusiasmado. --- Irei sem falta, espere
por mim perto daquela imundície... ``Transformaremos espadas em
arados...'' \footnote{Nome da escultura (retirado de Isaías 2:4)
  projetada por Evguéni Vutchétitch (1908\emph{--}1974) para o prédio da
  ONU, em Nova Iorque (1957). Umа cópia se acha em frente à Nova Galeria
  Tretiakóv (\emph{Krýmski val}).} Perto da estátua de Vutchétitch... Ou
melhor, perto dos caixas... Só que não irei sozinho... Estarei
acompanhado por uma dama... --- e Saviéli deu uma risadinha pudica.

Sómov também estava em casa e concordou em ir.

--- Precisamos nos ver, conterrâneo --- disse. --- Tenho um assunto a
tratar com você.

Depois disso, Andrei hesitou em ligar para Vássia, de quem não gostava e
tinha medo.

Vássia Korobkóv era de fato uma figura perigosa e estranha, mas nada
excepcional. Ela era pobre, desajustado, não se sabia do que vivia e
bebia, como somente na Rússia um homem pode viver, e beber, de
honorários literários. Esses honorários eram bastante consistentes no
país e alimentavam uma classe bastante heterogênea. Alguns de excessos
desmedidos e luxuosos, outros até se fartarem, alguns com parcimônia, de
restos, e outros ainda apenas ocasionalmente. No entanto, todos que se
serviam desses rendimentos sobreviviam: os altos dignitários e os
velhacos, que, se, perto dos saciados, não tinham o que comer de dia,
sempre tinham o que petiscar de noite. Assim, de petiscos livres, vivia
também Vássia, que escrevia versos estranhos em russo e em ucraniano. Em
russo, sua lírica era dirigida às massas:

\emph{Eu} \emph{tomo nas mãos um lápis de bétula, }

\emph{E escorre dele um verso terno e rosado}

\emph{Sobre a folha branca do campo nevado...}

Em ucraniano ele escrevia versos mais individuais e religiosos:

\emph{Em Kíev deu o Senhor}

\emph{O ar de sua graça}

\emph{E foi grande a sua dor...}

--- Pois eu sou da região de Khárkov --- dizia ele ---, da vila de
Chagaro-Petróvskoie, do sítio Lugovoi. Ou melhor, eu nasci em Kertch,
onde minha falecida mãe, Maria, trabalhava com minha avó, também chamada
Maria, como recrutada. Mas todos os meus parentes são de Khárkov. Na
realidade, meu verdadeiro sobrenome é ucraniano --- Korobko... O ``v''
foi acrescentado depois, no orfanato... Fui criado no orfanato até os
dez anos, depois minha tia se incumbiu de minha criação, logo após a
guerra, quando ela me achou. Tia Ksiénia era de Vorónej. Eu não conheço
meu pai, mas Ksiénia dizia que ele era um marinheiro, um ucraniano da
Crimeia. E, na Crimeia, cada ucraniano carrega algo de turco, tártaro e
grego... Eis que ele me premiou com este focinho de \emph{jid}... Mas
meus parentes são diferentes, tipicamente ucranianos. Na vila de
Chagaro-Petróvskoie vivem minha tia Chura e os filhos dela e vivia
também meu tio Kólia, que morreu na guerra, e ainda o tio Vássia, que
desapareceu ainda pequeno, durante a coletivização, e meu nome me foi
dado em sua homenagem. E se vocês vissem minha tia Ksénia, de Vorónej,
ela não tem nada de judia, uma típica ucraniana. Só eu tenho o nariz
torto e os olhos e os cabelos pretos. Certa vez, um \emph{jid} se
aproximou de mim na rua e puxou conversa na língua dos \emph{jides}. Eu
estava bêbado, claro, mas não muito, e em resposta recitei uns versos:

\emph{Não há nada mais bonito que a nossa Ucrânia, }

\emph{Lá não há} jides \emph{nem nobres,}

\emph{E uniatas}\footnote{Cristãos ortodoxos que reconhecem a autoridade
  do Papa.} \emph{jamais... }

Então ele reclamou: ``Ai, \emph{vei}'', mas eu lhe respondi:
``Desculpe-me, mas é permitido pela censura, Tarás Grigórievitch
Chevtchenko,\footnote{Tarás Chevtchenko (1814\emph{--}1861), poeta
  ucraniano. Os dois últimos poemas foram citados no orginal em
  ucraniano.} volume tal, página tal, evidentemente numa edição de antes
da revolução''. Ainda por cima, meus velhos, eu tinha recebido meus
honorários bem naquele dia e tinha tomadо no restaurante ``Ucrânia'' um
bom prato de \emph{borsch} ucraniano, com pãezinhos de alho, para
acompanhar a vodca. Eu me virei para o \emph{jid}, que teve o
descaramento de tomar a mim, um ucraniano, por um deles, talvez por
causa do cheiro de alho. ``Mas,'' digo eu, ``um ucraniano não fede a
alho como um \emph{jid}.'' Virei-me para ele, ameacei-o com o pé, e eu
mesmo fiquei surpreso com o que havia feito. Apavorado, o \emph{jid}
fugiu de mim como se fugisse do temível espírito pagão de um cossaco.

Vássia sempre ria borbulhando e com modulação de tom, e sua capacidade
de soltar gases era conhecida por vários círculos, assim como seu
antissemitismo veemente e contínuo. O gás saía de seu intestino de
diversas maneiras, refletindo seu estado interior. Às vezes era como uma
palavra breve e clara, às vezes um lamento calmo e prolongado, às vezes
ainda como um grito de horror selvagem...

Andrei Kopóssov temia Vássia de corpo e alma: sua alma sentia repulsa,
seu corpo se resguardava da fúria de uma personalidade infeliz que, por
não ter nada a perder, era mais perigosa para os outros. Quando, durante
uma discussão religiosa, Vássia gritou ``fascista!'' para Andrei, que já
havia expressado sua opinião, este imediatamente foi embora. Ele sabia
que, numa recente discussão religiosa sobre Cristo, Vássia dera um soco
no olho do velho Ilováiski, conhecedor da Antiguidade. Mas também havia
outra razão.

Um dia, havia muito tempo, ainda antes das rodas de discussão sobre o
Cristo, nos primeiros dias de contato, Vássia convidara Andrei para ir a
sua casa, na periferia industrial de Moscou, onde tinha um quarto,
resultado de uma troca de moradia com a ex-esposa. Andrei, na época, não
tinha um Evangelho, e Vássia havia prometido lhe emprestar um. Ele
encontrou Vássia vestido numa camisa jogada por cima da calça, manchada
de tinta, com um pincel na mão. Ele retocava um ícone, de aspecto
antigo, postado em sua frente. Vássia o convidou para sentar,
ofereceu-lhe um chá ruim e \emph{priánikes}\footnote{Bolo de mel,
  tradicionalmente da cidade de Tula.} secos. No início, serviu-lhe
modestamente. Mas depois lhe trouxe pão e uma vasilha com banha de
porco, muito cheirosa.

--- Minha tia de Vorónej me mandou --- disse ele. --- Ela desperdiça seu
dinheiro comigo, ainda não sabe que vou acabar mal --- e sorriu.

Talvez graças a esse acontecimento Andrei agora tivesse decidido ligar
também para Vássia. Andrei sentiu uma vontade súbita de que, no dia em
que lhe fora revelado o que ele queria preservar, este homem estivesse a
seu lado.

--- Sei, sei --- respondeu Vássia, felizmente com voz sóbria ---, tenho
certeza que tudo isso não passa de um rebuliço criado por nossos
franceses locais, assim como puseram nas alturas Malévitch, Tátlin e
todos esses perseguidores do realismo russo. Mas eu irei por
curiosidade.

Após comer rapidamente os ovos fritos já frios e tomar uma garrafa de
quefir, Andrei saiu para o dia quente moscovita. Ele tinha ouvido dizer
que o público ia em massa à exposição, que havia longas filas de espera,
e por isso saiu muito antes da hora combinada, pensando que a estação de
metrô Novokuzniétskaia estaria lotada. No entanto, a Novokuzniétskaia
estava vazia e fresca, e perto do gradil da galeria Tretiakóv havia, de
fato, uma fila, mas pequena, de não mais de vinte minutos. ``O que
fazer?'' pensou Andrei. ``Vou sozinho e depois volto com os rapazes.''
Após haver assim decidido, ele se dirigiu à fila do caixa, onde não
ficou nem vinte minutos, e de repente alguém o chamou perto do gradil.
Era Sómov, seu conterrâneo, que também chegara antes.

--- É ele --- disse o poeta satírico Sómov, sorrindo e olhando para
Andrei ---, eu o estou reconhecendo, mas não nos pires de meus óculos
salvadores, bom dia, como estou contente por você estar vivo...

--- Os rapazes ainda não chegaram --- disse Andrei, cumprimentando-o e
se alegrando com o fato de que o primeiro a chegar fosse o mais tolo,
não alguém emocionalmente doentio, como Saviéli, nem raivoso, como
Vássia.

--- Vamos sem eles --- disse Sómov ---, eu queria lhe mostrar uma
coisa... Compus um poema, claro que não é para publicação. ``Os
fenômenos colaterais do instinto de reprodução.'' Eis --- ele resfolegou
perto da bochecha de Andrei e sussurrou:

\emph{Eu comi a alface e fui ao} samizdat\emph{,}\footnote{Publicações
  clandestinas que circulavam na URSS.}

\emph{O editor falou: Velhaco, vamos lá .}

\emph{Eu respondi: Nem ``a'' nem ``b'',}

\emph{Nem ``a'' nem ``b'', nem ``KGB''.}

\emph{O editor se zangou: aonde pensa que vai?}

\begin{quote}
\emph{Com esta ocorrência, que vá para a agência...}
\end{quote}

``Eu me enganei,'' pensou Andrei, ``teria sido melhor se Vássia viesse
antes, já que não fui fadado a ver a exposição sozinho. Ele, pelo menos,
guardaria sua raiva para si mesmo... Realmente, foi um erro... O melhor
seria ver a exposição sozinho. Este tolo vai me atrapalhar mais do que
outros.''

O pintor francês nascido na Rússia\footnote{Alusão a Marc Chagall. A
  exposição se deu em junho de 1973, na última visita de Chagall, aos 86
  anos, à URSS, onde, então, ele era praticamente desconhecido.} causou
uma impressão marcante a Andrei, a despeito do desapontamento que
aguardava. A cadência do século XX tirou das pessoas um dos bens
fundamentais da vida --- a paciência. Os homens do século XX são
impacientes tanto em seu comportamento como em seu entendimento. Se não
compreendem algo de imediato, simplesmente seguem adiante.

A exposição do pintor francês, natural da Rússia, ocupava duas salas de
fundo, de modo que, ao se dirigir para lá, era necessário passar por uma
infinidade de quadros e de visitantes. Andrei estava agitado e muito
falante, mas interiormente, e esse estado o agradava.

--- Parece-me --- disse Andrei sobre o pintor francês --- que seus
desenhos, especialmente do período tardio, estão mais próximos da
literatura do que da pintura. Algo entre a literatura e a criação
pictórica. A percepção visual do espectador aqui é algo acessório. Como
se dá na leitura. As cores e as figuras são, em essência, letras de
algum alfabeto. É preciso aprender a lê-las para penetrar no
acontecimento, enquanto o pintor realista é acessível até a um
analfabeto. Não se trata de uma vantagem ou de uma falha, são apenas
coisas diferentes. Um iletrado olha para um quadro de Rembrandt ou de
Riépin\footnote{Iliá Riépin (1844\emph{--}1930), pintor russo de cunho
  realista.} e vê árvores, pessoas, o céu --- tudo o que é possível ser
distinguido numa fotografia ---, e, ao mesmo tempo, sabe que o pintor é
muito conhecido e orgulha-se de compreender todos esses objetos, sendo
grato ao artista. Outra coisa seria se esse sujeito pegasse um livro de
Shakespeare ou mesmo se um homem letrado pegasse um livro de Shakespeare
em inglês. Nem lendo sílaba por sílaba, será possível compreendê-lo.
Vocês já notaram que um livro escrito numa língua incompreensível nos
irrita interiormente? O mesmo se dá com a obra de um pintor realista.
Ele nos irrita, de forma aberta ou secreta...

Diante de desenhos abstratos e surrealistas Sómov ficava entediado, mas
nas salas de obras russas ultrapassadas mostrava um verdadeiro
interesse, e seu rosto adquiria o aspecto sôfrego e estúpido de um homem
intelectualmente limitado que quer compreender o que lhe é inatingível.
Porém, em salas de época ele se sentia mais à vontade. Salas com
retratos do tempo de Catarina... Rostos de perucas, mas, se lhes
tirassem as perucas, seus possuidores se achariam hoje sentados nas
poltronas de diretores, de chefes de construtoras, de vice-ministros,
das libertinas dos altos comitês, das esposas dos membros das instâncias
superiores. Andariam de ``Volga'',\footnote{Carro soviético, cuja
  primeira linha (1956) se chamava \emph{Gaz-21}. Algumas linhas eram
  reservadas para uso do alto ecalão.} e o conde Orlóv poderia usufruir
perfeitamente do bonde ou do metrô. Catarina II\footnote{Conde Orlóv
  (1734\emph{--}1783) era um dos ``favoritos'' de Catarina II, a Grande.}
faria geleias na datcha vestindo um \emph{sarafan}\footnote{Vestido
  típico sem mangas, usado na Rússia e em alguns países do norte.}
russo. Eis quem construiu a Torre de Babel, transferindo-a às mãos
sólidas de seus sucessores. Mais adiante, achava-se um enorme quadro de
Ivánov, \emph{O aparecimento do Cristo ao povo},\footnote{Quadro feito
  entre 1837 e 1857 por Aleksándr Ivánov (1806\emph{--}1858), pintor
  acadêmico.} na frente do qual sempre havia uma multidão, formada
principalmente de pessoas provincianas. Os que se apressavam à exposição
do francês não se detinham ali ou se detinham rapidamente. No entanto,
Andrei ficou um bom tempo examinando o quadro e o público. Sómov bufava
ao seu lado, e seu rosto era dominado pelo esforço criativo que aparece
no rosto de um homem na privada. Aliás, rostos assim encontrados também
nas igrejas. Andrei notou ao lado uma mulher insignificante, de uns
quarenta anos, ou até menos, mas envelhecida por causa dos frequentes
partos e abortos espontâneos. Sua fisionomia não era nem citadina, nem
camponesa, um rosto miúdo e trivial. As bochechas vermelhas, ou melhor,
uma vermelhidão doentia; o nariz pequeno e arrebitado. Nada feminina,
com os seios caídos. Assim são as mulheres devotas --- e ela era uma
devota --- que acreditam em rumores e no governo, caso seja o governo
delas, o russo. Perto dela se postava um garoto de nove ou dez anos, de
rosto redondo e queixo pesado, parecendo um mau aluno de uma escola de
província ou de subúrbio. A julgar por seu comportamento, não era um
menino peralta, obedecia a sua mãe e fazia perguntas. Ele perguntou
sobre o quadro:

--- O que é isso, mamãe?

--- É Cristo --- respondeu a mãe, baixinho ---, ele queria que todos os
homens vivessem bem e por isso os judeus o mataram.

O garoto consentiu com um aceno de cabeça e se dirigiu aos outros
quadros. A mulher estava rodeada por moças russas desengonçadas e
desproporcionais, que podiam tanto ser suas filhas quanto moças vindas
dos ``cafundós''. Tinham vindo para visitar seus parentes ou para
comprar produtos alimentícios. E na sua lista constava: visitar o
Krêmlin, o Mausoléu de Lênin, a galeria Tretiakóv, o GUM, o TSUM, e o
``O Mundo das Crianças''.\footnote{GUM (\emph{Glávnyi universsálnyi
  magazin}), centro comercial ao lado da Praça Vermelha; TSUM
  (\emph{tsentrálnyi universálnyj magazin}), centro comercial perto do
  Teatro \emph{Bolchói}; ``O Mundo das Crianças'' (\emph{Diétskii mir}),
  rede de lojas de produtos infantis fundada em 1957.} As lojas de
alimentos, evidentemente, eram as primeiras da fila, e não contavam. A
mulher olhava para o \emph{Aparecimento do Cristo ao povo}, e Andrei
olhava para ela e pensava: ``Eis o crente russo. Nas rodas de religião,
agora muitos falam que o ateísmo perdeu adeptos e começou uma renascença
religiosa. Muito bem, suponhamos que o ateísmo tenha perdido, mas será
que isso levou a religião a vencer na Rússia? Sem ninguém ter aprendido
nada, a religião renasce com a antiga histeria no lugar do sentimento,
com pessoas teimosas discutindo sobre Cristo e com o povo simples que,
em vez de refletir sobre Cristo, espera dele o mesmo que esperava do
georgiano Stálin, do turco Rázin ou de algum outro atamã\footnote{Stepan
  (Stenka) Rázin (1630\emph{--}1671), líder cossaco (atamã) que
  organizou uma série de insurreições contra o poder tsarista. Foi
  enforcado barbaramente por ordem do tsar Aleksei. Figura de Rázin se
  tornou mítica, gerando obras como о poema \emph{A execução de Stepan
  Rázin,} de Ievtuchenko, que inspirou música homônima (1964) de
  Chostakóvitch.} russo. E se a Rússia, no futuro, tentar se salvar
através da consciência popular nacionalista, ela não será nem
materialista nem ateísta. A consciência nacional religiosa será a
máscara do fascismo salvador russo. Em primeiro lugar, o que era chamado
de ``ateísmo'', na realidade, se comprometeu na Rússia, tornou-se
inoportuno, perdeu a novidade. Em segundo, o ateísmo não mostrou a
devida flexibilidade no âmbito nacional, revelou-se desajeitado, ao
passo que a ortodoxia demonstrou inúmeras vezes, no passado, seu
desprendimento ao engrandecer abertamente o poder nacional, e hoje, para
a juventude, ela é uma novidade atraente''.

Mas eis uma sala totalmente diferente. Os quadros \emph{Púchkin}, de
Kipriénski, \footnote{O retrato de Púchkin, feito por Oriest Kipriénski
  (1782\emph{--}1836), é de 1827.} e \emph{Lêrmontov}, de Peróv,
provocavam a mesma sensação de suas reproduções folheadas na revista
\emph{Luzinha}.\footnote{\emph{Luzinha} (\emph{Ogoniók}), revista
  ilustrada semanal política e literária de grande tiragem. Começou a
  circular em Moscou em 1923.} Na mesma sala se encontravam
\emph{Tolstói} e \emph{Dostoiévski}.\footnote{O retrato de Lêrmontov é
  de 1869 e o de Dostoiévski de 1872, ambos feitos pelo pintor Vassíli
  Peróv (1833/1834\emph{--}1882), um dos Itinerantes
  (\emph{peredvíjniki),} ou membro dа Sociedade de Exposições de
  Pintores Itinerantes, que romperam com a temática clássica da arte
  acadêmica, voltando-se, dentro da estética realista, para motivos
  sociais e paisagens russas.} \emph{Tolstói} tinha um olhar vazio, mas
nele isso parecia natural, algo budista, pois, entre os humanistas do
século XIX, sobressaía a paixão por atingir a perfeição pelo caminho
mais curto, o que inevitavelmente levava ao esquematismo espiritual e
poético tão característico do budismo. Na parede oposta, pendurava-se o
quadro de Peróv \emph{O peregrino}. Peróv pintou o retrato de
Dostoiévski em 1872 e \emph{O peregrino} em 1870. São surpreendentemente
parecidos, em particular no olhar. Tanto Dostoiévski como \emph{O
peregrino} têm uma tensão penetrante e um aprofundamento no olhar e no
porte. Como se esses olhos se fixassem nas mais profundas criações
divinas, mas, na realidade, caso se observe com atenção, eles estão
concentrados nas velhas alpargatas de fibra trançada e nas dívidas não
saldadas. Mas isso foi ecleticamente reunido a pensamentos elevados e
mais gerais. Não foi à toa que Dostoiévski elevou o tipo ``peregrino'' a
santo. O peregrino, especialmente o russo, é eclético até o último fio
de cabelo, ele combina, de forma mecânica, suas necessidades essenciais
com as necessidades do mundo. Ele sonha que tudo o que ele elaborou se
realize. \emph{O peregrino} de Peróv tem um guarda-chuva nas costas e
uma caneca pendurada no cinto. Já Dostoiévski segura o joelho dobrado
com as mãos. Ambos estão compenetrados e meditam sobre a mesma coisa.

Mas eis que surgiu o francês, um emigrado russo. Andrei teve a impressão
de que foi um erro, um erro imposto, o francês ser visto em tamanho
natural, na parede do museu. Ele deveria ser folheado num álbum, num
livro. A reprodução não perderia nada do original, assim como nada perde
a imagem Tolstói quando é impressa em tipografia ao pé de um manuscrito.
Em compensação, haveria a possibilidade de se concentrar, já ali era
impossível. Havia poucas pessoas vindas da província; raramente eram
atraídas para lá. Mas muitos judeus, basicamente o público que forma o
prosélito moderno, religioso ou civil.

O prosélito de antes da revolução era, essencialmente, o mercador, o
comerciante ou o engenheiro, o médico --- um homem calculista que não
tinha nada contra Moisés se este lhe garantisse algum lucro. O prosélito
atual é o intelectual, o filósofo, o místico, e ele está conscientemente
descontente com Moisés. ``Tudo é proibido: não se pode, não se deve, não
é permitido. Mas com Cristo tudo é possível, tudo é permitido.'' De
Moisés ele sabe basicamente: ``Olho por olho...''. De Cristo: ``Ame seu
inimigo...''. Os judeus da exposição eram claramente moscovitas, já
tinham estado nas outras salas inúmeras vezes e não se detiveram ali,
assim como o restante do público que viera ver o francês. O público da
exposição temporária era homogêneo, enquanto das outras salas variado.
Era tedioso. A animação vinha dos provincianos.

--- O que é isso? --- perguntou alguém da província. --- Por que tem um
homenzinho na bochecha?

--- Porque o pintor quis assim --- respondia uma mulher de nariz grande,
com os olhos brilhando e sorrindo de forma enigmática.

``É pouco provável,'' pensava Andrei, ``uma pintura realista é muito
mais difícil de explicar, há mais segredos nela. Já aqui tudo foi
disposto como as frases de uma obra apurada. Não há nada de supérfluo.''
Um velho extremista da província, magro e de cabelos castanho-claros,
disse ao seu filho em voz deliberadamente alta:

--- Vamos embora, depois de Riépin e de outros bons quadros, não se deve
ver isso.

Ninguém reagiu. Ele não suscitou polêmica e foi embora, mas queria muito
ter discutido na fila, ter defendido a mãe \emph{Rus}...

Mais adiante ficava a sala de Vrúbel. O conhecido
\emph{Demônio},\footnote{\emph{O demônio sentado} (1890), um dos quadros
  mais famosos de Mikhail Vrúbel (1856\emph{--}1910), expoente da
  pintura simbolista.} de 1890, parece mais frágil que o \emph{Demônio}
corpóreo, estendido numa pose violentamente passional, porém sozinho,
sem mulher... Preto, azul, lilás... Adiante o mártir Falk\footnote{Robert
  Falk (1886\emph{--}1958), pintor e professor do VKhUTEMAS
  (\emph{Víschie khudójestvenno-tekhnítcheskie másterskie,} Oficinas
  Superiores de Arte e Técnica). Um dos fundadores do grupo de vanguarda
  ``Valete de ouros'', do qual fizeram parte pintores como
  Kontchalóvski, Machkóv e Lariónov.}... Kontchalóvski em \emph{O
retrato de Iakúlov}:\footnote{\emph{O retrato do pintor Iakúlov} foi
  feito por Piótr Kontchalóvski (1876\emph{--}1956) em 1910.} o
homenzinho alegre de bigodinhos de bufão, sentado em pose oriental e
vestindo uma gravata, parece parte do ornamento, assim como os iatagãs
pendurados na parede... Tudo é como num tapete; o homem e o iatagã têm o
mesmo direito de estar ali... A obra de Falk traz uma sensação de
fragilidade. As cores são acanhadas, enquanto o talento de Kontchalóvski
se impõe de forma senhorial. Não se trata simplesmente da distribuição e
da organização do espaço. É um sentimento interior --- fragilidade e
pudor em Falk, força e tenacidade viva em Kontchalóvski. A fragilidade e
o pudor são necessários à noite, a portas fechadas; a força e a
tenacidade, necessárias ao dia, em meio a uma multidão de iguais... A
fragilidade transforma-se numa leveza etérea, não do corpo, mas da
essência, e conduz ao céu; a força e a tenacidade lançam suas raízes a
terra. A força e a tenacidade não se acomodam no céu; a fragilidade e o
pudor não se encaixam na terra... Em seguida, as naturezas-mortas... O
pão russo, a carne... Aqui também, tirado do acervo, o francês em sua
juventude, quando era um judeu russo... Eis a \emph{Lua de mel}. Ele e
ela, dois corpos compridos e nebulosos, em forma de arco-íris, elevam-se
por detrás do horizonte... O céu está coberto de flores e a terra da
lama bielorrussa. E as faces judias de bode dos amantes... Era a sala
mais triste. Tudo era colorido, tudo era jovial, e volta e meia surgiam
lágrimas nos olhos, mas não em todos. Para Sómov, o conterrâneo de
Andrei, era simplesmente agradável. Ele não estava entediado, como
diante dos desenhos abstratos e surrealistas, nem tinha aquele ar
estúpido e compenetrado, como diante das telas realistas. Tudo era de
seu interesse, como numa festa de rua... Abstracionismo e realismo são
artes de autoafirmação, mas o impressionismo é a arte do sacrifício... O
pintor aqui é como um gladiador, que morre para exaltar a multidão. O
impressionismo, e não o abstracionismo ou o realismo, seria o estilo
mais apto a iniciar as almas imaturas e rudes à arte se um dia ele
prevalecesse como arte oficial... Mas, para um homem de sentimento, tudo
ali parecia pesado, como em um cemitério que lhe era querido. Muito
longe dali, estava o realismo socialista, que acalmava a alma com
ninharias sólidas, na linha da clareza cotidiana, engessada pela
eternidade. Se Sómov se entediava diante da abstração, tinha um ar
estúpido e compenetrado em meio ao velho realismo e se animava com o
impressionismo, nas salas do realismo socialista se sentia como em um
trólebus. Aqui tudo era conhecido, tudo era habitual, aqui ele era o
guia, andava na frente e se perdia nas salas dos pintores acadêmicos,
artistas do povo.\footnote{Na União Soviética, os artistas de destaque
  recebiam o título honorário de ``Artista do Povo da URSS'' (hoje seria
  o ``Artista do Povo da Federação Russa'').} E Andrei foi para o pátio,
perto da escultura ``Transformaremos as espadas em arados'', de
Vutchétitch.

No banco perto do café, do qual emanavam, sem nenhuma consideração pelo
lugar sagrado, os cheiros costumeiros do sistema de alimentação popular,
Saviéli estava sentado ao lado de uma jovem mulher, com a qual, como
logo entendeu Andrei, seu amigo passava as noites sonhando, e sob
diversos aspectos. Sim, Saviéli ficava sempre nesse estado; mesmo um
frango assado e colocado inteiro numa travessa, com as coxas separadas,
provocava-lhe, em vez de apetite, desejo sexual... A mulher tinha um
rosto despretensioso, não o redondo de traços tártaros, tão comum na
Rússia, mas o rosto russo do Norte, desprovido de marcas asiáticas...
Seus olhos eram particularmente incomuns. Os olhos claros russos são
geralmente aguados, mas os dela eram de um azul encorpado, em tom
escuro.

Assim que Andrei, um homem reservado, olhou para ela, veio-lhe algo de
sua irmã, Tássia --- que se apaixonara pelo Anticristo com o terceiro
tipo de amor, nem carnal nem platônico --- e de sua mãe, Vera --- a
amante abnegada do Anticristo. E Andrei alegrou-se, pois, sabendo que
suas ideias bíblicas foram conduzidas intocadas pelas salas da galeria
Tretiakóv, ele sentiu, através de uma sensação que repentinamente lhe
inflamou, sua alma se fortalecer ainda mais.

--- O que aconteceu? --- perguntou Andrei a Saviéli.

--- Atrasamos --- disse Saviéli ---, foi minha culpa.

Pelo visto, eles chegaram muito depois da hora combinada, sem saber que
Andrei, tendo chegado bem mais cedo, não os havia esperado.

--- Ilováiski apareceu em casa --- disse Saviéli ---, e ficamos
discutindo sobre Cristo... Sou culpado...

--- O culpado tem meu agrado --- gritou Sómov, que havia aparecido ---,
o culpado tem meu agrado, e o inocente minhas saudações!

Sómov passou pelas salas do realismo socialista como se tomasse uma
ducha, lavando-se do tédio do abstracionismo, da tolice compenetrada do
realismo clássico e da festividade do impressionismo, e saiu como tinha
entrado: em nada havia mudado e estava pronto para seguir a vida em sua
realidade atual. As salas do realismo socialista eram como uma sala de
banhos, na qual o homem era limpo de todas as camadas desnecessárias,
tanto da arte passada como da realidade que se encerrava entre os muros
da galeria.

--- Esta é Ruthina, minha vizinha --- disse Saviéli ---, e este é Andrei
Kopóssov, meu colega de curso.

Assim eles foram unidos pelo acaso, que, na realidade, era o desígnio
divino. Logo no início da conversa, reconheceram um ao outro como
conterrâneos. Descobriram que Ruthina, na infância, era amiga de Ústia,
a irmã de Andrei, e que havia conhecido Tássia, sua outra irmã, e Vera,
sua mãe. Sómov contou que também era da cidade de Bor, filho de um
operário da casa das caldeiras a gás do hospital central de Bor e de uma
contadora aposentada. E eles decidiram fazer um brinde a esse acaso.

Assim, o que deveria acontecer aconteceu. Mas ainda faltava algo. Vássia
Korobkóv ainda não havia aparecido, estava muito atrasado. No entanto,
assim que ele apareceu, o quadro se completou. A profetisa Pelágia o
notou de longe e entendeu: eis a semente ruim do Anticristo que deveria
ser destruída, assim como Tamar havia destruído as sementes ruins de
Judá, seus filhos Her e Onã...

Vássia, bêbado, aproximou-se e disse:

--- Atrasei-me, sou culpado!

E Sómov repetiu:

--- O culpado tem meu agrado e o inocente minhas saudações!

Mas Vássia não gostou dos versos de Sómov, assim como Pávlov, o ferido
de guerra da cidade de Bor, não havia gostado. Naquela época, Pávlov
batera em Sómov no parque, perto da pista de dança. Agora, em Moscou,
Korobkóv deu-lhe um golpe no pátio da galeria Tretiakóv... A Tretiakóv é
um lugar bem vigiado, repleto de policiais. Por isso todos se afastaram
correndo da exposição do célebre pintor francês e, quando se reuniram
novamente, num jardim público nos arredores, Sómov não estava entre
eles, havia se ofendido... A profetisa Pelágia disse a Vássia:

--- Para que você bate nas pessoas?

Mas Vássia, que sempre ficava alegre ao bater em alguém impunemente, não
respondeu nada, só olhou para a profetisa Pelágia e notou, por sua vez,
que seus olhos se fixavam nele.

--- Por que você olha para mim desse jeito? --- perguntou Vássia. ---
Será que me conhece de algum lugar?

--- Sim --- disse a profetisa Pelágia, conhecida por Rute ---, você é
muito parecido com meu pai... Surpreendentemente parecido.

--- E seu pai por acaso é judeu? --- perguntou Vássia com sarcasmo. ---
Srul\footnote{Diminutivo de Israel.} Samuílovitch?

--- Ele é judeu --- respondeu a profetisa Pelágia ---, mas se chama Dã
Iákovlevitch... Você se enganou...

--- Desculpe --- disse Vássia sarcasticamente e continuou a falar em
ucraniano ---, \emph{desculpe, perdão, como dizem na Ucrânia... ``Em
Kíev deu o Senhor o ar de sua graça e foi grande a sua dor''...}
\emph{Você é assim sensível? }

--- Apareça em casa --- disse a profetisa Pelágia --- e verá como é
parecido com meu pai... Tomaremos chá.

E ela o fitou de novo. Seu segundo olhar já era mortal, revelando muito
de Tamar, que havia matado a semente ruim de Judá, seus filhos, o
primogênito Her e Onã...

O rosto de Vássia, da tribo de Dã, se desfigurou e ele disse, repetindo
o destino do filho de Sulamita, da tribo de Dã:

--- Eu cuspo em sua vendinha \emph{jid} e em seu Deus \emph{jid}...

Então a profetisa Pelágia pronunciou mentalmente: ``Que isso se realize.
Aquele que blasfemar sobre o nome do Senhor deverá morrer. Seja
estrangeiro, seja nativo, quem blasfemar sobre o nome do Senhor será
entregue à morte''.\footnote{Levítico 24:16.}

Assim ela disse a si mesma, olhando para Vássia, que se afastava. Andrei
e Saviéli, que temiam Vássia por sua disposição para o mal, disseram:

--- Ainda bem que ele foi embora --- este foi Andrei.

E Saviéli acrescentou:

--- Somente agora eu percebi que Vássia se parece com o pai de Rute.

Andrei disse:

--- A culpa é minha, foi tolice convidá-lo.

--- O dia começou mal --- disse Saviéli ---, mas pode terminar bem...
Ilováiski está na minha casa e convidou para irmos à datcha de seus
amigos. A datcha pertence a um cirurgião que estudou com Ilováiski no
seminário. O sobrenome do cirurgião é Vsesviátski.\footnote{``Vsesviátski''
  vem de \emph{sviátki,} que se refere ao Natal.}

--- Isso é perigoso --- disse Andrei ---, eles vão falar de Cristo e,
para mim, será difícil ouvir sobre isso hoje.

--- Não tem importância --- sorriu Saviéli ---, esses velhos falam de
Cristo de outro modo... Falam de maneira cômica e alegre... Ilováiski
fala hilariamente de Cristo com eles... Vamos... Você, eu e Ruthina,
além de minha mãe e Ilováiski.

--- Vamos --- concordou Ruthina-Pelágia.

Andrei logo consentiu, pois começou a valorizar cada minuto que passava
ao lado dessa mulher de olhos azuis. Enquanto Saviéli foi para casa,
atrás de sua mãe, Andrei passou mais de uma hora a sós com Ruthina,
cercado, evidentemente, de um público casual: primeiro, os transeuntes;
depois, os passageiros do trólebus; em seguida, os viajantes indo à
estação Saviólovski. Conversavam sobre a cidade de Bor e a região de
Górki, da qual a profetisa Pelágia tinha muitas lembranças, embora
tivesse saído de lá ainda menina.

--- Como vai Ústia? --- perguntou Pelágia.

--- Ústia tem dois filhos pequenos --- disse Andrei ---, e Tássia,
também: meu xará, Andrei, e seu irmão, Varfolomei. Andrei está no
exército e Varfolomei trabalha como motorista.

--- E como vai sua mãe, Vera? --- perguntou a profetisa Pelágia.

--- Minha mãe é uma pessoa boa, mas não tem firmeza. Todo mundo grita
com ela, e ela obedece a todos, às filhas e aos netos; e até a velha
Vesselova, a mãe do marido de Tássia, a maltrata. Mamãe tem medo de tudo
e, mesmo quando está rezando, seu rosto se mostra assustado, como se
Deus ralhasse com ela...

Assim eles conversaram e, aparentemente, não tinham mais sobre o que
falar, mas, felizmente, tiveram tempo para se aproximarem, e sentaram-se
com prazer, lado a lado, em silêncio, como às vezes a profetisa Pelágia
ficava com seu pai, o Anticristo. Pelágia se surpreendeu com isso, pois
ainda não sabia que Andrei Kopóssov era também uma semente do Anticristo
(assim como Vássia Korobkóv), entretanto era uma semente saudável,
apesar de não ser a principal.

O olhar fixo é fecundo quando o objeto não influi sobre a personalidade
de quem observa, à diferença do que ocorre no budismo... O olhar budista
contém o frio épico da união com a natureza, o mesmo que se apodera,
cada vez mais, do cristianismo em decadência, porém o olhar fixo bíblico
é lírico. A sabedoria da lei são os lábios de Deus, mas o corpo de Deus
é uma lírica elevada. A profetisa Pelágia fitou Andrei Kopóssov em meio
à agitação da estação e o reconheceu. Ela compreendeu que a vida dele se
comporia liricamente. Pois, quando uma vida se compõe liricamente, não
importa de que matéria, muitas vezes da mais baixa e ignóbil, Deus
sempre permanece ao lado desse destino. Esse homem viverá uma vida
longa, e será uma vida tensa e perigosa, mas a de um trabalhador
espiritual, e ela não conhecerá o castigo divino, somente o castigo
humano, que a alma não precisa temer...

Quando compreendeu tudo sobre Andrei Kopóssov, a profetisa Pelágia não
tinha mais por que ficar ao lado dele, em silêncio, e logo apareceu
Saviéli, sua mãe, Klávdia, agora uma velha jovial com os lábios
pintados, e o velho Ilováiski, conhecedor da Antiguidade. O velho
Ilováiski se tornava desagradável quando, ao encontrar alguém, se
esforçava, com os lábios sujos senis e o rosto descuidado de um
solitário desleixado, por beijar sua boca; o desafio consistia em
desviar o beijo, da boca para a bochecha, virando desajeitamente a
cabeça, como que sem querer, obrigando Ilováiski a beijar o ar, mas sem,
com isso, ofender o velho. A profetisa Pelágia realizou isso com
delicadeza e sabedoria, no entanto Andrei foi apanhado e sentiu em seus
lábios a carne morta do velho. Além disso, a mãe de Saviéli, que agora
imitava Ilováiski em tudo, cravou seus lábios pintados nele. Saviéli se
agitava:

--- O trem para o subúrbio chegará logo --- e correu para cuidar das
passagens.

--- Meu filhinho é um verdadeiro Ívolguin --- disse Klávdia. --- Quando
o vejo agitado, me lembro de seu falecido pai, sempre alarmado --- e,
como era seu costume, ela deixou cair umas lágrimas.

O tempo subitamente piorou. Em Moscou isso acontecia com mais frequência
no verão do que no inverno. De repente, no meio de um céu quase sem
nuvens, trovejou uma vez, depois outra, e, quando eles embarcaram, já
havia vento e um ar fresco e, após uns dez minutos de viagem, as janelas
se cobriram de chuva. As conversas no trem eram conduzidas
principalmente pelos moradores dos subúrbios, enquanto as pessoas da
cidade, cansadas de Moscou, que se torna muito entediante quando surge
continuamente diante de nossos olhos, se esforçavam por ver as datchas
locais pela janelinha. A exceção era Ilováiski, que falava sem parar e
não deixava ninguém em paz.

--- Vocês, jovens --- dizia Ilováiski ---, certamente não ouviram, muito
menos leram, os escritos do sacerdote Petróv... Um filósofo do
cristianismo --- Ilováiski deu uma risadinha ---, o amor como base da
vida em sociedade. Ele rejeitava a propriedade privada e a desigualdade
econômica e provou que a propriedade privada é uma criação judia, não
cristã... Sob sua influência, os seminaristas decidiram ir ao povo com
um novo Evangelho... O populismo religioso foi omitido da história da
revolução... Mas Petróv foi excomungado... Sim, sua tolice foi recebida
com repressão, como é costume na Rússia.

--- Acalme-se, Gavriil --- disse Klávdia a Ilováiski.

--- Mas o que foi que eu disse? --- Ilováiski respondeu com um tom de
supresa e provocação. --- Eu, ao contrário, estou zombando das tolices
antigovernamentais.

--- Não pronuncie a palavra ``antigovernamental'' --- sussurrou Klávdia.

--- Ah, sua alma se tornou judia após seu casamento com Katz... ---
disse Ilováiski.

E entre Ilováiski e Klávdia começou uma disputa inesperada, o que
revelava a intimidade da relação dos dois.

--- Vou voltar agora mesmo --- sussurrou Klávdia na primeira parada. ---
Isso é uma indelicadeza com Saviéli... E com Ruthina...

--- Qual é o problema? --- disse Ilováiski. --- Ruthina sabe que eu não
sou um antissemita e respeito o pai dela, não é verdade?

--- É verdade --- concordou a profetisa Pelágia.

Mas Saviéli realmente empalideceu, e sabe-se lá o que aconteceria se
eles não tivessem chegado nesse momento ao seu destino. A chegada e a
mudança de ambiente agradaram a todos, mesmo ao impulsivo Ilováiski, que
entendeu que passara dos limites. Ele sabia que tinha essa fraqueza, mas
não podia se privar do prazer da maledicência quando estava seguro de
que, por isso, seria apenas xingado e ninguém bateria nele, como
costumava bater Vássia Korovkóv.

O subúrbio moscovita, úmido e coberto de datchas, acolheu os citadinos
com ameaças vindas de cercas alheias, latidos de cães, ausência de
policiamento nos cruzamentos, e vultos ameaçadores perto do quiosque de
cerveja. No entanto, quando eles encontraram a datcha do cirurgião
Vsesviátski, o amigo de Ilováiski, e entraram no quintal, defendendo-se
das patas sujas de um cachorro grande e amistoso, ficaram mais alegres.
E, ao verem na mesa da varanda um prato com maçãs, colhidas no jardim da
datcha, umas com cabinhos e outras com folhas, e um prato de framboesas
frescas, vindas do mesmo jardim, o encanto do subúrbio moscovita
dissipou de vez a primeira impressão desagradável que tiveram.

À mesa, além do dono da casa, o cirurgião Vsesviátski, um velho bem
cuidado de bochechas rosadas, sentavam-se sua esposa, Varvara Davýdovna,
e um velho da idade deles, também conhecido de Ilováiski, e que, ao se
apresentar, disse:

--- Belogrúdov...\footnote{``Belogrudov'', de \emph{biélyi} (``branco'')
  e \emph{grud} (``peito'', ``seio'').} Um sobrenome épico, porém mais
apropriado para o gênero feminino --- o que logo revelou seu espírito
brincalhão.

E ele indicou também sua profissão: professor de literatura.

Ilováiski imediatamente se pôs a beijar os três, primeiro o cirurgião,
depois a esposa, em seguida o professor de literatura, depois de novo o
cirurgião. A empregada trouxe o samovar e Varvara Davýdovna uma garrafa
empoeirada de aguardente de ginja. ``Agora vão começar a falar de
Cristo,'' pensou Andrei, alarmado. Mas, enquanto não tomaram a
aguardente, não o fizeram, e, depois de beberem, começaram a falar sobre
isso docemente, como ocorre geralmente quando os velhos se recordam da
remota juventude, sonhando com o passado como se ele não tivesse
existido.

--- Lembram? --- diziam. --- Lembram? --- e seus olhos semicerravam-se,
como se sonhassem com algo agradável, e, ao despertarem, sentiam um peso
no coração.

--- Lembram-se da homilética? --- disse com doçura o professor de
literatura Belogrudov, semicerrando os olhos. --- A homilética, a teoria
da arte oratória religiosa...

--- A liturgia, o estatuto da igreja --- continuou amavelmente
Ilováiski.

--- A igreja possui um estatuto? --- Klávdia olhava, surpresa como uma
gansa. --- Gavriil, será que há mesmo um estatuto? --- ela também havia
tomado a aguardente de ginja e estava coquete.

Da esposa maldosa do crítico de arte Ívolguin que fazia se passar por
inteligente, comedida, cerimoniosa, uma mulher assegurada que expulsara
com mão de ferro os filhos de sua irmã, vítima da repressão, não sobrara
vestígio. Klávdia agora se zangava e se irritava como as mulheres tolas
e levianas, perdoava rapidamente e se contentava com pouco. Para
Saviéli, seu filho, fazia tempo que ela deixara de ser uma ameaça, que
não era mais aquela mãe severa que constrangia seu pecado juvenil, e ele
passou a ser exigente com ela, como um educador, rivalizando com
Ilováiski no domínio de sua frágil alma, e não com a intenção de
preservá-la, mas para marcar sua presença, por meio dela, diante de seu
rival masculino.

--- O estatuto da igreja --- disse Ilováiski, em tom de sermão --- é o
estudo da ordem da execução de todos os seus ofícios.

--- E os textos evangélicos, sobre os quais se escreviam sermões em casa
--- continuou Belogrúdov ---, o estudo de João Crisóstomo, lembra-se,
Gavriil? Lembra-se, Semion? --- voltou-se para o cirurgião.

--- Como não?! --- respondeu o cirurgião Vsesviátski. --- Nós
praticávamos nas igrejas paroquiais. Mas eu preferia a teologia e a
medicina... Era o que estudávamos nas classes superiores...

--- E como os católicos demonstram... --- disse Ilováiski, já bastante
embriagado. --- Sim... O pensamento católico é a Europa, com todas as
suas fraquezas... Mas os irmãos e as irmãs, na concepção da Trindade...
--- ele tentou se levantar, mas Klávdia, abraçando-o pelo ombro, o
conteve. --- Na concepção da Trindade... Entre nós o Espírito Santo
provém somente do Pai, na Europa descende também do Filho... O
pensamento católico é livre... Quanto a nós, somos escravizados pelo
pensamento judaico, nascido de Moisés. Chega a ser hilário, nós, russos,
e as ideias de Moisés...

``Agora vai começar,'' pensava Andrei, preocupado. Se não fosse por
Ruthina, que estava sentada ao seu lado, ele seria invadido pela
tristeza, mas seu amor por ela tinha amadurecido rapidamente, e um rapaz
de vinte e poucos anos gosta de obedecer com submissão a uma mulher
bonita de trinta, sem exibir sua masculinidade e esforçando-se por
imitar suas maneiras. Já Ruthina permanecia tranquila e observava os
velhos seminaristas bêbados.

--- Kant equiparava a religião à moral --- dizia Belogrudov, solene,
como se estivesse em cima de uma tribuna ou de um púlpito. --- Para
Hegel, a religião era o estágio inicial da filosofia, a qual surgiu dos
homens selvagens como uma necessidade de pensamento e de conhecimento; a
religião é uma ilusão do homem que adora a si mesmo... A divinização do
espírito humano... --- então ele mudou repentinamente de assunto e
declarou: --- No seminário eram proibidos Turguêniev, Gontcharóv,
Tolstói, Belínski, Dobroliúbov, Píssarev, Tchernychévski, Gontcharóv...
Pensando bem, falei Gontcharóv duas vezes...

--- Eis um cálice --- disse Ilováiski, pegando com os dedos reumáticos
uma bonita xícara com borda dourada do serviço de chá ---, ela é
simples...

--- Mãe --- disse Saviéli ---, tire a xícara de Ilováiski, senão ele
quebrará o que não é dele...

--- Você, meu jovem, tem complexo de Édipo --- disse Ilováiski,
virando-lhe a cabeça desgrenhada de intelectual-farrista russo.

--- Se não fosse tão fraco, eu lhe daria um soco --- disse Saviéli, com
lágrimas de indignação juvenil brilhando nos olhos, porém, ao ver o
rosto assustado e sofrido de sua mãe, deu-se por satisfeito e se
acalmou.

--- Já chega --- desconcertados, disseram ao mesmo tempo os anfitriões,
os Vsesviátski ---, beberam além da conta e agem como crianças.

--- Não foi nada, eu já estou calmo --- disse Saviéli ---, vou dar uma
volta no jardim.

--- Temos um belo jardim, deixe-me acompanhá-lo --- disse Varvara
Davýdovna, e eles saíram.

--- Eis o legado insolente de Moisés --- disse Ilováiski quando Saviéli
saiu.

--- Exatamente --- acrescentou Belogrudov ---, lembrem-se, a
revolução... Houve um comício no seminário... Um pregador do Velho
Testamento entrou na sala, e nós lhe dissemos: a Bíblia é um dogma...
Por que, foi a nossa pergunta, nós, russos, devemos estudar a história
do povo judeu, por algum motivo eleito por Deus, estudar todos os seus
pormenores? Será que devemos estudar a história dos judeus mais a fundo
do que a história da nossa própria pátria? Em 1952, no seminário, eu
mandei um artigo sobre esse exemplo de patriotismo russo para uma
revista antirreligiosa, mas não deixaram passar...

--- Em 1952 --- disse Vsesviátski --- aconteceu uma história da qual
sempre me recordo... No hospital do nosso campo, estavam fazendo a
autópsia de um prisioneiro... A autópsia era feita pelo médico-chefe na
presença de todos os médicos prisioneiros do campo. O cadáver era de um
homem de idade, e no seu peito havia uma grande cruz de cobre. A cruz e
o cordão foram entregues à despensa do campo, e o médico-chefe, o major
Baránov, aproveitando a ocasião, perguntou aos médicos prisioneiros se
eles acreditavam em Deus. Todos responderam que sim. Somente um
respondeu: ``Sim, mas no sentido filosófico''. ``Dá no mesmo!'' disse
Baránov... Eu penso --- acrescentou Vsesviátski --- que, se eles
estivessem em liberdade, não falariam tão corajosamente que sim... Mas
lá, com uma condenação de dez ou quinze anos, não tinham nada a perder.

--- Eis o cálice --- de novo Ilováiski agarrou a xícara ---, ele é
simples, mas, ao jogá-lo no chão, ele se quebrará e se tornará um cálice
complexo... Vocês se lembram do cálice de Moisés... Moisés é uma
personagem cuja importância foi claramente exagerada --- ele continuou
---, sou especialista em Antiguidade. Com o perdão da palavra, mas a mim
ninguém enganará. Foi o escriba Esdras quem atribuiu essa grandeza a
Moisés, numa época posterior... Isso foi provado... Os profetas da época
dos Juízes ou dos Reinos não mencionam Moisés, e os grandes profetas
também não, com exceção de Jeremias... Mesmo assim, ele só o faz de
passagem. O culto de Moisés surgiu num período posterior, na época dos
profetas Neemias e Esdras... Foi Esdras quem escreveu o
Pentateuco\footnote{Primeiros cinco livros do Antigo Testamento:
  Gênesis, Êxodo, Levítico, Números e Deuteronômio.} de Moisés,
conferindo-lhe artificialmente um caráter antigo.

--- Mas o que isso importa? --- não se conteve Andrei Kopóssov, pálido e
agitado. --- O quê?... Desculpe-me, mas o senhor usou o termo de forma
imprecisa. Ele não ``escreveu'', mas ``anotou''. Eu li um tratado
filosófico que tentava rebaixar o Pentateuco afirmando que ele tinha
sido criado posteriormente... Mas para que reforçar o que já é de
conhecimento geral? E a vida dos patriarcas não é uma crônica. Lá, por
exemplo, menciona-se que Abraão chegou à região de Dã, enquanto Dã veio
ao mundo quatro gerações após Abraão, e a região de Dã só surgiu após o
êxodo do Egito, ou seja, muitos séculos depois dos patriarcas. Esdras
reforçou a figura de Moisés em momento histórico análogo ao da fuga do
jugo babilônico, que, por sua vez, repetiu a fuga do jugo egípcio. Esse
é um exemplo de imitação genial que Púchkin colocava no ponto mais alto
de sua criação... A imitação de grandes paradigmas exige mais talento
que inovação... O estágio inferior da criação é o epigonismo, depois vem
a inovação e, em seguida, a imitação dos grandes paradigmas... Isso é o
classicismo... A grandeza da Bíblia está na imitação, na imitação de
Deus... Pode ser que o genial imitador Esdras tenha anotado o poema no
Pentateuco de Moisés através de antigas lendas orais e depois tenha
colocado Moisés como figura principal, papel que lhe convém, pois a
verdade da poesia está acima da verdade histórica... Mas isso eu não
li... Essa ideia eu compreendi sozinho, só depois a li em Aristóteles e
fiquei contente ao achar uma confirmação. Aristóteles afirmava que,
mesmo se a obra do historiador Heródoto tivesse sido escrita em versos,
continuaria sendo história, e não poesia. A diferença é que o
historiador fala sobre o que de fato aconteceu, enquanto o poeta fala
sobre o que poderia ter acontecido. Por isso a poesia é mais filosófica
e mais séria do que a história. A poesia fala sobre o geral e a história
sobre o particular. O geral abrange o que deve ser feito e o que se
almeja, enquanto o particular histórico fala do que aconteceu e de seu
resultado... E a criação do mundo do ponto de vista bíblico, sobre a
qual popes atoleimados discutem num palavreado científico-filosófico, é
um poema que não permite uma análise histórico-científica...

Assim se expressando, de maneira prolixa e até lhe arder a garganta,
Andrei entendeu que, se ele tentou dizer algo em que tinha convicção e
acreditava sinceramente, Ilováiski iria replicar de maneira mais
inteligente e irrefutável, conforme a capacidade do polemista russo,
cuja fala é mais inteligente do que a ideia. Nesse momento, a empregada
taciturna trouxe o samovar aquecido e o brincalhão Belogrúdov, vermelho
da aguardente caseira, intrometeu-se:

--- Juventude, crianças... Lembram-se... --- disse, alegre e rindo. ---
Diga-me, filho, será que você não se deflorou na infância pela
excitação, não se estimulou sozinho?...

--- Diga-me, filho --- tomou a palavra o intelectual-farrista Ilováiski,
todo desgrenhado ---, será que você não cometeu atos impuros com outro
homem ou deixou que ele cometesse em você, não pecou com uma
mulher?...\footnote{Os dois diálogos usam fórmulas de confissão com
  termos do eslavo eclesiástico.}

--- Gavriil, pare de dizer essas coisas na frente dos jovens... ---
disse Klávdia, vermelha, piscando tolamente os olhos.

--- Não foi flagrado com um animal no pasto ou com uma ave? ---
Ilováiski estava descontrolado.

--- A circuncisão do Cristo se deu no oitavo dia de sua existência
carnal --- dizia também o velho Belogrudov, atrevido ---, no oitavo dia
dignaram-se a circuncidá-lo para a salvação da nossa tribo.

--- Vocês se lembram do incêndio na igreja? --- disse Vsesviátski. ---
Uma caixa com tocos de velas pegou fogo no coro... O sacerdote corria
segurando uma cruz e gritava: ``Apaguem, apaguem!...''. Depois o piso se
incendiou...

--- E quando nós estávamos escondidos nuns arbustos? --- ria Belogrudov.
--- Uns gritavam: ``Bosta!'', outros: ``Pegue!'', ``Idiota!''...

--- E a prece para a fundação da casa --- ria Ilováiski ---, para a
escavação do poço... Abençoai os ovos e o queijo... A prece para os que
traziam espigas de grãos.

E o ambiente em volta da mesa se tornou nocivo, com uma alegria típica
de monastério, mas a profetisa Pelágia mantinha-se em silêncio, pois ela
sabia como era difícil para um russo acreditar em Deus... Se lhe
oferecessem algo de útil na descrença, no ateísmo, ele ficaria
contente... No início, parecia que ele havia achado um substituto para
sua fé, e ele alegrou-se, mas não por muito tempo, pois isso passou
ainda mais rápido... E ele regressou, mas para onde? Um russo é capaz de
acreditar diante da vastidão de sua terra e de sua história? Se não em
Deus, ao menos ``naquele que foi crucificado para nos salvar na época de
Pôncio Pilatos''... O profeta Isaías dizia que não se deve procurar
sempre por Deus, mas apenas quando Ele está próximo. Mas Ele só se
aproximará de uma jovem nação não religiosa quando ela se cansar de sua
frivolidade alegre, livre e ruidosa. Ele se aproximará de uma jovem
nação na desgraça, mas se afastará dela na alegria. Uma nação adulta
cederá à tentação na opressão --- como, privados do Pai, foram tentados
os judeus na opressão egípcia ---, mas na alegria ela conhecerá o
florescimento divino... É grande o pranto bíblico, o lamento dos
profetas, o choro de Jeremias, mas o homem está mais próximo de Deus ao
louvá-lo. Não por acaso o Livro dos Salmos é chamado, na versão original
hebraica, de Livro dos Louvores... Conseguirá o povo russo sentir Deus
plenamente na alegria, e não na desgraça, a fé russa irá se tornar
adulta? Ou ele voltará ao mesmo círculo sem nada aprender? O ateísmo
russo perdeu, mas terá a fé russa ganhado algo por isso?...

Agora os três velhos amantes do riso, ex-seminaristas, davam sinais de
cansaço, e nos rostos exauridos transparecia devoção. Eles passaram a
falar sobre a prece de outra maneira.

--- E da oração com três profundas reverências, lembram? --- disse
Ilováiski. --- ``Senhor, soberano do meu ventre, livra-me do espírito da
ociosidade, da ambição e da vaniloquência, mas concede a mim, teu
escravo, o espírito da castidade, da humildade, da paciência e do
amor... Senhor, Rei, faça com que eu veja minhas faltas e não condene
meu irmão, pois tu és bendito pelos séculos dos séculos,
amém.''\footnote{Trata-se da Oração de Santo Efrém, o Sírio (traduzida
  do texto russo).}

--- E que cantos havia no seminário --- disse Belogrudov, num tom calmo
e sonhador ---, o coro do prelado... O chantre... --- e se pôs a cantar
com uma voz surpreendentemente jovial: --- Creio em Ti, Pai Nosso...

Os outros dois velhos o acompanharam, e o canto soou de forma
harmoniosa. Varvara Davýdovna, voltando do jardim com um prato de maçãs
molhadas, disse:

--- Vocês, beatos, cantem mais baixo, basta de louvores --- e ela se
sentou com um sorriso doce e tolo, o mesmo que adornava o rosto de
Klávdia, entre lágrimas.

E os velhos, que até então só haviam blasfemado, cantavam com
sentimento, mesmo o filosófo conhecedor da Antiguidade, Ilováiski, que,
após assoar o nariz com modos de bêbado, disse:

--- Na igreja do seminário havia dois coros, cada um com seu chantre...
Vocês se lembram do regente Kolka, que se casou com a mulher de um pope
rico?... Ela tocava piano de cauda e Kolka violino...

Para não perturbar a boa disposição conquistada com dificuldade pelos
velhos, a profetisa Pelágia levantou-se cuidadosamente da mesa e
dirigiu-se ao quintal, de onde foi ao jardim por um caminho revestido de
tijolos. Andrei, atraído pelas orações e pelos salmos cantados pelos
velhos, não notou a saída de Ruthina, mas, quando voltou a si e não a
encontrou por perto, sentiu de repente a dor de uma perda irreparável,
ficando atordoado, pois, pela primeira vez em três horas, ela não estava
a seu lado. Levantando-se de um salto e chamando a atenção de todos, de
modo que até os velhos pararam de cantar, Andrei desceu correndo os
degraus da varanda e olhou ao redor, sem saber para onde ir. De repente
alguém se atirou às suas costas e o empurrou, e ele, de susto, deu um
grito estranho.

--- O que você tem? --- assustada, Varvara Davýdovna apareceu nos
degraus com uma lanterna na mão, pois já estava escuro.

Surgiu também o desgrenhado Ilováiski, novamente com uma expressão
maliciosa, maldosa e descarada no rosto.

--- A juventude tem suas questões... O ciúme... Ele tem ciúme de
Saviéli...

--- Foi o cachorro que o assustou --- disse Varvara Davýdovna ---, ele
não morde, meu jovem.

--- Qual é o caminho da estação? --- disse Andrei, sofrendo pela súbita
mudança que lhe acontecera, pois, pouco antes, ele se mostrava seguro de
si diante daquelas pessoas, defendendo o que lhe era caro com palavras
firmes, sentindo-se maduro, mas agora, ao se assustar como um tolo, ele
revelara seus sofrimentos mais íntimos, que, aos olhos desses velhos,
pareciam infantis, assim como agora se tornaram as palavras tão apuradas
que ele dissera durante a discussão...

--- Mas espere --- apareceu também Klávdia ---, talvez possamos pegar o
trem juntos... Ou pode ir com Saviéli... Saviéli! --- chamou. --- Mas
onde está ele? Provavelmente passeando com Ruthina.

--- Não, eu vou --- disse Andrei, apressado, sentindo sobre si o olhar
irônico e descarado de Ilováiski ---, está na hora de ir...

Ele atravessou o portão e começou a andar a esmo sobre a grama molhada
e, quando se virou, percebeu que, mesmo se quisesse voltar, não saberia
por onde ir. Todas as datchas surgiam na escuridão e pareciam iguais.
Afastando-se o quanto pôde, Andrei sentou-se numa grande pedra, das que
emergem da terra ou jazem sem motivo à beira das estradas dos subúrbios,
e começou a refletir, mas, por qualquer razão, não sobre seu amor por
Ruthina, que era intenso, embora tivesse nascido três horas antes, e que
já havia lhe causado tanto sofrimento e tanta vergonha diante de todos.
Ele começou a pensar sobre o começo da sua vida em Moscou, quando tudo o
que o afligia nesse instante ainda lhe era festivo e agradável.

Ao chegar à capital, Andrei descobrira entre muitas das pessoas
respeitadas por ele naquele tempo um sentimento russo
nacional-religioso, e este sentimento fora justamente o primeiro degrau
de sua iniciação ao mundo espiritual. É possível refletir de várias
maneiras sobre os acontecimentos atuais; no entanto, deve-se reconhecer
que a renovação da juventude começou com um sem-número de crucifixos de
grande consumo, feitos do mesmo material dos porquinhos com uma abertura
em cima para pôr moedas. Ele também sonhava em conseguir um crucifixo
desses, assim como antes sonhara com uma faca finlandesa, que ele via
nas mãos dos poderosos. Uma vez que, mesmo antigamente, tudo o que era
digno de imitação era russo, da mesma forma que era russo tudo o que era
coroado e recompensado, esses crucifixos russos ajudaram a renunciar ao
passado e a mudar muitas coisas, mas não mudaram nada em essência.
Andrei começou a ler o Evangelho que Vássia Korobkóv lhe emprestara, e
no Evangelho tudo também era russo e negava o que não era, mas o que
havia de mais não russo, naturalmente, eram o judaísmo e Moisés... Tudo
o que vinha de Moisés era maldoso, enquanto de Cristo bondoso... Muitas
mulheres da \emph{intelligentsia}, algumas de origem judia, que haviam
se iniciado na renovação russa, reforçaram ainda mais a afeição dele
pelo Cristo russo... A alegre lua de mel entre Andrei e o cristianismo
russo não foi destruída por dúvidas espirituais, para as quais, naquela
época, ele não estava preparado, mas por acontecimentos à primeira vista
insignificantes e cotidianos, e pelo caráter desagradável dos cristãos
da capital. Um caráter definido pela obviedade e pelo consumismo,
correspondendo mais às emoções nacionais do que ao desejo de penetrar
verdadeiramente nos preceitos evangélicos. Quando os jovens começaram a
fazer cópias à mão dos textos evangélicos e a repassá-los um para outro,
como se fossem panfletos, ele entendeu definitivamente que a religião
não salvaria a Rússia no futuro, assim como o ateísmo não a salvara no
passado. Não há como se salvar sozinho: o homem é indefeso diante de si
próprio. O caráter nacional --- eis seu verdadeiro opressor. Não foi
concedido ao homem modificar a si mesmo, mas compreender-se e advertir
os outros por meio de suas palavras. O que será, só Deus o sabe, mas o
que não deve ser, o homem também pode sabê-lo. Não deve haver esperança
demasiada na religião, como houvera no ateísmo, pois, agora, a religião
cristã não pode confiar em si mesma. O cristianismo, que começou seu
caminho histórico com a conspiração dos apóstolos contra Cristo,
compreende, certamente, que o que o homem mais espera receber da
religião é paz de espírito, em troca da qual ele está disposto a dar sua
submissão. Ele espera o mesmo que um filho espera de sua mãe --- apenas
se eu me tranquilizar, serei obediente... E ela o tranquilizará com o
amor pelos sofrimentos e a recompensa será a vida após a morte. No
entanto, se o amor ao sofrimento for substituído pelo amor à proeza ---
o que, em princípio, é a mesma coisa --- e a recompensa, em vez da vida
após a morte, tornar-se a glória da nação, isso estará plenamente de
acordo com o desafio terreno contra Deus --- a construção das torres
nacionais de Babel. O cristianismo apostólico se orgulha de seu amor
pelo homem, mas, na realidade, em sua base moral dá uma importância e um
sentido exagerados ao homem no mundo divino, aproximando-se dos ateus.
Não, não é isso que ensinam os profetas bíblicos, não é assim que eles
trazem tranquilidade, mas com a verdade bíblica, com a verdade de Deus.
A verdade consiste no fato de o homem ser uma criatura amaldiçoada desde
sua expulsão do paraíso, do Éden. Compreender a verdade sobre si está ao
alcance de qualquer um, no entanto nem todos estão dispostos a
compreendê-la. Poucos concordam em fazê-lo. Mas essa verdade tanto
aliviará sua vida como a reforçará. Assim, cada minuto bem vivido,
qualquer felicidade, toda boa ação será recebida como algo não merecido,
por isso a recompensa será duas vezes mais valiosa; toda desgraça, todo
fracasso, ao contrário, será recebido como algo merecido, por isso o
castigo será menos ofensivo. Não esperar por recompensas, que devem ser
sempre inesperadas e recebidas como algo que não lhe pertence, e não
temer castigos, que devem ser sempre aceitos como algo natural --- eis o
verdadeiro destino de uma personalidade religiosa.

Há uma célebre passagem no Segundo Livro de Moisés, ``Êxodo'', em que os
filhos de Israel, amedrontados diante do faraó que os perseguia, em
lugar de lutas e feitos, dirigiram-se a Deus com uma prece, e a Moisés
com maldições, pois ele os forçava a lutar e os afastava das preces. E o
grande profeta, com o coração aflito, dirigiu-se ao povo que rezava e
prometeu-lhe, por meio de uma prece, a clemência de Deus: ``Não temais,
ficai firme, e vós vereis a salvação que o Senhor hoje realizará.
{[}...{]} O Senhor lutará por vós e vós vos tranquilizareis''.
\footnote{Êxodo 14:13, 14.} Então o Senhor deu uma lição a Moisés. ``E
disse o Senhor a Moisés: Por que clamas a mim? Dize aos filhos de Israel
que partam.''\footnote{Êxodo 14: 15.} O desígnio de Deus não é
suficiente, sem o homem nada acontecerá e nada se realizará.

Andrei Kopóssov se lembrou de quando, algum tempo antes, visitara o
mosteiro de Zagórski, nos arredores de Moscou, conhecido como Mosteiro
da Trindade-São Sérgio,\footnote{Mosteiro (ou Lavra) da Trindade-São
  Sérgio (\emph{Tróitse-Sérguieva lavra}), mosteiro mais importante da
  Rússia situado em Sérguiev Possád (região de Moscou), cidade parte da
  rota do Anel de Ouro.} e voltara de lá com um peso no coração. Ele
sempre teve medo de cemitérios, e o mosteiro parecia exatamente um
cemitério, com túmulos escavados para apreciação geral. Tudo ali tinha a
aparência desses túmulos velhos, que, atraindo turistas, eram lucrativos
--- os muros do mosteiro, os campanários, o refeitório, que, construído
no estilo rústico do ``barroco russo'', lembrava um
\emph{kalátch}\footnote{Pão branco redondo trançado, em forma de argola.}
corado e petrificado, extraído de uma sepultura, um alimento ideal para
os mortos e assustador para as bocas vivas. E, como nos túmulos, tudo
estava coberto por inscrições: as religiosas com uma caligrafia
ornamentada, as públicas com uma escrita severa. Multidões de velhas
quase da mesma idade --- perto dos sessenta anos ---, quase da mesma
altura e vestidas da mesma cor --- preto ou cinza --- harmonizavam com o
cenário. Às vezes entre elas surgia um rosto de homem ou um rosto jovem,
que com frequência era de menina --- os meninos ali eram raros. Só que
esses rostos também se tornavam parecidos com o das velhas, e se podia
pensar que todos os defuntos tinham rostos assim, independentemente do
sexo e da idade. Os visitantes de fora olhavam para elas com curiosidade
e cautela, como um vivo olha para um defunto. Somente os monges --- uns
usando vestimentas pretas com insígnias de distinção, correntes e
cruzes, outros vestes cinza e cinturadas sem insígnias --- zanzavam pelo
pátio com os rostos saudáveis, vivos e brilhantes e se dirigiam aos
devotos com tranquilidade. Havia um quê de coveiro nesses monges, que
estavam habituados a lidar com corpos mortos como objetos de trabalho
cotidiano. Apesar de ser verão, o dia estava frio e ventoso, e os
devotos, como numa estação a céu aberto, se acomodavam sobre uma longa
fileira de bancos de jardim. Uns dormiam, deitados nos bancos, outros
petiscavam modestamente (pães e salsichões cozidos) e bebiam água de
garrafas de meio litro. No mosteiro também havia gatos, que
aparentemente se alimentavam das esmolas dos peregrinos, e nuvens de
pombos, que pousavam sobre quem dormia e flanavam.

Numa das antigas igrejas, diante do célebre iconóstase, considerado um
patrimônio do país, acontecia uma missa. O sacerdote, de óculos, com uma
vasta cabeleira branca, sentava-se à cabeceira daquilo que representava
o Caixão do Senhor --- o Leito de Deus ---, de um prata fúnebre e
reluzente, e o recitativo dos homens era acompanhado pela ``Aleluia!''
das mulheres. Os devotos andavam em fila e seus lábios tocavam o Leito
de Deus. Tudo isso acontecia a meia-luz e no aperto. Na outra metade da
igreja, estreita como uma sala de espera, havia bancos em que os
peregrinos se sentavam, espremidos, com suas trouxas e cestas, como numa
estação de trem. Apesar da ``Aleluia'', o local lembrava uma repartição
pública russa, com a morosidade das instituições russas, a igualdade
russa, a monotonia e o coletivismo. O caráter russo, das florestas e das
estepes, formou-se, desde tempos remotos, no espírito da coletividade,
consolidando-se assim até os dias de hoje. Por isso o individualismo é
tão frágil nele, por isso esse caráter é ateísta e comunitário, e a
igreja russa o confirma até com seu aspecto exterior. Quando o homem
russo tenta violar a si mesmo, isolando-se em mosteiros, vivendo a vida
de eremitas, tentações lhe aparecem com força singular, tentações das
quais só é possível se refugiar no coletivo. Lev Tolstói, que tinha uma
percepção de mundo bastante sensata, representou essa ideia de modo
pleno em \emph{Padre Sérgio}. Uma menina de oito anos que sua mãe, uma
turista, havia levado para ver uma missa, disse de modo tipicamente
tolstoiano:

--- Vamos embora, isso me dá medo --- sussurrou a menina, farta de
escutar a ``Aleluia'' e de ver o caixão de prata coberto de beijos.

Não, a religião não renovará o caráter russo, pois ela mesma é fruto
desse caráter e precisa de renovação. Porém, justiça seja feita, a
religião russa, por força de sua morosidade, apenas expressa de modo
evidente o que é característico na situação atual da religião como um
todo. Hoje é especialmente compreensível o temor de Tolstói da igreja,
que tornou a fé pública e coletiva. Todas as religiões se formaram
quando massas de pessoas eram ignorantes e precisavam, tal como as
ovelhas, de um pastor. Entretanto, a intimidade não é menos necessária à
religião, e talvez seja até mais, do que o amor. Nenhum outro homem, por
melhor que seja e por mais importante que seja o título que o paramenta,
deve ou pode violar a intimidade da fé, pois a exposição pública da fé
pela igreja, num grau ainda maior que a do amor, é o caminho da
desilusão e da ruína espiritual. E será que um homem que declara sua fé
abertamente está longe daquele que comete adultério abertamente? Se no
passado a expressão pública da fé era uma triste necessidade, no futuro
a intimidade da fé se tornará uma necessidade inevitável. A intimidade é
o único caminho para a renovação religiosa. As pessoas podem saber que
um sujeito está apaixonado, mas não devem saber como ele ama, podem
apenas imaginá-lo. O mesmo vale para a religião. A importância do rito,
que priva a religião de intimidade, deve enfraquecer, ao passo que o
significado da fé íntima deve aumentar...

Assim se tranquilizou Andrei Kopóssov, o filho ilegítimo do Anticristo,
da tribo de Dã, e de uma mulher russa, Vera Kopóssova, da cidade de Bor,
da região de Górki. Ele compreendeu o que o atormentava e determinou com
clareza seu caminho. Ele sabia que acreditava em Deus e por isso se
sentia no direito de prevenir os outros da tentação religiosa que se
aproximava da Rússia em meio ao tédio de ateístas oficiais e desprovidos
de talento, advertindo de que no futuro a religião seria o grande perigo
na Rússia. Ele seria odiado por sua posição antirreligiosa --- em
círculos antigovernamentais não oficiais despertaria risos, e em
círculos oficiais tentariam se aproveitar dele, como Pilatos tentara se
aproveitar da advertência de Cristo contra a envelhecida Lei de Moisés,
Lei em que Jesus Cristo, um judeu, acreditava com todas as forças de sua
grande alma.

Quando Andrei Kopóssov, filho do Anticristo, entendeu a orientação
antirreligiosa a que fora predestinado, a profetisa Pelágia, filha
adotiva do Anticristo, que estava com Saviéli no jardim escuro da
datcha, sentiu isso por meio de um pequeno baque no coração e disse,
sorrindo:

--- Saviéli, eu gostei do seu amigo Andrei, mas ele é muito jovem, ele
não serve para mim.

Fazia tempo que entre ela e Saviéli havia se estabelecido uma relação
sincera e amistosa, como entre duas amigas, com ajuda da qual uma mulher
inteligente impede o homem que não ama de dar passos arriscados. Se isso
dura muito tempo, o homem realmente começa a entender cada sensação
dessa mulher, e sua vida torna-se praticamente um prosseguimento da
dela.

--- Ele caiu de amores por você --- disse Saviéli ---, apaixonou-se num
instante e, mesmo se encontrasse outra mulher, não seria feliz.

--- Ele não vive para ser feliz ao lado de uma mulher --- disse a
profetisa Pelágia.

--- Vássia também se apaixonou por você --- disse Saviéli ---, apesar de
ser um desbocado... É um homem infeliz.

--- Eu sei --- disse a profetisa Pelágia ---, mas ele não sofrerá ou se
atormentará por muito tempo.

De repente nos olhos dela surgiu um reflexo impuro, no tom rubro-escuro
da ginja que matiza o ferro incandescente e as brasas de uma fogueira se
extinguindo. Era a cor cruel do castigo celestial que a profetisa
recebera como legado de seu pai adotivo, o Anticristo. Toda vida,
bondosa ou perversa, termina sob domínio dessa cor...

Saviéli nunca havia visto nada parecido, pois o jardim se iluminou e as
macieiras bem cuidadas, de troncos caiados, ficaram visíveis. Um homem
são, vendo isso surgir da mulher amada, teria perdido o juízo, mas
Saviéli já havia passado por um tratamento na clínica psiquiátrica e
agora se sentia atraído pela alquimia com a mesma paixão que o atraía,
na tenra juventude, para o doce pecado dos solitários. Ele também tinha
seus acessos, mas não se afligia com eles: simplesmente, em tais
momentos, os pensamentos contínuos que o atormentavam tornavam-se mais
insistentes. Assim, ele começou a pedir a Ruthina-Pelágia que lhe desse
um pouco de seu sangue.

--- Qualquer pessoa pode fazer uma análise de sangue na policlínica ---
dizia Saviéli ---, eu já combinei tudo, eu pagarei e me darão uma
proveta com seu sangue... Não oficialmente, claro... Eu tenho uma prima,
Nínotchka, até pensei em pegar o sangue dela quando ela veio nos
visitar, mas depois eu soube que ela havia se casado. Preciso do sangue
de uma virgem.

No jardim da datcha estava fresco e úmido depois da chuva intempestiva
típica do subúrbio, e no ar pairava um odor rico que parecia o perfume
da própria vida. ``O perfume da vida terrena, em sua raiz, deveria ser
exatamente assim,'' pensava a profetisa Pelágia, ``depois da chuva da
noite, num jardim de macieiras de subúrbio.''

Esse perfume inspirou também Saviéli. Ele passou a falar do motivo de
suas noites insones, da ideia que tivera nos últimos meses, de seu sonho
de uma alquimia moderna e renovada, a única capaz de resolver o mistério
de tudo, o mistério dos mistérios, o mistério da vida.

A profetisa entendeu que nem isso o salvaria. Os livros o arruinaram, o
que acontece com frequência a naturezas impressionáveis, sensíveis, cuja
vida emocional serviria a um gênio, enquanto a intelectual se limita a
de um adolescente. São capazes de abraçar uma imagem artística com força
e profundidade, embora de forma limitada, mas um livro, que exige uma
generalização madura, lhes é nocivo. Sem dúvida, Saviéli era um homem de
extremos perigosos e fruto de uma má mistura de sangue, no entanto,
muitos traços que lhe eram peculiares vinham, em geral, de uma leitura
religiosa imatura. É mais provável que um jovem ardente descubra Deus em
Púchkin do que no Evangelho... Como surgira, no início do século, um
entusiamo generalizado pelos livros inteligentes do materialismo
econômico que dera início ao declínio de muitas almas talentosas, hoje
surge o perigo do entusiasmo pelos livros sagrados, graças aos quais
algumas almas, ou até de muitas, já começaram a decair. O início da
queda de Saviéli se dera com o Santo Evangelho, um fragmento da Bíblia.
A Bíblia para pessoas dessa natureza é menos nociva ou menos sedutora.
Tudo é muito claro --- ``olho por olho'' ---, o que é difícil de ser
compreendido aqui? Porém, o Evangelho --- ``ame seu inimigo...'' ---
arrebatará corações, fará promessas, cativará e levará não para a
clareza de Púchkin, mas para os livros místicos. Assim, para o jovem
crente, o ascetismo cristão inevitavelmente se transforma em uma mística
erótica. Isso é particularmente perigoso numa época de aridez
espiritual, provocada pelo predomínio de um ateísmo medíocre e
inconsequente, o ateísmo idealista.

Fazia algum tempo que Saviéli passara a frequentar um círculo de jovens
que se reuniam secretamente na residência de um deles e se entregavam
com entusiasmo à Idade Média, graças à moda atual de cantar louvores a
ela por qualquer pretexto. Gógol tinha o direito, num século
envelhecido, de admirar a Idade Média, de alegrar-se com o jogo livre e
juvenil da alma e do pensamento; no entanto, os homens de nosso tempo,
que viram e sentiram o fim, inevitavelmente medíocre, do talentoso jogo
medieval do homem-deus, não devem, em seu fim sem talento, tentar imitar
seu começo fecundo e notável... Nenhum jogo infantil deve ser levado até
seu fim tedioso, pois qualquer criança sabe que a parte mais
desinteressante de um jogo é o fim. A Idade Média foi uma infância
renovada, alegre e temporária, que surgiu depois da velhice bíblica,
sábia e eterna. Foi precisamente na Idade Média que o cristão tornou-se
em definitivo um pagão alegre. O fascismo, como todo movimento popular,
é um alegre jogo infantil que começou com os gênios medievais. Mas um
gênio é dotado da grande capacidade salvadora de realizar o horror
somente em sua alma e pensamento, poupando o mundo de sua mais terrível
ameaça: a materialização de fantasias e extravagâncias humanas
inconcebíveis. Quando desse jogo participam crianças de sangue impuro,
fica claro que as fantasias de Shakespeare e de Dante têm seus atores e
particantes. Um simples enigma atormentava a \emph{intelligentsia}
liberal: de onde surgiu o fascismo na culta Europa? O fascismo surge
quando uma infinidade de pobres crianças de sangue impuro inicia-se nos
jogos notáveis da Idade Média. E um homem adulto, aparentemente um
representante de uma nação adulta, liga-se facilmente a esses alegres
jogos, afasta as amarras da razão que o privaram de tantos prazeres
selvagens, amaldiçoa o ``é proibido'' de Moisés, atribui um sentido
pagão ao ``é permitido'' de Cristo, alegra-se, transforma-se numa
criança medieval do século XX; no entanto, alegra-se já com beliscões e
mordidas, como folgam e brincam os abortos espinhentos cheirando a
urina. Mas o que confere pompa medieval a esse jogo são as
quinquilharias místicas.

Em outros tempos, na Rússia, decadentes que tinham desaprendido a crer
em Deus eram atraídos pelo mistiscismo. E o que ocorrerá se hoje por ele
são atraídos os filhos do tedioso ateísmo dos anos anteriores que ainda
não aprenderam a crer em Deus? Em que se transformará o misticismo russo
popular de massa? Que jogo será jogado pelos russos para sua própria
perdição e a dos outros? Há muitos pecados na alma russa, pois esse é
seu destino; uma nação que dominou tantos espaços não pode viver sem
causar tormentos a si e aos outros. No entanto, será que não se prepara
para o futuro um terrível pecado, que já não será perdoado por Deus?
Quando o Santo Evangelho começar a ensinar a maldade às almas imaturas e
cansadas do ateísmo...

Eis os livros que eram lidos no círculo frequentado por Saviéli:
\emph{Da condição do homem após a morte e a transformação do corpo
perecível em imperecível, assim como foi criado no Éden, e da condição
dos corpos imperecíveis condenados pelos princípios das trevas};
\emph{As portas da natureza oculta e de suas qualidades que agem sobre o
bem e o mal. O que é a essência das coisas e a matéria-prima que todos
os químicos desejam há tempos conhecer: a matéria-prima do remédio
filosófico universal, em prol dos que buscam verdadeiros conhecimentos
espagíricos e médicos}.

No entanto, fazia algum tempo que Saviéli quase não aparecia nesse
círculo, ficando mais tempo em casa em meio aos matrazes e as retortas
que havia arranjado no depósito de uma farmácia. Pensava também em
abandonar o Instituto de Literatura para entrar no curso de bioquímica
da universidade. Por ora, abasbacava-se com o livro \emph{Sobre os
homúnculos filosóficos, o que eles são na realidade e como
criá-los}.\footnote{A pessonagem faz clara alusão ao \emph{Fausto,} de
  Goethe (segunda parte).} Na página de rosto havia um adendo que lhe
agradava especialmente: ``Edição impressа, ilustrada e comunicada ao
mundo''. A origem dos homúnculos filosóficos foi divulgada ao mundo com
simplicidade e segurança, sem excesso de lirismo e com convicção
científica.

``É engendrado da seguinte maneira. Pegue um matraz do melhor vidro de
cristal, coloque uma parte do melhor orvalho de maio, colhido na lua
cheia, duas partes de sangue de homem e três partes de sangue de mulher.
É necessário notar que os participantes, se possível, devem ser puros e
castos. Em seguida, feche o matraz contendo esse material com uma rolha
opaca e conserve-o num lugar quente para putrificar; então, no fundo, se
precipitará terra vermelha. Depois filtre o mênstruo,\footnote{No caso,
  termo da alquimia para solvente.} acumulado na parte de cima, em um
vidro limpo e preserve-o cuidadosamente.''

Assim começava a descrição do processo de criação dos homúnculos
filosóficos, homens e mulheres...

A profetisa Pelágia sabia que o que Saviéli queria realizar era pecado
e, tendo ouvido mais de uma vez seus relatos entusiasmados e seus
pedidos insistentes de que ela doasse um pouco de seu sangue para o
experimento, pensava em como poderia advertir o rapaz atormentado que
havia se apaixonado por ela. Ela sabia que palavras não adiantariam
nesse caso, mas não podia achar um jeito de impedi-lo. Ela poderia
recusar seu sangue, o que já havia feito, mas isso só reforçaria o
desejo de Saviéli de executar o planejado: ele procuraria sangue em
outro lugar, viveria por isso e tornaria seu pecado ainda maior. Ou ela
poderia lhe dar seu sangue e ele realizaria sua experiência, que,
certamente, não daria em nada ou, pelo menos, não no que ele esperava,
como acontece em todo experimento alquímico. Então ele daria mostras de
sua verdadeira obstinação mística, se lançaria a novаs experiências, a
novos fracassos e, se estivesse fadado a viver uma vida longa,
envelheceria no pecado. E agora, parada no jardim escuro da datcha,
entre as macieiras, respirando profundamente o perfume rico, excitante e
úmido da vida, vendo perto de si o rosto pálido de Saviéli ---
apaixonado, de feições eslavas, com o nariz curto de Klávdia e os olhos
assustados e noturnos de Aleksei Ióssifovitch ou mesmo de seu avô,
Ióssif Cháimovitch ---, vendo e sentindo tudo isso, a profetisa Pelágia
de repente decidiu lutar contra o pecado ajudando-o a se realizar e a se
revelar, decidiu lutar contra Satanás indo ao seu encontro...

É preciso notar, a propósito, que a profetisa Pelágia, havia tempo, era
atormentada por sua feminilidade e sentia plenamente em seu corpo o
terceiro flagelo do Senhor --- o animal selvagem. O sinal de seu dom
profético lhe fora dado através da tentativa de estupro que sofrera
quando era menina, perto da cidade de Bor, e ela se lembrava disso. Ela
também sabia que a façanha da castidade, que agora guardava por amor ao
Senhor, era fortalecida por Satanás, participante inevitável de qualquer
drama arriscado do Senhor... No início, quando Pelágia ainda era uma
adolescente e, depois, uma jovem, o pudor e o amor filial a ajudaram, e
fora o período menos complicado de sua luta. Porém, quando ela começou a
ler a Bíblia e o Evangelho e a rezar com frequência, por algum motivo se
tornou particularmente difícil manter seu voto de castidade. Ela
rejeitara, sem esforço e sem luta, os que lhe fizeram propostas, na
maior parte rapazes de seu círculo, pois as relações de Pelágia e de seu
pai, zelador do JEK,\footnote{JЕК, acrônimo de
  \emph{Jilischno-ekspluatatsiónnaia kontora} (Escritório de Gestão
  Habitacional).} eram restritas... Mas na fase mais difícil, entre 25 e
30 anos, ela se viu muitas vezes diante de homens perigosos...

Uma vez, o JEK a enviou para ajudar na colheita de batatas, fora da
cidade. No caminho, o motorista que conduzia a profetisa Pelágia ao
entreposto tentou violentá-la na cabine. Nela transparecia algo de muito
feminino que incitava uma natureza indômita à violência... Eles lutaram
numa pequena floresta, aonde tinham ido tomar um pouco de ar fresco, e
subitamente a profetisa Pelágia teve vontade de se entregar. Mas
Satanás, que estava ao lado e tinha seus próprios desígnios, viu e
compreendeu tudo. Esse motorista era um conhecido arruaceiro da aldeia
que tinha sido preso por causa de uma briga com faca, mas um belo rapaz.
Ele já havia violentado algumas mulheres na aldeia, no entanto todas
tinham medo de apresentar queixa. Ele não gostava apenas de violar, mas,
primeiramente, de assustar e de humilhar, sobretudo nesse dia, em que
uma mulher se encontrava inteiramente sob seu poder, sozinha numa
floresta no cair da tarde. Claro que ele não viu Satanás, que estava ao
lado da profetisa. No entanto, quando o motorista bateu em seu rosto e a
agarrou, Pelágia não quis usar seu poder profético, mas apenas sua força
física. Se, na época em que Pávlov tentara violentá-la, ela era uma
menina frágil, agora era uma mulher madura e robusta do norte da Rússia.
Ela deu um chute no ventre do motorista e foi embora com a blusa
rasgada, cobrindo com as mãos os seios desnudos. Assim ela se salvou da
tentação pela primeira vez. Na segunda vez, tudo deveria acontecer de
boa vontade: um homem bom e bonito havia lhe agradado, mas um ferido de
guerra, um mutilado. Tudo também aconteceu muito rapidamente, e o grande
perigo era justamente a pressa. Houve um pedido de casamento, tudo em
ordem e conforme a lei, mas contra a ordem e a lei havia seu forte voto
de castidade. Ela temia somente a desordem e o acaso. E esse acaso
perigoso se dera durante um funeral, ao qual Pelágia havia ido com seu
pai, o Anticristo. Seu pai, Dã, a Áspide, o Anticristo precisou sair
mais cedo devido aos seus afazeres de zelador, e na volta Pelágia foi
acompanhada pelo ferido de guerra. Claro que houve lágrimas no funeral
--- embora o morto não fosse muito conhecido, as almas estavam
enternecidas. Eles caminhavam nesse estado de espírito, e não era ele
que a conduzia, mas o contrário, pois as ruas estavam cobertas de gelo e
ele usava uma prótese e uma bengala. Quando se aproximaram da casa dele,
ele pediu que a profetisa Pelágia entrasse:

--- Ruthina, vamos tomar um chá depois desse frio gelado...

Ele agiu como normalmente agem os homens nessa situação... Ela entrou e
ele se pôs a mostrar as fotografias do \emph{front} em que aparecia com
o morto. Mostrando as fotos e chorando, seu rosto ganhou uma expressão
totalmente infantil, e ela sentiu pena desse homem que havia sacrificado
na guerra sua juventude e masculinidade e agora não era plenamente
viril. E ela de novo teve vontade de se entregar. Mas Satanás, como
antes, estava ao lado. A luz fora apagada com antecedência, o ferido de
guerra visivelmente sentia vergonha, diante da jovem, de sua mutilação,
do que sobrara de sua perna... A profetisa Pelágia já havia se deitado
na cama quando, de repente, no escuro, esbarrou a mão na bengala dele,
que caiu, e o barulho da queda fez com que a profetisa voltasse a si de
longe, aonde havia ido por um instante, apoiada no travesseiro, ao lado
de um corpo estranho e tenso que ela deveria salvar, assim como a si
mesma... Ela se levantou de um salto, pois tudo o que acontecera, mesmo
que o irreparável não tivesse sucedido, formara uma história. Assim que
se compôs um enredo, restabeleceu-se a ordem, e, com a ordem,
restabeleceu-se o voto de castidade que ela fizera ao Senhor. A
profetisa se vestiu, desculpou-se e foi embora, e apenas pediu que ele
não acendesse a luz até que ela saísse, tateando no escuro... Nessa
época, Pelágia estava com 27 anos e, então, pareceu-lhe que sua
virgindade era suficientemente forte e não estava tão sujeita a
tentações. No entanto, as tentações logo reapareceram: primeiro em
sonhos, depois na realidade. Por isso, agora, parada num jardim escuro,
ao lado de um pecador apaixonado por ela, a profetisa resolveu lutar
contra o pecado indo ao encontro dele, de Satanás, mas, mesmo assim, sem
quebrar seu voto de castidade.

--- Está bem --- disse ela ---, eu lhe darei sangue para a experiência.

Saviéli não podia acreditar em sua sorte, pôs-se a rir alegremente e
pediu para beijar sua face. Ela permitiu. Então ele pediu para beijar
sua mão. E ela consentiu de novo. Mas ele não teve coragem de ir adiante
em seu pedido, e eles saíram do jardim.

--- Ruthina, talvez possamos passar a noite aqui --- disse Saviéli. ---
A datcha é grande, e acharão um quarto para nós.

--- Não --- disse a profetisa ---, meu pai está sozinho em casa... E eu
também sinto falta dele...

--- Então eu vou também, mas vamos embora sem nos despedirmos, minha mãe
entenderá, senão eles vão nos segurar aqui. Só não sei como chamaremos
Andrei...

--- Andrei foi embora há muito tempo --- disse Ruthina-Pelágia. --- Eu o
vi sair.

--- Ele está sofrendo --- disse Saviéli. --- Tenho pena dele.

--- Mas de Vássia você não tem pena --- disse de repente Ruthina-Pelágia
--- e ele também está sofrendo.

--- De Vássia? --- repetiu Saviéli, surpreso. --- Eu o conheço há muito
tempo. Ele é perigoso, vive de uma forma assustadora, é como se todos
fossem culpados de algo perante ele. Eu tenho medo de Vássia ---
confessou Saviéli ---, é um terrível antissemita, um antissemita doentio
e agitado.

--- Não é verdade que ele se parece muito com meu pai? --- perguntou a
profetisa Pelágia.

--- Realmente --- respondeu Saviéli. --- Eu estava pensando nisso agora
mesmo. Provavelmente porque os ucranianos do Sul se misturaram com os
turcos. Aliás, ele sabe que parece um judeu e sofre por isso. Se ele
tivesse outra aparência, possivelmente seria um bom rapaz e um
antissemita mais moderado. Hoje, perto da galeria Tretiakóv, ele ficou
muito nervoso e bateu em Sómov tolamente. Pode-se bater em Sómov com
mais inteligência, ele merece. Vássia nem sempre age como um tolo;
quando ele se distrai, se esquece de si, bondade transparece nele, e é
agradável ficar em sua companhia. Mas hoje ele pode causar problemas.

Isso era verdade. Depois de se separarem, perto da galeria Tretiakóv,
depois de ele falar seus disparates, das vendas dos \emph{jides} e do
Deus \emph{jid}, Vássia não conseguia se acalmar nem ficar sentado,
andava de um lado para outro na esperança de se cansar e de se
tranquilizar. Mas ele não se cansou nem se acalmou. E não compreendia o
que lhe passava: ele odiava os judeus a ponto de ter um ataque de nervos
e de repente se viu apaixonado por uma judia de olhos azuis. Em relação
às mulheres, Vássia era mais calmo e ponderado do que Saviéli e Andrei e
considerava as paixões e os suspiros amorosos sinais de fraqueza que não
convinham a um homem, a não ser que fosse um judeu. Vássia havia se
casado com uma lavadora de pratos, mas se separara; agora ele mantinha
um relacionamento com uma uma professora de inglês da escola que ficava
em frente à casa dele... Eis que, desde manhã, ele foi invadido pela
tristeza. Vássia sabia onde morava Saviéli e ouvira dizer que no mesmo
apartamento morava a judia que lhe tirava o sossego.

``Eu vou lá,'' decidiu Vássia, ``faz tempo que preciso fazer isso. Farei
um escândalo na casa dessa \emph{jidovka} e depois me acalmarei e me
esquecerei dela.''

Vássia passou antes pela Casa dos Escritores e entrou no famoso
restaurante em que o público literário privilegiado tinha permissão de
aspirar o cheiro picante da carne em decomposição e do molho de tomate
azedo... Sentado à mesa de um cantor judeu rico que tinha pavor de seus
escândalos e que, dois anos antes, havia apanhado dele nas festas de
Maio, Vássia tomou, de graça, uns trezentos gramas de vodca com
\emph{sprats}...\footnote{\emph{Sprats} (em russo, \emph{chpróty}),
  peixes da família \emph{Clupeidae}, normalmente defumados, do Mar
  Báltico.} Vássia comia pouco... Da Casa dos Escritores até o bulevar
em que morava a judia era bem perto, e Vássia andava rápido, mas os
trezentos gramas de vodca gratuita subiram-lhe mais à cabeça e
deformaram ainda mais o mundo divino diante dele. Ele chegou nesse
estado ao prédio do bulevar. Era uma construção antiga, distinta,
pré-revolucionária, e a Vássia parecia haver um cheiro adocicado de
\emph{jid} na escadaria. No entanto, no andar de cima farreavam
ruidodamente, ao som de \emph{tchastuchkas}, e isso o acalmou: era sinal
de que incomodavam os \emph{jides} e de que não se deixavam dominar...
Com os olhos por algum motivo lacrimejantes, ele achou o número do
apartamento e tocou a campanhia. A porta se abriu.

--- Posso ver a Rute?... Ruthina, por favor?... A moça Ruthina? ---
começou Vássia, enrolando a língua, e subitamente se calou.

O que ele viu no vão da porta deixou-o pasmo. Ele viu sua própria imagem
envelhecida, iluminada pela luz fraca e amarela da pequena lâmpada do
corredor. Vássia se viu velho, grisalho, com uma corcunda judia. Pois
foi seu pai, Dã, a Áspide, o Anticristo, quem abrira a porta.

--- Rute não está em casa --- disse o Anticristo e, ao fitar Vássia,
também o reconheceu.

Era seu primogênito, concebido perto da cidade de Kertch, à beira-mar,
com Maria, uma alma bondosa, uma jovem prostituta da vila de
Chagaro-Petróvskoie, da região de Dimítrov, perto da cidade de Khárkov.
Então o Anticristo desencostou mais a porta e Vássia, da tribo de Dã,
uma semente ruim, entrou. Pai e filho sentaram-se à mesa, frente a
frente, e começaram a se olhar. E, quanto mais se olhavam, mais se
reconheciam.

--- Então --- disse o Anticristo ---, conte, filho, como você pôde
ofender seu Deus judeu?

--- Está mentindo, \emph{jid} --- gritou Vássia. --- Meu pai é
ucraniano... Ucraniano com turco. E minha mãe é da vila de
Chagaro-Petróvskoie. E meu Deus é ortodoxo. Eu odeio o Deus \emph{jid}.
E também odeio o pão impuro dos \emph{jides} --- e ele pegou um pedaço
de pão de cima da mesa e o jogou no chão.

Era realmente o pão impuro do exílio, legado pelo profeta Ezequiel. E os
suaves olhos judeus do Anticristo se desfiguraram, irradiaram algo
mortal e que a profetisa Pelágia havia herdado de seu pai adotivo. No
mesmo instante, inflamaram-se os olhos de Pelágia, que estava a
quilômetros dali, entre as macieiras do jardim escuro de uma datcha.
Quando os olhos do Anticristo afoguearam e o quarto matizou-se de
rubro-escuro, entre a cor da ginja e da framboesa, como nuvens celestes
antes do crepúsculo, Vássia se assustou, e seu coração, que, alguns
instantes antes, estava tomado pela segurança eslava, começou a doer, e
pela primeira vez ele sentiu a única e verdadeira culpa judia perante o
mundo decaído, a Fraqueza. Vássia se levantou e, sem que o pai a
acompanhasse, foi até a porta de entrada e a abriu sozinho, saindo para
o patamar da escadaria. Nesse momento, escancarou-se a porta do andar de
cima, onde farreavam ruidosamente, e brutamontes de rostos vermelhos,
todos os que havia, saíram para o patamar. Isso se chama: ``os homens
vão fumar''. E um deles disse a Vássia:

--- \emph{Jid,} aonde você vai com esses olhos esbugalhados?

Vássia não respondeu nada e chegou não se sabe como ao seu prédio. Assim
que entrou em casa, começou a procurar uma forma de enforcar-se. No
começo, pensou em usar o cinto da calça, no entanto percebeu que o cinto
poderia não aguentá-lo; então encontrou embaixo da banheira, no meio da
poeira, uma corda de varal que estava lá não se sabe desde quando,
talvez estivesse lá, à espera dele, desde o tempo dos antigos donos da
casa, para que agora ele pudesse realizar aquilo a que fora
predestinado. Ele fez um nó na corda e pôs-se a procurar um gancho, mas
não conseguiu achar um bom, nem no quarto nem na cozinha, também não
havia um prego decente, tampouco um martelo, pois Vássia vivia em total
desalinho. Havia garrafas e vidros sujos no parapeito da janela, meias
sujas sobre o radiador para calefação, montes de lixos em todo canto; e,
com exceção de dois ícones --- Cristo Salvador e São Nicolau ---, Vássia
não tinha objetos de valor.

``A venda dos ícones dará para meu enterro,'' pensou Vássia, ``e, se
tiverem a sorte de vender a um estrangeiro, ainda será possível colocar
uma cruz. Escreverei a tia Ksiénia para que ela venda os ícones para o
enterro e a cruz do túmulo.''

Sentado à mesa, com a corda de varal enrolada na mão, Vássia escreveu
uma carta a tia Ksiénia, assim como um pedido a quem o encontrasse morto
para enviar um telegrama a Vorónej em nome de Ksiénia Korobko ---
sobrenome de casada Gussakóvaia ---, e anotou o endereço. E também um a
Aleksandra Korobko --- sobrenome de casada Nalivaiko ---, à região de
Khárkov, distrito de Dimítrov, vila de Chagaro-Petróvskoie, sítio
Lugovoi. Ele acrescentou ao bilhete uma nota amassada de três rublos,
terminando os preparativos, e começou de novo a procurar um gancho. Não
o achando, resolveu simplesmente se atirar da sacada, no entanto ficou
constrangido em poder provocar uma algazarra de curiosos, uma multidão
de tolos. Ele continuou suas buscas e achou, finalmente, um gancho num
canto perto da janela, coberto de teias de aranha e de tinta. Pelo
visto, os antigos moradores usavam o gancho para fixar a barra em que
penduravam as cortinas. Convencido de que o gancho era forte, molhou um
pedaço de sabão na torneira, ensaboou a corda e largou o sabão ali
mesmo, no meio do quarto. Vássia fez o nó de novo, pegou um banquinho
bambo, sentiu fortes cólicas no estomâgo e, de pé no bamquinho, urinou
no chão; deu um salto no ar com o nó envolto no pescoço, pisando na
beirada do banquinho, que caiu. O nó apertou na hora e ele começou a
gemer e a babar; e Vássia morreu de modo impuro, soltando um sonoro
peido de Khárkov.

Assim a semente podre do Anticristo, o enviado do Senhor, foi separada
do mundo.

Vássia foi achado depois de três dias pelos vizinhos, que certamente
ficaram assustados. Não há como um enforcado não causar espanto, no
entanto aqui o susto fora reforçado por um incidente. Depois de muitos
gritos e de muitos ais e sem tocarem no defunto, telefonaram para a
polícia e a ambulância e, de repente, antes da chegada das autoridades,
Vássia se soltou e caiu no chão, na presença dos vários vizinhos que se
aglomeraram ali, e dele rolou uma fina roda endentada, como um reloginho
de um grande relógio de bolso, fez um semicírculo, oscilou um pouco e
tombou, deitada. Com isso, conclui-se a parte insólita da morte de
Vássia e começou a rotineira. Chegaram Ksiénia e Chura, que, prevenidas
pelos telegramas, encomendaram um caixão e os músicos.

Ksiénia, como não raro acontece a mulheres que foram libertinas na
juventude, transformara-se numa velhinha caridosa e compassiva, sem
filhos. Era uma rica viúva graças aos recursos que o falecido marido lhe
deixara, morava na periferia de Vorónej numa casinha própria com jardim.
Para Vássia, sua tia sempre fora uma espécie de tutora e lembrava sua
mãe, Maria, que ainda menina, em 1933, o ano da fome, fora morar com
ela. Ksiénia enviara a irmã de volta à aldeia devido a um escândalo
familiar, mas por Vássia esforçava-se para fazer o melhor. Ela organizou
o funeral do sobrinho à própria custa, pois Chura não deu um copeque. De
fato, ela não tinha nada. Como antes, Chura morava na vila de
Chagaro-Petróvskoie, da qual pouco saía, era pobre, tinha uma penca de
filhos, todos crescidos e mal arrumados na vida, e, como antes, ela
tinha um olhar maldoso, inexpressivo e extenuado. O velho sobretudo de
Vássia, suas sandálias gastas, sua chaleira coberta de fuligem, tudo,
salvo o que era enterrado com o sobrinho, ela embalou em trouxas e levou
para sua casa, na vila de Chagaro-Petróvskoie. Ksiénia pegou apenas os
dois ícones, de Cristo Salvador e de São Nicolau. Ela queria levar os
ícones consigo, mas, a conselho de um vizinho, vendeu-os a contento a um
sujeito barbudo, dando, evidentemente, uma comissão ao conselheiro.

Eis que levaram o caixão de Vássia. Enquanto o conduziam, ficou logo
visível o constrangimento que uma morte comum pode causar. Num dia de
verão, em plena manhã de trabalho, de repente, sem mais nem menos,
ressoaram em meio à monotonia cotidiana sons de uma marcha fúnebre
tocada por alguns músicos contratados. Saíram levando de casa as coroas
e a tampa do caixão, que não estava apoiada nos ombros, mas nas cabeças
dos homens que a carregavam. Finalmente, levaram o defunto, cujo rosto
não mostrava inteligência, como, aliás, é o rosto da maioria das pessoas
colocadas num caixão. De modo que, quando dizem ``O defunto tinha o
rosto inteligente'', estão enganando a si mesmos com as lembranças do
tempo em que ele vivia e era querido.

Poucas pessoas foram ao funeral. Uns poucos velhos e velhas e alguns
jovens, evidentemente os vizinhos. Entre eles estava Andrei Kopóssov,
que soubera da morte de Vássia e viera acompanhar seu irmão. Embora não
o soubesse, Andrei sentia uma estranha compaixão por Vássia, como por um
irmão de fato, mas um irmão miserável, fracassado... E era de fato
assim, como depois ele pôde se convencer. O pai de Vássia e de Andrei, o
Anticristo, e sua filha adotiva, a profetisa Pelágia, observaram o
funeral de longe.

Foi um funeral alegre, e essa alegria foi produzida pelas crianças. Na
escola em que havia em frente ao prédio de Vássia, ele era conhecido,
provavelmente por ter lá aparecido bêbado várias vezes à procura da
professora de inglês, Ekaterina Anastássievna... A professora ou não
estava em Moscou nesse dia ou havia se desentendido com Vássia em
virtude de alguma atitude violenta, como era do feitio dele. Era
evidente que toda a rua conhecia seu comportamento, e as crianças se
divertiam com isso. Eis que também agora a criançada corria, alegre e
travessa, pelo funeral. Meninas de mãos dadas davam pulinhos e gritavam:

--- Enterraram o Alho. Enterraram o Alho...

Acontece que Vássia ganhara ali o apelido de ``Alho''. Um dia, um garoto
travesso, querendo divertir as meninas, aproximou-se correndo dele e,
dando um salto para trás, franziu o nariz como se sentisse um cheio ruim
e disse:

--- Credo, que fedor!

As crianças corriam por todo lado ao longo da rua.

--- Lá está o caixão --- elas gritavam, animadas.

Afinal, crianças são insensíveis, pois ainda não foram atormentadas por
sua consciência, ainda precisam amadurecer; seus corações são duros e
rudes como raízes de plantas jovens cravadas na terra. No entanto, de
uma lavanderia ao lado saíram duas funcionárias de jalecos brancos ---
ouviram os sons da marcha fúnebre, viram o caixão de um estranho e
enxugaram algumas lágrimas. A vida não lhes parecia ser tão infinita
como para a criançada, e qualquer morte era para elas uma ameaça. Elas
sofriam por si mesmas, lamentavam a si mesmas.

Então o Anticristo, pai do primogênito rejeitado pelo Senhor, disse uma
passagem do sexto salmo de Davi:

--- Volta-te, Senhor, liberta a minha alma; salva-me com Tua
graça.\footnote{Salmos 6:5.}

E a profetisa Pelágia, a filha adotiva do Anticristo, continuou:

--- Pois na morte não há lembrança de ti; e no túmulo quem Te
louvará?\footnote{Salmos 6:5.}

No entanto, o Anticristo ainda não sabia que sua filha era uma
profetisa, pensava apenas que ela havia estudado o livro dos Salmos e a
elogiou.

Nesse momento, o defunto foi colocado num caminhão e levado para o
enterro. Poucos o acompanharam até o cemitério. Basicamente, Chura,
Ksiénia e os homens que ela havia pago para segurar as coroas. Apenas
Andrei Kopóssov, filho do Anticristo e de Vera Kopóssova, da cidade de
Bor, da região de Khárkov, e irmão de Vássia, acompanhou o caixão sem
receber nada em troca. O funeral de Vássia foi acanhado, quase deserto,
mas, alguns dias depois, começaram a falar dele como de um talento que
desaparecera de forma trágica e prematura. No restaurante dos literatos,
almoços e jantares transformaram-se em refeições fúnebres, todos estavam
enternecidos e, por alguns dias, trataram-se com consideração. No
entanto, havia também outras pessoas a quem a morte de Vássia causou
efeito, embora em outro sentido. Com ainda mais força se agarraram à
conhecida postura ``Quem está arruinando a Rússia?'', as mãos apoiadas
nas bochechas, às vezes mexendo o maxilar, os olhos cravados na toalha
de mesa suja de vinho. Andrei Kopóssov olhou à sua volta, observou os
rostos variados, rostos que tinham conseguido tudo ou, em todo caso,
muitas coisas, e compreendeu que, com a evolução natural da vida, cedo
ou tarde, veria seus necrológios. ``Quem está vivo morrerá,'' pensou,
``mas já que eu não estou vivo, não morrerei.'' Incutiu a si essa ideia
--- ``não morrerei'' --- e fim. Incutiu a si um pensamento pecaminoso.
Pois ele já sabia muita coisa de si mesmo. Suspeitava vagamente, como
num sonho, de que era filho do Anticristo, o enviado do Senhor. Mas logo
sua mãe, Vera Kopóssova, uma velhinha piedosa, o confirmou.

Depois das paixões que marcaram sua vida, ela envelhecera precocemente e
dera de ler o Evangelho, e parecia bem mais velha do que seus cinquenta
e poucos anos. Pelo menos dez anos mais. Ela usava uns óculos baratos de
velha com uma armação de ferro; quando pegava o Evangelho nas mãos, seu
rosto adquiria uma expressão tola e solene e sua nuca parecia a de um
animal doméstico que olha para qualquer objeto humano com curiosidade.

É surpreendentemente belo o rosto de uma pessoa que pensa e que lê
sinceramente, por si mesma, um livro profundo. Já o rosto de uma pessoa
que não pensa e que lê um livro que lhe inquieta de forma insensata,
conforme o que lhe fora incutido de fora, perde com frequência seus
traços humanos, que são substituídos por traços animais, sempre
desagradáveis no semblante de um homem. Algo de simiesco transparecia no
rosto de Vera enquanto ela lia o Evangelho. Mas, mesmo tola nos
pensamentos, ela às vezes era inesperadamente habilidosa nas palavras.
Quando Vera foi visitar seu filho, ele resolveu levá-la à Praça
Vermelha, aonde frequentemente os ex-provincianos levam seus parentes,
para lhes suscitar respeito por sua posição atual.

Nesse dia, Andrei tinha um colóquio no instituto para o exame do dia
seguinte, por isso ele e sua mãe chegaram cedo à praça, o sol ainda se
levantava. O centro de Moscou, de dia, atormenta com o barulho e o
tumulto, no entanto a aurora silenciosa sobre o Krêmlin é mais solene do
que qualquer oração. Um brilho rosado e celestial cobre as velhas pedras
da fortaleza. A \emph{Rus} fica meditativa nesses minutos e a alma se
aconchega, sentindo a tranquilidade da casa dos pais, e qualquer um que
apareça verá ali uma mãe que não faz diferença entre o seu e o outro,
pois se apieda de todos, como a Mãe de Deus... Eram breves esses
instantes de comunhão na alvorada de verão sobre a Praça Vermelha. No
alto do céu azul, cristalino e solene, ressoa o tinido dos relógios da
Torre de Salvador\footnote{Principal torre do Krêmlin de Moscou.} e,
marcando o passo sobre a calçada ecoante, como sob os arcos de uma
catedral, surgem os rituais da troca da guarda, perto do caixão do
senhor marxista, o Mausoléu de Lênin.

Andrei Kopóssov e sua mãe, Vera Kopóssova, contemplavam tudo o que
acontecia. De repente Andrei olhou para trás e viu os olhos de sua mãe
cheios de lágrimas, das que correm pela face de forma inconsciente e
imperceptível.

--- Não é nada, mamãe --- disse Andrei Kopóssov ---, é só a troca da
guarda em frente ao Mausoléu de Lênin. Ela acontece todos os dias e
várias vezes por dia.

--- Que honra para um homem --- disse Vera Kopossóva, em voz baixa e
tomada por lágrimas, uma mulher continuamente humilhada, tanto por seus
pecados como pelos pecados dos outros ---, que honra para um homem...
--- disse sem pensamentos sensatos, mas com palavras repletas de
sabedoria.

Assim se manifesta o verdadeiro caráter popular. Há tempos, o termo
``caráter popular'' se tornou uma expressão idolatrada na Rússia. Há
tempos, seu sentido foi canonizado pela \emph{intelligentsia}
eslavófila:\footnote{Em linhas gerais, o pensamento ``eslavófilo''
  defendia um caminho voltado para a própria tradição russa, por vezes
  revelando um pensamento messiânico; ao contrário dos
  ``ocidentalistas'', que queriam modernizar o país conforme modelos
  europeus. No século XIX, o tema marcou debates acalorados entre
  literatos russos.} o caráter popular é o povo simples. Os eslavófilos
têm até a sua própria Bíblia, que estudam com o afinco de monges
fanáticos, na qual acreditam incondicionalmente, da qual se vangloriam e
a qual contrapõem, em discussões, à Bíblia dos hebreus. Essa Bíblia é a
aldeia russa.

--- Vocês têm a Bíblia e nós a aldeia russa: eis a nossa Bíblia. Vocês
não podem compreendê-la.

Aqui se revela o sonho secreto dos eslavos de interromper a história. E
também o sábio Herzen, com suas esperanças absurdas na
\emph{obschina}.\footnote{Comunidade autônoma de camponeses.} E
Dostoiévski, o profeta da \emph{intelligentsia} russa servil, o qual
afirma ter descoberto o caráter popular, em seu melhor aspecto, entre
prisioneiros. Mas o que ele é, o caráter popular, não conforme
Dostoiévski, mas conforme Púchkin? Para Púchkin, o caráter popular não
vem do povo simples, mas do caráter nacional. ``O caráter nacional de um
escritor,'' escrevia Púchkin, ``é uma qualidade que só pode ser
plenamente apreciada por seus compatriotas.''\footnote{Retirado de um
  pequeno texto de Púchkin, ``O caráter popular na literatura''
  (\emph{Naródnost v literature}), achado entre seus manuscritos dos
  anos 1820.} Segundo Púchkin, o aristocrata Racine é popular para um
francês, mas não para um alemão. Púchkin, como sempre, é genialmente
claro, no entanto nem sua genialidade profética poderia compreender o
que ainda não foi dito pelo Senhor por meio do tempo contínuo. Pois o
tempo é a língua que o Senhor usa para falar com os homens. Na época de
Púchkin, a questão do caráter popular ainda não era uma questão trágica.
A questão do povo não era interpretada com a tragicidade de hoje. Além
disso, a matéria popular autêntica era profícua, parecia transbordar, um
oceano inesgotável como os minérios do planeta. Mas quem o secou, quem o
esgotou? A consciência popular, através da qual o povo passou a conduzir
a direção da história. É fecundo o instinto popular, essa inteligência
eterna das massas vinda dos ancestrais, na qual o homem parece agir e
falar à sua maneira, mas, na realidade, fala como seu bisavô falava, age
como seu avô agia. A fala do homem não vem dele, mas é essencial e
eterna. Assim que o homem começa a falar à sua maneira, privado de
cultura, torna-se estéril. O povo não pode ensinar, mas pode-se aprender
com o povo, para depois lhe explicar o que ele é. Essa é a obrigação
sagrada de um indivíduo. O povo não é capaz de entender seu instinto
fecundo com sua consciência humilde e estéril, pois, para entender os
instintos nacionais, é necessário possuir uma consciência supranacional
e comum a toda humanidade. Quando o povo tenta compreender, com sua
consciência humilde, seus instintos profundos, surge a filosofia do
\emph{lubok},\footnote{Estampa popular com uma sequência de imagens
  acompanhada por textos simples.} a filosofia da \emph{tchastuchka,} a
que se curvam os eslavófilos da Rússia. Um criminoso leviano, um
oposicionista ou um governante --- eis o produto final da consciência
popular. Mas o pior é quando a cultura, que tem a obrigação de servir ao
povo, explicando-lhe o que ele é, ou seja, explicando o elemento popular
ao povo, tenta ouvir dele, de maneira covarde e submissa, as verdades
sobre si mesma, sobre sua cultura e personalidade. Dessa maneira, ela
corrompe o povo e, honrando a consciência popular estéril, destrói o
instinto fecundo que existe nele. E do povo pouco restou, e ele se
conserva somente quando, de forma inconsciente, nascem palavras
essenciais e sagradas, quando o homem raciocina tolamente, mas fala com
sabedoria... Se, no século XIX, a Rússia conseguiu criar uma grande
cultura, foi porque as reformas de Pedro, o Grande, separaram a
\emph{intelligentsia} do povo, e, explorando o oceano fecundo do
instinto popular, a cultura não ficou escravizada pela consciência
popular. Somente mais tarde, perto do fim do século, graças aos esforços
de acusadores \emph{raznotchinets}, a consciência popular começou a
escravizar a cultura, e os adeptos desses acusadores levaram o processo
ao extremo.

Assim pensava Andrei Kopóssov, durante o colóquio, lembrando-se das
palavras de sua mãe. No Instituto de Literatura, antiga casa de Herzen,
partidário da \emph{obschina} rural --- a salvadora da Rússia ---, já
haviam começado as reformas de verão: pairava cheiro de tinta, os
corredores estavam entulhados de móveis e o chão forrado de jornais.
Apenas a sala de conferências estava intacta, onde davam continuidade ao
processo de educação dos partidários do realismo socialista. Após ter
pensado algum tempo em seus próprios assuntos e ter feito anotações e
observações rápidas numa folha de papel, ele queria prestar atenção no
que diziam ao seu redor, no entanto se falava tanto da consciência
popular eslavófila e o palestrante, um conhecido poeta com um pseudônimo
puramente russo e um acento de Riazán, tinha a voz tão sonora que Andrei
se distraiu de novo e começou a olhar para os lados.

A sala de conferências era revestida de fragmentos de literatura, de
todos os tempos e de todos os povos, como órgãos separados, extraídos de
um corpo. Andrei longamente pensou em que se pareciam aqueles painéis
que cobriam completamente as quatro paredes da sala com capas de
clássicos do passado e dos que são considerados clássicos hoje, mas
também de livros de várias categorias, de primeira, segunda e terceira.
Ao redor, havia perfis e silhuetas. E Andrei entendeu que era uma sala
de dissecação literária, um necrotério para partes isoladas de um mesmo
corpo. As citações e as capas conservadas pareciam um fígado, pulmões,
mãos e pés colocados em potes de vidro com álcool. As partes do corpo
assim conservadas estão mais afastadas do homem do que uma pedra na rua
ou um galho de árvore. Uma pedra e um galho lembram mais um homem vivo
do que seu próprio fígado e pulmões extraídos de si. De forma análoga,
os fragmentos de literatura de um necrotério literário estão longe da
literatura. Além disso, havia algo de médico, de científico, nessa
instituição, onde a literatura parecia uma cobaia, um coelho torturado
por experimentos, onde à literatura foi destinado o papel de vítima em
nome do bem-estar da humanidade, conforme os princípios humanistas do
realismo socialista.

Ao terminarem as aulas, Andrei Kopóssov voltou rapidamente para casa,
pois ele e sua mãe tinham que visitar inúmeros estabelecimentos, onde os
provincianos se abastecem de produtos difíceis de achar. Andrei
precisava comprar uns jeans a Varfolomei Vesselóv, filho de sua irmã
Tássia; uma combinação à Tássia, antiga paixão de seu pai, o Anticristo
--- fato que aquele ignorava ---; à velha sentinela Serguéievna, sogra
de Tássia, torrões de açúcar natural para o chá --- que na cidade de Bor
não se encontravam ---; aos filhos de Ústia roupas de baixo e
guloseimas; e também, na medida do possível, conservas de carne para
estocar e limões e laranjas, frutos sagrados, para se dar um pouco de
prazer... No entanto, ao chegar, Andrei descobriu que tudo já havia sido
comprado, empacotado em papel de embrulho branco, cinza e azul e em
papel colorido com a marca da loja. E havia uma sacola cheia de frutos
sagrados, limões e laranjas. Sua mãe, Vera, sentava-se com seu lencinho
branco e lia o Evangelho, mas com um ar astuto, alegre e enigmático.

--- Adivinhe, filhinho, quem esteve aqui e me ajudou a fazer as
compras?...

--- Mas, mamãe, será que você conhece alguém em Moscou?

--- Eu conheço e me conhecem --- disse Vera ---, eu não queria lhe dizer
logo, fiquei constrangida, mas a \emph{velha crente} Tchesnokova, que
mora na casa n\textsuperscript{o} 30 da Rua Derjávin, ainda se
corresponde com seus antigos inquilinos. Ela me deu o endereço de Dã
Iákovlevitch e sua filha, Ruthina. E eu pedi para sua vizinha (que
mulher simpática!) ligar para eles... Ruthina veio num instante. Ela nos
convida para visitá-los, aqui está o endereço deles.

Então Andrei se sentou na cadeira e sentiu uma estranha inquietação com
o que ouvira.

--- Eu conheço o endereço --- disse ele ---, e conheço Ruthina. Eu a
amo, mamãe, não posso mais esconder.

A essa altura, a aparência astuta desaparecera do rosto de sua mãe,
deixando lugar para certo temor, submisso, tolo e solene, como quando
ela lia o Evangelho.

--- Você é muito desajeitado, meu filho --- disse Vera e fez um rápido
sinal da cruz ---, muito agitado e inseguro, mas será possível amar a
própria irmã? Seu pecado lhe será perdoado, pois você não sabia, mas a
culpa recairá em mim por eu não ter lhe contado. Oh, sou uma pecadora
sem salvação!

--- O que você está dizendo, mamãe? --- Andrei ficou surpreso e
assustado. --- Por acaso ela é sua filha?

--- Ela não é minha filha, mas é filha do seu pai... Seu pai é Dã
Iákovlevitch, um judeu... De modo que você não é russo... Não é à toa
que seus parentes do lado de Tássia, os Vesselóv, um antigo clã do
Volga, não gostam de você... Especialmente a velha Serguéievna. Ela tem
um faro animal para judeus, apesar da idade. Assim, eu me retrato do que
fiz, meu filho, e peço-lhe perdão pelo meu grave pecado.

Ela quis se ajoelhar diante de Andrei, no entanto ele a conteve a tempo
e disse:

--- Não é nada, mamãe. O terrível não é saber de quem eu sou filho de
verdade, mas não conseguir me acostumar a essa ideia. Vamos nos abraçar,
mamãe, para que eu aceite isso mais rápido.

Eles ficaram assim abraçados até anoitecer. À noite, Andrei Kopóssov
disse:

--- Vou visitar meu pai.

--- Obrigada, meu filho --- disse Vera. --- E eu vou com você. Ele pode
não ser meu marido perante os homens, mas é meu marido perante Deus.

Ao chegarem, foram recebidos na antessala por Ruthina, que disse em voz
baixa:

--- Nosso pai hoje celebra uma festa triste. O início do jejum judaico
\emph{Shiv'ah Asar B'Tamuz},\footnote{O jejum \emph{Shiv'ah Asa B'Tamuz}
  ocorre em memória da quebra de partes do muro de Jerusalém, antes da
  destruição do Segundo Templo pelos romanos, e da quebra das Tábuas da
  Lei por Moisés.} que é o jejum em memória das Tábuas da Lei que foram
quebradas...

Quando eles entraram no apartamento e Vera Kopóssova, agora uma velhinha
piedosa, viu o objeto de sua última paixão envelhecido e com os cabelos
cinza, com as costas permanentemente curvas, ela girou a cabeça
jovialmente e disse:

--- Mas seria você, meu querido? Aqui estou eu, sua amada... E aqui está
seu filho, Andrei, que não leva seu nome, mas foi gerado por você...

A mãe e o pai, que por longo tempo não se viram, abraçaram-se; depois o
filho e o pai, que nunca haviam se visto; então o irmão e a irmã, que se
viam sem saber quem eram e, por isso, quase pecaram... Então chegou a
hora de acender as velas --- na véspera de datas religiosas, as velas
são acesas em um momento estritamente determinado.

Assim, em sua família terrena, Dã, a Áspide, o Anticristo, o enviado do
Senhor, começou o jejum. Eis a composição de sua sagrada família. De um
vagão de carga e de uma mãe anônima, que era conduzida à escravidão
alemã, caiu nas mãos do Anticristo, ainda bebê, a profetisa Pelágia,
nascida na vila de Brussiány, perto da cidade de Rjév. Através do
adultério, o terceiro flagelo do Senhor, Vera Kopóssova uniu-se à
sagrada família, assim como Tamar unira-se à família de Judá. E Vera e
Anticristo tiveram, na cidade de Bor, um filho, Andrei, a semente boa. A
semente ruim, o primogênito Vássia, gerado por Anticristo e Maria
Korobko perto da cidade de Kerch, foi rejeitada e se tornou o Irmão
perdido para sempre... Pois nem todos os fragmentos do Cálice serão
colados, alguns serão eliminados, no entanto, graças à força divina, o
Cálice ficará como novo...

O jejum \emph{Shiv'ah Asar B'Tamuz,} no dia 17 de Tamuz,\footnote{O
  jejum dura três semanas, do dia 17 de Tamuz até 9 de Av. Tamuz, no
  calendário judaico, é o quarto mês, com 29 dias, e Av o quinto, com 30
  dias.} é um dos mais tristes, e a mágoa não vem da violência de fora,
comum na história dos judeus, mas das maldades que o povo lançou contra
si mesmo, um povo que rejeitou seu Deus e ofendeu seu profeta Moisés, o
qual, em fúria e sofrimento, renegou os insensatos e quebrou as Tábuas
da Lei. Depois se seguiu o conhecido diálogo entre o Senhor e Moisés.
Cada vez que Moisés tentava renegar seu povo ingrato, o Senhor o
persuadia a dominar sua justa fúria, não em nome do povo, que era tão
ruim quanto todos os outros, mas em nome da realização do vaticínio do
profeta. No entanto, quando o Senhor quis renegar o povo, foi o próprio
Moisés quem o persuadiu, e, de novo, não em nome do povo, mas do
Desígnio do Senhor ligado a esse povo. Assim, no intervalo entre as
Primeiras e as Segundas Tábuas, a relação de Moisés com seu povo se
fortaleceu, tornando-se mais verdadeira e mais simples. Foi dito: ``As
Tábuas eram obra de Deus; e as escrituras esculpidas nas Tábuas eram
sinais de Deus''.\footnote{Êxodo 32:16.}

Quando Moisés e Josué, filho de Num, aproximaram-se do acampamento, este
disse:

--- Há gritos de guerra no acampamento.\footnote{Êxodo 32:17.}

Mas Moisés respondeu:

--- Não são gritos de vitória nem de derrota; eu ouço a voz dos que
cantam.\footnote{Êxodo 32:18.}

Assim, entre cantos e danças ao redor do bezerro de ouro, um ídolo
pagão, o povo renegou Deus. A arte, um dom divino, foi lançada contra
quem a dera.\footnote{``Quando se aproximou do acampamento e viu o
  bezerro e as danças, Moisés acendeu-se em ira; lançou das mãos as
  tábuas e quebrou-as no sopé da montanha.'' (Éxodo 32: 19, \emph{Bíblia
  de Jesuralém,} Ed. Paulus, 2016, p. 149)} E o pecado foi duplo, pois,
além da arte, não há nada de divino no homem. A ciência é fruto dos
homens, essencial, indispensável para a satisfação dos bens humanos. Ela
não necessita de Deus, e a ciência religiosa não pode nem deve existir.
Como a ciência, a filosofia é fruto dos homens e tem um motivo claro
para existir: a filosofia é necessária para uma criatura racional
exercitar a mente. Da mesma forma que um esquilo, em sua corrida inútil
na roda, realiza uma ação útil, conservando a força dos músculos, a
filosofia preserva a força dos músculos da mente, força indispensável
para a satisfação dos bens humanos na luta por sua existência. Por essa
razão, a filosofia religiosa tem, no fundo, a mesma finalidade da
filosofia ateísta, e qualquer tentativa de compreender Deus de forma
coerente conduzirá inevitavelmente ao ateísmo. Deus também não pode ser
compreendido através da moral, pois qualquer moralista sistemático e
honesto, mesmo Lev Tolstói, deve responder às clássicas perguntas
ligadas à moral: Por que o homem é mortal? Por que o mal existe no mundo
de Deus e por que ele reina nos limites da vida humana?

Mas existe algo inútil e incompreensível para a vida, para a satisfação
de bens, para a luta pela existência, e que, em oposição à ciência,
enfraquece com frequência as possibilidades físicas; em oposição à
filosofia, nem sempre aumenta a inteligência; e, em oposição à moral,
obscurece as questões eternas... No sétimo dia da Criação, aconteceu seu
Nascimento, quando o Senhor pediu ao homem que desse um nome a tudo que
ele havia criado...

Dessa forma, iniciou-se o jogo do Senhor com o homem, e o homem chamou
esse jogo de arte. O que é a arte senão a imitação instintiva do
Criador? Claro que, mesmo através da arte, não se pode enxergar Deus nem
compreendê-lo. Pois o Senhor disse a Moisés: ``Não poderás ver a minha
face, porque o homem não pode me ver e continuar vivendo''.\footnote{Êxodo
  33:20.} Mas a arte é a chama ardente da sarça que Moisés, ainda um
pastor desconhecido, vira num deserto longínquo, perto do Monte
Horeb.\footnote{Monte Horeb ou Sinai, ``o cenário da aparição da sarça
  ardente {[}...{]} e da adoração do bezerro de ouro''.
  (\emph{Dicionário bíblico,} Ed. Paulus, 1984, p.427)} Nenhuma arte,
por superior que seja, pode compreender Deus, mas ela é um sinal
semelhante à chama da sarça. Um sinal de que Deus está por perto. Quando
a alma humana é comovida e iluminada pela arte, Deus está ao lado dela,
e esse momento não pode ser desperdiçado, assim como o pastor Moisés não
desperdiçara seu instante de comoção. Nesses instantes, o Senhor permite
que se fale diretamente com Ele, face a face, pois assim disse o profeta
Isaías: ``Não fala sempre com Deus, mas somente quando Ele estiver ao
lado...''.\footnote{Possível alusão a Isaías 55:6: ``Procurai Iahweh
  {[}Senhor{]} enquanto ele se deixa encontrar, invocai-o enquanto está
  perto''. (\emph{Bíblia de Jerusalém,} Ed. Paulus, 2016, p. 1343)} No
entanto, para que não se deixe passar o momento em que Deus está ao
lado, é necessário ao menos de uma fração de talento, e Moisés, que a
possuía, disse: ``Irei ver de perto esse grande fenômeno que impede a
sarça de queimar...''.\footnote{Êxodo 3:3.}

Na sagrada família do Anticristo, o enviado do Senhor, todos eram
dotados dessa fração de talento, e nenhum deles havia perdido seu
momento. Nem a profetisa Pelágia, nem Vera Kopóssova, nem Andrei
Kopóssov. Já a semente ruim, Vássia, filho de Maria Korobko, fora
rejeitada.

Ao se despedirem, Vera Kopóssova levantou os olhos ao seu marido e disse
de forma repentina:

--- É você, Senhor?

Ele respondeu:

--- Não me chame de Senhor, pois só temos um Senhor. Todos nós
chegaremos e todos nós partiremos. Que diferença faz se o que nos leva
para o outro mundo são circunstâncias exteriores inescapáveis ou nossas
próprias artimanhas?

Com essas palavras, eles se despediram. Cada um foi viver sua própria
vida. Vera, a esposa do Anticristo, tornou-se ainda mais tola nos
pensamentos e mais sábia nas palavras, e partiu levando para a cidade de
Bor os embrulhos e as frutas sagradas, limões e laranjas, que havia
comprado; Andrei, o filho do Anticristo, terminou o ano letivo e foi
descansar em sua cidade, perto de sua mãe e longe das reflexões da
capital; a profetisa Pelágia se ocupou da realização da promessa que
havia feito a Saviéli, que queria usar seu sangue virginal parar criar
homúnculos filosóficos; o Anticristo, esperando as ordens do Senhor,
continuou a trabalhar como zelador no JEK; e Vássia, a semente
rejeitada, jazia no cemitério numa sepultura abarrotada de flores que
eram trazidas por inúmeros admiradores que inesperadamente surgiram.

Nesse meio-tempo, a profetisa Pelágia foi tirar sangue na policlínica
local para que fosse analisado no laboratório. Saviéli recuperou a
proveta com o sangue dela, claro que de forma ilegal, por meio de uma
enfermeira que trabalhava no laboratório e tinha o costume de beber.
Também de forma ilegal, ele recuperou a proveta com seu próprio sangue,
pois queria, nem que fosse num matraz, misturá-lo ao sangue de sua
amada. Ele poderia, então, começar a segunda experiência --- Saviéli
escondeu da profetisa que já havia realizado a primeira, que fracassara:
por meio de uma enfermeira permissiva do laboratório, ele conseguira
sangue de homens e mulheres desconhecidos, misturara-o na proporção
necessária, acrescentara nele o puro orvalho de maio, colhido na aurora
que caíra sobre o bulevar Tverskoi, fechara a mistura com uma rolha
opaca e a colocara num lugar quente para putrefação. No entanto, depois
de ele filtrar a película formada na parte de cima --- o mênstruo --- e
a transferir para um matraz limpo, a bolha que deveria testemunhar a
concepção de uma vida filosófica artificial não se formou. Embora
Saviéli, por um lado, tenha ficado frustrado, por outro se alegrou. Não
por ter decidido acabar com essa história estéril e pecaminosa, mas
porque já tinha dúvidas quanto à atitude arriscada de usar sangue de
pessoas desconhecidas na experiência. De fato, haviam dito: ``Se o
sangue usado para preparar o \emph{Otzer}, do qual sairão um homem e uma
mulher, for tirado de pessoas não castas, o homem sairá metade animal e
as partes inferiores da mulher terão um aspecto horrendo''.

Agora ele fazia a experiência pela segunda vez, trancado em seu quarto,
pois, no quarto de sua mãe, Ilováiski, conhecedor da Antiguidade,
discutia em voz alta sobre Cristo ora com um amigo, ora com outro. Pouco
tempo antes, Ilováiski havia se mudado para o apartamento deles, virando
padrasto de Saviéli, e agora discutia sobre Cristo vestido sem
cerimônia.

O classicista Ilováiski se tornara um chefe em casa e, durante uma
discussão, andava de um lado para outro com passos miúdos, pisando no
parquete encerado por Klávdia com seus pés de velho reflexivo, brancos
com manchas vermelhas. Seus artelhos não eram igualmente torturados por
calos, porém nenhum deles tinha um aspecto saudável. Ele vestia calças
curtas e largas de uma cor indefinida e uma camiseta verde-clara sem
mangas, cujas alças largas ficavam caindo de seus ombros brancos e
ossudos enquanto gesticulava. As aberturas da camiseta eram tão grandes
que mostravam seus flancos e suas costelas magras e, na frente, a blusa
encurtava, pois, no corpo magro de Ilováiski, pronunciava-se uma barriga
redonda.

--- Aqui está o cálice --- gritava ele pegando uma xícara com cheiro de
vodca do serviço de chá comprado na época de Ívolguin e do qual só
restara a metade. --- E agora vou jogá-lo no chão e ele se tornará um
cálice complexo...

Saviéli pegava uma garrafa térmica com chá, sanduíches de queijo e
embutidos e passava o dia inteiro trancado em seu quarto, saindo apenas
parair ao toalete. Nem sua tola mãe, nem mesmo o indelicado Ilováiski o
perturbavam. No entanto, uma noite, bateram repentinamente na sua porta.

Foi uma noite difícil e agitada. A experiência aproximava-se do ponto em
que a primeira fracassara. O sangue já havia sido misturado na proporção
necessária --- duas partes do sangue de Saviéli e três partes do de
Ruthina ---, já havia sido reservado num local quente, fechado por uma
rolha opaca, e também havia sido diluído pelo orvalho, mas, de fato, não
pelo orvalho do mês de maio, o que o preocupava; terra vermelha se
fomara no fundo, separara-se o mênstruo, que fora filtrado e colocado
num matraz limpo; uma parte de tintura vinda do reino animal, um ovo
cru, fora inserida em outro matraz, no entanto a bolha-embrião ainda não
havia aparecido.

Quando bateram na porta, Saviéli estava sentado com as mãos na cabeça e
sua nuca formigava. Ele queria gritar de raiva, xingar sua mãe quando,
de repente, ouviu a voz de sua amada Ruthina, cujo sangue participava da
experiência com o dele. Seu coração começou a palpitar, a respiração
ficou acelerada. Saviéli destrancou a porta.

--- Como está abafado aqui --- disse ao entrar a bela Ruthina de olhos
azuis ---, a janela está fechada... --- e abriu a janela.

O calor noturno de julho entrou arejando suavemente, como um pássaro, o
quarto sufocante de Saviéli, parecendo sussurrar algo inaudível em seu
ouvido... No centro da pétrea e estéril Moscou, sentiu-se de repente um
odor de maçã, não da maçã podre vendida nos tabuleiros, mas perfume de
maçã viva, irrigada pela chuva da noite. Era o perfume da vida. E,
quando o perfume da vida coincide com o olhar da mulher amada, nasce a
loucura sem a qual não há frutificação. A loucura ergueu o ânimo de
Saviéli, atormentado desde a infância por seu pecado de menino solitário
e tímido, e ele foi de braços abertos ao encontro da mulher que amava.
No entanto, no quarto apertado, entulhado de matrazes e de provetas, ele
tropeçou num objeto --- que depois não conseguiu localizar --- e caiu,
batendo fortemente o joelho. Ruthina deu uma risadinha, passou sua mão
delicada pelos cabelos dele --- fazendo-о arrepiar como uma galinha
exposta ao vento --- e saiu. Agoniado, Saviéli deitou-se sem trocar de
roupa e adormeceu sem fechar a janela. Acordou de súbito, como se
tivesse ouvido um tiro. Era o classicista Ilováiski que batia na porta.
Depois de brigar com Klávdia, Ilováiski, bêbado e com ares voltairianos,
saíra vagando pela cidade. Entrando no metrô, ele se largou num assento.
Assim que o trem de pôs em marcha, Ilováiski começou a virar a cabeça
grisalha furiosamente, mas de maneira involuntária, ora para a esquerda,
ora para a direita, apertando na mão os óculos sem estojo e assustando,
com seu rosto amarelo, os passageiros pacíficos ao redor. Descendo na
parada final, ele passou cambaleando por entre a multidão, mas não
chegou até a saída; fitou feito um carneiro, arregalando os olhos de
cima a baixo, as mulheres sentadas num dos bancos e acomodou-se ele
mesmo ao lado, numa beirada livre, apoiando na mão a cabeça pendente,
prestes a cair do pescoço.

Saviéli, que só despertara por um momento, caiu de novo num sono
profundo. E teve um sonho que foi, no começo, terrivelmente cômico e
depois simplesmente terrível. No início do sonho, ele andava por uma rua
e viu sobre uma cerca uma inscrição feita com giz: ``Eu irei
esfaqueá-lo''. Ele virou a esquina e viu de novo uma inscrição:
``Acredite, será esfaqueado''. Em seguida, teve outro sonho: sentia
náuseas devido a uma substância parecida com algodão, e partículas dessa
substância pairavam ao redor dele. Fazendo esforço para acordar, como
faz uma pessoa se afogando para vir à tona, Saviéli realmente sentiu uma
náusea subindo de seu estômago. Acendeu a lâmpada de cabeceira,
levantou-se e foi ver depressa o matraz em que se achava o mênstruo e o
matraz em que colocara uma partícula de ovo de galinha borrifada com o
mênstruo produzido de sangue e orvalho. A bolha-embrião subira até a
superfície, porém não era mais uma simples bolha, mas algo que se
desenvolvera durante a noite, com veias finas. Então, com as mãos
trêmulas e os dedos enrijecidos, morrendo de medo de deixar cair o
matraz com o mênstruo, Saviéli retirou a rolha e verteu no embrião, em
outro matraz, um pouco de mênstruo, que havia sido aquecido com uma
espiriteira.

Desde então, para Saviéli, a vida não tinha nenhum interesse além da
experiência. Obedecendo rigidamente à prescrição, esforçava-se por não
mexer no matraz bem fechado. Sem sair de casa, pálido e mais gordo pela
vida sedentária, ele acompanhava o desenvolvimento da matéria
concentrada no matraz, e o embrião ficava cada vez maior. Durante um
mês, ele verteu mênstruo quatro vezes, aumentando a dose pouco a pouco.
Eis que aconteceu o que fora previsto no livro de alquimia. ``Após esse
período, quando ouvir algo chiando e assobiando, aproxime-se do matraz
e, para sua grande alegria e surpresa, verá duas criaturas vivas. E, se
forem de sangue casto, você se alegrará com elas e as contemplará com
uma felicidade sincera. Mas elas não terão mais do que um palmo de
altura, no entanto se mexerão, movendo-se de um lado para outro dentro
do matraz. Entre elas, crescerá uma pequena árvore repleta de frutos.''

E assim tudo se deu. Saviéli vertia o mênstruo através de um tubinho com
um elástico de borracha, pois sabia que o ar que o homem comum respira é
prejudicial ao homem e à mulher minúsculos que viviam no matraz. Em
volta deles, cresceram muitas ervas e árvores que lhes serviam de
alimento, e eles se relacionavam com Saviéli com temor e respeito.
Saviéli decidiu se aproveitar desse temor e tirar desses homúnculos
filosóficos o que desejava saber. Ele peguntou:

--- Quais são as maiores ideias do mundo?

O homem filosófico respondeu, enquanto a mulher filosófica, sentada ao
lado dele no matraz, acariciava-o:

--- As principais são as ideias do Tempo e do Espaço. A ideia do Tempo é
religiosa e a do Espaço ateísta. A ideia do Espaço produziu a filosofia
e a ciência; a ideia do Tempo, a religião e a arte. No entanto, depois
as ideias se cruzaram. A ideia do Espaço é contemplativa e o homem é
capaz de alcançar com ela a ilusão de sua igualdade com Deus. A ideia do
Tempo é enérgica e o homem sente nela sua fraqueza diante do Futuro, sua
dependência do Futuro, e necessita da ajuda de Deus. O budismo e a
Antiguidade são ideias do Espaço, enquanto a Bíblia é a ideia do Tempo.
Quando o Cálice foi quebrado, o mundo cristão temporal tornou-se cada
vez mais espacial. Apoiando-se na ideia do Espaço, na ideia do presente
e da beleza, um gênio pode atingir a grandeza, mas o ponto máximo dessa
grandeza, ele só o atingirá com a ideia do Tempo, do Futuro.

Então Saviéli perguntou:

--- O que é o mundo filosófico e o que é o mundo religioso?

O homem respondeu do matraz:

--- O mundo filosófico é o mundo da União; o mundo religioso, o da
Polaridade. No mundo filosófico tudo se origina do Uno e retorna para o
Uno. Esse é o mundo do Homem-Deus. No mundo religioso, o essencial está
permanentemente separado por um abismo. Esse é o mundo de Deus. O Céu e
a Terra, Deus e o Homem, a Vida e a Morte... O que está de um lado do
abismo pode ser compreendido, o que está do outro lado só pode ser
conjecturado. Mas as ligações entre Deus e o Homem, o Céu e a Terra, a
Vida e a Morte são inacessíveis, tanto ao entendimento quanto à
conjectura. A mistura de ideias religiosas e filosóficas consiste num
método convencional, científico, fecundo no particular, mas que
obscurece a essência...

Então Saviéli perguntou:

--- Quais são os caminhos que levam a Deus?

O homúnculo filosófico respondeu de dentro do matraz:

--- Três caminhos levam a Deus: a Fé, a Descrença e a Dúvida. A Fé é o
caminho mais simples, mais difundido e mais frágil. É o caminho da
Igreja. A Descrença é o caminho mais perigoso, apesar de fecundo. É o
caminho dos gênios terrenos que, trilhando seu próprio caminho para
Deus, semeiam o ateísmo entre os fracos. O caminho da Dúvida é o dos
justos, o caminho de Jó. É o mais difícil e se dá através de um trabalho
espiritual cotidiano. Embora lento, é um caminho sólido.

Então Saviéli perguntou:

--- Como distinguir uma Boa ação de uma Má ação quando, no mundo, com
frequência o Mal traz a máscara do Bem e o Bem a máscara do Mal?

O homem respondeu de dentro do matraz:

--- Se o que você faz e ensina lhe é difícil, você faz e ensina o Bem.
Mas, se seus ensinamentos são facilmente assimilados e seus atos lhe são
fáceis, você ensina e faz o Mal.

Então Saviéli perguntou:

--- O que é a Verdade?

O homúnculo respondeu:

--- Não há apenas uma Verdade para o homem, mas também não há três.
Existem duas Verdades: a autêntica e seu reflexo no espelho. Não cabe ao
homem distinguir a Autêntica da Lendária, no entanto ele deve fazer uma
escolha: ao procurar pela Autêntica, não deve passar para a Lendária; ao
procurar pela Lendária, não deve passar para a Autêntica. Não abdique da
sua verdade nem procure por um terceiro caminho, pois ele não existe...

Nesse momento, a conversa do homem filosófico do matraz com Saviéli foi
interrompida, pois sua mãe o chamou para almoçar e ele, sentido uma fome
repentina, não pôde recusar. Ao sair do quarto, viu somente a mulher do
matraz aconchegar-se no homem, cansado depois de sua fala, e começar a
acariciá-lo.

Quando o crítico de arte Ívolguin, Aleksei Ióssifovitch estava vivo e
sua fotografia se achava sobre a escrivaninha (e não pendurada na parede
como agora), Klávdia não cozinhava bem. A bem da verdade, ela sabia
fritar um bom bife de carne bovina, mas fazia \emph{borsch} como no
exército, com repolho duro, e, como segundo prato, costumava preparar
salsichões ou almôndegas com macarrão. Já ela paparicava com pratos
saborosos seu novo marido, Ilováiski, que adorava, apesar de se
desentenderem por causa do gênio terrível dele. Ela dizia com a voz
tomada por lágrimas:

--- No campo de concentração, ele só comia peixe em conserva, passou
muita fome.

Ela cercava o marido de pratos variados, mas saía-se especialmente bem
no preparo de sua comida nacional, a bielorrussa. \emph{Schi} azedo com
cogumelos ou com trigo-sarraceno cozido; fígado à moda de Gómel;
rocambole de carne estufado recheado com toucinho; cebola e cheiro
verde, \emph{drotchiona}\footnote{Prato russo e bielorrusso. Espécie de
  fritada em que os ovos podem ser misturados com leite, batata ralada
  ou farinha. Além disso, pode-se acrescentar bacon e outros recheios.}
de batata com porco; batata cozida com carne de porco; farinha com
cheiro verde assada.

Dessa vez, a comida também estava boa, de modo que Saviéli, apesar do
esforço mental, comeu com apetite e sua dor de cabeça diminuiu um pouco.
No entanto, ele se lembrava de que não havia entendido todas as
explicações do homem no matraz, nem lhe perguntado tudo. Por isso,
Saviéli comeu apressadamente e, aprumando-se com um guardanapo após a
ingestão dessa comida pesada e gordurosa, retirou-se para seu quarto e
se trancou.

--- O que é um homem bom? --- perguntou ao homem no matraz.

--- Um homem bom não é um homem de Deus --- respondeu o homem ---, a
bondade não é divina, é um sentimento por demais insignificante para
Deus, mas necessário para o pequeno homem pecador. Muito mais necessária
do que a Verdade é a Riqueza espiritual. O Bem e a Bondade são coisas
diferentes. Um gênio não pode ser uma pessoa boa, pois ele serve a Deus,
assim como uma pessoa boa não pode ser um gênio, pois ela serve ao
homem. Um homem bom raramente traz bondade ao mundo, pois ele atrai
pessoas ruins, que se estragaram, se perderam, caprichosas, ávidas,
exigentes; e um homem bom não é um curandeiro, mas um cuidador de
doentes espiritualmente incuráveis. Um homem bom é um anônimo justo,
pronto para renunciar a si mesmo; por esse motivo um gênio e um profeta
não podem ser bons, pois, nesse caso, pecariam contra Deus, renegando o
elemento divino que lhes fora legado do alto em prol de homens
imperfeitos e efêmeros. Pois o bem que surgiu no mundo não veio de
homens bons, mas de profetas, que curam, e de gênios, que criam riquezas
espirituais. O que cura o mundo é a amargura da verdade, a clarividência
implacável do gênio, e não a bondade. A bondade não cura o mundo, mas o
consola e salva o pecador da solidão, ou seja, fortalece o mundo
decaído, não permite que ele se desfaça fisicamente, pois a bondade não
é sentimento espiritual, mas material. Ela geme ao lado do doente, tem
sede ao lado do sedento, tem fome ao lado do faminto, escuta as queixas
e os infortúnios de estranhos. Aproximam-se dela e, quanto mais ela
oferece, mais exigem, sem darem nada em troca nem agradecerem. O mundo
permanece mau, mas, graças à bondade, ele existe e não irá sucumbir à
sua própria maldade. Um verdadeiro cristão é um homem bom em qualquer
religião, mas um verdadeiro judeu, em qualquer religião, é um gênio e um
profeta. Analisem não importa que gênio, acharão nele algum princípio
judaico, ainda que ele renegue o judaísmo. O judaísmo está muito mais
próximo de Deus do que o cristianismo, que está mais próximo do homem.
Mas um homem bom, assim como um gênio, é um fenômeno raro, por isso há
tão poucos cristãos verdadeiros como judeus verdadeiros. Os homens, em
sua maioria, somente se chamam judeus ou cristãos, com frequência por
força do nascimento e, mais raramente, das circunstâncias. A principal
inverdade cristã está na afirmação de, que servindo ao homem, pode-se
servir a Deus. Coisa diferente é o fato de o Senhor, em virtude dos
pecados dos homens, aprovar esse caminho, embora ele esteja longe do
divino. O Senhor também mudou suas decisões várias vezes. Ele criou o
homem sem prever as consequências. Depois de criá-lo e de ver em que
resultou, resolveu aniquilar Sua criação. Em primeiro lugar, expulsou o
homem do Paraíso, do Éden, depois viu que isso só reforçou o pecado e
resolveu destruir totalmente a vida. Mas, após o primeiro justo, Noé ---
o primeiro Salvador que Deus não ousou destruir e graças ao qual salvou
o restante do mundo ---, o Senhor compreendeu que o homem não é capaz de
amá-Lo, e não por causa de um intento maldoso, mas por sua própria
insignificância. Só os gênios e os profetas são capazes disso. Então Ele
resolveu enviar o Messias, o Cristo, para mudar o ideal de Amor em prol
do pecador. Se não conseguem amar a Deus, que ao menos amem uns aos
outros. E sob esse ideal uma civilização foi erigida. O papel principal
não foi desempenhado por um gênio ou por um profeta, mas por um homem
bom que não estava a serviço de Deus. Cego e insensato, ele se deu
igualmente a todos, porém os homens maldosos souberam se aproveitar com
mais habilidade. Então, a bondade gera a maldade, pois ela não escuta o
que vem de Deus, mas seu coração cego. Os que mais necessitam da bondade
são os mais privados dela. O cristianismo construiu sua civilização
porque foi o que mais se afastou do divino e, graças ao ideal da
bondade, atraiu para si os mais tenazes, os mais fortes, famintos e
maus, ou seja, os que foram mais privados desse ideal. Somente os gênios
não participaram desse jogo. Um jogo baseado na mentira de que, servindo
ao homem, serve-se a Deus. Um homem que vive sob os Mandamentos de Deus,
que são extremamente simples, não precisa do cristianismo. Visto que o
pecador não é capaz de cumprir o ``não matarás'', ``não roubarás'' e
``não cometerás adultério'', ele se salva nas ambiguidades cristãs.
Entre a massa e o indivíduo há sempre um abismo. A massa vive conforme o
hábito e para ela o cristianismo é um bem. No entanto, isso é uma
tragédia para quem tenta ser um cristão consciente. De fato, mesmo aqui
os ardilosos acharam uma saída: ``Eu anseio, mas ainda não estou
pronto''. O cristianismo, no limiar do ateísmo, é um jogo mais hábil. O
judaísmo não é capaz de criar um jogo tão flexível, é sério demais. No
cristianismo, é possível acreditar com ardor, mesmo sendo um incrédulo,
e usufruir dessa vantagem, pois a fé cristã é extremamente dialética. A
luta e a busca pelo Eterno, pelo Imutável, em que não deveria haver nem
luta nem busca --- eis a dramaturgia da vida cristã. À primeira vista, o
cristianismo pode parecer uma doutrina idealista que não leva em
consideração a natureza do homem. O homem é mau, mas ele prega um bem
idealista. Na realidade, não é assim. Uma doutrina idealista é capaz de
criar uma religião ou uma cultura, mas não impérios poderosos nem
civilizações terrenas. O cristianismo se utilizou justamente, e com
bastante habilidade, da verdadeira natureza do homem. Pois a essência do
homem pode não estar baseada na maldade, mas está na imprudência. A
imprudência extrema é a base do sentimento cristão e corresponde ao
mundo decaído. É evidente que Cristo não era um cristão, sequer ouvira
falar desse termo em sua vida, mas ele compreendera o que a natureza
frívola do homem precisava. Não foi Cristo que construiu a civilização,
mas o cristianismo. Cristo mesmo era um homem profundo e espirituoso que
se relacionava com Deus. Ele se considerava um judeu e era de fato um
judeu da seita dos fariseus. Mas o grande mérito do cristianismo
consiste em, sem mudar nada em essência do mundo pagão que lhe era
odioso e ruim, ter criado a aparência de uma mudança completa. Por sua
vez, o social-ateísmo, em seu auge, aprendeu isso com o cristianismo,
conseguindo manter a ordem no mundo decaído, muito mudado na forma e
nada na essência. O judaísmo não poderia ser assim, pois era enorme a
ruptura entre ele e o paganismo, a idolatria, assim como sua rejeição
mútua. Deus é grande e o homem pecador, eis por que o judaísmo --- a
religião do gênio e do profeta --- preserva Deus para o homem, enquanto
o cristianismo --- a religião do homem anônimo, bom e insensato, do
mártir voluntário que se sacrifica em nome dos outros, de ingratos ---
salva o homem para Deus num mundo leviano e decaído. Se não o salva
espiritualmente, ao menos o salva materialmente. O mundo não apenas se
habituou à materialidade cristã como se afeiçoou a ela. Esse caráter
material não deve ser mudado, mas hoje a essência do cristianismo deve
ser compreendida e modificada. Sua essência consistiu, ao longo de
quinze séculos, na luta contra suas raízes bíblicas.

Saviéli percebeu que o homem no matraz estava esgotado, assim como ele
mesmo. No entanto, sabia que o homúnculo era submisso e o respeitava,
por isso continuou a questioná-lo.

--- Diga-me --- perguntou Saviéli, afundando-se pesadamente na cadeira e
fechando os olhos ---, por que eu não consigo crer em Deus
racionalmente, apesar de ter lido muitos livros inteligentes que
provavam a existência de Deus?

--- Porque --- respondeu o homenzinho de dentro do matraz, com voz baixa
e cansada --- Deus não está na inteligência, mas no instinto. O homem
nasceu com o instinto de Deus do mesmo modo que nasceu com o instinto de
comer, beber e se multiplicar. Mas esses instintos são simples,
concretos e acessíveis à verificação empírica e racional. A inteligência
do selvagem não era capaz de compreender os fenômenos físicos
científicos do céu e da terra, que, mesmo atingíveis pela razão, estavam
fora do alcance da experiência. Na mesma situação se encontra a
inteligência do homem civilizado em relação ao complexo instinto de
Deus, que também está fora do alcance da experiência. Imaginemos um caso
fantástico: se o desejo de beber não fosse amparado pela presença
acessível de um líquido, a existência da água seria para a razão um
problema análogo ao da existência de Deus. A sede obrigaria o homem a
procurar e a imaginar a água, mas a razão provaria com mais facilidade
sua ausência do que sua presença. Imaginemos ainda um homem que nunca
tenha visto uma mulher, um mundo sem mulheres, seu desejo e sua luxúria
o obrigariam a imaginar a mulher, no entanto sua razão com mais
facilidade rejeitaria a existência dela do que a provaria. O desejo
seria forte e provavelmente torturaria mais os sensatos do que os
insensatos, de modo que seriam escritos muitos livros inteligentes sobre
a existência da mulher. Quando a inteligência ficasse extenuada dessas
tentativas de encontrar a mulher através da análise, os mais honestos e
consequentes dos homens sensatos, com dois ou três livros claros e
inteligentes, provariam que é um absurdo condicionar a existência da
mulher à da luxúria ou a existência da água à da sede. E, se
considerarmos que a sede e a luxúria surgiram em tempos selvagens, elas
poderiam facilmente ser explicadas como resultado daqueles tempos
bárbaros, até hoje não de todo superados. ``Creio porque é um absurdo'',
exclamou, em desespero, Tertuliano, escritor do cristianismo primitivo.
Ele teve inteligência para reconhecer a impotência da razão para
compreender Deus, mas a inteligência não lhe foi suficiente para
rejeitar a razão, pois o absurdo é uma noção racional e científica. Só
um artista pode, como Moisés, ouvir Deus falar através de uma sarça
ardente. A razão exige uma prova racional, mas a única prova do instinto
é a necessidade. A necessidade de Deus é a única prova da presença de
Deus, assim como a sede seria a única prova da presença da água se ela
não existisse na terra, e a luxúria a única prova da presença da mulher,
caso Deus, tendo criado Adão, não tivesse criado Eva...

Após essas palavras, o silêncio invadiu o quarto, e de repente Saviéli
ouviu chiados e assobios, como no início da concepção dos homúnculos.
Assustado, ele abriu os olhos e viu que o homem e a mulher do matraz
estavam provando os frutos da primeira árvore que havia crescido e
florescido. E, na rolha do matraz, acumulou-se uma névoa que lembrava
uma nuvem. A nuvem se condensava a olhos vistos, até tornar-se vermelha
como sangue. Saviéli aqueceu rapidamente o mênstruo na espiriteira,
ainda que não fosse a hora de vertê-lo. No entanto, assim que verteu uma
grande dose do mênstruo que conservava a vida dos homens minúsculos no
matraz, uma chama ardente exalou da nuvem cor de sangue, e os dois
homúnculos começaram a se arrastar, tentando esconder-se do fogo.
Saviéli sentiu um aperto no coração. Diante de seus olhos, as cores do
matraz desbotaram, as ervas murcharam e as árvores secaram, como
acontece durante uma seca. A terra do matraz se abriu, chamas ardiam com
força, e os homúnculos, o homem e a mulher, caíram imóveis e foram
engolidos pela erupção. Aterrorizado, Saviéli começou a soluçar, mas já
não era seu coração que doía, mas sua alma, algo muito maior do que o
coração e situado no peito inteiro, do ventre até a garganta. Ele ouviu
sua mãe e Ilováiski baterem na porta, no entanto não a abriu; ele ficou
observando como no matraz se formaram quatro partes, uma sobre a outra.
A parte superior tinha um brilho tão intenso que ofuscava o olhar; no
meio havia uma parte cristalina; depois uma vermelha como sangue; e a
parte mais baixa era formada por uma fumaça negra que se reproduzia sem
cessar.

--- Saviéli --- gritava a mãe ---, abra a porta, menino, nós vamos
ajudá-lo.

No entanto, Saviéli sabia que não deveria destrancar a porta até que
tudo estivesse terminado.

--- Não vá fazer uma bobagem, meu velho --- ele ouviu a voz de Ilováiski
---, só se deve se fingir de louco para seu proveito.

--- Gavriil --- disse a mãe ---, vá buscar o zelador, vamos arrombar a
porta --- e ela se desfez em lágrimas.

Saviéli notou que atrás da porta havia muitas pessoas aglomeradas,
alguém aproximou alguma coisa dela, pressionou-a com o ombro, e ouvi-se
o ranger de um objeto metálico. Nesse momento, ocorreu uma grande
explosão, queimando Saviéli, e ele sentiu algo pontudo lacerando sua
bochecha e sua mão esquerda, pois o matraz estava do seu lado esquerdo.
Ele estava em pé, sentindo uma dor dilacerante, mas, quando sangue
começou a jorrar, como numa torrente, de sua bochecha e de sua mão, ele
caiu e perdeu a consciência. Mas, no instante em que perdia os sentidos,
ele compreendeu seu erro. A explosão fora consequência da falta de
solidez do matraz e de seu formato, que não fora escolhido
adequadamente: ele era alongado, quando teria sido necessário usar um
matraz redondo, como uma bola.

Quando no quarto irromperam Ilováiski, Klávdia, um serralheiro da
administração de moradias e a profetisa Pelágia, que estava substituindo
seu pai, o zelador, viram uma cena terrível: tudo estava envolvido em
uma fumaça tóxica, parte escura, parte amarela; o chão estava inundado
por uma solução oleosa e escorregadia, cujos respingos salpicaram a
mobília; estilhaços do matraz que explodira estalavam sob os pés; e do
matraz saía um tipo de massa que lembrava lodo e cheirava a charco.
Saviéli estava deitado no chão, em meio a esse caos, ferido pelos
estilhaços, ensanguentado.

É escusado falar do sofrimento de Klávdia, a mãe do insensato coberto de
feridas, ou da preocupação e perplexidade de todos os que presenciaram o
acontecido. Por sorte, os socorristas apareceram a tempo de prestar
ajuda a Saviéli. Ele foi transferido para o divã da sala, as feridas
foram tratadas e se verificaram inofensivas, embora sangrassem em
abundância. Saviéli abriu os olhos.

--- O que aconteceu com você, filhinho? --- perguntou Klávdia,
ajoelhando-se diante dele.

--- Mamãe --- disse Saviéli, baixinho ---, eu tenho a sensação de que
minha cabeça ficou tão pequena como a cabeça de um alfinete, e que
querem enfiar algo muito grande nela --- e ele pressionou a mão
enfaixada contra a testa.

Pouco tempo depois, levaram Saviéli. Assim que o levaram, a profetisa
Pelágia voltou a si, se ajoelhou e disse:

--- Eu pequei, Senhor, contra Seu escravo Saviéli... Como posso me
redimir?

E a profetisa Pelágia entendeu que ela não teria feito isso se Satanás
não estivesse ao seu lado. Mas Satanás se aproximava apenas de sua
femilinilidade, e a aparição dele agora não fora em vão. De súbito ela
compreendeu o motivo de Satanás ter aparecido e teve medo. Ela se
lembrou das filhas de Ló,\footnote{Ló, sobrinho de Abraão, foi poupado
  da destruição de Sodoma, perto de onde morava, passando a viver numa
  caverna com suas filhas, que lhe deram filhos.} que, em nome da
continuidade de sua estirpe depois da destruição da pecadora Sodoma,
embebedaram seu pai e se deitaram com ele, perpetuando a linhagem dos
moabitas. Ela também se lembrou dа grande moabita Tamar, que,
disfarçando-se de meretriz, deitou-se com seu sogro, Judá, dando
continuidade à tribo de Judá e fundando a Casa de Davi, de onde
descendiam o sábio Salomão e Cristo, o Messias. Assim, com a ajuda de
Satanás, Pelágia teve o sinal de realizar sua Ideia através da
violência, pois no amor de uma filha por um pai há ternura, mas na
paixão de uma mulher por um homem há crueldade, e o Senhor não pode ser
cruel.

Eis que seu pai, Dã, a Áspide, o Anticristo, voltou para casa, e eles se
sentaram para jantar. A exemplo das filhas de Ló, a profetisa Pelágia
havia preparado uma garrafa de aguardente caseiro, envelhecido com ervas
da floresta, a qual a \emph{velha crente} Tchesnokova, da cidade de Bor,
na região de Górki, lhe enviara através de Vera. Ela tinha pensado em
guardar essa garrafa para a festa feliz de
\emph{Simchat-Torá},\footnote{A festa de \emph{Simchat-Torá,} celebrada
  depois de \emph{Sukkot} (outono), marca o fim do ciclo anual da
  leitura da Torá e o início de um novo.} a alegria da leitura da Torá,
mas compreendeu que o momento havia chegado: ela deveria realizar a
Ideia. E Satanás já havia aparecido parcialmente. Satanás tem o hábito
de aparecer aos poucos; é como se espiássemos pelo vão de uma porta que
se abre paulatinamente. Primeiro, aparecem os cascos, depois a eles se
junta o corpo hirsuto, então surge o rosto de bode com o ar de sabedoria
do astuto pessimista.

A aguardente da velha, envelhecida com ervas da floresta, era boa. O
Anticristo a tomou, assim como Ló de Sodoma, e viu o corpo preservado e
robusto, repleto de feminilidade, de sua filha querida. Seus braços eram
redondos, os ombros largos, mas não de um jeito masculino --- ela não
tinha a ossatura máscula de um homem, mas a força femínea e tenaz de uma
mulher fecunda. O Anticristo sabia que ela era uma mulher de físico
vigoroso, como as belas camponesas do Norte. Para uma moça, ela já não
era tão jovem, e, como um pai querido que amava uma filha em quem
confiava plenamente, ele sabia que ela ainda não tinha sido tocada. Há
velhas virgens que, por não frutificarem, secam. Mas Pelágia não secou,
o que foi um milagre de uma floração longa, assim como é o milagre de
uma vida longa. Mas até pessoas longevas morrem, e todo milagre tem um
limite. Através de Satanás, um pessimista astuto, o Anticristo entendeu
que ele, o pai, estava fadado a dar um limite à floração infecunda de
sua filha. Ela não era sua filha de sangue, mas sua filha de alma ---
ele a pegara ainda bebê das mãos de sua mãe, que estava prestes a
morrer, e a criara, e agora ele mesmo tinha o dever de realizar o que
era inconcebível sem a ajuda de Satanás. Ele não via Satanás, apenas
sentia o cheiro acre de um corpo úmido e tépido, o cheiro maturado,
desagradável e intenso do arenque, o cheiro de algo que esteve sempre
escondido bem fundo, decompondo no calor, mas que agora se revelava...
Era o cheiro da sedução de Satanás, que já havia mostrado uma parte de
seu corpo, pois Satanás aparece aos poucos, paulatinamente, para
preparar o terreno e habituá-lo à sua presença.

Dã, a Áspide, o Anticristo, compreendeu que era um caminho sem volta,
que atrás havia apenas a Maldição, entendeu que arruinaria seu sonho se,
nesse instante, abraçasse Ruthina de forma afetuosa e paternal, em vez
de agarrá-la com força, como um homem pronto para seu ato. E se, ao
agarrá-la, ele hesitasse em derrubá-la, arruinaria definitivamente sua
esperança. Mas Dã, a Áspide, era astuto e decidiu deixar que sua filha
lhe desse as costas para, então, agarrá-la. Quando ela se virou para o
bufê, criou-se o momento perfeito, mas ele hesitou e tomou-a num
instante inesperado para si mesmo. Ela fora buscar alguma coisa do outro
lado do quarto e, como as camas ficavam longe dali --- a dela, de moça,
ficava atrás de um biombo, e a dele era dobrável ---, ele a derrubou no
chão mesmo. No entanto, em seguida aconteceu algo que ele realmente não
esperava. Ele pensava que ela iria resistir com as mãos e os joelhos,
mas ela cravou os dentes na mão dele, mordendo-a não como uma pessoa o
faria, mas como um animal selvagem, de forma irracional, para perfurar,
desprezando a dor de sua vítima. O Anticristo gemeu de dor e de
surpresa, e sua mão entorpeceu instantaneamente, até o antebraço, e ele
só conseguia pensar em salvá-la. Mas, quando ele estava a ponto de
desistir, Satanás ajudou-o a distrair-se da dor terrível que sentia e a
entender que a resistência de Ruthina se limitaria a isso, pois ela não
poderia simplesmente se entregar ao seu querido pai. No entanto, seus
joelhos fortes, a principal defesa de uma mulher, ficaram imóveis e
resignados. Então, ajudado pela mão livre de mordidas, o Anticristo fez
tudo o que queria e sonhava.

Assim tudo se deu, e chegou o momento em que Satanás se mostrou
inteiramente, parte por parte, e o prazer cruel da queda transpassou
como uma onda seus corpos, com a esperança de que seus corações parassem
ao mesmo tempo e de que ambos morressem em felicidade. Porém, por mais
que os dois se esforçassem para permanecer nesse prazer fatal, a mesma
força que os fazia mergulhar na inumanidade os trouxe de volta, de volta
à vida, à dor e ao medo da morte, e seus corações deram uma guinada
brusca, superando o deleite do Sono Eterno...

Alguma coisa cintilou no quarto ainda escuro; era a face de Satanás que
desaparecia, bela e triste, e não com a expressão raivosa ou sarcástica
que surge em seu rosto durantes as tentações, quando o homem luta contra
ele.

O relógio deu duas horas da madrugada. Os dois sentiam muita sede, como
doentes; a boca, quase sem saliva, estava viscosa. Ruthina se levantou
da cama e, sem acender a luz, farfalhou na escuridão com a roupa que
fora amarrotada e até rasgada em alguns lugares pelo Anticristo. Ela se
despiu e se deitou. O Anticristo também tirou a camisa e se deitou ao
lado.

--- O que vai acontecer agora? --- ele perguntou, aflito.

--- Não diga nada, pai --- disse Ruthina, pois, mesmo se tornando sua
mulher, continuava a chamá-lo de pai.

O Anticristo obedeceu à filha que violara, pois não havia outro caminho
para alcançar a Ideia. Eles estavam deitados, mas a noite, como de
hábito, vivia e trabalhava, buscava seu fim. No início, a noite o
buscava de modo invisível, imperceptível, sem nenhuma mudança, depois o
fazia empalidecendo, embranquecendo, movendo-se devagar.

--- O que vai acontecer agora? --- voltou a dizer o Anticristo quando
surgiu um reflexo avermelhado e nervoso no céu, estranho à tranquilidade
noturna. Já não era a noite, mas a aurora.

--- Não diga nada, pai --- voltou a ordenar sua filha, que se tornara
sua mulher.

Agora eles estavam deitados em meio ao labor frenético e apressado das
forças matinais que limpavam o céu e a terra sob o pipilar renovado dos
pássaros. Quando tudo se tornara claro e radiante, e nada se ocultava da
luz, ele perguntou pela terceira vez:

--- O que vai acontecer agora?

Ela não respondeu. Ela dormia, e o frescor e a pureza da manhã
iluminavam seu rosto belo e bondoso. Apenas nesse instante o Anticristo
soube do Senhor que sua filha Ruthina era na realidade a profetisa
Pelágia, da vila de Brussiány, da região de Rjév.

Da mesma forma que o Senhor entregara às mãos de Satanás o justo Jó para
que, passando por sofrimentos, sua fé se fortalecesse, pai e filha, para
o bem de Deus, foram entregues às mãos de Satanás, participante
permanente e indispensável da dramaturgia trágica do Senhor. Dã, a
Áspide, o Anticristo, se lembrou do profeta Isaías: ``Então Isaías
disse: `Escutai agora, Casa de Davi! Será que vós não embaraçastes os
homens o suficiente, e ainda quereis embaraçar meu Deus? Pois o Senhor
mesmo vos dará um sinal: a Virgem conceberá e dará à luz um Filho que
receberá o nome de Emanuel. Ele se alimentará de leite e de mel, até que
aprenda a rejeitar o mal e a escolher o bem'''.\footnote{Isaías
  7:13\emph{--}15} E Isaías disse adiante: ``E eu me aproximei da
profetisa, e ela concebeu e deu à luz um filho''.\footnote{Isaías 8:3}
Sendo um judeu instruído --- assim como seu Irmão ---, o Anticristo
sabia que esse não era ainda o Filho, mas um filho-sinal. Pois, sem um
sinal, nada de divino pode acontecer. Depois da Casa de Davi, foi
concedido à Casa de Dã cobrir-se de glória, que se anunciou: ``Um menino
nasceu para nós, um Filho nos foi dado''.\footnote{Isaías 9:5.}

Logo que o Anticristo compreendeu isso e tudo se realizou, sentiu
saudade de seu passado e de sua terra. Assim ele se afligira no início,
quando, como um adolescente judeu, quase um garoto, aparecera, em 1933,
trazendo o segundo flagelo do Senhor --- a fome ---, na região de
Khárkov, distrito de Dimítrov, vila de Chagaro-Petróvskoie. Naquela
época, ele sussurrava com frequência o juramento secular, a maldição de
sua Cidade Santa: ``Se eu me esquecer de Ti, que minha língua se grude
na goela''.\footnote{Salmos 137:5, 6.}

A verdadeira pátria do homem não é a terra em que vive, mas a nação a
que pertence. Não existe terra russa, judia, inglesa, turca, ou seja
qual for. Toda a terra é do Senhor, e o Senhor é seu único habitante
originário. E o direito legítimo a um ou a outro pedaço da terra do
Senhor não é dado por conquistas ou deslocamentos históricos, nem por um
domínio secular, mas é dado a uma nação que tornou essa porção de terra
do Senhor frutífera e estabeleceu nela uma ordem justa, ou, à semelhança
do Pliúchkin\footnote{Personagem de \emph{Almas Mortas,} Stepán
  Pliúchkin era um ``\emph{pomiêchtchik,} um proprietário rural
  {[}...{]} dono de mais de mil almas''. (\emph{Almas Mortas,} ed.
  Abril, 1987, p. 137, tradução de Tatiana Belinky)} de Gógol, a uma
nação que ocupou avidamente os vastos espaços do Senhor que lhe caíram
nas mãos. De uma nação como essa o Senhor cobrará Seu Bem com crueldade.
Mas a nação que o preservar será por Ele recompensada.

E então o Anticristo avistou a Cidade, que não estava coberta de flores,
mas renascia depois dos quatro flagelos enviados pelo Senhor. Assim ela
ficou depois do jugo babilônico, conforme o Livro de Neemias, pois, ao
renascer, não houve um único Moisés, como se dera após o jugo egípcio,
mas houve Neemias, que retirou seu povo da Babilônia, e Esdras, que lhe
ensinou a Lei.

``E se levantou Eliasib, o sumo sacerdote, com seus irmãos, os
sacerdotes, e juntos construíram a porta das Ovelhas. Eles a consagraram
e colocaram as vigas, e a consagraram da torre de Meá até a torre de
Hananeel. E, ao lado deles, os homens de Jericó construíram; e, ao lado
deles, Zacur, filho de Imri. E os filhos de Asená construíram a porta
dos Peixes; colocaram as vigas e fixaram os batentes, as fechaduras e as
trancas. Ao lado deles, fez os reparos Meremot, filho de Urias, filho de
Acus; ao lado dele, Masolam, filho de Baraquias, filho de Mesezebel; ao
lado dele, Sadoc, filho de Baana. E, ao lado deles, fizeram os reparos
os homens de Técua, porém seus nobres não baixaram a nuca ao serviço do
Senhor. E a porta Velha foi repararada por Joiada, filho de Fasea, e
Mesolam, filho de Besodias; colocaram as vigas e fixaram os batentes, as
fechaduras e as trancas {[}...{]}''\footnote{Neemias 3:1\emph{--}6.}

Assim, com a obstinação de formigas, reconstruíam o Eterno com suas mãos
frágeis humanas.

``Melquias, filho de Herem, e Hasub, filho de Faat-Moab, repararam o
segundo setor, assim como a torre dos Fornos {[}...{]} A porta do Vale
foi reparada por Hanun e os habitantes de Zanoa {[}...{]} e ainda
restauraram mil côvados do muro, até a porta do Esterco. E a porta do
Esterco foi reparada por Malquias {[}...{]} E a porta da Fonte restaurou
Salum {[}...{]} Ele mesmo fez os reparos no muro do reservatório de
Siloé, em frente ao jardim do Rei, até a escadaria que desce da Cidade
de Davi. Ao lado dele, restaurou Neemias, filho de Azboc {[}...{]} até
os túmulos de Davi, até a cisterna construída e até a Casa dos Valentes
{[}...{]} Ao lado dele, fez os reparos Azer, filho de Jesua {[}...{]} em
frente à subida do Arsenal, na Esquina {[}...{]} Ao lado dele, Falel
{[}...{]} diante da Esquina e da torre que sobressai na casa do Rei, que
fica no alto, perto do pátio do cárcere {[}...{]} Fadaías {[}...{]} até
a porta das Águas {[}...{]} O sacerdotes fizeram os reparos a partir da
Porta dos Cavalos {[}...{]}.''\footnote{Neemias 3:11, 13, 15
  \emph{--}17, 19, 25, 26, 28.}

No entanto, em um mundo decaído, ao lado dos Construtores, sempre há os
Destruidores, e eles também devem ser compreendidos. Hoje em dia, um
liberal e humanista sempre compreenderá melhor a grande verdade do
Destruidor do que a fina verdade do Construtor. Não é à toa que, desde o
fim do século XIX, as palavras de ouro do liberal sempre precedem a faca
de um assassino. De fato, enquanto o Construtor trabalha egoisticamente
por si mesmo, o Destruidor se esforça abnegadamente para todos. O
Destruidor sempre parece desprendido, embora viva na abundância. Ele
sempre provoca piedade, sempre perde alguma coisa. Pois, no mundo
decaído, não achar significa perder.

Os Destruidores que viviam nas redondezas de grandes espaços disseram
com suas habituais queixas, invariáveis desde os tempos babilônicos:

--- Acabarão a obra algum dia? Farão reviver dos escombros cheios de pó
as pedras queimadas?\footnote{Neemias 3: 34 pela \emph{Bíblia de
  Jerusalém}, 4:2 em outras versões consultadas.}

No entanto, o experiente Construtor sabe o que deve esperar dos
sofrimentos do Destruidor e quanto é difícil, para o Destruidor, ver os
bens dos outros. Naquele tempo, os Destruidores eram Sanabalat, o
horonita, e Tobias, o amonita, que viviam em vastos espaços, recebidos
gratuitamente após a invasão babilônica.\footnote{Neemias 2:10
  ``Sanabalat é conhecido como governador de Samaria. Tobias era, sob
  suas ordens, judeu governador de Amon.'' (\emph{Bíblia de Jerusalém,}
  ed. Paulus, 2016, p. 642)} Eis as palavras de Neemias, filho de
Hacalias, antigo copeiro do rei persa Artaxerxes e quem liderou os que
participavam da construção:

--- Nós construíamos o muro, e do muro se ergueu a metade; porque o povo
trabalhava com afinco {[}...{]} Nossos inimigos, porém, disseram: ``Eles
não saberão nem verão nada até aparecermos entre eles e os matarmos,
então faremos cessar a obra. {[}...{]} Olhei ao redor, levantei-me e
disse aos nobres, aos magistrados e ao resto do povo: ``Não tenhais
medo, devei vos lembrar do Senhor, grande e terrível, e combater por
vossos irmãos, vossos filhos e vossas filhas, vossas mulheres e vossas
casas {[}...{]} Desde esse dia, metade dos meus homens trabalhava na
obra e a outra metade empunhava lanças, escudos, arcos e couraças
{[}...{]} Os que edificavam o muro e os que carregavam as cargas, com
uma mão, cuidavam da obra e, com a outra, seguravam a lança {[}...{]}
Mas nem eu, nem meus irmãos, nem meus homens, nem os guardas que me
acompanhavam tiramos nossas vestes; cada um mantinha a espada e água ao
alcance da mão.\footnote{Neemias 3: 38; 4: 5, 8, 10, 11, 12, 17 pela
  \emph{Bíblia de Jerusalém}; em outras versões, Neemias 4 6, 11, 14,
  16, 17, 18.}

O Anticristo se lembrava de tudo isso com frequência e caía em
meditação. O amor filial da profetisa Pelágia não diminuiu desde o dia
em que seu pai, através de Satanás, tornara-se também seu marido. Mas,
durante a noite, aparecia também a paixão por seu marido. Assim, Pelágia
engravidou e, conforme seus cálculos, teria o bebê no início da
primavera, na época de \emph{Purim}, uma festa alegre. Desde que
engravidou, ela passou a estar sempre em companhia de seu pai, pois
sabia que ele não estaria com ela para sempre. Seu pai cuidava bem dela
e, sabendo que uma grávida precisa do ar puro do campo, saía
frequentemente com ela da cidade, indo para as florestas outonais dos
subúrbios, pois o outono já havia chegado.

Certa vez, eles foram a uma região pouco habitada, cheia de ravinas e
com uma colina coberta pela floresta. Com eles estava Andrei Kopóssov,
que também já sabia quem eram na realidade seu pai e sua irmã, a qual se
tornou para Andrei uma espécie de mãe adotiva e esperava um filho de seu
pai. Enquanto subiam a colina, Dã, a Áspide, o Anticristo, o enviado do
Senhor, disse através de Mateus, o evangelista mais confiável, usando as
palavras de seu Irmão da tribo de Judá:

--- Não penseis que eu vim abolir a Lei ou os profetas; eu não vim para
destruí-los, mas para realizá-los, pois, em verdade, eu vos digo:
enquanto houver o céu e a terra, nem um iota, nem uma linha se ocultarão
da Lei, até que tudo seja cumprido.\footnote{Mateus 5:17, 18.}

Então Andrei Kopóssov, que ia a Deus pelo caminho mais difícil, pelo
terceiro caminho --- não pela Fé ou a Descrença, mas pela Dúvida, como o
profeta Jó ---, perguntou, abrindo o pequeno Evangelho de bolso do qual
não se separava:

--- Pai, por que seu Irmão Jesus, da tribo de Judá, diz claramente nos
versículos 17 e 18 do capítulo 5 do Evangelho de Mateus que veio para
executar a Lei de Moisés, mas, a partir do versículo 21, começa a falar
coisa diferente e nos versículos 38 e 39 diz: ``Vós ouvistes o que foi
dito: olho por olho, dente por dente, mas eu vos digo: não resistais ao
mau; se alguém ferir-te a face direita, estenda-lhe também a outra''. E
nos versículos 43 e 44 diz: ``Vós ouvistes o que foi dito: amarás o teu
próximo e odiarás o teu inimigo. Mas eu vos digo: amai os vossos
inimigos, fazei o bem aos que vos odeiam e orai pelos que vos ofendem e
vos perseguem''.

Dã, a Áspide, o Anticristo, o enviado do Senhor, respondeu:

--- Tudo foi dito corretamente e não há aqui nenhuma contradição. Como
um judeu crente, como um gênio que se dirige a Deus, ele preserva e
executa a Lei divina, para guardar Deus ao homem; é do que se fala nos
versículos 17 e 18. Essa é a primeira Verdade, a de Moisés. Mas, como
sábio, Salvador e Messias, Ele sabe que o pecador neste mundo decaído
não é capaz de amar a Deus conforme os Mandamentos da Lei de Moisés, não
é capaz de cumprir os mais simples preceitos divinos: ``não matarás'',
``não roubarás'', ``não cometarás adultério''. Os pecadores não serão
convencidos nem mesmo pelos profetas de Deus, cujas vozes clamam no
deserto. Por isso, para a salvação do mundo decaído, Ele não se apoiou
na Lei divina dos profetas, estranha ao mundo, mas nos preceitos do
homem bom, compreensíveis a qualquer pecador que vive de sua abnegação e
de seu sacrifício, assim como o verme vive da maçã. Dessa maneira, é
através do humano, e não do divino, que o mundo decaído se salva para
Deus. É justamente disso que fala meu Irmão Jesus, da tribo de Judá.
Esses preceitos são para poucos, mas salvam muitos. Essa é a segunda
Verdade. E uma terceira não pode existir... E meu irmão Jesus disse,
concluindo seu Sermão sobre a Montanha: ``Vós deveis ser perfeitos como
vosso Pai celestial...''.\footnote{Mateus 5:48.} São palavras de quem
compreendeu Deus, ditas àqueles que, em virtude de seus pecados, são
incapazes de compreendê-Lo e devem se salvar por meio de outra
perfeição, a humana, pois a bondade é também uma perfeição.

A essa altura, a profetisa Pelágia, filha adotiva e mulher de Dã, o
Anticristo, perguntou:

--- Pai, então a quem seu Irmão Jesus Cristo trouxe a salvação: aos
perseguidos ou aos perseguidores, aos odiados ou aos que odeiam?

Dã, o Anticristo, respondeu:

--- Claro que Cristo trouxe a salvação aos perseguidores e aos que
odeiam, pois seus tormentos são terríveis. São terríveis os sofrimentos
de um perseguidor miserável.

--- Pai --- replicou a profetisa Pelágia ---, mas como podem se salvar
os perseguidos e os odiados?

Dã, o Anticristo respondeu:

--- Para os perseguidores, Cristo é o Salvador; para os perseguidos, o
Anticristo. Foi para isso que fui enviado pelo Senhor. Vocês ouviram o
que foi dito: amem os seus inimigos, abençoem quem os amaldiçoa, façam o
bem a quem os odeia e orem por quem os ofende e os persegue. Mas eu lhes
digo: não amem seus inimigos, mas o ódio de seus inimigos; não abençoem
quem os amaldiçoa, mas as maldições que lhes foram lançadas; não orem
por quem os ofende e os persegue, mas por suas ofensas e perseguições.
Pois o ódio de seus inimigos é a Marca da bêncão Divina. Se o ódio é
secular e universal, se a paixão por esse ódio é sincera, se o homem que
odeia não odeia por vontade própria, mas como se algo dentro dele
odiasse, se às vezes nem a razão de quem odeia consegue dominar seu
ódio, se ao redor desse ódio se erigem ideologias e impérios --- o
Senhor, através desse ódio, envia um sinal a quem é odiado. O ódio de um
homem contra outro homem não é raro no mundo decaído e com frequência
este ódio é tão mesquinho como este mundo. Mas somente o povo do Senhor
mereceu a honra de ser odiado invariavelmente, por um ódio universal e
fecundo, por mais de dois mil anos e ao longo de mais de dez impérios
--- as Torres de Babel. Esse povo em nada se destaca dos outros, em nada
é melhor que os outros, mas por esse ódio invariável ele se destaca e
nisso ele é melhor.

Assim terminou de falar Dã, o Anticristo, o enviado pelo Senhor, que já
sabia que não ficaria aqui por muito tempo, pois os quatro flagelos
atuais do Senhor terminaram, mas quando novos sofrimentos serão
enviados, isso só o Senhor sabe. Claro que essas punições nunca
abandonam o mundo decaído, mas há períodos pecadores em que elas se
renovam e adquirem força incomum. Então o Anticristo pode aparecer de
novo ou não, pois isso depende do plano do Senhor. Desse modo, a
profetisa Pelágia sabia que teria que se separar de seu pai e marido por
longo tempo, se não para sempre. No entanto, ela não sabia quando e como
essa separação aconteceria, e rezava ao Senhor que ao menos se desse
após o nascimento do bebê. Com isso, eles passaram os dias até o Natal
com amor e inquietação.

Nesse ano, o Natal não foi dos mais frios, mas cheio de vento e de
preocupações. Dã, o Anticristo, comemorou modestamente o aniversário de
seu Irmão, apenas na companhia da profetisa Pelágia, sua filha e mulher.
Ele celebrou a data pensando no seu Irmão e conversando com sua filha,
que lhe daria um filho. Ele disse:

--- Todo homem nasce espiritualmente miserável, tolo e mau. Mas,
enquanto é um bebê insensato, vive no paraíso divino. Quando aparecem os
germes da consciência, ele é imediatamente expulso do paraíso, entregue
à própria sorte, então a miséria, a tolice e a maldade o espreitam. Como
retornar a Deus vivendo com sua consciência e entregue à própria sorte?
Contra a miséria há os gênios; contra a tolice, os profetas e os sábios;
e contra a maldade, pessoas boas, anônimas. Púchkin não é um gênio
terreno, nem Shakespeare, nem o divino Moisés, nem os profetas Jeremias
e Isaías. Só são terrenos aqueles que se entregam totalmente ao presente
e dos quais, no futuro, nada sobrará. Se um homem bom deixar rastros
atrás de si, se ele se tornar conhecido e for coberto de glórias --- ele
não ofereceu a verdadeira bondade, não realizou até o fim o que fora
idealizado, mesmo se disserem que ele era uma pessoa boa. Apenas um
justo anônimo, que nunca recebe nada em troca, realiza a bondade em
completude. Eis para que foi concebido meu Irmão da tribo de Judá, pois
Ele é o único consolo e a única recompensa dos justos anônimos que vivem
para a salvação dos perseguidores. Já eu vim para recompensar e salvar
os perseguidos.

Anticristo despertou no meio da noite, soergueu-se sobre seu cotovelo,
lembrou-se das palavras sobre si que o Senhor colocara em seus lábios,
sorriu, olhou para sua filha, que dormia ao lado, quente, volumosa, com
bonitas manchas amarelas de gravidez, olhou pela janela noturna e viu
brilharem as estrelas natalinas --- e uma delas brilhava mais que todas
---; ele olhou ao redor e se despediu suavemente do mundo de Deus e
também de sua filha, roçando-lhe cuidadosamente os lábios para não a
acordar. Depois disso, Dã, a Áspide, o Anticristo adormeceu; ele viveu
ainda por três horas e morreu dormindo ao amanhecer, esquecendo
instantaneamente tudo o que havia de terreno em si, da mesma forma que
às vezes nos esquecemos por completo de um sonho noturno no despertar da
manhã.

A profetisa Pelágia, mesmo depois de seu pai ter despertado do sonho
terreno, continuou a dormir ao lado do corpo já frio que pertencera a
ele. Ela sonhou com um funeral, assim como sonhara Ánnuchka Emeliánova,
a mártir ímpia, num chiqueiro alemão. No entanto, Ánnuchka Emeliánova
sonhara com o caixão de sua mãe, sob um temporal, no meio de um pátio
localizado na cidade de Rjév, 3º setor, barracão n\textsuperscript{o}
3...

Já no sonho da profetisa Pelágia o lugar não foi indicado com precisão,
apesar de ela ser quase conterrânea de Ánnuchka Emeliánova, pois a vila
de Brussiány ficava perto de Rjév. Além disso, não chovia diante da
profetisa Pelágia, mas fazia um dia ensolarado. Eis que passava uma
densa multidão de pessoas carregando quatro caixões. A multidão
embrenhou-se numa ponte estreita, mas comprida. Andou um pouco, deixando
um caixão na ponte, andou mais um pouco, deixando o segundo. Quando
atravessaram a ponte, colocaram o terceiro caixão na água e, um pouco
adiante, caminhando pela margem do rio, colocaram também o quarto. Só
que os caixões não foram levados pela corrente, mas se agitavam perto da
margem. De repente, do caixão que se achava mais perto da margem se
levantou uma moça forte e saudável e caiu, ficando com água até o
pescoço. Então, do caixão que estava um pouco mais distante um jovem se
ergueu, precipitou-se na água e andou em direção à moça; ele a conduziu
pela mão até a margem onde estava a multidão e, ao voltar, deitou-se de
novo em seu caixão, que começou lentamente a se afastar. Quanto à moça,
encharcada, assim que ela pisou na margem, começou a falar com uma voz
muito alta, como uma lunática, mas não na língua que havia falado até
morrer, ou seja, não em russo. E ela também havia mudado, seu rosto
escurecera, assim como seus cabelos; a forma redonda do corpo
desaparecera; e seus gestos tornaram-se rápidos, como das pessoas do
Sul. Os que estavam na margem a pegaram pelas mãos molhadas, com cuidado
e respeito, e levaram-na a um quarto. Lá a moça colocou roupas secas,
que deixaram seus joelhos visíveis, e pegou uma bolsinha branca, bordada
de miçangas. Seu monólogo, porém, continuava, apesar de já não parecer
tão desproprositado nem tão ruidoso. Ela falava numa língua
incompreensível, primitiva, selvagem, talvez das cavernas, que não
lembrava nenhuma outra. Mesmo assim, dessa torrente de palavras guturais
e ininteligíveis, surgia vez ou outra uma palavra russa habitual. No
entanto, a palavra solta não possibilitava nenhuma interpretação. Em
todo caso, as pessoas ouviam avidamente a moça, da mesma forma que
observavam seus gestos. Os que não conseguiam achar um lugar no quarto,
olhavam pelas janelas ou pelo vão da porta e também se aglomeravam na
entrada. Ouviram-na por horas, embora não a compreendessem. No início, a
profetisa Pelágia receou entrar no quarto, mas depois pensou: ``Que mal
ela poderia me fazer?'' --- e entrou, dizendo à morta:

--- Bom dia...

--- Bom dia, Pelágia --- respondeu em russo a morta, retomando sua fala
desconhecida, da qual ocasionalmente brotava uma palavra russa.

Entretanto, quanto mais as pessoas ouviam a moça morta, menos a
entendiam, embora a aprovassem com vontade.

--- Sim... Ora essa... Cada coisa... É preciso...

Ao redor, o público estava diferente, as pessoas já não pareciam de
luto, num enterro. Havia muitos jovens, com roupas multicoloridas e
rostos que não eram sombrios nem pensativos...

Assim, depois desse pesadelo, Pelágia acordou, sentindo a alma aliviada,
e viu pela janela a manhã gélida, ensolarada e alegre de Natal. Abraçou
o pai para acordá-lo e lhe contar o sonho terrível que tivera, no
entanto se afastou subitamente dele, sentindo aversão. Um instante antes
de sua consciência ser surpreendida por uma mágoa humana, a morte de um
ente querido, ela sentiu uma aversão verdadeiramente bíblica por um
corpo morto. Ela sabia que havia mais de seu pai em cada palavra que ele
lhe dissera e ela memorizara e mesmo em qualquer objeto que ele vira e
tocara do que nesse corpo inexpressivo que ele abandonara para sempre.
Não é à toa que, nos antigos tempos bíblicos, os homens de Nazaré,
dedicados a Deus, eram proibidos de tocar num cadáver. Enquanto o corpo
estivesse insepulto, não podia haver lembrança viva do morto. O corpo
devia ser entregue, sem demora, a terra, para que o ente querido pudesse
renascer.

Assim ela o fez, de maneira modesta e discreta, com a ajuda de seu
irmão, Andrei Kopóssov. Seus filhos, com o coração mortificado,
enterraram o pai num caixão modesto e barato, num cemitério acessível a
todos e de grande consumo. No entanto, enquanto voltavam do cemitério,
seus corações renasceram. Seu pai estava de novo com eles. Desde então,
raramente eles se separavam do pai ou um do outro, mas isso não se
tornou um peso para eles, pois não se cansavam de ficarem juntos.

Pelágia teve seu filho no início de março, exatamente na festa de
\emph{Chuchan Purim},\footnote{A festa de \emph{Purim} --- que comemora
  a salvação dos judeus persas da conspiração de Amã --- é celebrada nos
  14º (\emph{Purim}) e 15º (\emph{Chuchan Purim}) dias do mês de Adar, o
  12º (ou 13º) mês do calendário judaico (coincide normalmente com o mês
  de março).} no dia 15 de Adar, segundo o calendário judaico. Essa
festa celebra a libertação dos judeus da ameaça de extermínio total
tramada pelo grego Amã, um estrangeiro no império persa, que trezentos e
cinquenta e sete anos antes do Nascimento de Cristo tentara resolver
definitivamente a questão judaica: salvar a humanidade dos judeus e, ao
mesmo tempo, do Nascimento de Cristo, de modo que foi dito no decreto:
``{[}...{]} para que essas pessoas não nos impeçam, em tempos futuros,
de seguir a vida com tranquilidade e sem preocupações até o
fim''.\footnote{Ester 3:13g.}

No entanto, graças aos esforços da judia Ester,\footnote{Segundo Livro
  de Ester, ela revelara a Assuero, rei da Pérsia, que Amã era o inimigo
  dos hebreus. Então o rei ``{[}...{]} ordenou que Amã fosse enforcado
  na forca que mandara preparar para Mardoqueu'', tio de Ester
  (Hadassa). (\emph{Dicionário bíblico,} ed. Paulus, 1983, p. 309)} a
humanidade pacífica não se livrou do nascimento de Cristo. Quanto ao
próprio Amã, o grego libertador, ele foi enforcado por ordem do rei.
Assim fracassara a primeira conspiração grega contra o Cristo, que ainda
não havia nascido. Mas a segunda conspiração grega, realizada depois da
morte de Cristo, foi em parte cumprida. O cálice foi quebrado. E agora,
após os quatro flagelos do Senhor terem terminado, foi de novo uma
mulher que se opôs a essa conspiração: a profetisa Pelágia, da vila de
Brussiány, perto da Rjév, que deu à luz uma criança-sinal, concebida por
seu pai, o Anticristo, irmão de Jesus Cristo.

Essa criança, chamada Dã em homenagem ao pai, tinha os traços judeus do
lado paterno, mas os olhos nórdicos de Rjév de sua mãe. Como todos os
bebês saudáveis, ele estava no paraíso divino, no entanto já se
revelavam pequenos sinais --- perceptíveis apenas a Pelágia, sua
mãe-profetisa --- de que, ao abandonar o paraíso na infância, Dã se
destacaria de muitos e, ao tornar-se um adolescente, se destacaria de
todos. Ele passará rapidamente pela fase dos questionamentos e
acreditará em seu achado assim que o encontrar. Ele amará do fundo do
coração todos os profetas bíblicos, mas principalmente o profeta que
ficou para sempre desconhecido, condicionalmente incluído no livro do
profeta Isaías e chamado Segundo Isaías.\footnote{Corresponde aos
  capítulos 40 e 55 do Livro de Isaías, já que estes ``não podem ser
  obra do profeta do século VIII. {[}...{]} Esses capítulos contêm a
  pregação dum anônimo, continuador de Isaías {[}...{]} o qual na falta
  de um nome melhor, chamamos de Dêutero-Isaías ou de Segundo Isaías
  {[}...{]}''. (\emph{Bíblia de Jerusalém,} ed. Paulus, 2016, p. 1238)}
Por isso, a profetisa Pelágia agora lia com frequência o Segundo Isaías,
o profeta desconhecido, e cada palavra ardia sem se consumir, semelhante
à sarça de Moisés, como que assinalando a Divindade do sentido que ela
continha.

``Eis meu Servo que eu levo pela mão, eleito por mim, e que é simpático
à minha alma. Eu pus meu espírito sobre Ele, e Ele anunciará o
julgamento aos povos,''\footnote{Isaías 42:1, 2.} assim lia a profetisa
Pelágia, a mãe do bebê Dã. ``Por muito tempo eu me calei,'' ela lia,
``tive paciência e me contive, mas agora irei gritar como uma mulher em
trabalho de parto, irei a todos destruir e engolir.''\footnote{Isaías
  42:14.} ``Eu ofereci Meu dorso aos que me feriram e Minhas faces aos
que me golpearam, e não ocultei Meu rosto dos insultos e dos
escarros.''\footnote{Isaías 50:6.}

Assim falava o profeta desconhecido Segundo Isaías, quinhentos anos
antes da estrela de Belém. Foi também dito: ``E o Senhor Deus irá Me
socorrer, por isso Eu não Me sinto intimidado, por isso fiz do Meu rosto
uma rocha e sei que não serei humilhado. Perto está quem Me justifica.
Quem quer disputar comigo? Compareçamos juntos! Quem quer me questionar
em juízo? Que venha até Mim!''.\footnote{Isaías 50:7, 8.}

E há o curto capítulo 53 do Segundo Isaías, com apenas doze versículos.
Todo o espírito do Evangelho --- a dramaturgia do Evangelho e mesmo seu
enredo principal --- está contido nesse pequeno capítulo escrito pelo
Segundo Isaías quinhentos anos antes do Natal. Tudo o que há de criativo
no Evangelho se acha nesse capítulo. Ele só omite os adornos e o sentido
pagão que, mais tarde, através do tutor grego, foram usados para
rebaixar o Evangelho. Eis o Evangelho do Segundo Isaías, o mais antigo,
o principal, o mais poético --- não é um Evangelho-crônica, como todos
os outros, mas um Evangelho-profecia:

``Senhor! Quem acreditará no que ouvimos, e a quem se revelou o braço do
Senhor? Ele cresceu como um rebento diante Dele, como uma raiz de terra
árida; Ele não tinha aparência nem brilho para atrair nosso olhar. Ele
era desprezado e rejeitado pelos homens, um homem de dor e conhecedor do
sofrimento, e nós Lhe viramos o rosto. Ele era desprezado, e nós fizemos
pouco caso Dele. Ele tomou para Si nossas dores e carregou Consigo
nossas doenças, e nós O considerávamos derrotado, ferido e humilhado por
Deus. Mas Ele foi ferido por nossas transgressões e torturado por nossas
iniquidades; o castigo que nos traria paz caíra sobre Ele, e fomos
curados por Suas feridas. Todos nós andávamos sem rumo como ovelhas,
cada um seguia seu próprio caminho; mas o Senhor fez cair sobre Ele a
iniquidade de todos nós. Ele foi maltratado, mas sofria voluntariamente
e não abria a boca; como um cordeiro foi conduzido ao matadouro e, como
uma ovelha, ficou mudo diante dos tosquiadores; Ele não abriu a boca.
Foi preso depois de detido e julgado, mas quem se preocupará com Sua
sina? Ele foi separado da terra dos vivos, foi punido pelos crimes de
Seu povo. Colocaram Seu caixão com os ímpios e Seu sepulcro com os
ricos, ainda que não tivesse pecado, nem houvesse mentiras em Sua boca.
Mas o Senhor quis esmagá-lo e entregou-o ao sofrimento. Quando Sua vida
se fizer expiação do pecado, Ele verá descendentes, prolongará seus
dias, e por Suas mãos a vontade do Senhor se realizará. Graças aos
esforços de Sua alma, Ele verá a luz e se sentirá satisfeito; por Seu
conhecimento, Ele, Meu Servo, o Justo, justificará multidões e tomará
para Si suas transgressões. Eu Lhe darei Seu quinhão entre os grandes e
Ele repartirá os despojos entre os poderosos, já que Ele se despojou até
a morte e foi contado com os transgressores, quando, na verdade, Ele
carregou Consigo o pecado de muitos e intercedeu pelos transgressores.''

Esse é o Evangelho do Segundo Isaías, o único Evangelho profético.
Apesar de ele ser profético, ou seja, de ter sido escrito bem antes de
seus vaticínios se realizarem, ele traz, em essência, mais significação
do que os Evangelhos escritos consideravelmente depois da realização do
que foi predito. Em sua última frase, assinala-se quem é o Cristo:
aquele que intercede pelos criminosos, que são a maioria, e não pelas
vítimas.

Certamente, no mundo da filosofia, no mundo da Unidade, no mundo
espacial de conceitos gerais, o criminoso e a vítima são inseparáveis, e
é por isso que o Cristo dos filósofos intercede por todos. No entanto,
no mundo religioso, no mundo da Polaridade de conceitos essenciais, no
mundo móvel, temporal, bíblico, o criminoso, em cada momento concreto, é
separado nitidamente de sua vítima, e Cristo, na religião, aparece
somente como intercessor do criminoso. Pela vítima quem intercede é o
Anticristo. Eis por que, no mundo antigo espacial, o Cristo e o
Anticristo estão como que fundidos num só homem, pois, no mundo antigo,
a vítima não pode ser separada do criminoso.

No Segundo Isaías, não se fala apenas do Cristo, mas também do
Anticristo: ``E conduzirei os cegos por um caminho que eles não
conhecem, por veredas desconhecidas; diante deles Eu farei as trevas
virarem luz e os caminhos tortuosos virarem planos, eis o que lhes farei
e não os abandonarei''\footnote{Isaías 42:16.} --- assim dizia não
apenas o intercessor dos criminosos, o Cristo, mas também o intercessor
das vítimas, o Anticristo. ``Quando tu atravessares as águas, Eu estarei
contigo; quando passares pelos rios, eles não te engolirão; quando
andares pelo fogo, não te queimarás e a chama não te
alcançará''.\footnote{Isaías 43:2.}

Através do Anticristo, Seu enviado, o Senhor se dirige à vítima: ``Eu
sou o Senhor e, além de Mim, não há Salvador.\footnote{Isaías 43:11.}
{[}...{]} Assim diz o Senhor, teu redentor, e que te formou desde o
ventre materno. Eu sou o Senhor que criou todas as coisas, que sozinho
estendeu os céus e firmou a terra com Sua Própria Força. {[}...{]} Que
diz ao abismo: seca-te!''.\footnote{Isaías 44:24, 27.} E aos
perseguidores, cujo intercessor é o Cristo, diz o Anticristo, o
intercessor dos perseguidos: ``Eu alimentarei teus opressores com sua
própria carne, e eles se embriagarão pelo seu próprio sangue, como um
mosto.\footnote{Isaías 49:26.} {[}...{]} Eis que Eu tiro de tuas mãos o
cálice do atordoamento; a levedura do cálice da minha cólera, tu nunca
mais a provarás. Mas eu o colocarei nas mãos dos que te atormentavam
dizendo: Abaixa-te, para que andemos sobre ti! E tu fizeste tuas costas
como chão, rua para os passantes''.\footnote{Isaías 51:22, 23.}

Assim dizia o Anticristo, o pai da profetisa Pelágia e do filho dela, o
Anticristo que no mundo filosófico, o mundo da Unidade, é o inimigo do
Cristo, mas que no mundo religioso, o mundo da Polaridade, é o Irmão do
Cristo, seu complemento no justo julgamento de Deus. Assim a profetisa
Pelágia lia e compreendia o Segundo Isaías.

Quando seu filho cresceu um pouco, a profetisa Pelágia passou a levá-lo
com frequência para fora da cidade, para a região pouco habitada, cheia
de ravinas e com uma colina coberta pela floresta em que o pai de
Pelágia passara seus ensinamentos, tanto a ela como a seu irmão Andrei
Kopóssov. Não raro, ela era acompanhada por Andrei Kopóssov e Saviéli
Ívolguin, seu vizinho, que, conforme concluiu a medicina, estava
praticamente curado de sua doença do espírito. Com efeito, seu rosto
perdera aquele entusiasmo perigoso que revelava seu mundo interior
polifônico; ele se tornou menos solitário e começou a encarar as coisas
de modo mais confiante, sem mais imaginar que o mundo tramava algo
contra ele ou que lhe ocultava alguma coisa. E a questão do conhecimento
deixou de ser para ele uma questão marcada por contrários. Ele já sabia
que no mundo não havia Unidade, por isso a questão do conhecimento se
tornou parte da vida trivial, e não algo trágico e fatal, como ela seria
se houvesse uma unidade geral dos fenômenos e das noções. Ele também se
lembrava do principal preceito do desconhecido profeta, o Segundo
Isaías, que resumia suas revelações. Ele aparece no versículo 6 do
capítulo 55: ``Procurai o Senhor quando se pode achá-Lo, chamai o Senhor
quando Ele está perto''.

Assim, graças aos mais recentes tratamentos de saúde, graças a sua
renúncia à busca da unidade do mundo, conforme lhe ensinara o homem
morto do matraz, e graças ao grande mandamento do profeta, a alma de
Saviéli se aquietou, ele se tornou agradável ao convívio, e a profetisa
Pelágia convidava-o com prazer para passeios fora da cidade.

Quando chegou o primeiro aniversário da partida de seu pai, o
Anticristo, exatamente no dia do Nascimento de Cristo, a profetisa
Pelágia se preparou para ir ao campo com seu filho. Seus zelosos
companheiros de viagem, Andrei e Saviéli, foram com ela. Pelágia
agasalhou bem seu bebê, pois, embora nesse Natal não fizesse muito frio,
como fora no anterior, estava gelado e uma neve espessa caía sem cessar
desde manhã.

No inverno, especialmente no campo, num dia nevado, predominam duas
cores: o branco e o preto, pois, sobre fundo branco, tudo o que é escuro
parece negro. Por isso, ao se olhar para os troncos das árvores do lado
que, na noite anterior, foi varrido por um vento rasante e pela neve,
todos os troncos parecerão irmãos brancos, cobertos de neve, enquanto,
do outro lado, todos parecerão irmãos negros. E essas duas cores
conferem algo de sagrado à floresta invernal, e se anda por ela com a
alma entorpecida, como num templo de Deus. Há uma austeridade sagrada e
viril no branco e no preto que torna todas as outras cores secundárias e
apagadas. A floresta invernal sob a cúpula branca será divina enquanto o
sol não aparecer, enquanto não despertarem as cores terrenas e frívolas,
os reflexos femininos e alegres, o azul radiante no céu. Isso é bonito e
agradável, mas traz uma inquietação feminina. Revela-se por um instante
algo da alienação estival, doce ao corpo, da sensação frenética de perda
produzida pelo verão, quando sentimos a vida escapar, dia após dia. Mas
no bom inverno nevado do campo e dos subúrbios, Deus como que dá uma
trégua ao homem: diminui o desassossego da existência, reforça a
imobilidade, despersonaliza agradavelmente os dias, de modo que o homem
não sente os dias lhe fugirem. Mesmo entre os pássaros, as criaturas que
mais animam a natureza, predomina a ave negra, morosa e invernal, sobre
a neve branca --- o corvo e a gralha. O pintassilgo brilhante vem
voando, como uma pequena nuvem casual, fugidia, como um pigmento
feminino e vibrante em meio à alvura, como o profano num templo...

Com tais sentimentos, chegaram à floresta invernal: Andrei Kopóssov,
Saviéli Ívolguin e a profetisa Pelágia com seu filho Dã, no qual já
haviam despertado os primeiros germes da consciência, alegrando sua mãe,
apesar de sua expulsão do paraíso divino ainda estar longe. Eles
caminhavam, embrenhando-se na neve, em direção à colina onde, no outono
anterior, o pai de Pelágia e Andrei os instruíra. Olharam em volta e
tudo o que viram na Terra, no plano inferior, era sagrado, mas, no plano
superior, não havia nem o sol --- que era cultuado por adoradores, mas
que Abraão se recusava a cultuar ---, nem o céu --- ídolo pagão que hoje
foi obliterado pelo dia terreno sagrado do Natal ---, nem o espaço
ilimitado das estrelas --- provavelmente as principais responsáveis pelo
politeísmo, desviando por longo tempo, com sua beleza multifacetada, o
homem antigo do Deus Único que existe. Mas agora Ele estava próximo, era
o sagrado momento em que Ele poderia ser achado, pois a floresta negra
invernal ardia em sua alvura, como a sarça ardera no deserto diante do
pastor Moisés. Andrei Kopóssov, o filho do Anticristo, Saviéli Ívolguin,
o pecador-alquimista, e a profetisa Pelágia, a mulher do Anticristo,
ouviram a voz de Deus com mais clareza que nunca. No entanto, o pequeno
Dã, de fralda nos braços da profetisa e olhando, com seus olhos azuis
nórdicos de Rjév, para os galhos suspensos cobertos de neve, frescos e
perfumados, ouviu a voz do Senhor como se estivesse diante Dele e
apoiasse Sua mão em seu rosto. Claro que o que ele ouviu ficará guardado
por muito tempo em seu coração, porém, quando chegar a hora, tudo lhe
será revelado, se ele levar a vida que lhe fora destinada pelo Senhor e
que fora criada por seu pai.

No poema bíblico sobre a criação do mundo foi dito que o Senhor criou,
enquanto o homem inventou os nomes para a criação, pois, em virtude da
fraqueza humana, o Divino deve ser rebaixado através da palavra e do
nome, isto é, através da arte. Da mesma forma, as ideias insondáveis do
Senhor, para que se tornem acessíveis ao homem, devem ser rebaixadas
através da grande palavra do profeta. Mas a palavra do profeta também é
muitas vezes rebaixada se não vem com um sinal, como o sinal que eles,
nesse instante, receberam da floresta invernal sagrada. E disse o
profeta Isaías: ``Pede um sinal ao Senhor, pede seja nas profundezas,
seja nas alturas''.\footnote{Isaías 7:11.} Eis o que disse a eles o
Senhor na floresta invernal sagrada, por meio do profeta Isaías: ``Se o
ímpio é favorecido, ele não aprende a justiça. {[}...{]} O Senhor sai de
Sua morada para punir os habitantes da terra por sua iniquidade, e a
terra descobrirá o sangue que ela absorveu e não mais esconderá seus
mortos''.\footnote{Isaías 26:10, 21.}

Na floresta invernal, eles entenderam que o malfeitor só pode confiar no
Cristo e que por Cristo ele será perdoado e consolado, com o sangue
vertido por Ele. Mas o Senhor não o perdoará, pois o Cristo é o
Salvador, enquanto o Senhor é o Criador.

Qualquer vida e qualquer destino, mesmo uma vida amarga e um destino
cruel, à medida que se desenrola, deve compor um Salmo. Um louvor ao
Senhor por ela ter se realizado, à diferença das vidas que não nasceram
e dos destinos que não se realizaram. Qualquer vida, mesmo amarga, é uma
sorte e um privilégio. Por isso, simplesmente com seu nascimento, o
malfeitor, o renegado, engana o Criador. Já o Cristo Salvador tem o
coração puro, pois é puro o coração de quem desconhece o tormento da
criação; o Cristo foi enviado pelo Senhor para não abandonar aqueles que
Ele mesmo, o Criador, abandonou. A essência do Senhor é infinita e
inalcançável ao homem, mas nesse infinito existe apenas um aspecto do
Senhor, talvez não o principal nem o mais importante, que é acessível à
compreensão humana: o aspecto Criador.

--- Eis que virão os dias --- disse o Senhor através de Amós, o mais
antigo dos profetas, o fundador da profecia ---, eis que virão os dias
em que enviarei fome à terra; não fome de pão, nem sede de água, mas
sede de ouvir as palavras do Senhor.\footnote{Amós 8: 11.}

Esses tempos se aproximam, e a fome da Palavra do Senhor talvez seja o
mais terrível flagelo do Senhor, o quinto, que foi anunciado pelo
profeta Amós, assim como os quatro flagelos anteriores foram anunciados
pelo profeta Jeremias. O ímpio foi perdoado através de Cristo, foi salvo
dos quatro flagelos: do primeiro --- a espada ---, do segundo --- a fome
---, do terceiro --- o animal selvagem do adultério ---, e do quarto ---
a doença, a peste. Mas do quinto flagelo, a sede e a fome da Palavra do
Senhor, nenhum ímpio se salvará, nem Cristo, intercessor dos criminosos,
o salvará. Da fome da Palavra do Senhor e da sede da consolação do
Senhor, o ímpio morrerá atormentado. Em compensação, o justo se saciará
da Palavra do Senhor. E foi dito no Livro do profeta Isaías:

--- E será assim: antes de me chamarem, Eu responderei; enquanto
estiverem falando, Eu já os terei compreendido {[}...{]}\footnote{Isaías
  65:24.}

E foi dito:

--- Sedentos! Vinde às águas, e mesmo vós que não tendes dinheiro,
vinde, comprai e comei {[}...{]}\footnote{Isaías 55:1.} Aguçai vossos
ouvidos e vinde a Mim, escutai e vossa alma viverá; Eu vos darei o
preceito perpétuo, as graças imutáveis prometidas a Davi
{[}...{]}\footnote{Isaías 55:3.} Como a chuva e a neve que caem do céu
sem nunca retornar, mas regam a terra e a tornam capaz de reproduzir e
de germinar, dando a semente a quem semeia e o pão a quem come, assim
será a Minha Palavra que sair da Minha boca; ela não voltará em vão para
Mim, antes cumprirá o que Me convém, realizará o motivo de Eu a ter
enviado.\footnote{Isaías 55:10, 11.}

O gênio repete o gênio, e o Livro do profeta Isaías repete o
Deuteronômio de Moisés, em que foi dito sobre a Palavra divina:

--- Que Minha Doutrina caia como a chuva, Minha fala como o orvalho,
como o chuvisco sobre a folhagem, como a tempestade sobre a relva
{[}...{]}\footnote{Deuteronômio 32:2.}

Eles entenderam através do sinal --- as árvores negras da floresta
ardendo sobre o branco nevado sagrado --- que, após os quatro pesados
flagelos do Senhor, virá o quinto, a sede e a fome da Palavra de Deus, e
somente um trabalhador espiritual poderá lembrá-la ao mundo e dela o
salvar, matando a sede do mundo, alimentando-o da Palavra divina. E
também compreenderam a essência do grito do coração do profeta Isaías:

--- E vós, que lembrais o Senhor, não fiqueis em silêncio!\footnote{Isaías
  62:6.}

\emph{Outubro, novembro, dezembro de 1974,}

\emph{janeiro de1975}
